\documentclass[runningheads,evcountsame,a4paper,11pt,orivec]{llncs}
\usepackage[margin=2.5cm,includefoot]{geometry}

%=================Begin:Packages===================
\usepackage{graphicx}
\usepackage{bmpsize}
\usepackage{algorithm}

\usepackage[noend]{algpseudocode}
% \usepackage{enumitem}
\usepackage{url}
%\usepackage{amsthm}

\usepackage{amsfonts}
\usepackage{amsmath}
\usepackage{hyperref}
% \usepackage[capitalize,nameinlink]{cleveref}
% \usepackage{framed}
% \usepackage{fancybox}
\usepackage[utf8]{inputenc}
\usepackage{mathtools}
% \usepackage{tcolorbox}
\usepackage{tikz}
% \usepackage{todonotes}
\usepackage{xspace}
\usepackage{xcolor}
\usepackage[advantage,
adversary,
asymptotics,
complexity,
ff,
mm,
operators,
probability,
]{cryptocode}
\usepackage[linewidth=1pt]{mdframed}
\usepackage{ulem}
\usetikzlibrary{arrows,chains,matrix,positioning,scopes}

% TODO: Clean up macros

% \newcommand\doubleplus{+\kern-1.3ex+\kern0.8ex}
\newcommand\doubleplus{\ensuremath{\mathbin{+\mkern-10mu+}}}


%% oracles
\newcommand{\ora}[1]{\ensuremath{\mathcal{O}\mathsf{#1}}\xspace}
\newcommand{\oramsg}[1]{\ensuremath{\mathtt{#1}}\xspace}
%% algorithm
\newcommand{\algo}[1]{{\textsc{#1}}}
%%primitive algo
\newcommand{\primalgo}[1]{{\ensuremath{\mathsf{#1}}}\xspace}
%%primitive
\newcommand{\prim}[1]{{\ensuremath{\mathsf{#1}}}\xspace}
%%set
\newcommand{\setsym}[1]{{\ensuremath{\mathcal{#1}}}}
%%array
\newcommand{\arraysym}[1]{{\ensuremath{\mathsf{#1}}}}

\newcommand\N{\mathbb{N}}
\newcommand\F{\mathbb{F}}
% \newcommand\Gr{\mathbb{G}}


\def\mathperiod{.}
\def\mathcomma{.}



\newcommand*\set[1]{\{ #1 \}} % in text, we don't want {} to grow
\newcommand*\Set[1]{\left\{ #1 \right\}}
\newcommand*\setst[2]{\{ #1 | #2 \}}
\newcommand*\Setst[2]%
        {\left\{\,#1\vphantom{#2} \;\right|\left. #2 \vphantom{#1}\,\right\}}
% ``set such that''; puts in a vertical bar of the right height


\section{Primitives}
\label{sec:lambda}

\def\ecE{{\mathbb{E}}}
\def\grE{{\mathbf{E}}}
\def\genE{E}
\def\genG{G}
\def\genB{K} %{\genE_{\mathrm{bind}}}

\def\ecJ{{\mathbb{J}}}
\def\grJ{{\mathbf{J}}}
\def\genJ{J}

% As our ring VRF is built by composing them, 
We briefly recall the primitives and security assumptions underlying both Chaum-Pederson proofs and pairing based zkSNARKs. 


\subsection{Elliptic curves}

We obey mathematical and cryptographic implementation convention by using additive notation for elliptic curve and multipicative notation for eliptic curve scalar multiplications. 

We always implicitly have a paramater generation procedure $\mathtt{Params}$ that takes a security level $\lambda \in \N$ and returns elliptic curve paramaters including some prime numbers and efficient algorithms for computing elliptic curve operations.  As customary, we treat $\lambda$ and the output of $\mathtt{Params}$ as fixed paramaters, which makes sense because humans run $\mathtt{Params}$ manually in practice. 

As implicit outputs of $\mathtt{Params}$, we work with an elliptic curve $\ecE[\F]$ over some base field $\F$ of (prime) characteristic $q_{\grE}$, along with a distinguished subgroup $\grE \le \ecE[\F]$ of prime order $p_{\grE} \approx 2^{2\lambda}$.  As $\grE$ distinguishes $\ecE[\F]$, we let $h_{\grE}$ denote the cofactor of $\grE$ in $\ecE[\F]$, meaning $\ecE[\F]$ has $h_{\grE} p_{\grE}$ points.
% but abbreviate $h = h_{\grE}$, $p = p_{\grE}$, and $q = q_{\grE}$ when $\grE$ is clear from context.
We write $\grE$ without subscript, and abbreviate $h = h_{\grE}$, $p = p_{\grE}$, and $q = q_{\grE}$, when $\ecE$ is either our uinque pairing friendly curve or else the only curve in view.

We let $H_p : \{0,1\}^* \to \F_p$ or $H_q : \{0,1\}^* \to \F_q$ denote random oracles (RO) with a range $\F_p$ or $\F_q$.  We let $H_\ecE : \{0,1\}^* \to \ecE$ or $H_\grE : \{0,1\}^* \to \grE$ denote a hash-to-curve for $\ecE$ or hash-to-group for $\grE$, which we model as a random oracle too.  We note $H_\grE(x) = h H_\ecE(x)$ always works, although more efficent exist.

\smallskip

Almost all SNARKs like \cite{groth16} or \cite{plonk} work on a pairing friendly elliptic curve $\ecE$ over a field $\F_q$ of characteristic $q \approx 2^{2\lambda}$, which comes equipped with a type III pairing on subgroups of prime order $p \approx 2^{2\lambda}$:  We let $q_1,q_2,q_T$ denote small powers of $q$, and let $\grE_1 \le \ecE[\F_{q_1}]$ and $\grE_2 \le \ecE[\F_{q_2}]$ and $\grE_T \le \F_{q_T}^\times$ denote subgroups all of prime order $p$.  We also let $e : \ecE_1 \times \ecE_2 \to \ecE_T$ denote a type III pairing, meaning a computable bilinear map without known efficiently computable maps between $\ecE_1$ and $\ecE_2$.  Also $q_i = q_{\grE_i}$ for $i=1,2$ in our above notation.  

Any pairing friendly elliptic curve $\ecE$ provides solutions to the decisional Diffie-Hellman problem (DDH).  We do however assume the computational Diffie-Hellman problem (CDH) remains hard in $\ecE$.  We remark that $H_\grE$ being a random oracle plus CDH hardness prevents computable relationships between $H_\grE$ outputs.

% TODO: Pairing assumptions required by Groth16

\smallskip

% We shall require ZCash Sapling style ``Jubjub'' Edwards curves, whose base field characteristic divides of the order of a pairing friendly elliptic curve, thereby making Jubjub base field arithmetic SNARK friendly, and hence Jubjub elliptic curve operations as well \cite{}.

We let $\ecJ$ denote a ZCash Sapling style ``JubJub'' Edwards curve associated to the pairing friendly elliptic curve $\ecE$, meaning $\ecJ$ has base field $\F_p$ whose characteristic $q_{\grJ} = p$ matches the group order $p$ of $\grE_1 \cong \grE_2 \cong \grE_T$.  As in ZCash Sapling, we now prove statements about elliptic curve operations inside $\ecJ$ by proving base field arithmetic in $\F_p$, which our $q_{\grJ} = p$ makes reltively inexpensive inside SNARKs on $\ecE$.

As above, $\grJ \le \ecJ[\F_p]$ has large prime order $p_{\grJ}$ and a small cofactor $h_{\grJ}$.  We always support $4 p_{\grJ} < p$ because if $\ecJ$ is an Edwards curve then $h_{\grJ} \ge 4$ which imposes this by the Hasse bound.

\smallskip

We ask that deserialization prove that putative elements of $\grE$ lie in $\ecE[\F]$ by verifying curve equations, perhaps including twist checks.  We do sometimes require that deserialization checks membership in the prime order subgroup $\grE$ by checking $|\grE| X = 1$ or similar.  Yet, we also sometimes work in $\ecE[\F]$ directly when judicious multiplications by $h_{\grE}$ suffice.  

Anytime $\ecE$ represents a pairing friendly curve then we do ask that deserialization prove elements of $\grE_1$, $\grE_2$, and $\grE_T$ lie inside the correct subgroup of order $p$.  As our SNARKs handle with points in $\ecJ$ directly, we prefer writing $\grJ$ equations in $\ecJ[\F_p]$ and explicitly describe where one clears the cofactor $h_{\grJ}$.  We handled $\grE$ with $\ecE$ not necessarily pairing friendly similarly to $\ecJ$.  We scrape by with only CDH hardness for $\grJ$ for convenience, although DDH winds up hard in $\grJ$.



\subsection{zkSNARKs}

We require details about Groth16 \cite{groth16} becuase our zero-knowledge continuations demand rerandomizing existing zkSNARKs. % which only Groth16 supports.

We thus have an implicit setup procedure $\mathtt{Setup}$ that take our paramaters generated by $\mathtt{Params}$ and produces a structured reference string (SRS).  In practice, our SRS consists of points in $\ecE_1$ and $\ecE_2$ with specific relationships.  We suppress both $\mathtt{Setup}$ and $\mathtt{Params}$ whenever convenient, but $\mathtt{Setup}$ makes an appearance in \S\ref{sec:srs_second_stage}.

TODO: AGM

TODO: Describe Groth16 \cite{groth16} 

































\endinput



BROKEN BOLOW THIS




We fix $J \in \ecJ$ as a generator for public keys.  Any $\KeyGen$ algorithm randomly samples a secret keys $\sk \in \F_q$ and then computes its associate public keys $\pk = \sk J$.  We shall not discuss infrastructure that authorizes public keys.  Yet although our results do not require proof-of-knowledge on $\pk$ per se, we still strongly recommend that back certifications accompany any certificates that authorize $\pk$.

\smallskip






\newcommand{\tab}[1]{\hspace{.03\textwidth}\rlap{#1}}
\newcommand{\tabdbl}[1]{\hspace{.05\textwidth}\rlap{#1}}
\newcommand{\tabdbldbl}[1]{\hspace{.07\textwidth}\rlap{#1}}
\newcommand{\tabdbldbldbl}[1]{\hspace{.19\textwidth}\rlap{#1}}


\newcommand{\Gen}{\ensuremath{\mathsf{Gen}}}

\newcommand{\anonymouskeymap}{\ensuremath{\mathtt{anonymous\_key\_map}}}
\newcommand{\anonymouskeylist}{\mathcal{W}}
\renewcommand{\sim}{\simulator}
\newcounter{FunCond}
\newcommand{\game}[3][]{\operatorname{#2}^{#1}_{#3}(\secpar)}
\newcommand{\transcript}[1]{\langle #1 \rangle}
\newcommand{\eppt}{\pccomplexitystyle{EPPT}}
\newcommand{\pt}{\pccomplexitystyle{PT}}

% \renewcommand{\pcadvstyle}[1]{\ensuremath{\mathsf{#1}}}
% \newcommand{\zdv}{\pcadvstyle{Z}}

% \newcommand{\msg}[1]{\mathsf{#1}}

\newcommand{\simulator}{\ensuremath{\mathsf{Sim}}}
%\newcommand{\minote}[1]{\todo[color=green!30,inline]{\textbf{Michele says:} #1}}

\newcommand{\fvrf}{\mathcal{F}_{\textsf{vrf}}}
\newcommand{\fgvrf}{\mathcal{F}_{\textsf{rvrf}}}
\newcommand{\fcpke}{\mathcal{F}_{\mathsf{CPKE}}}
\newcommand{\pvrf}{\mathsf{\Pi}_{\textsf{rvrf}}}
\newcommand{\svrf}{\simulator_\mathsf{gvrf}}
\newcommand{\fnizk}{\mathcal{F}_{\textsf{nizk}}}
\newcommand{\fkes}{\mathcal{F}_{\textsf{sgke}}}
\newcommand{\fcom}{\ensuremath{\mathcal{F}_{\mathsf{com}}}}
\newcommand{\fsec}{\ensuremath{\mathcal{F}_\mathsf{ED-SMT}}}
\newcommand{\frsc}{\ensuremath{\mathcal{F}_{\mathsf{rSC}}}}
\newcommand{\fsasle}{\ensuremath{\mathcal{F}_{\mathsf{sle}}}}
\newcommand{\finit}{\ensuremath{\mathcal{F}_{\mathsf{init}}}}
\newcommand{\fsig}{\mathcal{F}_{\mathsf{sig}}}
\newcommand{\fros}{\mathcal{F}_{\mathsf{ros}}}
\newcommand{\fzkvrf}{\mathcal{F}_{\mathsf{zkvrf}}}
\newcommand{\fcommit}{\mathcal{F}_{\mathsf{commit}}}
\newcommand{\gclock}{\mathcal{G}_{\mathsf{clock}}}
\newcommand{\fcrs}{\mathcal{F}_{crs}}
\newcommand{\env}{\mathcal{Z}}
\newcommand{\stake}{\mathsf{st}}
\newcommand{\stakeset}{\setsym{ST}}

\newcommand{\sid}{\textsf{sid}}
\newcommand{\pid}{\textsf{pid}}
\newcommand{\user}{\mathsf{P}}
\newcommand{\defeq}{\coloneqq}


\newcommand{\evaluationslist}{\texttt{evaluations}}
\newcommand{\evaluationsecretlist}{\texttt{secrets}}
\newcommand{\vklist}{\texttt{signing\_keys}}
\newcommand{\siglist}{\texttt{signatures}}
\newcommand{\prooflist}{\texttt{proofs}}
\newcommand{\proofzklist}{\texttt{zkproofs}}
\newcommand{\Linklist}{\texttt{links}}
\newcommand{\emptylist}{\emptyset}
\newcommand{\fail}{\mathbf{fail}}
\newcommand{\R}{\mathsf{R}}
\newcommand{\bool}{\textit{bool}}
\newcommand{\lst}{\setsym{L}}
\newcommand{\distribution}{\setsym{D}}

\newcommand{\weak}{\ensuremath{W}}
\newcommand{\inbox}{\ensuremath{\setsym{I}}}
\newcommand{\dqueue}{\ensuremath{\setsym{Q}^\D}}
\newcommand{\wqueue}{\ensuremath{\setsym{Q}^\weak}}
\newcommand{\weaklist}{\ensuremath{\setsym{\weak}}}
\newcommand{\mID}{\ensuremath{\mathsf{mid}}}
\newcommand{\plist}{\ensuremath{\setsym{P}}}
\newcommand{\timeoutlist}{\ensuremath{\setsym{T}}}
\newcommand{\anony}{\ensuremath{\mathfrak{a}}}
\newcommand{\dleqr}{\R_\textsf{dleq}}
\newcommand{\view}{\mathsf{view}} 
\newcommand{\preoutputlist}{\arraysym{pre\-outputs}}




\sloppy


\title{Ethical identity, ring VRFs, and zero-knowledge continuations}

\author{Jeffrey Burdges \and Handan Kilinc-Alper \and Alistair Stewart \and Sergey Vasilyev}
\date{}
\institute{Web 3.0 Foundation}


\begin{document}
	
\maketitle

\begin{abstract}
Verifiable random functions (VRFs) weld a pseudorandom function (PRF) into a signature scheme, so that valid signatures prove correct PRF evaluation.  VRFs have diverse applications in consensus, protocols, randomness beacons, DNS via NSEC5, certificate transparency, and online games.

Anonymized ring VRFs or group VRFs built from ring or group signatures prove correct evaluation of an authorized signer's PRF while hiding the signer's identity.
As a natrual application, anonymized VRF outputs act like a unique pseudonym for its signer within whatever context the input determines, which then remains unlinkable vs other inputs.  These unlinkabile but unique pseudonyms provide an excellent balance between user privacy and service provider interests \cite{pop2008}.  Identity systems lacking this unlinkability property like OAuth wind up being unethical.
% TODO: ``one person, one persona'' \cite{pop2008} https://bford.info/pub/net/sybil.pdf

We discuss ring VRFs because they integrate with distributed systems more cleanly than threshold designs required by  group signatures and hence group VRFs.  As a rule, ring signatures require signer time logarithmic in the ring size, while group signatures take only constant signer time.  
%
We explore a continuation abstraction for composing rerandomizable zkSNARKs like \cite{groth16}, which dramatically improves {\it amortized} constraint count and hence prover time.  We implement a ring VRF with an amortized constraints count below 800 (TODO), which makes ring VRFs more competitive with group VRFs, and makes distributed identity systems less impractical.  
\end{abstract}


% \section{Introduction}

\def\qaudbreak{\eprint{\quad}{\\}}

% A {\it ring verifiable random function} (ring VRF) is a ring signature that proves correct evaluation of some pseudo-random function (PRF) determined by the actual key pair used in signing. 

We introduce ring verifiable random functions (ring VRFs) as a natural
fulcrum around which anonymous credentials turn, in formalization,
in optimizations, in the nuances of use-cases, and in miss-use resistance.
%
Along with some formalizations, we explain portions of their unfolding
story which address three questions:
\begin{enumerate} 
\item
What are the cheapest SNARK proofs?  \qaudbreak
Ones users reuse without reproving.
% \item
% How can credentials use be contextual?  \qaudbreak
% Prove evaluation of a secret function.
\item
How can identity be safe for general use?  \qaudbreak
By revealing nothing except users' uniqueness.
\item
How can ration card issuance be transparent?  \qaudbreak
By asking users trust a public list, not certificates.
\end{enumerate}

% First
\paragraph{Zero-knowledge continuations:}

Rerandomizable zkSNARKs like Groth16 \cite{Groth16} admit a
transformation of a valid proof into another valid but unlinkable
proof of the exact same statement.  In practice, rerandomization
was never deployed because the public inputs link the usages.

We demonstrate in \S\ref{sec:rvrf_cont} a simple transformation of
any Groth16 zkSNARK into a {\it zero-knowledge continuation} whose
public inputs become opaque Pedersen commitments, with cheaply
rerandomizable blinding factors and proofs.
These zero-knowledge continuations then prove validity of the contents
of Pedersen commitments, but can now be reused arbitrarily many times,
without linking the usages. 

As recursive SNARKs shall remain extremely slow,
we expect zero-knowledge continuations via rerandomization become
essential for zkSNARKs used outside the crypto-currency space.

% \smallskip 
\paragraph{Ring VRFs:}

A ring signature proves only that its actual signer lies in a ``ring''
of public keys, without revealing which signed.
A {\it verifiable random function} (VRF) is a signature that proves
correct evaluation of a PRF defined by the signer's key.
% but nuances exist .

A {\it ring verifiable random function} (ring VRF) is a ring signature, in
that it anonymizes its actual signer within a ring of plausible signers,
but also proves correct evaluation of a pseudo-random function (PRF)
defined by the actual signer's key. % (see \S\ref{sec:rvrf_games}).
%
Ring VRF outputs provide linking proofs between different signatures iff
the signatures have identical inputs, as well as pseudo-randomness.

We build extremely efficient and flexible ring VRFs by amortizing a
zero-knowledge continuation that unlinkably proves ring membership
of a secret key, and then cheaply proving individual VRF evaluations.

As the PRF output is uniquely determined by the signed message and
signer's actual secret key, we can therefore link signatures by the
same signer if and only if they sign identical messages.
In effect, ring VRFs restrict anonymity similarly to but less than
 linkable ring signatures do, which makes them multi-use and contextual.

We define the security of  ring VRFs in both the standard model
and in the universally composable (UC) \cite{canetti1,canetti2} model. 
We show that our ring VRF protocol is secure in the UC model.

% Second
% \smallskip
\paragraph{Identity uses:}

As an identity system, ring VRFs evaluated on a specific context or
domain name output a unique identity for the signer at that domain or
in that context (see \S\ref{sec:app_identity}).
In this way, they permit banning specific users and prevent Sybil behavior.
Yet users' activities remain unlinkable across distinct contexts or
domains, meaning correctly deployed ring VRF based credentials
only prevent users being Sybil, nothing more.  We argue this yields
diverse legally and ethically straightforward identity usages.

As a problematic contract, attribute based credential schemes like
IRMA (``I Reveal My Attributes'') credentials \cite{IRMAcredentials}
are being marketed as an online privacy solution, but they cannot prevent
users being Sybil if they reveal numerous attributes, meaning they provide little or no privacy.
IRMA might improve privacy in narrow situations of course, but
overall attribute based credentials should {\it never} be considered
fit for general purpose usage, aka the prevention of Sybil behavior.

Aside from general purpose identity, our existing offline
verification processes often better protect user privacy and human
rights than adopting online processes like IRMA.
%
In particular, there are many proposals by the W3C for attribute based
credential usage in \cite{w3c_vc_use_cases}, but broadly speaking they
all bring matching harmful uses.  % https://www.w3.org/TR/vc-use-cases/
As an example, if users could easily prove their employment online, which
the W3C claims helps users apply for bank accounts, then job application
sites could similarly demand proof of current employment, a clear injustice.
Average users apply for jobs far more often than they open bank accounts.

In general, abuse risks like this dictate that IRMA verifiers should be
tightly controlled by legislation, which becomes difficult internationally. 
%
Ring VRFs avoid these abuse risks by being truly unlinkable, and thus
yield anonymous credentials which safely avoid legal restrictions.

{\it Any ethical general purpose identity system should be based
upon ring VRFs, not attribute based credentials like IRMA.}

We credit Bryan Ford's work on proof-of-personhood parties \cite{pop2008,pop2017}
% https://bford.info/pub/dec/pop-abs/  https://bford.info/pub/net/sybil-abs/
with first espousing the idea that anonymous credentials should produce
contextual unique identifiers, without leaking other user attributes.

As a rule, there exist simple VRF variants for all anonymous credentials,
including IRMA \cite{IRMAcredentials} or group signatures \cite{group_sig_survey}.
We focus exclusively upon ring VRFs for brevity, and because alone
ring VRFs contextual linkability covers the most important use cases.
% and our optimizations make ring VRFs extremely efficient.

% Third
% \smallskip
\paragraph{Rationing uses:}

Ring VRFs yield rate limiting or rationing systems, which work
similarly to identity applications, except their VRF inputs should also
include an approximate date and a bounded counter, and
 then their outputs should be tracked as nullifiers.
Yet, these nullifiers need only temporarily storage, which improves 
efficiency over anonymous money schemes like ZCash and blind signed tokens.

We expect a degree of fraud whenever deploying purely certificate
based systems, as witnessed by the litany of fraudulent TLS and covid
certificates.  Ring VRFs help mitigate fraudulent certificate concerns
because the ring is a database and can be audited.

We know governments have ultimately little choice but to institute
rationing in response to shortages caused by climate change, ecosystem
collapse, and peak oil.  Ring VRFs could help avoid ration card fraud,
and thereby reduce social unrest, while also protecting essential privacy.

Ring VRFs need heavier verifiers than single-use token credentials
based on OPRFs \cite{PrivacyPass} or blind signatures.
Yet, ring VRFs avoid these schemes separate issuance phase entirely,
and sometimes even their registration phase.  Instead, fresh tokens
merely require adjusting the approximate date in the VRF input.
This reduces complexity, simplifies scaling, and increases flexibility.

In particular, if governments issue ration cards based upon ring VRFs
then these credentials could safely support other use cases, like
free tiers in online services or games, and advertiser promotions,
as well as identity applications like prevention of spam and online abuse.

In this, we need authenticated domain separation of products or identity
consumers in queries to users' ring VRF credentials.  We briefly discuss
some sensible patterns in \S\ref{subsec:app_ration_carts} below, but
overall authenticated domain separation resemble TLS certificates except
simpler in that roots of trust can self authenticate if root keys act as
domain separators.

% \paragraph{Outline:}




\endinput




As a field, anonymous credentials come in myriad flavors,
many of which exist to limits the anonymity provided, ala
 attribute based credentials and group signatures. % \cite{group_sig_survey}.
% aka anonymized signatures
%
Ring VRFs by weakening anonymity only contextually provide a safer,
more private, more flexible, more powerful, and more ethical
choice for all everyday anonymous credential use cases.  % needs:  ???



% 
\section{Identity}

% “We can judge our progress by the courage of our questions and the depth of our answers, our willingness to embrace what is true rather than what feels good.” 
% - Carl Sagan

% https://twitter.com/IdentityZack/status/1480631954689216516

% bryan ford https://twitter.com/brynosaurus/status/1460094634567344133

% answer https://twitter.com/valkenburgh/status/1442894421289103361
% https://twitter.com/harryhalpin/status/1443053685219725315
% https://twitter.com/OR13b/status/1442964741022830594
% https://twitter.com/jeffburdges/status/1443539630033362948
% https://twitter.com/Steve_Lockstep/status/1448653579330342916

% https://github.com/dckc/awesome-ocap/issues/17

% https://twitter.com/smdiehl/status/1459825936757493770

% https://twitter.com/edri/status/1483818492646281225

% reply https://twitter3e4tixl4xyajtrzo62zg5vztmjuricljdp2c5kshju4avyoid.onion/matthew_d_green/status/1511437992740761601
%   https://twitter3e4tixl4xyajtrzo62zg5vztmjuricljdp2c5kshju4avyoid.onion/RaulACarrillo/status/1545038906776764416
%   https://twitter3e4tixl4xyajtrzo62zg5vztmjuricljdp2c5kshju4avyoid.onion/rohangrey/status/1511808597742612481
%   https://twitter3e4tixl4xyajtrzo62zg5vztmjuricljdp2c5kshju4avyoid.onion/coloradotravis/status/1545252419671572481
%   https://twitter3e4tixl4xyajtrzo62zg5vztmjuricljdp2c5kshju4avyoid.onion/RaulACarrillo/status/1545038906776764416/retweets/with_comments

% https://twitter3e4tixl4xyajtrzo62zg5vztmjuricljdp2c5kshju4avyoid.onion/PeterSweden7/status/1552729179568865280

% Zeroth law:  A robot may not harm humanity, or, by inaction, allow humanity to come to harm.
% First law:  A robot may not injure a human being or, through inaction, allow a human being to come to harm.


An identity system must not harm humanity or its human users, to do otherwise is clearly unethical.  

Identity systems for human users have three participants, an identity provider, an identity consumer, and the user being identified.  There exist two methods by which ethical identity systems avoid harming users, either 
\begin{itemize}
\item special identity systems enforce that identity consumers owe users some legal duty that prevents miss-using the user's details, or else
\item general identity systems merely constrain user activity, often only rate limiting, but avoid providing identity consumers with any user details.
\end{itemize}
In other words, identity consumers should always first prove to the identity provider that they owe the user a legal duty appropriate to the details being revealed by the identity provider.

\subsection{Legal duties}

In this paper, we discuss only cryptographic protocols for general identity systems that avoid legal entanglements by only proving user uniqueness and not providing user details.  We first in this section briefly discuss wider examples that help motivate this problem by clarifying the legal and ethical complexities that arise when revealing user details.

As an unethical example, our largest advertising companies like Google and Facebook track private users using OAuth \cite{oauth}, with the intent to waste users time with increased advertising engagement, manipulate public opinion, ensnare users into unnecessary purchases, often by harming users' psyche, and accumulate personal data users might otherwise wish kept hidden.

As an only moderately harmful example, websites often prevent abuse by demanding commenters identify themselves by email address, which creates moral hazards and should expose the website operators to legal risks.

As beneficial identity examples, financial institutions act as an identity provider for their own identity consumer logic by issuing login credentials, but then owe their customers some fiduciary duty and strongly discourage using the same login credentials elsewhere.  

As a more nuanced example, an employer identifies employees to a personnel management service by way of an external OAuth service, but the employer has some legal relationship with the personnel management service, the OAuth service, and the employee, so any resulting harms rest upon the employer-employee relationship.  

We think Google Single Sign-on or Facebook Connect cannot play the role of OAuth service even in this employer-employee example, and indeed cannot ever be used ethically, because they aggressively track the employee outside the employer-employee relationship.  At the same time, an employees' Github account might or might not serve this role depending upon the specific employee and how they use Github outside work.  

... passports or medical ...

\subsection{Unlinkable identity}

We now lay aside such identity systems that represent a distinguished purpose tied to onerous three-way legal relationships between the parties.  Instead we turn our attention towards the range of identity systems that avoid providing any user details.  

At present, CAPTCHAs provide a popular defense against automated abuse.  There also exist cryptographic tools that amplify defenses against automated abuse, like blind signatures or verifiable oblivious pseudo-random functions (VOPRFs), as used in Privacy Pass \cite{privacypass}.  These dispense single-use tokens within some limits imposed by other identity sources, rate limits, payments, or CAPTCHAs.  

We think single-use tools like CAPTCHAs, blind signatures, and VOPRFs adequately deter abuse in most use cases.  Yet, there also exist situations where abusers cannot be dissuaded by solving another CAPTCHAs or spending another token, like when abuse takes a personal character, or due to a larger profit motive.  

In such harder cases, we still need an anonymous credential so that identity consumers and providers cannot collude to track users, but identity consumers banning problematic users seemingly demands that users have different stable identities with each distinct identity consumer.  
To our knowledge, this identity formulation originates with proof-of-personhood parties \cite{pop2008,pop2017}.
% https://bford.info/pub/dec/pop-abs/
% https://bford.info/pub/net/sybil-abs/

We expect stable identities arise from multi-use anonymous credentials, like group signatures or ring signatures.  In group signatures, an identity provider holds a group manger secret key, with which they both issues credentials and deanonymize users.  We only want identity consumers to recognize returning users, making the deanonymization operation unacceptable.  

Ring signatures have classically given signers' control over their anonymity set aka ``ring'', which turns out mostly useless in practice.  Instead, realistic ring signatures like Zcash's circuits \cite{zcash_prorocol} have a shared public commitment to their ``ring'', so then users need only an opening for their own public key's presence in the ring. 



% sharing economy 
% business-to-business 



%   We think identity consumers should avoid imposing unnecessary constraints upon users and that rate limiting tools usually suffice.  Yet, there exist identity consumers who depend upon stronger Sybil defenses or an ability to ban problematic users.   


\section{Primitives}
\label{sec:lambda}

\def\ecE{{\mathbb{E}}}
\def\grE{{\mathbf{E}}}
\def\genE{E}
\def\genG{G}
\def\genB{K} %{\genE_{\mathrm{bind}}}

\def\ecJ{{\mathbb{J}}}
\def\grJ{{\mathbf{J}}}
\def\genJ{J}

% As our ring VRF is built by composing them, 
We briefly recall the primitives and security assumptions underlying both Chaum-Pederson proofs and pairing based zkSNARKs. 


\subsection{Elliptic curves}

We obey mathematical and cryptographic implementation convention by using additive notation for elliptic curve and multipicative notation for eliptic curve scalar multiplications. 

We always implicitly have a paramater generation procedure $\mathtt{Params}$ that takes a security level $\lambda \in \N$ and returns elliptic curve paramaters including some prime numbers and efficient algorithms for computing elliptic curve operations.  As customary, we treat $\lambda$ and the output of $\mathtt{Params}$ as fixed paramaters, which makes sense because humans run $\mathtt{Params}$ manually in practice. 

As implicit outputs of $\mathtt{Params}$, we work with an elliptic curve $\ecE[\F]$ over some base field $\F$ of (prime) characteristic $q_{\grE}$, along with a distinguished subgroup $\grE \le \ecE[\F]$ of prime order $p_{\grE} \approx 2^{2\lambda}$.  As $\grE$ distinguishes $\ecE[\F]$, we let $h_{\grE}$ denote the cofactor of $\grE$ in $\ecE[\F]$, meaning $\ecE[\F]$ has $h_{\grE} p_{\grE}$ points.
% but abbreviate $h = h_{\grE}$, $p = p_{\grE}$, and $q = q_{\grE}$ when $\grE$ is clear from context.
We write $\grE$ without subscript, and abbreviate $h = h_{\grE}$, $p = p_{\grE}$, and $q = q_{\grE}$, when $\ecE$ is either our uinque pairing friendly curve or else the only curve in view.

We let $H_p : \{0,1\}^* \to \F_p$ or $H_q : \{0,1\}^* \to \F_q$ denote random oracles (RO) with a range $\F_p$ or $\F_q$.  We let $H_\ecE : \{0,1\}^* \to \ecE$ or $H_\grE : \{0,1\}^* \to \grE$ denote a hash-to-curve for $\ecE$ or hash-to-group for $\grE$, which we model as a random oracle too.  We note $H_\grE(x) = h H_\ecE(x)$ always works, although more efficent exist.

\smallskip

Almost all SNARKs like \cite{groth16} or \cite{plonk} work on a pairing friendly elliptic curve $\ecE$ over a field $\F_q$ of characteristic $q \approx 2^{2\lambda}$, which comes equipped with a type III pairing on subgroups of prime order $p \approx 2^{2\lambda}$:  We let $q_1,q_2,q_T$ denote small powers of $q$, and let $\grE_1 \le \ecE[\F_{q_1}]$ and $\grE_2 \le \ecE[\F_{q_2}]$ and $\grE_T \le \F_{q_T}^\times$ denote subgroups all of prime order $p$.  We also let $e : \ecE_1 \times \ecE_2 \to \ecE_T$ denote a type III pairing, meaning a computable bilinear map without known efficiently computable maps between $\ecE_1$ and $\ecE_2$.  Also $q_i = q_{\grE_i}$ for $i=1,2$ in our above notation.  

Any pairing friendly elliptic curve $\ecE$ provides solutions to the decisional Diffie-Hellman problem (DDH).  We do however assume the computational Diffie-Hellman problem (CDH) remains hard in $\ecE$.  We remark that $H_\grE$ being a random oracle plus CDH hardness prevents computable relationships between $H_\grE$ outputs.

% TODO: Pairing assumptions required by Groth16

\smallskip

% We shall require ZCash Sapling style ``Jubjub'' Edwards curves, whose base field characteristic divides of the order of a pairing friendly elliptic curve, thereby making Jubjub base field arithmetic SNARK friendly, and hence Jubjub elliptic curve operations as well \cite{}.

We let $\ecJ$ denote a ZCash Sapling style ``JubJub'' Edwards curve associated to the pairing friendly elliptic curve $\ecE$, meaning $\ecJ$ has base field $\F_p$ whose characteristic $q_{\grJ} = p$ matches the group order $p$ of $\grE_1 \cong \grE_2 \cong \grE_T$.  As in ZCash Sapling, we now prove statements about elliptic curve operations inside $\ecJ$ by proving base field arithmetic in $\F_p$, which our $q_{\grJ} = p$ makes reltively inexpensive inside SNARKs on $\ecE$.

As above, $\grJ \le \ecJ[\F_p]$ has large prime order $p_{\grJ}$ and a small cofactor $h_{\grJ}$.  We always support $4 p_{\grJ} < p$ because if $\ecJ$ is an Edwards curve then $h_{\grJ} \ge 4$ which imposes this by the Hasse bound.

\smallskip

We ask that deserialization prove that putative elements of $\grE$ lie in $\ecE[\F]$ by verifying curve equations, perhaps including twist checks.  We do sometimes require that deserialization checks membership in the prime order subgroup $\grE$ by checking $|\grE| X = 1$ or similar.  Yet, we also sometimes work in $\ecE[\F]$ directly when judicious multiplications by $h_{\grE}$ suffice.  

Anytime $\ecE$ represents a pairing friendly curve then we do ask that deserialization prove elements of $\grE_1$, $\grE_2$, and $\grE_T$ lie inside the correct subgroup of order $p$.  As our SNARKs handle with points in $\ecJ$ directly, we prefer writing $\grJ$ equations in $\ecJ[\F_p]$ and explicitly describe where one clears the cofactor $h_{\grJ}$.  We handled $\grE$ with $\ecE$ not necessarily pairing friendly similarly to $\ecJ$.  We scrape by with only CDH hardness for $\grJ$ for convenience, although DDH winds up hard in $\grJ$.



\subsection{zkSNARKs}

We require details about Groth16 \cite{groth16} becuase our zero-knowledge continuations demand rerandomizing existing zkSNARKs. % which only Groth16 supports.

We thus have an implicit setup procedure $\mathtt{Setup}$ that take our paramaters generated by $\mathtt{Params}$ and produces a structured reference string (SRS).  In practice, our SRS consists of points in $\ecE_1$ and $\ecE_2$ with specific relationships.  We suppress both $\mathtt{Setup}$ and $\mathtt{Params}$ whenever convenient, but $\mathtt{Setup}$ makes an appearance in \S\ref{sec:srs_second_stage}.

TODO: AGM

TODO: Describe Groth16 \cite{groth16} 

































\endinput



BROKEN BOLOW THIS




We fix $J \in \ecJ$ as a generator for public keys.  Any $\KeyGen$ algorithm randomly samples a secret keys $\sk \in \F_q$ and then computes its associate public keys $\pk = \sk J$.  We shall not discuss infrastructure that authorizes public keys.  Yet although our results do not require proof-of-knowledge on $\pk$ per se, we still strongly recommend that back certifications accompany any certificates that authorize $\pk$.

\smallskip






\subsection{VRF-AD-KC security}
\label{subsec:vrf_def}


Any signature scheme requires a \KeyGen algorithm of course, but we also
support hiding public keys inside a commitment scheme \CommitKey and \OpenKey.
Zero-knowledge continuations then work by running \OpenKey inside yet
another zero-knowledge proof. 

\begin{definition}
A {\em verifiable random function with auxiliary data and key commitments} (VRF-AD-KC) consists of several algorithms:
\begin{itemize}
\item $\VRF.\KeyGen$ and returns a public key \pk and a secret key \sk, which one typically instantiates via come commitment scheme. 
%
\item $\VRF.\CommitKey : (\pk,\ctx) \mapsto (\compk,\openpk)$ takes a public key \pk and a commitment context \ctx, and returns a public key commitment \compk to \sk and secret opening data \openpk.
\item $\VRF.\OpenKey : (\compk,\openpk) \mapsto \pk$ opens a public key commitment \compk given the specified opening data \openpk.
%
\item $\VRF.\Eval : (\sk,\msg) \mapsto \Out$ takes a secret key \sk and an input $\msg$, and then returns a VRF output $\Out$.
\item $\VRF.\Sign : (\sk,\openpk,\msg,\aux) \mapsto \sigma$ takes a secret key \sk, a public key opening \openpk, an input \msg, and auxiliary data \aux, and then returns a VRF signature $\sigma$.
\item $\VRF.\Verify$ takes $(\compk,\msg,\aux,\sigma)$ for a public key commitment \compk, an input \msg, and auxiliary data \aux, and then returns either an output $\Out$ or else failure $\perp$.
\end{itemize}
\end{definition}
% PPA vs DPA ?

% We typically define VRFs for secret keys 
% We say secret keys are equivalent whenever their evaluation map $F_\sk$ given by
%  $\msg \apsto \Eval([\sk],\msg)$ defines the same function.
% We also define an equivalence relation upon secret keys with classes denoted $[\sk]$
% because secret keys could contain a public key opening data with only limited impact upon the VRF output. 

% \subsection{VRF-AD-KC security}

We say a VRF-AD-KC denoted \VRF is {\em secure} if it satisfies
 correctness, uniqueness, and pseudo-randomness as defined below,
 as well as being existentially unforgeable as a signature on $(\msg,\aux)$
 and being binding in one of the senses discussed blow.
We caution that VRF security remains complex, in part due to
signer and forger each being adversarial in some security properties,
and that ring VRFs make this worse by verifiers being adversarial.

We follow \cite{agg_dkg} by distinguishing an algorithm $\VRF.\Eval$,
 instead of defining it by the equality in correctness,
which simplifies requiring that verifying honest signatures gives a well-defined function.
$\VRF.\Eval$ always has more optimized instantiations anyways.
% We merge correctness of commitment and VRF here because
% our VRF correctness invokes $\CommitKey$ by necessity.

% \subsection{VRF-AD-KC security}

\begin{definition}
We say a VRF-AD-KC satisfies {\em commitment correctness} if
 $\OpenKey \circ \CommitKey$ returns the same public key \pk.
\end{definition}

\begin{definition}
We say a VRF-AD-KC satisfies {\em VRF correctness} if
$(\pk,\sk) \leftarrow \KeyGen$ and $(\compk,\openpk) \leftarrow \CommitKey(\pk,\ctx)$
imply
$\Verify(\compk,\msg,\aux,\Sign(\sk,\openpk,\msg,\aux)) = \Eval(\sk,\msg)$.
% perhaps except with odds negligible in $\secparam$.
\end{definition}

We demand unforgability on $(\msg,\aux)$ because alone
 the usual VRF conditions only yield unforgeability for \msg.
We need the usual EUF-CMA game here, except attackers access
\CommitKey in our EUF-CMA-KC game, so if desired
 they could obtain multiple signatures under the same \compk,
 and our signing oracle \ora{Sign} enforces commitments.

\begin{definition}\label{def:vrf_sign_oracle}
We let \ora{Sign} denote a CMA oracle, which creates and stores
a key pair $(\pk,\sk) \leftarrow \KeyGen$ and thereafter
answers oracle calls $\ora{Sign}(\compk,\openpk,\msg,\aux)$ by 
logging $(\msg,\aux)$ and returning $\Sign(\sk,\openpk,\msg,\aux)$,
provided $\pk = \OpenKey(\compk,\openpk)$, or aborting otherwise.
\end{definition}

% TODO \eprint
\begin{definition}
We say a VRF-AD-KC satisfies {\em existential unforgeability (EUF-CMA-KC)} if
any PPT adversary \adv has only a negligible advantage in $\secparam$
in the usual chosen-message game adapted to key commitments:
\begin{itemize}
 \item \adv receives $\pk$ from \ora{Sign}, % of Definition \ref{def:vrf_sign_oracle}
 repeatedly queries \ora{Sign},
  and finally produces $\compk,\msg,\aux,\sigma,\openpk$.
 \item \adv wins if $\Verify(\compk,\msg,\aux,\sigma)$ succeeds,
  $\OpenKey(\compk,\openpk) = \pk$, and
   \adv never queried $\ora{Sign}(\cdot,\cdot,\msg,\aux)$.
\end{itemize}
\end{definition}

We do not demand the commitment scheme $\CommitKey$ and $\OpenKey$
be hiding in the definition of VRF-AD-KC security.
Yet, we do internally employ the usual hiding definition from
\cite[pp.8]{cryptoeprint:2019:1185} for commitment schemes however.
% We could employ a weaker chosen-message-like hiding property, but
% this full strength versions suffices.

\begin{definition}
We say a VRF-AD-KC is {\em key hiding} if any PPT adversary \adv
who creates a pair of public keys $\pk_1,\pk_2$
has only negligible advantage for identifying which lies behind a commitment
 $\compk \leftarrow \CommitKey(\pk_i,\ctx)$.
\end{definition}

We want a commitment binding property with unique openings,
analogous to \cite[pp.9]{cryptoeprint:2019:1185}.
% except weakened to require the signature verify too.

ADD ANOTHER KEY BINDING PROPERTY

\begin{definition}\label{def:vrf_key_binding}
We say a VRF-AD-KC is {\em key binding} if no PPT adversary \adv
produces \compk, \msg, \aux and $\sk_i,\openpk_i$ for $i=1,2$
so that
 % $\CommitKey(\pk_1,\openpk_1) = \compk = \CommitKey(\pk_2,\openpk_2)$ and
 $\OpenKey(\compk,\openpk_1) \ne \OpenKey(\compk,\openpk_2)$ and
 $\Verify(\compk,\msg,\aux,\Sign(\sk_i,\openpk_i,\msg,\aux))$
both succeed for $i=1,2$, except with odds negligible in $\secparam$.
\end{definition}

We weaken binding like this because, from \S\ref{subsec:vrf_pederson}
onward, verification provides a proof-of-knowledge that turns
Pedersen commitments to secret keys into commitments to public keys.
% We always take $\ctx = \emptyset$ when using this ``unique'' key binding
% condition, like in the subsequent two sections, but
%  \ctx becomes important later in analogous ring properties.

VRFs should be well-defined functions of their public key too.
% at least up to cryptographic assumptions. 

\begin{definition}
We say a VRF-AD-KC satisfies {\em uniqueness} if anytime some PPT adversary \adv
produces $\msg$ and $\compk_i,\openpk_i,\aux_i,\sigma_i$ for $i=1,2$
 with $\OpenKey(\compk_1,\openpk_1) = \OpenKey(\compk_2,\openpk_2)$, then also
$\Verify(\compk_1,\msg,\aux_1,\sigma_1) = \Verify(\compk_2,\msg,\aux_2,\sigma_2)$
unless either $\Verify$ returns failure,
except with odds negligible in $\secparam$.
\end{definition}

\begin{lemma}
THIS IS FALSE
Assume a VRF-AD-KC satisfying completeness and uniqueness.
For any PPT adversary \adv,
there exists a function $F : (\msg,\pk) \mapsto \Out$ such that
\adv produces $\msg$, $\compk$, $\openpk$, $\aux$, and $\sigma$
$\Verify(\compk,\msg,\aux,\sigma) \in \{ F(\msg,\OpenKey(\compk,\openpk)), \perp \}$
except with odds negligible in $\secparam$.
\end{lemma}



If desired, one easily simplifies VRF-AD-KC to a VRF-AD by
 taking $\compk = \pk$ and fixing $\openpk = \mathtt{""}$,
 which makes $\VRF.\CommitKey$ and $\VRF.\OpenKey$ trivial.

\smallskip

We say VRFs are public key analogs of PRFs, but actually this PRF analogy
fails badly in the ``residual pseudo-randomness'' definitions by
Micali, et al. \cite{vrf_micali}, which employs \ora{Sign} in EUF-CMA-like
games, but says nothing for adversarially generated keys.
%
In \cite{praos}, there exists a UC functionality which captures the
desired PRF analogy, but brings unnecessary restrictions.

We now provide a (black-box) game-based definition which works by
counterintuitively treating \msg as the PRF key, and adversarially
supplied keys as PRF inputs, ala $\PRF_\msg : \pk \mapsto \Eval(\sk,\msg)$.

\begin{definition}
We say a VRF-AD-KC satisfies {\em pseudo-randomness} if 
any PPT adversary \adv has only a negligible advantage in $\secparam$
in this chosen-message game:
\begin{itemize}
\item[]
 Sample a random \msg, a random function $\rho$ with the same range as \Eval, and a bit $b$.
 \adv queries \ora{Verify} by providing both a public key \pk and
 a PPT (black-box) secret key algorithm $f_\sk$ such that
 if $(\openpk,\compk,\aux,\sigma) \leftarrow f_\sk(\msg)$ then
 $\pk = \OpenKey(\compk,\openpk)$ and
 $\Out \leftarrow \Verify(\compk,\msg,\aux,\sigma)$ succeeds (or aborts otherwise).
 \ora{Verify} always returns \Out and $\rho(\pk)$ with their order depending upon $b$.
 \adv wins by guessing $b$, aka by distinguish \Verify from $\rho$.
\end{itemize}
\end{definition}

Our pseudo-randomness winds up independent from residual pseudo-randomness
in \cite{vrf_micali}, even adapted to the key committing framework.
As converse to residual pseudo-randomness' weaknesses noted above,
an ordinary PRF satisfies both pseudo-randomness and uniqueness, but
without being a signature.  Yet, residual pseudo-randomness plus
uniqueness yields unforgeability on \msg.  We caution that
residual pseudo-randomness plus uniqueness says nothing about \aux, so
even if residual pseudo-randomness users require explicit unforgeability.

As in \cite{vrf_micali}, 
there exists a weaker {\em unpredictability} notion where \adv queries
only once, which defines a verifiable unpredictable function (VUF).
%
At least some works like \cite{agg_dkg} squeak by Micali's VUF and VRF's
weaknesses by threshold security arguments when generating randomness.

Also, if $H'(\cdot,k)$ is a PRF then computing $\Out = H'(\Verify(\cdots), \msg)$
transforms a VUF into a VRF, similarly to \cite[Proposition 1]{vrf_micali}.
It follows implementers should prefer VRFs over more subtle VUFs, assuming $H'$ is cheap.

% We handle cofactors explicitly in this work.  In particular, we impose
% a one-to-one map from secret keys \sk to PRFs $F_\sk$, thanks to
%  pseudo-randomness, but doing so imposes some subtleties and maybe overkill.
% TODO: Do we want this?  If so, explain better.

\smallskip


Although \cite[\S3.2 $\fvrf$]{praos} handles pseudo-randomness better,
they formalize VRFs with detached ouputs via the two algorithms
% \begin{itemize}
% \item
$\VRF.\primalgo{EvalProve}(\sk,\msg,\aux) \mapsto (\Out,\sigma)$, in which $\sigma = \VRF.\Sign(\sk,\msg,\aux)$ and $\Out = \VRF.\Eval(\sk,\msg)$, and
% \item
$\VRF.\primalgo{VerifyProof}(\pk,\msg,\aux,\Out,\sigma)$ which returns true only if $\VRF.\Verify(\pk,\msg,\aux,\sigma)$ returns $\Out$.
% \end{itemize}
We strongly prefer the \Sign and \Verify formulation from \cite{agg_dkg}
primarily because the \primalgo{EvalProve}, and \primalgo{VerifyProof}
formulation causes implementation and deployment mistakes:

EC VRF signatures have the form $\sigma = (\PreOut,\pi)$ in which some
inner proof $\pi$ proves correctness of some associated VUF output $\PreOut$. % aka ``pre-output''.  % ``pre-pseudo-random''
It follows $\VRF.\Eval$ never corresponds to $\PreOut$, but if one describes
protocols with an \primalgo{EvalProve} formulation then exposing $\PreOut$
invariably confuses developers into miss-using $\PreOut$ as the output.
% In other words, actual code never corresponds to an \primalgo{EvalProve} and \primalgo{VerifyProof} formulation.

The ``pre-output'' $\PreOut$ preserves algebraic relationships between
secret keys, so protocols described by the \primalgo{EvalProve} formulation
have implementations with broken pseudo-randomness, and perhaps
 related key vulnerabilities and mishandled cofactors.
% We need $\PreOut$ to be exposed by implementations so ...
We avoided the VUF formalism taken by \cite{agg_dkg} in part because
 VUFs obfuscate this difficulty to developers.

As a caveat, there exist UC formalisms that appear simpler with
the \primalgo{EvalProve} and \primalgo{VerifyProof} formulation, like in \cite{praos}.
We therefore propose that VRFs and protocols using VRFs should always be
described using the the \Sign and \Verify formulation, which provides
implementers with a sensible description, but then if needed adopt
 \primalgo{EvalProve} and \primalgo{VerifyProof} only inside the UC formulation itself.
We feel imposing this mental translation upon paper authors and reviewers
 beats imposing the reverse upon developers with less cryptographic knowledge.



\endinput 



\smallskip

There exist VUFs like RSA-FDH or BLS signatures that lack auxiliary data
% There even exist bespoke VRFs that relax correctness to some non-trivial
% relation on the space of secret keys and messages,
%  seemingly including some Rabin variants. 
Yet, these all suffer from either large signature sizes (RSA) or
 slow verification (BLS).
%  VRFs like single-layer XMSS, .

Instead, one prefers instantiating VRFs similarly to
 \cite{nsec5} or \cite{VXEd25519} using Chaum-Pedersen DLEQ proofs \cite{CP92} % Or should it be CP93 ??
 because they provide both small signatures and fast verification.
In these, our auxiliary data \aux can be verified for free,
by binding \aux into the challenge hash, like a Schnorr signature.
VRF protocols could often reduce bandwidth and verifier time this way,
 but some like Sassafras depend upon \aux. 





\endinput % no UC VRF discussion here






Also, pseudo-randomness in \cite{vrf_micali} merges \Eval and \Sign.















\begin{definition}
We say a VRF-AD-KC satisfies {\em residual pseudo-randomness} if 
any PPT adversary \adv has only a negligible advantage in $\secparam$
in this chosen-message game:
\begin{itemize}
\item[]
  \adv receives $\pk$ from \ora{Sign} of Definition \ref{def:vrf_sign_oracle},
  repeatedly queries \ora{Sign}, and produces $\compk,\openpk,\msg,\aux$. 
  If \adv never queried $\ora{Sign}(\cdot,\cdot,\msg,\cdot)$ then
  \adv wins by distinguishing $\Eval(\sk,\msg)$ from random.
\end{itemize}
\end{definition}
% TODO: Actually not quite right!
%












\subsection{UC}

TODO:  Should we give a relatively simple non-harmful UC functionality here?

TODO:  Can we prove this simpler UC functionality from the game?  Can our proof be close to the Praos proof?  If not then why not?











\subsection{Thin batchable EC VRF-AD}
\label{subsec:vrf_thin}

There are VRFs built upon Chaum-Pedersen DLEQ proofs in elliptic curves,
 like \cite{nsec5} and \cite{VXEd25519}.
Yet typically their proofs have one challenge scalar and one signature
scalar, like a non-batchable Schnorr signature, while 
their verification demands two elliptic curve multi-exponentiations.
As a result, their naively batchable variant becomes ``fat'', requiring
one scalar and two separate nonces, which complicates batch verification.

We propose a new ``thin'' batchable EC VRF that merges these two nonces,
and requires only one elliptic curve multi-exponentiation, but
internally computes an extra delinearization challenge.
As such, our thin batchable variant both runs faster and simplifies heavy usage,
and also provides a less annoying interface.

Interestingly, our thin batchable EC VRF winds up literally being
a tweaked Schnorr-like signature, which opens new use cases.
After this section, we abandon this thin batchable VRF however because
our ``fat Pedersen'' variant introduced next in \S\ref{subsec:vrf_pederson}
fits zero-knowledge continuations better.

\smallskip

\newcommand{\ThinVRF}{\primalgo{ThinVRF}} 

%PoK: We achieve this by always providing a proof-of-knowledge in the public key,
%PoK:  either separately or implicitly.

We build only a VRF-AD here which lacks the key commitments of \S\ref{sec:vrf},
meaning $\ThinVRF.\CommitKey$ and $\ThinVRF.\OpenKey$ are trivial, and $\openpk = \emptyset$.

We work solely in $\ecE$ here because we need only a basic Chaum-Pedersen DLEQ proof.
As in \S\ref{sec:background} and throughout,
 $\ecE$ has order $h p$ with $p \approx 2^{2\secparam}$ prime and $h$ a small cofactor.

\begin{itemize}
\item $\ThinVRF.\KeyGen$ selects a secret key \sk uniformly at random from $\F_p$ and computes the public key $\pk = \sk \, \genE$.
%PoK:  and attaches a proof-of-knowledge of $\pk$ to $\pk$ given by a Schnorr signature.  
%PoK:  All public keys must contain a valid proof-of-knowledge, or else be rejected by verifiers.
% We define equivalence $\pk_1 \equiv \pk_2$ of public keys by $h \pk_1 = h \pk_2$.
\item $\ThinVRF.\Eval(\sk,\msg)$ takes a secret key \sk and an input $\msg$, and
 then returns a VRF output $H'(\msg,\pk,h \, \sk \, H_{\grE}(\msg,\pk))$.
\item $\ThinVRF.\Sign(\sk,\msg,\aux)$ takes a secret key \sk, an input $\msg$, and auxiliary data \aux, and then performs
\begin{enumerate}
    \item compute the VRF input $\In := H_{\grE}(\msg,\pk)$ and pre-output output $\PreOut := \sk \, \In$, 
    \item compute the delinearization challenge $c_1 = H_p(\aux,\msg,\pk,\PreOut)$,
    \item sample $r$ uniformly at random from $\F_p$ and compute $R = r (\genE + c_1 \In)$,
    \item compute the challenge $c = H_p(\aux,\msg,\pk,\PreOut,R)$, the proof $s = r + c \, \sk$, and return the signature $\sigma = (\PreOut,R,s)$.
\end{enumerate}
\item $\ThinVRF.\Verify$ takes $(\pk,\msg,\aux,\sigma)$, parses $\sigma = (\PreOut,R,s)$, and then 
\begin{enumerate}
%PoK:    \item abort unless either $\msg$ contains $\pk$ or else \pk has a valid the proof-of-knowledge,
    \item recomputes the VRF input point $\In := H_{\grE}(\msg,\pk)$,
    \item recomputes $c_1 = H_p(\aux,\msg,\pk,\PreOut)$ and $c = H_p(\aux,\msg,\pk,\PreOut,R)$, % the challenges
    \item aborts unless $s h (\genE + c_1 \In) = h R + c h (\pk + c_1 \PreOut)$ holds, and 
    \item returns $H'(\msg,\pk,h \PreOut)$ if nothing failed.
\end{enumerate}
\end{itemize}
As discussed above, if we omit this final hash $H'$ making
our output only $h \PreOut$, then we obtain only a VUF, not a VRF.
We caution that $h \ne 1$ ensures SUF-CMA fails
 by \cite[\S4.1.2]{cryptoeprint:2020:823}.

If desired, one could generalize \ThinVRF to $k$ messages by
computing for $j=1,\ldots,k$ the $k$ distinct
points $\In_j := H_{\grE}(\msg_j)$, pre-outputs $\PreOut := \sk \, \In$,
delinearization challenges
 $c_j = H_p(\aux,\msg_1,\ldots,\msg_k,\pk,\Out_{0,1},\ldots,\Out_{0,k},j)$,
and then running our Schnorr-like signature with
 base point $\genE + \sum_{j=1}^k c_i \In_j$ and
 public key $\genE + \sum_{j=1}^k c_i \Out_j$.

% TODO: Proof correct?  Use same citations as schnorrkel.

% We define $H_\grG(\msg) = h H_\grE(\msg)$ and observe 
%
% \begin{lemma}
% If $H_\grE$ is a random oracle then $H_\grG$ is also a random oracle.
% \end{lemma}

% \begin{lemma}
% $\primalgo{PreEval}(\sk,\msg) = h \sk H_{\grE}(\msg,\pk)$
% \end{lemma}

We discuss chosen message queries against only one key in pseudo-randomness.  
% TODO: What?
In \ThinVRF, we hash the public key \pk along with the message \msg
in $H_\grE$, aka injected \pk into \msg, to prevent
related but different keys having algebraically related input points \In.
We cannot employ this trick in key committing VRFs or ring VRFs however.
Although $H'$ being a PRF mitigates problems, we still recommend caution 
when combining identical inputs \msg with related secret keys,
 like ``blockchain'' users often produce with ``soft key derivation''.

\begin{proposition}\label{prop:thin_vrf}
Assuming AGM in $\grE$, % $\ecE$ modulo $h$,
our $\ThinVRF$ satisfies
 VRF correctness, uniqueness, pseudo-randomness,
 and existential unforgeability on $(\msg,\aux)$.
\end{proposition}



\endinput







\begin{proof}[Proof sketch]
	TODO: ???
\end{proof}
















We expect $\ThinVRF$ to be an EUF-CMA signature scheme on $(\msg,\aux)$ too,
but proving this requires techniques outside our scope, even assuming AGM.

\begin{proof}[Proof sketch]
Correctness holds trivially.

At any fixed $\msg$ we have a Schnorr signature on $\aux$
 over the basepoint $\genE + c_1 \In$, which we name $\primalgo{VRFInner}_{\msg}$.
According to \cite[\S5]{cryptoeprint:2020:823},
 $\primalgo{VRFInner}_{\msg}$ is EUF-CMA secure,
 thanks to our random oracle assumption on $H_p$.

We consider an adversary that produces $(\pk,\msg,\aux,\sigma)$
 that pass verification, without knowing $\sk$.  
%PoK:  Ignoring the first abort path, and employing our random oracle assumption on $H_p$, 
We know from EUF-CMA security of $\primalgo{VRFInner}_{\msg}$ that
our forger knows some $w$ such that
 $h (\pk + c_1 \PreOut) = h w (\genE + c_1 \In)$.
We deduce from AGM knowledge of $x,y,u,v \in \F_p$ such that
 $\pk = x \genE + y \In$ and $\PreOut = u \genE + v \In$
 with $x + c_1 u = w$ and $y + c_1 v = c_1 w$ in $\F_p$,
 so $c_1^2 u + c_1 (x-v) - y = 0$, except with odds negligible in $\secparam$.
At most two $c_1 \in \F_p$ satisfy this equation.
As our $c_1$ depends upon both \pk and $\PreOut$, 
it again follows from our random oracle assumption on $H_p$ that
 $u=0=y$ and $v = w = x \equiv \sk \bmod h$, meaning $\PreOut = \sk \In$,
 except with odds negligible in $\secparam$.
%TODO: What do we cite here?
%PoK:
%PoK: We know $y=0$ if we check the proof-of-knowledge for $\pk$ of course.  
%PoK: We usually suggest that \pk appear in $\msg$ as a defense against related keys, 
%PoK: which occur if say \pk represents some account key on a blockcahin.  
%PoK: In this case, we also know $y=0$ by the random oracle assumption on $h$.  
%PoK: We even deduce $y=0$ after verifying two VRF signatures with distinct
%PoK: inputs $\msg_1$ and $\msg_2$ and hence distinct $\In_1$ and $\In_2$.
%PoK: We know $y=0$ in all cases, as desired.
%PoK: 
It follows that $\ThinVRF$ satisfies uniqueness of course. 

An unpredictability adversary \adv guesses
 a \msg and corresponding pre-output $h \PreOut := h \sk H_\grE(\msg)$,
after making chosen message queries to \Sign.
In AGM, \adv must express its guess for $h \PreOut$
 in terms of $H_\grE(\msg)$ and points arising earlier.
???  So simple ???
As $H_\grE$ is a random oracle, we deduce that either
 \adv solved the discrete logarithm problem, or else
 unpredictability holds for $\ThinVRF$.
As $H'$ is a PRF, we now argue pseudo-randomness for$\ThinVRF$ similarly
 to \cite[Proposition 1]{vrf_micali}.
\end{proof}
% An adversary cannot discover $\PreOut$ without querying $\msg$,
% % \cite[Theorem 6]{coron02}
% % https://eprint.iacr.org/2001/062.pdf NOT https://www.iacr.org/archive/eurocrypt2002/23320268/coron.pdf
% but our EUF-CMA game permits doing so with alternative $\aux$. 
% ...
%TODO: Actually this gets really long winded. 

%PoK:  In this, we still have a VRF if $y=0$, but not exactly the one specified.  
We caution that omitting $c_1$ only demands $x + u = v$ even if $y=0$,
which does not give a VRF.

We need two scalar multiplications in the prover and
 then four scalar multiplications in the verifier
 just like \cite{nsec5} or \cite{VXEd25519} do.
We do incur an extra hashing operation and two field multiplications,
 but they cost relatively little.
%PoK: At frist blush, we incur two more scalar multiplications when verifying
%PoK: the proof-of-knowledge for \pk too,
%PoK:  which one implements by a Schnorr signature. 
%PoK: Yet, VRF applications always require their public keys be registered in advance,
%PoK: meaning the proof-of-knowledge should be checked in advance and amortizes.
%PoK: 
We believe this approach actually reduces verifier computation because
advanced multi-scalar multiplication algorithms become more efficient when
larger, which should outweigh the extra hashing and field operations.

We also support batch verification without altering the signature or
increasing the signature size.  We think this tips the scales because
avoiding a separate batchable VRF signature type simplifies interface
over naive batch verification methods for \cite{nsec5} or \cite{VXEd25519}.

Aside from batch verification, we might simplify interaction with
other protocols by building upon Schnorr signatures.




\endinput



\section{UC}





\subsection{Pedersen keyed VRF-AD-KC}
\label{subsec:vrf_pederson}

Anonymized VRF signatures could use Chaum-Pedersen DLEQ proofs that
hide the true public key, which our key committing definition permits.
Zero-knowledge continuations work by opening this key commitment
 inside another zero-knowledge proof. % DUP

We only describe the non-batchable variant analogous to
 \cite{nsec5} and \cite{VXEd25519}.
We know easily batchable verifiable flavors exist, but
 they enlarge the VRF signature by 32 bytes.
We need pairings to verify the SNARKs component of our ring VRF,
which dwarfs the CPU savings from batch verification,
so saving 32 bytes sounds more helpful for non-blockchain use cases.

\newcommand{\PedVRF}{\primalgo{PedVRF}} 

We define \KeyGen as in \ThinVRF in \S\ref{subsec:vrf_thin} but \Eval differs
slightly by not injecting \pk into \msg:
% by selecting a secret key \sk uniformly at random from $\F_p$ and
% computing the public key $\pk = \sk \, \genG$.
\begin{itemize}
\item $\PedVRF.\KeyGen$ selects a secret key \sk uniformly at random from $\F_p$ and computes the public key $\pk = \sk \, \genG$. 
\item $\PedVRF.\Eval(\sk,\msg)$ takes a secret key \sk and an input $\msg$, and
 then returns a VRF output $H'(\msg,h \, \sk \, H_{\grE}(\msg))$.
\end{itemize}

We select here an arbitrary base point $\genG$ for our public key
 because this avoids some confusion later.
%
We fix a second generator $\genB$ of $\grE$ independent from $\genG$,
perhaps created by applying $H_\grE$ to an input outside existing usages'
domain and also containing $\genG$. 
We now form Pedersen-like commitments to this public key \pk as follows.
\begin{itemize}
\item $\PedVRF.\CommitKey$ selects a blinding factor $\openpk$ uniformly
 at random from $\F_p$ and computes the commitment $\compk = \pk + \openpk \, \genB$.
\item $\PedVRF.\OpenKey$ just returns $\pk = \compk - \openpk \, \genB$.
\end{itemize}
In fact, these are technically only Pedersen commitments to
the secret key \sk, not to the public key \pk, because
 \OpenKey does not enforce the structure of \pk.
Instead, our \Verify proves correctness of \compk, as
 required for our key binding condition, so
zero-knowledge continuations then use \OpenKey with only minor caveats.
% In particular, any VRF-AD-KC prevents adversaries from obtaining
% extra valid outputs, but ring VRF protocols need a proof-of-knowledge
% for the public key \pk if they demand that different \pk represent
% different functions.

\begin{itemize}
\item $\PedVRF.\Sign(\sk,\openpk,\msg,\aux)$ takes a secret key \sk and blinding factor \openpk, an input $\msg$, and auxiliary data \aux, and then performs
\begin{enumerate}
    \item compute the VRF input point $\In := H_{\grE}(\msg)$ and pre-output $\PreOut := \sk \, \In$,
    \item Sample random $r_1,r_2 \leftarrow \F_p$ and compute $R = r_1 \genG + r_2 \genB$ and $R_\msg = r_1 \In$.
    \item Compute the challenge $c = H_q(\aux,\msg,\compk,\PreOut,R,R_m)$,
     along with $s_1 = r_1 + c \sk$ and $s_2 = r_2 + c \, \openpk$.
    \item Return the signature $(\PreOut,c,s_1,s_2)$.
\end{enumerate}
\item $\PedVRF.\Verify(\compk,\msg,\aux,\sigma)$, parses $\sigma = (\PreOut,c,s_1,s_2)$, and then 
\begin{enumerate}
    \item recompute the VRF input point $\In := H_{\grE}(\msg)$,
    \item computes $R = s_1 \genG + s_2 \genB - c \, \compk$ and $R_m = s_1 \In - c \PreOut$, and
    \item returns $H'(\msg, h \PreOut)$ if $c = H_q(\aux,\msg,\compk,\PreOut,R,R_\msg)$ or failure otherwise.
\end{enumerate}
\end{itemize}

We obtain roughly \cite{nsec5} or \cite{VXEd25519}
if we choose $\openpk = 0 = r_2$ in \Sign.

\begin{lemma}\label{prop:pedersen_vrf_hiding}
$\PedVRF$ is a correct key commitment and key hiding.
\end{lemma}

Although Pedersen commitments are perfectly hiding, our $\R_\msg$ makes $\sigma$ only computationally hiding.

\begin{proposition}\label{prop:pedersen_vrf}
Assuming AGM in $\grE$, % $\ecE$ modulo $h$,
our $\PedVRF$ satisfies VRF correctness, key binding, uniqueness,
pseudo-randomness, and unforgeability. % (EUF-CMA-KC) on $(\msg,\aux)$.
\end{proposition}

We need this second verification equation in \PedVRF, but not in \ThinVRF,
because otherwise our $s_2 \genB$ term provides enough freedom to tamper
with the pre-ouputs.  

We could however generalize \PedVRF to $k$ messages $\msg_1,\ldots,\msg_k$
similarly to \ThinVRF in \S\ref{subsec:vrf_thin}:  We compute for
$j=1,\ldots,k$ the $k$ distinct
points $\In_j := H_{\grE}(\msg_j)$, pre-outputs $\PreOut := \sk \, \In$,
delinearization challenges
 $c_j = H_p(\aux,\msg_1,\ldots,\msg_k,\compk,\Out_{0,1},\ldots,\Out_{0,k},j)$,
and then use the \PedVRF proof for
 $\In = \sum_j c_j \In_j$ and $\Out = \sum_j c_j \Out_j$.

% TODO: Proof correct?  Use same citations as schnorrkel.


\endinput










\begin{proof}[Proof sketch]
	Correctness holds trivially.
	
	???
\end{proof}







\section{Ring VRFs}
\label{sec:rvrf_def}

Ring VRFs are firstly ring signatures broadly interpreted, in that they
prove an associated public key lies inside some commitment \comring to
the plausible signer set,
 which anyone could construct from this set of public keys.
At the same time, ring VRFs prove correct output of a PRF keyed by
the signer's actual secret key, and evaluated on a supplied message \msg,
 which then links ring VRF signatures sharing the same \msg.

A {\em ring verifiable random function with auxiliary data} (rVRF-AD)
consists of the algorithms of a VRF-AD-KC, except with
 \compk and \openpk renamed to \comring and \openring,
 plus one additional algorithm:
\begin{itemize}
\item $\rVRF.\CommitRing : \ctx \mapsto \comring$ takes a set \ctx of
 public keys and returns a public key set commitment \comring.
\end{itemize}

\def\rSign{\Sign}
\def\rVerify{\Verify}

In this, we have renamed the commitment and opening to avoid confusion
when we build a rVRF-AD from a VRF-AD-KC.  This fresh notation leaves
$\rVRF.\KeyGen$ and $\rVRF.\Eval$ untouched, but
 changes the other methods' signatures:
\begin{itemize}
\item $\rVRF.\CommitKey : (\pk,\ctx) \mapsto (\comring,\openring)$
\item $\rVRF.\OpenKey : (\comring,\openring) \mapsto \pk$
\item $\rVRF.\rSign : (\sk,\openring,\msg,\aux) \mapsto \sigma$
\item $\rVRF.\rVerify : (\comring,\msg,\aux,\sigma) \mapsto \Out \, \lor \perp$
\end{itemize}

\subsection{rVRF-AD security}

We extend the VRF-AD-KC commitment correctness condition for \CommitRing:

\begin{definition}
We say rVRF satisfies {\em ring commitment correctness} if
commitment correctness holds, and also $\rVRF.\CommitRing$ is 
 compatible with $\rVRF.\CommitKey$ in that
  $\rVRF.\CommitRing(\ctx) = \rVRF.\CommitKey(\pk,\ctx).0$.
\end{definition}

We lack anonymity against full key exposure ala
 \cite[pp. 6 Def. 4]{cryptoeprint:2005:304} of course, due to the VRF output,
but instead demand a weaker anonymity condition similar to
 \cite[pp. 5 Def. 3]{cryptoeprint:2005:304}:

\begin{definition}\label{def:rvrf_sign_oracle}
We let \ora{Sign} denote a ring signature CMA oracle, meaning
\begin{itemize}
\item $\ora{Sign}(\mathtt{'keygen'})$ creates and logs a fresh key pair
 $(\pk,\sk) \leftarrow \KeyGen$ and adds $\pk$ to the set $\ctx_0$, and then returns \pk.
\item $\ora{Sign}(\comring,\openring,\msg,\aux)$ returns
 $\Sign(\sk,\openring,\msg,\aux)$, provided it logged $(\OpenKey(\comring,\openring),\sk)$ previously.
\end{itemize}
\end{definition}

\begin{definition}
We say \rVRF satisfies {\em ring anonymity} if
any PPT adversary $\adv$ has an advantage only
 negligible in $\secparam$ to win the game:
\begin{itemize}
\item[]
 Initially \adv outputs a message \msg, associated data \aux,
 two distinct public keys $\pk_0,\pk_1 \in \ora{Sign}.\ctx_0$ created by \ora{Sign},
 and a ring $\ctx \subset \ora{Sign}.\ctx_0$ containing $\pk_0,\pk_1$.
 Set $\comring = \CommitRing(\ctx)$.
 Next the challenger chooses $j=0$ or $j=1$ and gives
  \adv a signature $\sigma = \rSign(\sk_i,\openring,\msg,\aux)$ with $\OpenKey(\comring,\openring) = \pk_j$.
 %
 \adv called \ora{Sign} of Definition \ref{def:rvrf_sign_oracle} throughout,
 except \adv loses if they ever query $\ora{Sign}(\comring,\openring,\msg,\cdot)$
 on \msg with $\OpenKey(\comring,\openring) \in \{ \pk_0, \pk_1 \}$.
 Finally \adv guesses $j$ and wins if correct.
\end{itemize}
\end{definition}

We similarly a ring unforgeability resembling
 \cite[pp. 7 Def. 7]{cryptoeprint:2005:304} % their definition appears broken
while also capturing the uniqueness condition that limits even adversaries who know secret keys.

\begin{definition}
We say \rVRF satisfies {\em ring unforgeability and uniqueness} if
any PPT adversary $\adv$ has an advantage only
 negligible in $\secparam$ to win the game:
\begin{itemize}
\item[]
 \adv calls \ora{Sign} of Definition \ref{def:rvrf_sign_oracle} throughout,
 but also creates its own keys freely.
 %
 Finally $\adv$ outputs a ring $\ctx$ as well as
 $k + |\ctx \setminus \ora{Sign}.\ctx_0|$ valid ring VRF signatures,
  each with distinct outputs,    % $\sigma_i$
 for $\ctx$ on a message \msg and associated data \aux.
 $\adv$ wins if they invoked $\ora{Sign}$ strictly fewer than $k$ times on $\msg$  % and some $\aux_i$,
  and distinct $i$ with $\pk_i \in \ctx \cap \ora{Sign}.\ctx_0$.
\end{itemize}
\end{definition}
 
Any ring VRF becomes a non-anonymized VRF whenever
 the ring becomes a singleton $\ctx = \{ \pk \}$ of course.
In doing so this, our ring uniqueness reduces to
 the seperate unforgeability and uniqueness games for VRF-AD,
meaning our uniqueness with only a trivial key commitment.

We reuse the VRF-AD-KC pseudo-randomness definition for ring VRFs
because pseudo-randomness is strongest for singleton rings, i.e. $|\ctx| = 1$.

\smallskip 

Although our applications could mostly ignore key multiplicity. 
we briefly mention AML/KYC applications in \S\ref{subsec:AML_KYC},
which demands suspects prove non-involvement using ring VRFs.

\begin{definition}
We say \rVRF is {\em exculpatory} if we have an efficient algorithm
for equivalence of public keys, but a PPT adversary \adv cannot
find non-equivalent public keys $\pk_0,\pk_1$ with colliding VRF outputs.
% (perfectly or computationally)
% (either ever or with advantage negligible advantage in $\secparam$)
\end{definition} 


\subsection{Generic rVRF-AD instantiation}

We instantiate a rVRF-AD from a hiding VRF-AD-KC like \PedVRF plus
a ring commitment scheme
 $\rVRF.\{ \CommitRing, \CommitKey, \OpenKey \}$
for which some zero-knowledge ring membership proof handles both
 $\PedVRF.\OpenKey$ and $\rVRF.\OpenKey$
efficiently.

\begin{itemize}
\item $\rVRF.\rSign : (\sk,\openring,\msg,\aux) \mapsto \sigma$ takes
 a secret key \sk, a ring opening \openring, a message \msg, and auxiliary data \aux, and then \\
 \begin{enumerate}
 \item computes the ring membership proof $\piring$ and associated \openpk,
  $$ \piring = \NIZK \Setst{ \compk, \comring }{
  \exists \openpk,\openring \textrm{\ s.t.\ } 
  \genfrac{}{}{0pt}{}{\PedVRF.\OpenKey(\compk,\openpk) \quad}{\,\, = \rVRF.\OpenKey(\comring,\openring)}
  } $$
 \item computes the VRF-AD-KC signature
  $$ \sigma = \PedVRF.\Sign(\sk,\openpk,\msg,\aux \doubleplus \compk \doubleplus \piring), \quad\textrm{and} $$ % finally
 \item returns the ring VRF signature $\rho = (\compk,\piring,\sigma)$.
 \end{enumerate}
\item $\rVRF.\rVerify$ takes $(\comring,\msg,\aux,\rho)$,
 parses $\rho$ as $(\compk,\piring,\sigma,)$,  and then returns
 $$ \PedVRF.\Verify(\compk,\msg,\aux \doubleplus \compk \doubleplus \piring,\sigma) $$
 iff $\NIZK.\Verify(\piring,\compk,\comring)$ succeeds. 
\end{itemize}

\begin{proposition}\label{prop:pedersen_rvrf}
$\rVRF$ satisfies ring uniqueness and ring unforgeability.
\end{proposition}


\subsection{UC}

TODO: Should we gives Handan's UC functionality or similar here? 





\endinput










% \noindent
In this, we tie $\sigma$ to $\piring$ by expanding $\sigma$'s auxiliary data with $\piring$.

% \smallskip

We now prove security of \rVRF using that
\begin{itemize}
	\item \PedVRF is a secure hiding VRF-AD-KC, and that
	\item our ring commitment scheme satisfies ring commitment correctness.
\end{itemize}

Pseudo-randomness holds by pseudo-randomness of \PedVRF from
Proposition \ref{prop:pedersen_vrf}.

\begin{proposition}\label{prop:pedersen_rvrf}
	$\rVRF$ satisfies ring uniqueness and ring unforgeability.
\end{proposition}

\begin{proof}[Proof sketch]
	???
\end{proof}


TODO:  Should we give an abstract pure NIZK instantiation here?  I think later probably.






\section{Zero-knowledge continuations}
\label{sec:rvrf_cont}

\newcommand\rrSNARK{\primalgo{Groth16}\xspace}
\newcommand\pifast{\ensuremath{\pi_{\mathtt{fast}}}}
\newcommand\pisafe{\ensuremath{\pi_{\mathtt{safe}}}}

We now construct ring VRFs which achieves fast amortized prover time
by using a heavy zero-knowledge continuation for $\rVRF.\OpenKey$ but
which permits updating \openpk in the $\PedVRF.\OpenKey$ invokation
without reproving $\pi$.
$$ \pi = \NIZK \Setst{ \compk, \comring }{
 \exists \openpk,\openring \textrm{\ s.t.\ } 
 \genfrac{}{}{0pt}{}{\PedVRF.\OpenKey(\compk,\openpk) \quad}{\,\, = \rVRF.\OpenKey(\comring,\openring)}
} \mathperiod $$

% \smallskip
\subsection{Rerandomization}
% \label{sec:rvrf_groth16}

Zero-knowledge continuations need rerandomizable zkSNARKs
when being reused multiple times, meaning Groth16 \cite{groth16},
but unlinkability requires more than merely rerandomization.

In Groth16 \cite{groth16}, we have an SRS $S$ consisting of curve
points in $\grE_1$ and $\grE_2$ that encode the circuit being proven.
We follow \cite{groth16} in discussing the SRS $S$ in terms of
its ``toxic waste''
 $(\alpha,\beta,\delta,\gamma,\tau^1,\tau^2,\ldots) \in \F_q^*$.
In other words, we could write say $[ f(\tau)/\delta ]_1$ or $[\cdots]_2$
to denote an element of our SRS $S$ in $\grE_1$ or $\grE_2$ respectively,
computed by scalar multiplication from the toxic waste $\tau$ and $\delta$,
 but for which nobody knows the underlying $\tau$ or $\delta$ anymore.

In the SRS $S$, we distinguish the verifiers' string of elements
 $Y_1,\ldots,Y_k, [\alpha]_1 \in \grE_1$ and
 $[\beta]_2, [\gamma]_2, [\delta]_2 \in \grE_2$.
% as seperate from the provers' much longer string of elements in $\grE_1$ and $\grE_2$.
A Groth16 \cite{groth16} proof then takes the form 
 $\pi = (A,B,C) \in \grE_1 \times \grE_2 \times \grE_1$.
A verifier then produces a $X = \sum_i^k x_i Y_i \in \grE_1$ from
 the public inputs $x_i$ and then checks 
$$ e(A,B) = e([\alpha]_1, [\beta]_2) \cdot
 e(X, [\gamma]_2) \cdot e(C, [\delta]_2) \mathperiod $$

We need the rerandomization algorithm from \cite[Fig.~1]{RandomizationGroth16}:
% to build a zero-knowlege continuation:
% https://eprint.iacr.org/2020/811
% https://github.com/arkworks-rs/groth16/pull/16/files
% \algo{rerandomize}
An existing SNARK $(A,B,C)$ is transformed into a fresh
SNARK $(A',B',C')$ by sampling random $r_1,r_2 \in \F_p$ and computing
$$ \begin{aligned}
A' &= {1 \over r_1} A \\
B' &= r_1 B + r_1 r_2 [\delta]_2 \\
C' &= C + r_2 A \mathperiod \\
\end{aligned} $$
At this point, our $x_i$ remain identical after rerandomization,
so $X$ links $(A,B,C)$ to $(A',B',C')$.
Alone rerandomization cannot alter public inputs $x_i$, so
we instead need an opaque public input point $X$, which then becomes
part of our proof and incurs its own seperate proof of correctness.

We also need one fresh basepoint $\genB_\gamma$ independent from all others,
again perhaps created by applying $H_\grE$ to an input outside existing usages' domain.
We now give provers the additional SRS elements
$$ \genB_\delta := {\gamma\over\delta} \genB_\gamma $$
Although $\genB_\gamma$ is independent, 
we create $\genB_\delta$ during the trusted setup,
 so the toxic waste $\gamma$ and $\delta$ remain secret.
After this, subversion resistance could be checked like 
$$ e(\genB_\gamma, [\gamma]_2) = e(\genB_\delta, [\delta]_2) \mathperiod $$

We now have a zero-knowledge continuation $(X,A,B,C)$ from which
we produce an unlinkable $(X',A',B',C')$ by
 first sampling random $b,r_1,r_2 \in \F_p$ and then computing
$$ \begin{aligned}
X' &= X + b \genB_\gamma \\
A' &= {1 \over r_1} A \\
B' &= r_1 B + r_1 r_2 [\delta]_2 \\
C' &= C + r_2 A + b \genB_\delta \mathperiod \\
\end{aligned} $$
As our two $b$ terms cancel in the pairings, we wind up with the standard Groth16
 rerandomization construction above, except with $X$ opaque.

We still verify $(X',A',B',C')$ like 
$$ e(A',B') = e([\alpha]_1, [\beta]_2) \cdot
 e(X', [\gamma]_2) \cdot e(C', [\delta]_2) \mathcomma $$
As our verifier does not build $X$ themselves, we proves nothing
with this pairing equation unless the verifier seperately checks
 some proof-of-knowledge that $X' = \sum_i^k x_i Y_i$.
We foresee some $x_i$ being transperent elements that determine the
Merkle root of the ring $\ctx$, but any $x_i$ concerning the
 specific $\pk$ must be hidded by out blinding terms in $b$.

All told, our rerandomization trick transforms some conventional
Groth16 SNARK $\pi_{\mathtt{inner}}$ for $\rVRF.\OpenKey$
into a SNARK $\pi$ with an opaque and unlinkable input $X$.
We therefore explore two concrete $\pi_{\mathtt{inner}}$ proposals next.

Importantly, rerandomization requires only
 four scalar multplicaitons on $\ecE_1$ and
 two scalar multplicaitons on $\ecE_2$,
which  BLS12 curves make roughly equivlent to
 eight scalar multplicaitons on $\ecE_1$.

% Intutitively, an adversary cannot link $(X,A,B,C)$ with $(X',A',B',C')$ because 

\begin{proposition}\label{prop:unlinkable}
Let $\sigma$ and $\sigma'$ denote Chaum-Pedersen proofs-of-knowledge
 for $X$ and $X'$ respectively, with nonces chosen randomly.
Then $(X,A,B,C,\sigma)$ and $(X',A',B',C',\sigma')$ are unlinkable.
\end{proposition}

\begin{proof}[Proof stetch.]
???
\end{proof}

\subsection{Faster}
\label{subsec:rvrf_faster}

We describe the preferred faster choice $\pifast$ for $\pi_{\mathtt{inner}}$
that sets $x_1 := \sk$ and $x_0 = \CommitRing(\ctx)$ so that
taking $\genG := Y_1$, $\genB := \genB_\gamma$, and $\openpk := b$ in \PedVRF
yields and incredible fast amortized ring VRF prover.
Also, \PedVRF itself proves knowledge of $X$.
$$ X = \CommitRing(\ctx)\, Y_0 + \sk\, Y_1 + b \genB_\gamma $$

A priori, we do not know $Y_1$ during the trusted setup for $\pifast$,
which prevents computing $\pk = \sk\, Y_1$ inside $\pifast$.
Instead, we propose $\ctx$ contain commitments to $\sk$ over
some Jubjub curve $\ecJ$.  

We know $\grJ$ typically has smaller order than $\grE$,
due to $\ecJ$ being an Edwards curve, but 
if $\sk = \sk_0 + \sk_1 \, 2^{128}$ then our public key commitments could
take the form $\sk_0\, \genJ_0 + \sk_1\, \genJ_1 + b' \genJ_2$,
with independent $\genJ_0,\genJ_1,\genJ_2$.
Interestingly, we avoid range proofs for $\sk_1$ and $\sk_2$
by this independence. 

$$ \pifast = \rrSNARK \Setst{ \sk_0 + \sk_1 2^128, \comring }{
 \exists b',\openring \textrm{\ s.t.\ }
 % 0 < \sk_0,\sk_1 < 2^128 \textrm{\ and\ } 
 \genfrac{}{}{0pt}{}{\rVRF.\OpenKey(\comring,\openring)}{\,\, = \sk_0 \genJ_0 + \sk_1 \genJ_1 + b' \genJ_2}
} $$ % \mathperiod 

We explain later in \S\ref{sec:ring_hiding} how one could
choose $Y_1$ independent before doing the trusted setup,
 and then wire $Y_1$ into $\pifast$ inside $C$.
In ths case, we could prove $\pk = \sk\, Y_1$ inside $\pifast$, but then
non-native arithmetic makes $\pifast$ far slower.

At this point, \PedVRF requires four scalar multiplications on $\ecE_1$,
so together with rerandomization our amortized prover time
 approaches 12 scalar multiplications on typical curves. 
We expect the three pairings dominate verifier time.

As an aside, one could construct a second faster curve with the same
group order as $\grE$, which speeds up one scalar multiplication
 in both the prover and verifier. 

Importantly, our fast ring VRF' amortized prover time now rivals
group signature schemes' performance.  We hope this ends the temptation
to deploy group signature like constructions where the deanonymization vectors matter.

\begin{proposition}\label{prop:pifast_anonymity}
\rVRF using \pifast satisfies ring anonymity.
\end{proposition}

\begin{proof}[Proof stetch.]
???
\end{proof}

\subsection{Safer}

Although blindingly fast, we processed $\sk$ directly inside $\pifast$,
which annoys those wanting lightweight HSM provers, and
increases side channel attack risks.

We could easily build a safer circuit in which
\PedVRF runs on, and $\pk$ lies in, the Jubjub curve $\ecJ$, like 
$$ \pisafe = \rrSNARK \Setst{ \pk, \comring }{
 \exists \openring \textrm{\ s.t.\ }
 \pk = \rVRF.\OpenKey(\comring,\openring)
} \mathperiod $$
In this, our $\genG$ and $\genB$ have no relation with $\ecE$ or \pisafe,
so our $X$ nolonger plays less nicely with \PedVRF, 
$$ X = \CommitRing(\ctx) Y_0 + \pk.x Y_1 + \pk.y Y_2 + b \genB_\gamma \mathperiod $$

Instead we strip the blinding $b \genB_\gamma$ using a second SNARK $\pisafe'$.
As our second SNARK $\pisafe'$ cannot be another zero-knowledge continuation
itself, because it knows the $\openpk$ of \PedVRF.
Yet, we keep $\pisafe'$ working on $\ecE$ by reusing the external blinding
trick again.  In other works, we construct an inner SNARK $\pisafe'$ with
its own $\delta'$ and $\gamma'$ and that processes
 transperent public inputs $\compk.x Y'_1 + \compk.y Y'_2$ and an opaque
$$ X' = \pk.x Y'_3 + \pk.y Y'_4 + b \genB_{\gamma'} \mathperiod $$

We need $\pisafe'$ to apply the $\ecJ$ blinding required by \PedVRF
ala $\pk = \PedVRF.\OpenKey(\compk,\openpk) = \compk - \openpk \genB$, so
$$ \pisafe' = \rrSNARK \Setst{ \compk, \pk }{
 \exists \openpk \textrm{\ s.t.\ } \compk = \pk + \openpk \genB
} \mathperiod $$
In this, we nolonger prove knowledge of $X$ in \PedVRF like $\pifast$ did.
Instead, we employ another Chaum-Pedersen DLEQ proof to wire
 the $X$ of $\pisafe$ to the $X'$ of $\pisafe'$,
 thus proving knowledge of both.
We cannot avoid this proof-of-knowledge, but it becomes simpler if
we choose $Y'_3 = Y_1$ and $Y'_4 = Y_2$ using the trick of \S\ref{sec:ring_hiding}.

We take $\aux \doubleplus \pisafe \doubleplus \pisafe'$
 to be the \aux of \PedVRF of course.

\begin{proposition}\label{prop:pisafe_anonymity}
\rVRF using \pisafe satisfies ring anonymity.
\end{proposition}

\begin{proof}[Proof stetch.]
???
\end{proof}

% \smallskip

As \openpk appears inside $\pisafe'$, we recompute $\pisafe'$ with
every ring VRF signature, but the elliptic curve addition requires
only 5ish constraints, and the elliptic curve scalar multiplication
requires under 750 ??? constraints. 
All told our amortized prover runs faster than a Groth16 prover
with 800 constraints.

A priori, our safer verifier requires five pairings, along with
some additional $\ecE_1$ scalar multiplications.
We conjecture $\gamma$ and $\delta$ could safely be shared between
$\pisafe$ and $\pisafe'$, thereby requiring only four pairings,
but caution this result appars non-trivial.

% \smallskip


\section{Application: Identity}
\label{sec:app_identity}

Ring VRFs yield anonymous identity systems:
After a user and service establish a secure channel and
the server authenticates itself with certificates, then
the user authenticates themselves by providing an anonymous
VRF signature with input \msg being the service's identity,
thus creating an pseudonymous identified session with
a pseudonym unlinkable from other contexts.

We expand this identified session workflow with an extra
update operation suitable for our ring VRF's amortized prover.
We discuss only \pifast here but all techniques apply to \pisk and \pipk similarly. 

\begin{itemize}
\item {\em Register} --
 Adds users' public key commitments into some \ring,
 after verifying the user does not currently exist in \ring.
\item {\em Update} --
 User agents regenerate their stored SNARK $(\pk,\pifast^\inner)$ using
 $\SpecialG.\Preprove( (\sk_1,\sk_2,\openring); (\sk,\comring) )$
 each time \ring changes, perhaps even receiving \comring and \openring
 from some ring management service.
\item {\em Identify} --
 Our user agent first opens a standard TLS connection to a server \msg,
 both checking the server's name is \msg and checking certificate
 transparency logs, and then computes the shared session id \aux.
 Our user agent computes the user's identity
  $\mathtt{id} = \PedVRF.\Eval(\sk,\msg)$ on the server id \msg,
 Our user agent next rerandomizes \pifast, \compk, and \openpk using
 $\SpecialG.\Reprove( \pk, \pifast^\inner )$, computes
  $\sigma = \PedVRF.\Sign(\sk,\openpk,\msg,\aux \doubleplus \compk \doubleplus \pifast)$,
 and finally sends the server their ring VRF signature $(\compk,\pifast,\sigma)$
 % $\rVRF.\rSign(\sk,\openring,\msg,\aux)$ % $ = (\compk,\pifast,\sigma)$.
\item {\em Verify} -- 
 After receiving $(\compk,\pifast,\sigma)$ in channel \aux,
 the server named \msg checks $\SpecialG.\Verify( \comring, (\compk,\pifast) )$,
 checks the VRF signature, and obtains the user's identity $\mathtt{id}$, ala \\
 $\mathtt{id} = \PedVRF.\Verify(\compk,\msg,\aux \doubleplus \compk \doubleplus \pifast,\sigma)$.
\end{itemize}


\subsection{Browsers}

We must not link users' identities at different web sites, so user agents
should carefully limit cross site resource loading, referrer information, etc.
User agents could always load purely static resources, without metadata
like cookies or referrer information.
% especially purely content addressable resources.
At least Tor browser already takes cross site resource concerns seriously,
while Safari and Brave may limit invasive cross site resources too.
% In any case, one could always specify rules against cross site privacy invasions
% whenever writing ring VRF browser specifications.

We somewhat trust the CAs and CT log system with users' identities in
the above protocol, in that users could login to a site with fraudulent
credentials.  We think cross site restrictions limit this attack vector.
If stronger defenses are desired then instead of \msg being the site name,
\msg could be a public ``root'' key for the specific site, which then
also certifies its TLS certificate.  Ideally its secret key remains air gaped.


\subsection{AML/KYC}
\label{subsec:AML_KYC}

We shall not discuss AML/KYC in detail, because the entire field lacks
clear goals, and thus winds up being ineffective
 \cite{doi:10.1080/25741292.2020.1725366}.
% https://www.tandfonline.com/doi/full/10.1080/25741292.2020.1725366
% via https://twitter.com/ronaldpol/status/1491548352189587460
We do however observe that AML/KYC typically conflicts with security
and privacy laws like GDPR.  As a compromise between these regulations,
one needs a compliance party who know users' identities,
 while another separate service party knows the users' activities.
We propose a safer and more efficient solution:

Instead our compliance party becomes an identity issuer who maintains
a public \ring, and privately knows the users behind each public key.
As above, identity systems could employ \ring freely for diverse purposes.
If later asked or subpoenaed, users could prove their relevant identities
to investigators, or maybe prove which services they use and do not use. 

Interestingly \PedVRF could run ``backwards'' like
 $H_{\grE'}(\msg) \ne \sk^{-1} \, \PreOut$
to show a ring VRF output associated to $\PreOut$
does not belong to the user, without revealing the users'
identity $\Hout(\msg, \sk \, H_{\grE'}(\msg))$ to investigators. 

Our applications mostly ignore key multiplicity. 
AML/KYC demands suspects prove non-involvement using ring VRFs.

\begin{definition}\label{def:rvrf_exculpability}
We say \rVRF is {\em exculpatory} if we have an efficient algorithm
for equivalence of public keys, but a PPT adversary \adv cannot
find non-equivalent public keys $\pk_0,\pk_1$ with colliding VRF outputs.
% (perfectly or computationally)
% (either ever or with advantage negligible advantage in $\secparam$)
\end{definition}

A priori, our JubJub representations $\sk_0 \genJ_0 + \sk_1 \genJ_1$
used in \S\ref{subsec:rvrf_faster} and \S\ref{subsec:rvrf_side_channel}
costs us exculpability from Definition \ref{def:rvrf_exculpability}.
% Ad hoc rings ...
% Rings used for AML/KYC would be maintained by an authority and require
% some registration procedure, using government issued identity documents.

There is however a natural {\em exculpable public key} flavor $(\pk,\sigma)$,
in which
 $\sigma = \Sign(\sk, \CommitRing(\{ \pk \},\pk).\openring, \mathtt{ring\_name}, \mathtt{""})$.
The singleton ring $\{ \pk \}$ ensure that 
$\rVerify(\CommitRing(\{\pk\}), \mathtt{ring\_name}, \mathtt{""}, \sigma)$
uniquely determines the secret key, so exculpability holds
 if joining the ring requires $(\pk,\sigma)$.

% \begin{proposition}
% \end{proposition}


\subsection{Moderation}
\label{subsec:moderation}

All discussion or collaboration sites have behavioral guidelines and
moderation rules that deeply impact their culture and collective values.

Our ring VRFs enables a simple blacklisting operation:
If a user misbehaves, then sites could blacklist or otherwise penalizes
their site local identity $\mathtt{id}$.
As $\mathtt{id}$ remains unlinked from other sites, we avoid thorny
questions about how such penalties impact the user elsewhere, and thus
can assess and dispense justice more precisely. 

At the same time, there exist sites who must forget users' histories
eventually, like under some ``right to be forgotten'' principle, either
GDPR compliance or an ethical principle of social mistakes being ephemeral.

We obtain ephemeral identities if \msg consists of the site name plus
the current year and month, or some other approximate date.
In this way, users have only one stable $\mathtt{id}$ within the
approximate date range, but they obtain fresh $\mathtt{id}$s merely
by waiting until the next month.

We could adjust \PedVRF to simultaneously prove multiple VRF input-output
pairs $(\msg_j,\mathtt{id}_j)$.
As in \cite{PrivacyPass}, we merely delinearize \In and \PreOut in
\rSign and \rVerify like:
\begin{align*}
x &= H(\msg_j,\mathtt{id}_j,\ldots,\msg_j,\mathtt{id}_j) \\
\In &= \sum_j H_p(x,j) \, \In_j \\
\PreOut &= \sum_j H_p(x,j) \, \PreOut_j \\
\end{align*}

As doing so links these pairs together,
we could link together two or more ephemeral identities like this
to obtain a semi-permanent identity with user controlled revocation:
As login, our site demands two linked input-output pairs given by
 $\msg_1 = \mathtt{site\_name} \doubleplus \mathtt{current\_month}$ and
 $\msg_2 = \mathtt{site\_name} \doubleplus \mathtt{registration\_month}$,
so users could have multiple active pseudo-nyms given by $\mathtt{id}_2$,
but only one active pseudo-nym per month, enforced by deduplicating
 $\mathtt{id}_1$, which still prevents spam and abuse.

If instead our site associates pseudo-nyms to their most recently seen
$\mathtt{id}_1$, then we could link adjacent months, meaning $\msg_j$
is defined by the $j$th previous month, until reaching a previously used $\mathtt{id}_1$.
In this model, pseudo-nyms could be abandoned and replaced, but
abandoned pseudo-nyms cannot then be reclaimed without linking intervening dates.
Although more costly, sites could permanently bans a few problematic
users via the inequality proofs described in \S\ref{subsec:AML_KYC} too.

In these ways, sites encode important aspects of their moderation rules
into the ring VRF inputs they demand.  
% We expect this makes sites' values and culture more uniform, predictable, and transparent.


\subsection{Reduced pairings}
\label{sec:reduced_pairings}

At a high level, we distinguish moderation-like applications discussed
above, which resemble classic identity applications like AML/KYC, from
rate limiting applications discussed in the next section. 
%
In moderation-like applications, ring VRF outputs become long-term
stable identities, so users typically reidentify themselves many times
to the same sites, reusing the exact same \msg.

As an optimization, our zero-knowledge continuation could reuse the
same \compk and \pifast for the same \msg, so that verifiers could
memoize their verifications of \pifast.  We spend most verifier time
checking the Groth16 pairing equation, so this saves considerable CPU time. % assuming our cache wind up fast enough.

As a concrete example, our coefficients $r_1,r_2,b$ used for
rerandomization in \S\ref{sec:rvrf_cont} could be chosen
deterministically like $r_1,r_2,b \leftarrow H(\sk,\msg)$.
In this way, each (helpful) user's $\mathtt{id}$ has a unique \pifast,
which verifiers could memoize by storing
 $(\mathtt{id},H(\compk \doubleplus \pifast),\mathtt{dates})$
after their first verification, but then skipping the Groth16 check
 after merely rechecking the hash $H(\compk \doubleplus \pifast)$.

We could risk denial-of-service attacks by users who vary $r_1,r_2,b$ 
randomly however.  We therefore suggest $\mathtt{dates}$ record the last
several previous dates when $H(\compk \doubleplus \pifast)$ changed.
We rate limit or verify more lazily users with many nearby login dates






\bibliographystyle{plain}
\bibliography{refs_misc,refs_new,refs_sass}


\end{document}





\endinput




\newpage
MERGER IN PROGRESS
\newpage


\section{Security Model of Ring VRF}


In this section, we describe the security of our new cryptographic primitive ring VRF. First, we describe the basic ring VRF in the real world and in the ideal world. Second, we extend the basic ring VRF by adding a new property that we call secret evaluation. 


\paragraph{Ring VRF (Basic Definition):} A ring VRF operates like a VRF but only proves its key comes from a specific list without giving any information about which specific key. We define the ring VRF functionality $ \fgvrf $ in Figure \ref{f:gvrf}. The functionality lets parties generate a key (Key Generation), evaluate a message with the party's key (VRF Evaluation), prove that the evaluation is executed by one of the keys (VRF evaluation and proof) and verify the evaluation without knowing the key used for the evaluation (Verification). We also define linking procedures in $ \fgvrf $ to link an evaluation and a proof with its associated key. So, if a party wants to reveal its identity at some point, it can use the linking process to prove that the evaluation is executed with its key (Linking proof). Later on, anyone can verify the linking proof (Linking verification).
%TODO Secret Evaluation
\begin{figure}\scriptsize
	\begin{tcolorbox}
		{  $ \fgvrf $ runs two PPT algorithms $ \gen_W$ and $\gen_{sign} $ during the execution.
			
			 \begin{description}
				
				\item[Key Generation.] upon receiving a message $(\msg{keygen}, \sid)$ from a party $\user_i$, send $(\msg{keygen}, \sid, \user_i)$ to the simulator $\simulator$.
				Upon receiving a message $(\msg{verificationkey}, \sid, \pkrvrf)$ from $\simulator$, verify that $\pkrvrf$ has not been recorded before; then, store in the table $\vklist$, under $\user_i$, the value $\pkrvrf$.
				Return $(\msg{verificationkey}, \sid, \pkrvrf)$ to $ \user_i$.
				
				\item[Malicious Key Generation.] upon receiving a message $(\msg{keygen}, \sid, \pkrvrf)$ from $\simulator$, verify that $\pkrvrf$ was not yet recorded, and if so record in the table $\vklist$ the value $\pkrvrf$ under $\simulator$. Else, ignore the message.
				
				%\item[Honest Ring VRF Evaluation.] upon receiving a message $(\msg{eval}, \sid, \pkring, \pkrvrf_i, m)$ from $\user_i$, verify that 
				%$\pkrvrf_i \in \pkring$ 
				%and  
				%there exists $ \pkrvrf_i $ in $\vklist $ associated with $ \user_i $. If that was not the case, just ignore the request.
				%If there exists no $ W $ such that $ \anonymouskeymap[W] = (m, \pkring, \pkrvrf_i) $, let $ W \sample \bin^\secpar $ and  $y \sample \bin^{\ell_\rvrf}$. Then, set $ \evaluationslist[m, W] = y$ and $ \anonymouskeymap[W] = (m, \pkring,\pkrvrf_i) $.
				%Return $(\msg{evaluated}, \sid, \pkring, m, W, y)$ to $ \user_i $.
				%The functionality does not check whether the evaluater's public key is in the ring because here we consider m, \pkring as an input of the evaluation which is evaluated by a party who is not neccesarily in the ring. 
				\item[Malicious Ring VRF Evaluation.] upon receiving a message $(\msg{eval}, \sid, \pkring, \pkrvrf_i, W, m)$ from $\sim$, verify that $ \pkrvrf_i $ has not been recorded under an honest party's key.
			    If it is the case record in the table $\vklist$ the value $\pkrvrf_i$ under $\simulator$. Else, ignore the request.  If $ \counter[m,\pkring] $ does not exist, initiate $ \counter[m,\pkring] = 0 $.
			    If there exists $ W $ such that $ \anonymouskeymap[W] = (m',\pkring', \pkrvrf)$ then do the following:
			    \begin{itemize}
			    	\item if $(m', \pkring', \pkrvrf) \neq  (m, \pkring, \pkrvrf_i)  $, ignore the request,
			    	\item else obtain $ y = \evaluationslist[m, W]   $. 
			    \end{itemize}
				If there exists no $ W $ such that $ \anonymouskeymap[W] = (m, \pkring, \pkrvrf_i) $, let   $y \sample \bin^{\ell_\rvrf}$ and increment $ \counter[m,\pkring] $. Then, set $ \evaluationslist[m, W] = y$, $ \anonymouskeymap[W] = (m, \pkring,\pkrvrf_i) $ .
				Return $(\msg{evaluated}, \sid, \pkring, m, W, y)$ to $ \user_i $.
				
				
				
				
				\item[Honest Ring VRF Signature.] upon receiving a message $(\msg{sign}, \sid, \pkring, \pkrvrf_i, m)$ from $\user_i$, verify that $\pkrvrf_i \in \pkring$ and that there exists a public key $\pkrvrf_i$ associated to $\user_i$ in the table $ \vklist $. If that wasn't the case, just ignore the request. 	
				If there exists no $ W' $ such that $ \anonymouskeymap[W'] = (m, \pkring, \pkrvrf_i) $, run $ \gen_W(\pkring, \pkrvrf_i, m) \rightarrow W$. Then, let $y \sample \bin^{\ell_\rvrf}$ and set $ \anonymouskeymap[W] = (m, \pkring,\pkrvrf_i) $ and set $ \evaluationslist[m, W] = y$.
%					\begin{itemize}
%						%\item If there exists $ W \in  \anonymouskeymap  $, abort.
%						\item Else 
%						%TODO define what \in \anonymouskeymap mean
%					\end{itemize}
%			    \end{itemize}
				Obtain $ W, y $ where  $ \evaluationslist[m, W] = y$, $ \anonymouskeymap[W] = (m, \pkring,\pkrvrf_i) $ and run  $ \gen_{sign}(\pkring, W, m) \rightarrow \sigma $. Verify that $ [m, W, \sigma, 0] $ is not recorded. If not, abort. Otherwise, record $ [m, W, \sigma, 1] $. Return $(\msg{signature}, \sid, \pkring,W,m, y, \sigma)$ to $\user_i$.
				
				%\item[Malicious VRF evaluation.] upon receiving a message $(\msg{evalprove}, \sid, \pkring, m)$ from $\simulator$, check that $\vklist$ has a public key associated to $\simulator$. If not, ignore the request. If $\evaluationslist[\pkring, m][\simulator]$ is not set, sample $y \sample \bin^{\ell(\secpar)}$ and set $\evaluationslist[\pkring, m][\simulator] \defeq y$ (and $\signaturelist[\pkring,m]$ to $\emptyset$). If $\signaturelist[\pkring, m]$ contains a proof (i.e., if $\signaturelist[\pkring, m]$ is not empty), return $(\msg{evaluated}, \sid, y)$ to $\simulator$. Else, ignore the request.
				
				%\item[Verification.] upon receiving a message $(\msg{verify}, \sid, \pkring, m, y, \sigma)$, from any party forward the message to the simulator. If there exists a $\pkrvrf_i$ among the values of \texttt{verification\_keys}, and there exists $\sigma \in \signaturelist[\pkring, m]$, set $b = 1$. Else, set $b =0$. Finally, output $(\msg{verified}, \sid, \pkring, m, y, \sigma, b)$.
				\item[Ring VRF Verification.] upon receiving a message $(\msg{verify}, \sid, \pkring,W, m, \sigma)$ from a party, relay the message $(\msg{verify}, \sid, \pkring,W, m, \sigma)$ to $ \simulator $ and receive back the message $(\msg{verified}, \sid, \pkring,W, m, \sigma, b_{\simulator}, \pkrvrf_\simulator)$. Then do the following: 
				\begin{enumerate}[label={{Cond.-} }{{\arabic*}}, start = 1]
					\item If there exits a record $ [m,W,\sigma, 1] $, set $ b = 1 $. (This condition guarantees the completeness meaning that if  $ W $ is an anonymous key that is generated for the ring $ \pkring $ and  the message $ m $ and the signature $ \sigma $ is legitimately generated for $ m, W $, then the verification succeeds.)
					\item Else if $ \pkrvrf  $ is an honest verification key where $ \anonymouskeymap[W] = (.,., \pkrvrf) $ and there exists no record $ [m, \pkring, W, \sigma', 1] $ for any $ \sigma' $, then let $ b= 0  $.
					(This condition guarantees unforgeability meaning that if an honest party never signs a message $ m $ for a ring $ \pkring $, then the verification fails.)\label{cond:forgery}
					\item Else if there exists a record  such as $ [m,W,\sigma, b'] $, set $ b = b' $. (This condition guarantees consistency meaning that all identical verification requests will output the same $ b $) \label{cond:consistency}
					\item Else if $ \pkrvrf  $ is an honest verification key where $ \anonymouskeymap[W] = (.,., \pkrvrf) $ and there exists a record $ [m, W, \sigma', 1] $ for any $ \sigma' $, then let $ b= b_{\simulator} $ and record $ [m, W,\sigma, b_{\simulator}] $. (This condition guarantees that if $ m $ is signed by an honest party for the ring $ \pkring $ at some point and the signature is $ \sigma' \neq \sigma $, then the decision of verification is up to the adversary) \label{cond:differentsignature}
					\item \label{cond:forgerymalicious}Else if there exists $ \anonymouskeymap[W] = (m', \pkring',.)  $ where $ (m', \pkring') \neq (m, \pkring) $ or $ \counter[m, \pkring] > |\pkring_m| $ where $ \pkring_m $ is a set of keys in $ \pkring $ which are not honest or $ b_{\simulator} = 0 $ or $ \pkrvrf_\simulator $ belongs to an honest party, set $ b = 0 $ and record $ [m, \pkring,W,\sigma, 0] $. (This condition guarantees that if $ W $ is an anonymous key of a different message and ring or the number of anonymous keys of malicious parties in $ \pkring $ is more than their number or     $ \simulator $ does not verify $ \sigma $, then the verification fails.)
					\item Else set $ b = 1 $. Set $ \evaluationslist[m, W]\sample \bin^{\ell_\rvrf}$, $ \anonymouskeymap[W]  = (m, \pkring, \pkrvrf_\simulator)$ and $ \counter[m, \pkring, \pkrvrf_\simulator] = 0 $ if they are not defined before. Increment $ \counter[m, \pkring, \pkrvrf_\simulator]  $. \label{cond:advsignature}
				\end{enumerate}
				In the end, if $ b = 0 $, set $ \out = \emptyset $. Otherwise,  set $ \out = \evaluationslist[m, W]$. 		Finally, output $(\msg{verified}, \sid, \pkring,W, m, \sigma, \out, b)$ to the party.
				
			\end{description}
		
			
		}
	\end{tcolorbox}
	\caption{Functionality $\fgvrf$.\label{f:gvrf}}
\end{figure}
	


\begin{figure}\scriptsize
	\begin{tcolorbox}
		{  This part of $ \fgvrf $ for the parties who want to show that they generate a particular ring signature.
			
		
			\begin{description}
				\item[Linking signature.] upon receiving a message $(\msg{link}, \sid, \pkring, \pkrvrf_i, W, m,\sigma)$ from $\user_i$, check that $\pkrvrf_i $ is associated to $\user_i$ in $ \vklist $, $ \anonymouskeymap[W] = (m, \pkring, \pkrvrf_i) $ and 
				check whether $ [m, W, \sigma, 1] $ is stored. If any of them fails, ignore the request. Otherwise,
				send $(\msg{link}, \sid, \pkring, W, m, y)$ to $\simulator$. Upon receiving $(\msg{linkproof}, \sid, \pkring, W, m, y, \hat \sigma)$ from $\simulator$, verify that $ [m, \pkring, \pkrvrf_i, \sigma, \hat{\sigma}, 0] $ is not stored in $ \Linklist $. If not, abort. Otherwise,  record $\hat\sigma$ to $[m, \pkring, \pkrvrf_i,\sigma, \hat{\sigma}, 1]$ to $ \Linklist $ and return $(\msg{linked}, \sid, \pkring, \pkrvrf_i,W, m, y,\sigma, \hat\sigma)$ to $\user_i$.
				%\item[Malicious linking proof.] upon receiving a message $(\msg{link}, \sid, \pkring, m, y)$ from $\simulator$, check that $\vklist$ has a key set for $\simulator$, and that it is in $R$.
				%Check that $\evaluationslist[\pkring, m][\simulator] = y$.
				%If any of the above is not satisfied, ignore the request.
				%Return $(\msg{linked}, \sid, y)$ to $\simulator$.
				\item[Linking verification.] upon receiving a message $(\msg{verifylink}, \sid, \pkrvrf_i, \pkring, W, m,\sigma,\hat\sigma)$ from any party forward the message to the simulator and receive back  the message $(\msg{verified}, \sid, \pkrvrf_i, \pkring, W,m, \sigma,\hat\sigma,  b_{\simulator})$. Then do the following:
				
				\begin{itemize}
					\item If there exits a record $ [m, \pkring,\pkrvrf_i,\sigma,\hat\sigma, 1] $ in $ \Linklist $, set $ b = 1 $ and $ \pk = \pkrvrf $. (This condition guarantees the completeness.)
					\item Else if $ \pkvrf_i $ is a key of an honest party and there exits no record such as $ [m, \pkring,\pkrvrf_i,\sigma,\hat\sigma',  1] $ for any  $  \hat\sigma'$, then set $ b = 0 $ and record $ [m, \pkring,\pkrvrf_i,\sigma,\hat\sigma,  0] $. (This condition guarantees unforgeability meaning that if an honest party never signs a message $ m $ in the linking signature, then the verification fails.)
					\item Else if there exists a record  such as $ [m, \pkring,\pkrvrf_i,\sigma,\hat\sigma,  b'] $, set $ b = b' $. 
					\item Else set $ b = b_{\simulator} $ and record $ [m, \pkring,\pkrvrf_i,\sigma,\hat\sigma,  1] $. 
				\end{itemize}
				
				Return $(\msg{verified}, \sid, \pkrvrf_i, \pkring, m, \hat\sigma, b).$ to the party.
			\end{description}
		}
	\end{tcolorbox}
	\caption{Functionality $\fgvrf$.\label{f:gvrf}}
\end{figure}



In a nutshell, the functionality $\fgvrf$, when given as input a message $m$ and a key set $\pkring$ of participant, allows to create $n$ possible different outputs pseudo-random that appear independent from the inputs. The output can be verified to have been computed correctly by one of the participants in $\pkring$ without revealing who they are. At a later stage, the author of the VRF output can prove that the output was generated by them and no other participant could have done so.

Below, we define the real-world execution of the ring VRF.
\begin{definition}[Ring-VRF (rVRF)]\label{def:ringvrf}
	Ring VRF is a VRF with a deterministic function $ F(.,.):\{0,1\}^\kappa \times\{0,1\}^* \rightarrow \{0,1\}^{\ell_\rvrf} $ and with the following algorithms:
	
	\begin{itemize}
		\item $ \rvrf.\keygen(1^\kappa) \rightarrow (\skrvrf,\pkrvrf)$ where $ \kappa $ is the security parameter,
	\end{itemize}
	Given list of public keys $ \pkring = \set{\pkrvrf_1, \pkrvrf_2, \ldots, \pkrvrf_n}$, a message $ m \in \{0,1\}^* $
	\begin{itemize}
		\item $ \rvrf.\eval(\skrvrf_i, \pkring, m)\rightarrow y $ where $ y = F(\skrvrf,\pkring,m) $,
		\item $ \rvrf.\evalprove(\skrvrf_i, \pkring, m)\rightarrow (F(\skrvrf,\pkring,m),\pi) $ where  $ \pi $ is a proof for the evaluation.
		\item $ \rvrf.\verify(\pkring, m, y,\pi) \rightarrow  b$ where $ b \in \{0,1\} $. $ b =1 $ means verified and $ b = 0 $ means not verified.
		\item $ \rvrf.\link(\skrvrf_i, \pkring,m,y, \pi) \rightarrow \pi_{\link} $ where  $ \pi^\link $ is a proof linking the public key of the producer of $ y $. 
		\item $ \rvrf.\link\verify(\pkring, \skrvrf_i, m, y, \pi,, \pi_{\link})\rightarrow b$ where $ b \in \{0,1\} $. $ b =1 $ means verified and $ b = 0 $ means not verified.
	\end{itemize}
	
\end{definition}
\paragraph{Ring VRF with Secret Evaluation:} 


Below, we define the real-world execution of the ring VRF with secret evaluation.
\begin{definition}[Ring-VRF (rVRF)]\label{def:ringvrfse}
	Ring VRF with secret evaluation is two VRFs with a deterministic function $ F(.,.):\{0,1\}^\kappa \times\{0,1\}^* \rightarrow \{0,1\}^{\ell_\rvrf} $ and$ F_s(.,.):\{0,1\}^\kappa \times\{0,1\}^* \rightarrow \{0,1\}^{\ell_\rvrf} $. It consists of the algorithms of ring VRF defined in Definition \ref{def:ringvrf} and additionally the following algorithms:
	
	Given list of public keys $ \pkring = \set{\pkrvrf_1, \pkrvrf_2, \ldots, \pkrvrf_n}$, a message $ m \in \{0,1\}^* $
	\begin{itemize}
		\item $ \rvrf.\secreteval(\skrvrf_i, \pkring, m)\rightarrow \omega $ where $ \omega = F_s(\skrvrf,\pkring,m) $,
		\item $ \rvrf.\secretprove(\skrvrf_i, \pkring, m)\rightarrow \pi / \perp $ where  $ \pi $ is a proof for the secret evaluation. If $ (F_s(\skrvrf,\pkring,m), (\pkring, m)) \notin \rel  $, it outputs $ \perp $.
		\item $ \rvrf.\secretverify(\pkring, m,\pi) \rightarrow  b$ where $ b \in \{0,1\} $. $ b =1 $ means verified and $ b = 0 $ means not verified.
	\end{itemize}
	
\end{definition}

\begin{figure}
	\scriptsize
	\begin{tcolorbox}
	{
			%\par\hrulefill\\
			 $ \fgvrf^{zk} $ for a relation $ \mathcal{R} $ behaves exactly as $ \fgvrf $ and  it additionally does the following:
			\begin{description}
				\item[Secret Evaluation.] upon receiving a message $(\msg{secreteval}, \sid, \pkring,\pkvrf_i, m)$ from $\user_i$, verify that $\pkrvrf_i$ is in $ \pkring $ and that there exists a key $\user_i$ with an associated public key $\pkrvrf_i$ in $\vklist$. If that was not the case, just ignore the request.
				If $\evaluationslist[\pkring, m][\pkrvrf_i]$ is empty, then sample a new $y \sample \bin^{\ell(\secpar)}$ and  store $\evaluationslist[\pkring, m][\pkrvrf_i] = y$. In any case, sample a random element $ \eta  \sample \bin^{\ell(\secpar)} $ and store $ \evaluationsecretlist[\pkring, m][\pkvrf_i] = \eta $. Create a set $ \proofzklist[\pkring,m][\pkvrf_i] = \emptylist $, if it does not exist. Return $(\msg{evaluated}, sid, y, \eta)$ to $ \user_i $.
				
				\item[Secret evaluation and proof.] upon receiving a message $(\msg{secretprove}, \sid, \pkring, \pkvrf_i, m)$ from $\user_i$, verify that $\pkrvrf_i \in \pkring $ and  that there exists a key $ \user_i $ with an associated public key $ \pkrvrf_i $ in $ \vklist $. (If that wasn't the case, just ignore the request.). Obtain $ (y, \eta) $ from $\evaluationslist[\pkring, m][\pkvrf_i]$ and $\evaluationsecretlist[\pkring, m][\pkvrf_i]$. If they are not defined execute the steps in VRF evaluation and secret evaluation.
				If $ ((m, y, \pkring),(\eta,\pkvrf_i)) \notin \mathcal{R} $, ignore the request.  Otherwise, send $(\msg{zkprove}, \sid, \pkring, m, y)$ to $\simulator$. Upon receiving $(\msg{zkproof}, \sid, \pkring, m, y \pi^{zk})$ as a response from $\simulator$,  add $ \pi^{zk} $ to  the set $\proofzklist[\pkring, m][\pkvrf_i]$. Return $(\msg{zkproof}, \sid, y, \eta, \pi^{zk})$ to $\user_i$.
				
				
				\item[Secret Verification.] upon receiving a message $(\msg{secretverify}, \sid, \pkring, m, \pi^{zk})$, from a party, check whether there exists  $ \pkvrf_i \in \vklist $ such that $ \proofzklist[\pkring,m][\pkvrf_i] = \pi^{zk} $, $ \evaluationslist[\pkring,m][\pkrvrf_i] = y $ and $ \evaluationsecretlist[\pkring,m][\pkrvrf_i] $. If there exists, set $b = 1$. Else, set $b =0$. Finally, output $(\msg{verification}, \sid, \pkring, m, \pi^{zk}, b)$.
			\end{description}
		}
	\end{tcolorbox}
	\caption{Functionality  $ \fgvrf^{zk} $.\label{f:gvrfzk}}
\end{figure}

\section{A Ring VRF Construction without Secret Evaluation} 
\label{sec:ringvrfconstr}
Our ring VRF construction $\pvrf$ works in the $ \fcrs $-hybrid model described in Figure \ref{f:crs}. We need the common reference string (crs) in our scheme for SNARK proofs.
\begin{figure}
	\scriptsize
	\begin{tcolorbox}
		$ \fcrs $ is parametrized with the distribution $ \distribution $.
		{  \begin{description}
				\item [\textbf{CRS Generation:}] pick a value $ crs \sample \distribution $
				\item [\textbf{CRS Distribution:}] upon receiving the message $ (\msg{learn\-crs}, \sid) $, return the message $ (\msg{CRS}, \sid, crs) $.
			\end{description}
		}
	\end{tcolorbox}
	\caption{Functionality $\fcrs$.\label{f:crs}}
\end{figure}
We instantiate parameter generation by constructing a group $\GG$ in which discrete logarithm is hard, two generators $ G_1, G_2 \in  \GG$, and for which there exists a \textbf{s}uccinct
\textbf{n}on-interactive \textbf{ar}guments of \textbf{k}nowledge (SNARK) provides efficient proofs of elliptic curve arithmetic and efficient Merkle tree proofs.  We consider three hash functions: a hash-to-group function $\hashG : \{0,1\}^* \rightarrow \GG$ and $ \hash, \hash' : \{0,1\}^* \rightarrow \FF_p $. Our ring VRF construction works as follows:

\begin{itemize}
	\item $ \rvrf.\keygen(1^\kappa):  $ It selects randomly a secret key $ x \in \FF_p$ and computes the public key $ X = xG_1 $. It outputs $ \skrvrf = x $ and $ \pkrvrf = X $.
	\item $ \rvrf.\eval(\skrvrf, \pkring, m) $: It lets $ P = \hashG(m, \pkring) $ and computes $ W = xP  $. Then, it outputs $ y = \hash(m, \pkring, W) $. So, the VRF $ F $ in our protocol is $ F(\skvrf, \pkring, m) = H(m, \pkring, x\hashG(m,\pkring)) $.
	\item $ \rvrf.\evalprove(\skrvrf, \pkring, m) :$ It lets $ P = \hashG(m, \pkring) $ and computes $ W= xP, y = \hash(m, \pkring, W) $. The proving algorithm works as follows:
	\begin{itemize}
		\item It first commits its secret key $ \skvrf = x$ i.e., $ C = X + \beta G_2 $ where $ \beta \sample \FF_p $.
		\item It generates the first proof $ \pi_1 $ showing the following relation:
		$$\rel_1 = \{((x, \beta), (G_1, G_2, C,W,P)): C = xG_1 + \beta G_2, W = xP \}$$
		For this, it generates a non-interactive Schnorr proof with the Fiat-Shamir transform:  Sample random $r_1, r_2 \leftarrow \F_p$.
		Let $R = r_1 G_1 + r_2 G_2$, $R_m = r_1 P$, and
		$c = \hash'(\pkring, m, W,C,R,R_m)$.
		Set $\pi_1 = (c,s,\delta)$ where $s = r_1 - c x$ and $\delta = r_2 - c \beta$.
		\item It obtains $ crs $ from $ \fcrs $ for the second proof by sending the message $ (\msg{learncrs}, \sid) $ to $ \fcrs $. Then, it constructs a Merkle tree $ \mathsf{MT} $ of the public keys in $ \pkring $ whose root is denoted by $ \mathsf{root} $. In the end, it generates the second proof $ \pi_2 $ for the following relation with  the witness $ (\mathsf{copath}, x, \beta) $. 
		$$ \rel_2 = \{((\mathsf{copath}, x, \beta),(\mathsf{root}, C, G_1, G_2)): C-\beta G_2 = xG_1 = X, \mathsf{MT}.\verify(\mathsf{copath}, X, \mathsf{root} ) \rightarrow 1\} $$
		Here, $ \mathsf{copath} $ is a copath of the Merkle tree $ \mathsf{MT} $. $ \mathsf{MT}.\verify(\mathsf{copath}, X, \mathsf{root} ) $ is a verification algorithm of the Merkle tree which verifies whether $ X $ is the one of the leaves of $ \mathsf{MT} $ i.e., compute a root $ \mathsf{root}' $ with $ X $ and $ \mathsf{copath} $ and output 1 if $ \mathsf{root} = \mathsf{root}' $.
		
		The second proof $ \pi_2 $ is generated by running the proving algorithm of the SNARK i.e., $ \mathsf{SNARK}.\prove(\rel_2, crs,((\mathsf{root}, C, G_1, G_2)), (\mathsf{copath}, x, \beta)) $ \cite{groth16}.
	\end{itemize}
		In the end, $ \rvrf.\evalprove $ outputs $ y, \pi  $ where $ \pi = (\pi_1, \pi_2, C, W) $
	\item $ \rvrf.\verify(\pkring, m, y, \pi) $: It verifies the proofs $ \pi_1 $ and $ \pi_2 $.
	\begin{itemize}
		\item For the first proof $ \pi_1 = (c,s, \delta) $, it lets $R = s G_1 + \delta G_2 + c C$ and $R_m = s \hashG(m, \pkring) + c W$.
		Return true if $c = \hash'(\pkring,m,W,C,R,R_m)$.
		\item For the second proof $ \pi_2 $, it obtains $ crs $ from $ \fcrs $ by sending the message $ (\msg{learncrs}, \sid) $. Then, it constructs the Merkle tree $ \mathsf{MT} $ of the public keys $ \pkring $ and obtains $ \mathsf{root} $ which is the root of $ \mathsf{MT} $. In the end, it verifies $ \pi_2 $ by running the verification algorithm of the SNARK, i.e., $ \mathsf{SNARK}.\verify(\rel_2,crs, ((\mathsf{root}, C, G_1, G_2)), \pi_2) $.
	\end{itemize}
	If two of the verification algorithms output 1 and $ y =  \hash(m, \pkring,W)  $, $  \rvrf.\verify $ outputs 1. Otherwise, it outputs 0.
	\item $ \rvrf.\link(\skrvrf, \pkring, m, y,\pi): $ It generates a discrete logarithm equality proof $ \pi_{link} $ for the following relation:
	$$\rel_{dleq} = \{(x, (G_1, X,P,W)): X = xG_1, W = xP\}$$
	For this, it generates a non-interactive Schnorr proof with the Fiat-Shamir transform:  Pick $ r \sample \FF_p $ and let $ R = rG_1 $. Set $ c_\link = \hash'( \pkvrf,m,R,rP)) $ and $ s_\link = r_\link + cx $. In the end, it outputs the linking proof $ \pi_{link} = (c_\link,s_\link) $.
	\item $ \rvrf.\link\verify(\pkring, m, y, \pi, \pi_{link}) $: It obtains $ W $ from $ \pi $ and computes $ P = \hashG(m, \pkring) $. It outputs 1, if $ c = \hash'(\pkvrf,m, sG_1 - cX, sP-cW) $ and $ y = \hash(m, \pkring,W) $.
\end{itemize}


We next show that our ring VRF construction realizes $ \fgvrf $ in the $ \fcrs $-hybrid model and in the random oracle model under the assumption of the decisional Diffie Hellman (DDH) and Gap Diffie-Hellman (GDH) problems. The GDH problem is solving the computational DH problem by accessing the Diffie-Hellman oracle ($ \mathsf{DH}(.,.,.) $) which tells that given triple $ X,Y,Z $ is a DH-triple i.e., $ Z = xyG $ where $ X = xG $ and $ Y = yG $.

%\begin{definition}[$ n $-One-More Gap Diffie-Hellman (OM-GDH) problem]
%	Given   $ p $-order group $ \GG $ generated by $ G $, the challenges $ G, X = xG, P_1, P_2, \ldots, P_{n+1} $ and access to the DH oracle $ \mathsf{DH}(.,.,.) $ and the oracle $ \mathcal{O}_x(.) $ which returns $ xP $ given input $ P $, if a PPT adversary $ \mathcal{A} $ outputs $ xP_1, xP_2, \ldots, xP_{n+1} $ with the access of at most $ n $-times to the oracle $ \mathcal{O}_x $, then $ \mathcal{A}  $ solves the $ n $-OM-GDH problem. We say that $ n $-OM-GDH problem is hard in $ \GG $, if for all PPT adversaries, the probability of solving the $ n $-OM-GDH problem is negligible in terms of the security parameter.
%\end{definition}
\newcommand{\malkeys}{\mathsf{malicious}\_\vklist}
\newcommand{\hkeys}{\mathsf{honest}\_\vklist}
\begin{theorem}
	Assuming that $ \hashG, \hash, \hash' $ are random oracles, the GDH problem and the DDH problem is hard in the group structure $ (\GG, G_1, p) $ and SNARK is a zero-knowledge, $\pvrf$ UC-realizes $\fgvrf$ in the $\fcrs$-hybrid model according to Definition \ref{def:uc}.
\end{theorem}

\begin{proof}
	We construct a simulator $ \simulator $ that simulates the honest parties in the execution of $ \pvrf  $ and simulates the adversary in $ \fgvrf $. Moreover,
	$ \simulator $ behaves as $ \fcrs $ against the real world adversaries. 
	\begin{itemize}
		\item \textbf{[Simulation of $ \fcrs $:] }When simulating $ \fcrs $, it runs $ \mathsf{SNARK}.\mathsf{SetUp}(\rel_2) $ which outputs a trapdoor $ \tau $ and $ crs $ instead of picking $ crs $ randomly from the distribution $ \distribution $. Whenever a party comes to learn the common reference string, $ \simulator $ gives $ crs $ as  $ \fcrs $.
		
		\item \textbf{[Simulation of $ \msg{keygen} $:]} Upon receiving $(\msg{keygen}, \sid, \user_i)$ from the functionality $\fgvrf$, $ \simulator $ samples $x \sample \FF_p$ and defines the key $X \defeq xG$. It adds $ xG $ to $ \hkeys $ and $ \vklist $ as a key of $ \user_i $. 
		In the end, $ \simulator $ returns $(\msg{verificationkey}, \sid, X)$ to $\fgvrf$. %Whenever the honest party $ \user $ is corrupted by $ \env, $ $ \simulator $ moves the key of $ \user $ to $ \malkeys $ from $ \hkeys $.
		
		\item\textbf{[Simulation of the random oracles:]} We  describe how $ \simulator $ simulates the random oracles $ \hashG, \hash $ and $ \hash' $ against the real world adversaries. 	$ \simulator $ simulates  $ \hash' $  as a usual random oracle.
		
		$ \simulator $ simulates the random oracle $ \hashG $ as described in Figure \ref{oracle:Hg}. In short, it selects a random element  $ h $ from $ \FF_p $ for each new input and outputs $ hG_1 $ as an output. Thus, $ \simulator $ knows the discrete logarithm of each random oracle output of $\hashG  $.
		\begin{figure}
			\centering
			
			\noindent\fbox{%
				\parbox{7cm}{%
					\underline{\textbf{Oracle $ \hashG $}} \\
					\textbf{Input:} $ m, \pkring $ \\
					\textbf{if} $\mathtt{oracle\_queries\_gg}[m, \pkring] = \perp  $
					
					\tab{$ h \sample \FF_p $}
					
					\tab{$ P \leftarrow hG_1 $} 
					
					\tab{$\mathtt{oracle\_queries\_gg}[m, \pkring] := h$}
					
					\textbf{else}:
					
					\tab{$ h \leftarrow \mathtt{oracle\_queries\_gg}[m, \pkring] $}
					
					\tab{$ P \leftarrow hG_1$}
					
					\textbf{return $ P $}
					
			}}	
			\caption{The random oracle $ \hashG $}
			\label{oracle:Hg}
		\end{figure}
	
		The simulation of the random oracle $ \hash $ is less straightforward (See Figure \ref{oracle:H}). Whenever an input $ m, \pkring, W $ is given to $ \hash $, it first obtains the discrete logarithm $ h $ of $ \hashG(m, \pkring) $ from the $ \hashG $'s database. It needs this information to detect whether $ W $ could be the value which is generated for the evaluation (See $ \rvrf.\eval $ in $ \pvrf $). Remark that if it is $ W $-value (pre-output) of the evaluation, it should be equal to $ x^*\hashG(m, \pkring)= x^*hG_1 $  where $ x^* \in \FF_p$ is a some ring VRF secret key. For this, it obtains $ X^* = h^{-1}W $. If $ X^* \in \pkring $, $ \simulator $ consider this oracle call as  a computation of evaluation of the  message $ m $ with the public key $ X^* $ and the ring $ \pkring $.
		If $ X^* $ has not been registered as a malicious key, it registers it to $ \fgvrf $. Thus, $ \simulator $ has a right to ask to evaluation of the message $ m $ to $ \fgvrf $. It asks the evaluation of the message $ m $ with the key $ X^* $ and the ring $ \pkring $ and learns $ y $ from $ \fgvrf $ which is the evaluation of $ m $. Then, it sets $ y $ as the answer of the random oracle $ \hash $  for the input $ m, \pkring, W $. If $ X^* $ is registered as an honest party's key and $ \hash(m,\pkring,W) $ is not defined before, $ \simulator $ aborts and the simulation ends.
		
		\begin{figure}
			\centering
			
			\noindent\fbox{%
				\parbox{11cm}{%
					\underline{\textbf{Oracle $ \hash$}} \\
					\textbf{Input:} $ m, \pkring,W $ \\			
					$ P \leftarrow \hashG(m,\pkring) $\\			
					$ h \leftarrow \mathtt{oracle\_queries\_gg}[m, \pkring] $\\
					$ X^* := h^{-1}W $ // candidate key \\
					\textbf{if $ X^* \notin \pkring $} // it is not a evaluation query
					
					\tab{$ y \sample \FF_p $}
					
					\tab{$  \mathtt{oracle\_queries\_h}[m, \pkring, W]:=y $}
					
					\textbf{else:}
					
					\tab{\textbf{if} $ X^* \notin \vklist$// if the key has not been registered} 
					
					\tabdbl{\textbf{send} $ (\msg{keygen}, \sid, X^*) $ \textbf{to} $ \fgvrf $}
					
					\tabdbl{\textbf{add} $ X^* $ \textbf{to} $ \malkeys $ \textbf{and} $ \vklist $}
					
					\tab{\textbf{if} $ X^* \in \malkeys$}
					
					\tabdbl{\textbf{send} $ (\msg{eval}, \sid, \pkring, X^*, m) $ \textbf{to} $ \fgvrf $}
					
					\tabdbl{\textbf{receive} $ (\msg{evaluated}, \sid, \pkring, m, y) $ \textbf{from} $ \fgvrf $}
					
					\tabdbl{$  \mathtt{oracle\_queries\_h}[m, \pkring, W]:=y $}
					
					\tab{\textbf{else if:} $ \mathtt{oracle\_queries\_h}[m, \pkring, W]  = \perp$}
					
					\tabdbl{\textbf{return} \textsc{Abort}}
					
					\textbf{return $  \mathtt{oracle\_queries\_h}[m, \pkring, W] $}
					
			}}	
			\caption{The random oracle $ \hash $}
			\label{oracle:H}
		\end{figure}
	
	\item \textbf{[Simulation of $ \msg{evalprove} $]} 
	%The simulator has table  $\preoutputlist $ to keep the pre-outputs that it selects for each input and the ring of public keys. 
	Upon receiving $(\msg{evalprove}, \sid, \pkring, m)$  from the functionality $\fgvrf$, $ \simulator $ generates the proof $ \pi $ as follows:
	
	For the first proof, it samples $ c, s, \delta \in \FF_p $ and $ C, W \in \GG$. Then, it lets the first proof be $\pi_1 =  (c, s, \delta) $. 
	In addition, it sets $ R = sG_1+ \delta G_2- cC $ and $ R_m = s \hashG(m, \pkring)+ cW $ and lets $ \pkring,m, W,C, R, R_m$ be $ c $ in the table of the random oracle $ \hash' $ so that $ \pi_1 $ verifies in the real-world execution.  
	%It adds $ W $ to the list $ \preoutputlist[m, \pkring] $.
	
	 $ \simulator $ gets the trapdoor $ \tau $ that it generated during the simulation of $ \fcrs $ to simulate the second proof. Then, it runs $ \mathsf{SNARK}.\mathsf{Simulate}(\rel_2,\tau, crs) $ and obtains $ \pi_2 $.
	 
	 In the end, $ \simulator $  responds by sending the message $(\msg{evalprove}, \sid, \pkring, m, \pi = (\pi_1, \pi_2, C, W))$ to the $ \fgvrf $. 
	\item \textbf{[Simulation of $ \msg{verifiy} $]} Upon receiving  $(\msg{verify}, \sid, \pkring, m, y, \pi)$ from the functionality $\fgvrf$,  it runs $ \rvrf.\verify(\pkring, m, y, \pi) \rightarrow b $. If there exists no $ m, \pkring, W $ such that $ \mathtt{oracle\_queries\_h}[m, \pkring, W] = y $ and  $ b = 1  $, $ \simulator $ aborts. Otherwise, it returns $ (\msg{verified}, \sid, \pkring, m, y, \pi, b) $ to  $\fgvrf  $.
%	\par\hrulefill
	\item \textbf{[Simulation of $ \msg{link} $:]} Upon receiving $(\msg{link}, \sid, \pkring, X_i, m, y, \pi)$ from $\fgvrf$, $ \simulator $ obtains $ W $ from $ \pi $. Remark that $ \simulator $ gave $ \pi $ to $ \fgvrf $, so $ W $ must be the part of $ \pi $. Then, it picks random $ s, c $ and lets $ (X_i, m, sG_1-cX_i, s\hashG(m, \pkring)- cW) $ be $ c $ in the table of the random oracle $ \hash' $. It sends $ (\msg{link}, \sid, X_i,(s,c)) $ to $ \fgvrf $. Also, if $ \hash(m,\pkring, W) $ is not defined, it sets $ \mathtt{oracle\_queries\_h}[m, \pkring, W] = y$. If it is defined but not equal to $ y $, it aborts.
	
	%TODO \item receiving $(\msg{verifylink}, \sid, \vk_i, R, m, y, \hat\pi)$, send $(\msg{verify}, \sid, (X, H, Y), m, \hat\pi)$ to $\fnizk$.
	\item \textbf{[Simulation of outputs of ideal honest parties:]} Whatever an honest party outputs in the ideal-world, $ \simulator $ outputs the same in the real-world simulation as an output of the same honest party. If the honest party's output is an evaluation value $ y $ and a proof $ \pi $ of an input $ m, \pkring $, $ \simulator $ checks whether the proof $ \pi $ of the evaluation $ y $ with the input $ m, \pkring $ is valid. For this,
	% it checks whether $ \prooflist[\pi] $ is assigned to an input $ m $ and a ring key set $ \pkring $. If it is the case, 
	it sends $ (\msg{verify}, \sid, \pkring, m, y, \pi) $ to $ \fgvrf $. If $ \fgvrf $ verifies it, $ \simulator $ retrieves $ W $ from $ \pi $ and sets $ \mathtt{oracle\_queries\_h}[m, \pkring, W] $ as $ y $ if $ \mathtt{oracle\_queries\_h}[m, \pkring, W] $ is not defined. $ \mathtt{oracle\_queries\_h}[m, \pkring, W] $ was already defined with another value, $ \simulator $ aborts. 
	\end{itemize}
	
	
	\begin{claim} 
		The view of $ \env $ in its interaction with the simulator $ \simulator $ is indistinguishable from the view in its interaction with real honest parties.
	\end{claim}
	
	\begin{proof}
		We prove this claim via a sequence of games. The initial game corresponds to the real protocol, whereas the final game corresponds to the simulator $\simulator$ described above. In each game, we change one (or more) step of in the ring VRF construction with the steps which are different in the simulation above.
		\begin{enumerate}[label={{Game} }{{\arabic*}}, start = 0]
		%\setlength\itemsep{0.1em}
			\item The simulator $ \simulator $ simulates the honest parties as in the ring VRF protocol that we describe in Section \ref{sec:ringvrfconstr}. 
			
			\item This game is the same as the previous game except that $ \simulator $ simulates the $ \fcrs $ functionality.  When simulating $ \fcrs $, it runs $ \mathsf{SNARK}.\mathsf{SetUp}(\rel_2) $ which outputs a trapdoor $ \tau $ and $ crs $ instead of picking $ crs $ randomly from the distribution $ \distribution $ differently than the described $ \fcrs $ in Figure \ref{f:crs}. This game is indistinguishable from the previous game because 
			$ \mathsf{SNARK}.\mathsf{SetUp}(\rel_2) $ generates $ crs $ from the same distribution $ \distribution $. %TODO Is this true?
			
			\item This game is the same as in the previous game except that $ \simulator $ simulates the second proof of an input $ m, \pkring $ when running $ \evalprove $ by running  $ \mathsf{SNARK}.\mathsf{Simulate}(\rel_2, \tau, crs) $. Since SNARK has the zero-knowledge property, there exists a $ \mathsf{Simulate} $ algorithm which outputs a indistinguishable proof from the real proof. Therefore, this game is indistinguishable from the previous game.
			
			\item \label{game:roracles}This game is the same as in the previous game except that $ \simulator $ starts to simulate the random oracles  $ \hashG$  as described in Figure \ref{oracle:Hg}	and $ \hash' $ as a usual random oracle. Since $ \hashG $'s outputs are indistinguishable from a usual random oracle simulation of $ \hashG $, this game is same as the previous game.
			
			
			\item \label{game:DDH}This game is the same as the previous game except that $ \simulator $ simulates the first proof of an input $ m, \pkring $ when running $ \evalprove $ as follows: it samples $ c,s,\delta $ randomly from $ \FF_p $ and $ C, W \in \GG $. Then, it lets the first proof be $\pi_1 =  (c, s, \delta, C, W) $. 
			In addition, it sets $ R = sG_1+ \delta G_2- cC $ and $ R_m = s \hashG(m, \pkring)+ cW $ and lets $ \pkring,m, W,C, R, R_m$ be $ c $ in the table of the random oracle $ \hash' $. Since the discrete logarithm of $ W $'s generated for the first proofs are unknown to $ \simulator $ in this game, it simulates the link proof $ \pi_\link $ of an input $ \pkring, X_i, m, y, \pi $ as follows: it obtains $ W $ from $ \pi $ and picks random $ s_\link,c_\link \in \FF_p$ and sets $ X_i, m, s_\link G_1 - c_\link X_i, s\hashG(m,\pkring) - c_\link W$ to be $ c_\link $ in the table of the oracle $ \hash' $. Remark that $ s,\delta,C $ in $ \pi_1 $ and $ s_\link $ in $ \pi_\link $ are from the uniformly random distribution as in the previous game. Therefore, they are indistinguishable. However, $ W $ is not computed as $x \hashG(m, \pkring) $ as in the previous game. Therefore, we should show that selecting $ W $ randomly from $ \GG $ and computing $ W $ as $x \hashG(m, \pkring) $ are indistinguishable.
			We  show this under assumption that the decisional Diffie-Hellman (DDH) problem  is hard. We use the hybrid argument to show this.
			
			We define hybrid games $ H_{i} $ where $ \rvrf.\eval $ and $ \rvrf.\evalprove $ of the first $ i $ honest parties are computed as in the previous game and the rest are computed as in this game. Without loss of generality, $ \user_1, \user_2, \ldots, \user_{n_h} $ are the honest parties. Thus, $ H_0 $ is equivalent to \ref{game:DDH} and $ H_{n_h}  $ is equivalent to \ref{game:roracles}.  We construct an adversary $ \mathcal{B} $ that breaks the DDH problem given that there exists an adversary that distinguishes hybrid games $ H_i $ and $ H_{i + 1} $ for $ 0 \leq i < n_h $. $ \mathcal{B} $ receives the DDH challenges $ X,Y, Z \in \GG $ from the DDH game and simulates the game against $ \adv $ as follows: $ \mathcal{B} $ generates the public key of all  honest parties' key as usual by running $ \rvrf.\keygen$ except party $ \user_{i+1} $. It lets $ \user_{i+1} $'s public key be $ X $. Differently, it simulates the random oracle $ \hashG $ against $ \adv $ as described in Figure \ref{oracle:HgbyB}. Remark that this simulation is indistinguishable.
			
			\begin{figure}
				\centering
				
				\noindent\fbox{%
					\parbox{8cm}{%
						\underline{\textbf{Oracle $ \hashG $ in \ref{game:DDH} by the DDH adversary $ \mathcal{B} $}} \\
						\textbf{Input:} $ m, \pkring $ \\
						\textbf{if} $\mathtt{oracle\_queries\_gg}[m, \pkring] = \perp  $
						
						\tab{$ h \sample \FF_p $}
						
						\tab{\fbox{$ P \leftarrow hY $}}
						
						\tab{$\mathtt{oracle\_queries\_gg}[m, \pkring] := h$}
						
						\textbf{else}:
						
						\tab{\fbox{$ P \leftarrow hY $}}
						
						\textbf{return $ P $}
						
				}}	
				\caption{The simulation of the random oracle $ \hashG $ by $ \mathcal{B} $. The different steps than Figure \ref{oracle:Hg} are in the box.}
				\label{oracle:HgbyB}
			\end{figure}
			
			$ \mathcal{B} $ simulates the first $ i $ parties as in \ref{game:roracles} and the parties $ \user_{i+2}, \ldots, \user_q $ as in \ref{game:roracles}. The simulation of $ \user_{i + 1} $ is different. Whenever, $ \user_{i+1} $ needs to output evaluation of an input $ m, \pkring $, it obtains $ P = \hashG(m, \pkring) = hY $ from $ \mathtt{oracle\_queries\_gg} $, lets $ W = hZ $ and outputs $ \hash(m, \pkring, W) $. Remark that if $ (X,Y,Z)$ is a DH triple (i.e., $  \mathsf{DH}(X,Y,Z) \rightarrow 1 $), $ \user_{i+1} $ is simulated as in \ref{game:roracles} because $ W = xP $ in this case. Otherwise, $ \user_{i+1} $ is simulated as in \ref{game:DDH} because $ W $ is random. So, if $  \mathsf{DH}(X,Y,Z)  \rightarrow 1$, $ \mathcal{B} $ simulates $ H_{i+1} $. Otherwise, it simulates $ H_{i} $. In the end of the simulation, if $ \adv $ outputs $ i $, $ \mathcal{B} $ outputs $ 0 $ meaning $  \mathsf{DH}(X,Y,Z) \rightarrow 0$. Otherwise, it outputs $ i + 1 $. The success probability of $ \mathcal{B} $'s is equal to the success probability of $ \adv $ who distinguishes $ H_i $ and $ H_{i +1} $. Since DDH problem is hard, $ \mathcal{B} $ has negligible advantage in the DDH game. So, $ \adv $ has a negligible advantage too. Hence, from the hybrid argument, we can conclude that $ H_0  = $\ref{game:DDH} and $ H_q = $ \ref{game:roracles} are indistinguishable.
			
			
			
			\item \label{game:gdh}This game is the same as in the previous game except that $ \simulator $ starts to simulate the random oracle $ \hash $ as described in Figure \ref{oracle:Hgame}. We remark that the output of the simulation of $ \hash $ in this game and the output of the simulation of $ \hash $ in Figure \ref{oracle:H} are the same. The simulation of $ \hash $ in this game does not ask for the evaluation $ m, \pkring, W $ to $ \fgvrf $ for a malicious key $ X^* $ instead it picks the evaluation output randomly from $ \FF_p $. Since  $ \fgvrf $ also picks the evaluation output of $ m, \pkring, W $ from $ \FF_p $ randomly, this difference does not change the output distribution of the simulation of the random oracle $ \hash $ in Figure \ref{oracle:H} and $ \hash $ in Figure \ref{oracle:Hgame}.
			
				\begin{figure}
				\centering
				
				\noindent\fbox{%
					\parbox{8cm}{%
						\underline{\textbf{Oracle $ \hash$}} \\
						\textbf{Input:} $ m, \pkring,W $ \\			
						$ P \leftarrow \hashG(m,\pkring) $\\			
						$ h \leftarrow \mathtt{oracle\_queries\_gg}[m, \pkring] $\\
						$ X^* := h^{-1}W $ // candidate key \\
						\textbf{if $ X^* \notin \pkring $}
						
						\tab{$ y \sample \FF_p $}
						
						\tab{$  \mathtt{oracle\_queries\_h}[m, \pkring, W]:=y $}
						
						{\textbf{else if} $ X^* \in \malkeys$}
						
						\tab{$ y \sample \FF_p $}
						
						\tab{$  \mathtt{oracle\_queries\_h}[m, \pkring, W]:=y $}
						
						{\textbf{else if} $ \mathtt{oracle\_queries\_h}[m, \pkring, W]  = \perp$}
						
						\tab{\textbf{return} \textsc{Abort}}
						
						\textbf{return $  \mathtt{oracle\_queries\_h}[m, \pkring, W] $}
						
				}}	
				\caption{The random oracle $ \hash $}
				\label{oracle:Hgame}
			\end{figure}
			
			The only difference from the simulation of $ \hash $ in this game from the random oracle $ \hash $ in the previous game is aborting when $ X^* $ belongs to an honest party and $  \mathtt{oracle\_queries\_h}[m, \pkring, W]  $ is not defined before. Therefore, we analyse the probability that the abort occurs during the simulation of $ \hash $ in this game. Intuitively, if the abort occurs, it means that the adversary obtained $ W = x^*P $ which has not been published by the honest party with the public key $ X^* $ because $ W $ is randomly selected instead of computing $ W = x^*P $ in this game. In a nutshell, we next show that if the adversary is able to obtain $ W = x^*P $, then we can construct another adversary $ \mathcal{B} $ that breaks the Gap-Diffie-Hellman (GDH) problem.
			
			Consider a GDH game in the group $ \GG $ with the generator $ G_1 $, challenges $ X, Y \in \GG$ and access to the Diffie-Hellman oracle $ \mathsf{DH} $. The GDH challenges are given to the adversary $ \mathcal{B} $ and $ \mathcal{B} $ starts to simulate \ref{game:gdh} to the adversary $ \adv $. The simulation is exactly the same except that $ \mathcal{B} $ picks a random $ r_x \in \ZZ_p $ and generates the public key of each party as $ r_xX $ and also simulates the random oracles $ \hash $ and $ \hashG $ differently as described in Figure \ref{oracle:HbyB} and Figure \ref{oracle:HgbyB}. Remark that $ \mathcal{B} $ in this game never needs to know the secret key of honest parties to simulate them since pre-outputs are selected randomly. Therefore, generating the honest public keys as this is indistinguishable. 			
			$ \mathcal{B} $ simulates the random oracle $ \hashG $ as described in Figure \ref{oracle:HgbyB}. In short, it generates the random group element by exponentiating the challenge $ Y $  instead of the generator $ G_1 $.  
			
			$ \mathcal{B} $ simulates the random oracle $ \hash $ as described in Figure \ref{oracle:HbyB}. The only difference from Figure \ref{oracle:Hgame} is the way of finding the candidate key if it exists. Remember that $ \hashG $ simulation is different. Therefore, when $ \mathcal{B} $ computes $ h^{-1}W $, it does not find a candidate key $ X^* $ if $ W $ is generated from one of the public keys in $ \pkring $ i.e., $ W = x^*\hashG(m, \pkring) $. The reason of this is if $ W $ is generated from one of the public keys in $ \pkring $, $X =  h^{-1}W = h^{-1} x^*\hashG(m, \pkring)  = x^*Y \neq X^*$. Therefore, $ \bdv $ gets advantage of having the access to $ \mathsf{DH} $ oracle to learn whether $ W $ is generated from one of the public keys in $ \pkring $ i.e., check whether $ \mathsf{DH}(X^*, Y, Z) \rightarrow 1$ for each $ X^* \in \pkring $. This step is indistinguishable from checking whether $ X^* = h^{-1}W $ in the simulation in Figure \ref{oracle:Hgame} because it does the same computation in a different way.
			
			\begin{figure}
				\centering
				
				\noindent\fbox{%
					\parbox{8cm}{%
						\underline{\textbf{Oracle $ \hash$}} \\
						\textbf{Input:} $ m, \pkring,W $ \\			
						$ P \leftarrow \hashG(m,\pkring) $\\			
						$ h \leftarrow \mathtt{oracle\_queries\_gg}[m, \pkring] $
						\begin{mdframed}
							
							$ X^* := \mathsf{null} $
							
							$ Z := h^{-1}W $  
							
							\textbf{for $ X \in \pkring $}
							
							\tab{\textbf{if} $ \mathsf{DH}(X^*,Y,Z) \rightarrow 1 $}
							
							\tab{$ X^* := X $}
							
						\end{mdframed}
						\textbf{\textbf{if} $ X^* \notin \pkring $}
						
						\tab{$ y \sample \FF_p $}
						
						\tab{$  \mathtt{oracle\_queries\_h}[m, \pkring, W]:=y $}
						
						{\textbf{else if} $ X^* \in \malkeys$}
						
						\tab{$ y \sample \FF_p $}
						
						\tab{$  \mathtt{oracle\_queries\_h}[m, \pkring, W]:=y $}
						
						{\textbf{else if} $ \mathtt{oracle\_queries\_h}[m, \pkring, W]  = \perp$}
						
						\tab{\textbf{return} \textsc{Abort}}
						
						\textbf{return $  \mathtt{oracle\_queries\_h}[m, \pkring, W] $}
						
				}}	
				\caption{The simulation of the random oracle $ \hash $ by $ \mathcal{B} $. The different steps than Figure \ref{oracle:Hgame} are in the box.}
				\label{oracle:HbyB}
			\end{figure}
			
			If $ \mathcal{B} $ aborts during the simulation of $ \hash $, it means that $ X^* $ belongs to an honest party and $ DH(X^*,Y,Z) \leftarrow 1 $. We know the honest public key $ X^* $ is generated as $ r^*X $ for $ r^* \in \FF_p $. Therefore, $ (r^*)^{-1}Z $ is the CDH solution of $ X,Y $. This shows that if the probability of abort in the random oracle $ \hash $ is equal to the probability of solving the GDH problem. Therefore, \ref{game:gdh} and \ref{game:DDH} are indistinguishable.
			
%			Consider the one-more GDH game in the group $ \GG $ with the generator $ G_1 $, challenges $ X = xG_1, P_1, P_2, \ldots, P_{q_{h_g}} \in \GG$ and access to the Diffie-Hellman oracle $ \mathsf{DH}(.,.,.) $ and the oracle  $\mathcal{O}_x(.) $ which returns $ xP $ given $ P $. The OM-GDH challenges are given to the adversary $ \mathcal{B} $ and $ \mathcal{B} $ starts to simulate the Game 3 to the adversary $ \adv $. The simulation is exactly the same except that $ \mathcal{B} $ picks a random $ r_i \in \ZZ_p $ and generates the public key of each party as $ X_i = r_iX $. In this case, $ \mathcal{B} $ does not know the corresponding secret keys. Therefore, when it needs to run the evaluation, it obtains the pre-output value of the given input $ \pkring,m $ and public key $ X_i $ by asking $ \mathcal{O}_x(.) $ with the input $ P $ where $ P = \hashG(m, \pkring) $ and lets the pre-output be $ W = r_i\mathcal{O}_x(P) $ (See Figure \ref{sim:preoutputs}).  Since $ n_e \leq q_h $, $ \mathcal{B} $ has enough source to compute the pre-outputs. 
%			
%			
%			\begin{figure}
%				\centering
%				
%				\noindent\fbox{%
%					\parbox{5cm}{%
%						\underline{\textbf{Computation of  pre-outputs}} \\
%							\textbf{Input:} $ X_i, m, \pkring $ 
%							
%							$ P := \hashG(m, \pkring) $
%							
%							\textbf{if}   $ \lst[P] = \perp $
%							
%							\tab{$ Q \leftarrow \mathcal{O}_x(P) $}
%							
%							\tab{$ \lst[P] := Q $}
%							
%							\textbf{else}:
%							
%							\tab{$ Q := \lst[P] $}
%							
%							\textbf{return $ r_iQ $}
%							
%					}}	
%					\caption{The simulation of pre-outputs by $ \mathcal{B} $. }
%					\label{sim:preoutputs}
%				\end{figure}
%			
%			
%			
%			Besides, $ \mathcal{B} $ simulates the random oracles $ \hash $ and $ \hashG $ differently as described in Figure \ref{oracle:HbyB} and Figure \ref{oracle:HgbyB}.  
%			
%			The simulation of the random oracle $ \hashG $, $ \bdv $ returns $ P_i $ (one of the OM-GDH challenge). Since $ P_1, P_2, \ldots, P_{q_{h_g}} $ are generated randomly. the new simulation of $ \hashG $ is indistinguishable from the simulation in Game 3 (Figure \ref{oracle:Hg}).
%			
%			\begin{figure}
%				\centering
%				
%				\noindent\fbox{%
%					\parbox{8cm}{%
%						\underline{\textbf{Oracle $ \hashG $ in Game 2 by the CDH adversary $ \mathcal{B} $}} \\
%						\textbf{Input:} $ m, \pkring $ 
%						
%						$ \ell := \ell + 1 $
%						
%						\textbf{if} $\mathtt{oracle\_queries\_gg}[m, \pkring] = \perp  $
%						  
%						\tab{\fbox{$ P \leftarrow P_\ell $}}
%						
%						\tab{$\mathtt{oracle\_queries\_gg}[m, \pkring] := P$}
%						
%						\textbf{else}:
%						
%						\tab{\fbox{$ P \leftarrow \mathtt{oracle\_queries\_gg}[m, \pkring] $}}
%						
%						\textbf{return $ P $}
%						
%				}}	
%				\caption{The simulation of the random oracle $ \hashG $ by $ \mathcal{B} $. The different steps than Figure \ref{oracle:Hg} are in the box.}
%				\label{oracle:HgbyB}
%			\end{figure}
%			
%			$ \mathcal{B} $ simulates the random oracle $ \hash $ as described in Figure \ref{oracle:HbyB}. The only difference is the way of finding out whether the query is possibly done for the evaluation. Remember that $ \hashG $ is simulated differently by $ \mathcal{B} $. 
%			Therefore, computing $ h^{-1}W $ does not work in this simulation to deduce the candidate key because $ h $ which is the discrete logarithm of $ \hashG(m, \pkring) $ is not known. Instead, $ \bdv $ gets advantage of having the access to $ \mathsf{DH} $ oracle to learn whether $ W $ is generated from one of the public keys in $ \pkring $ i.e., check whether $ \mathsf{DH}(X^*, P, W) \rightarrow 1$ for each $ X^* \in \pkring $ where $ P = \hashG(m, \pkring) $. This step is indistinguishable from checking whether $ X^* = h^{-1}W $ in the simulation in Figure \ref{oracle:Hgame} because it does the same computation.
%			
%			
%			%, it does not find a candidate key $ X^* $ if $ W $ is generated from one of the public keys in $ \pkring $ i.e., $ W = x^*\hashG(m, \pkring) $. The reason of this, if $ W $ is generated from one of the public keys in $ \pkring $, $X =  h^{-1}W = h^{-1} x^*\hashG(m, \pkring)  = x^*Y \neq X^*$. Therefore, $ \bdv $ gets advantage of having the access to $ \mathsf{DH} $ oracle to learn whether $ W $ is generated from one of the public keys in $ \pkring $ i.e., check whether $ \mathsf{DH}(X^*, Y, Z) \rightarrow 1$ for each $ X^* \in \pkring $. This step is indistinguishable from checking whether $ X^* = h^{-1}W $ in the simulation in Figure \ref{oracle:Hgame}.
%				
%				\begin{figure}[h]
%				\centering
%				
%				\noindent\fbox{%
%					\parbox{8cm}{%
%						\underline{\textbf{Oracle $ \hash$}} \\
%						\textbf{Input:} $ m, \pkring,W $ \\			
%						$ P \leftarrow \hashG(m,\pkring) $\\			
%						\sout{$ h \leftarrow \mathtt{oracle\_queries\_gg}[m, \pkring] $}
%						\begin{mdframed}
%						
%						$ X^* := \mathsf{null} $
%						
%						%$ Z := h^{-1}W $  
%					
%						\textbf{for $ X \in \pkring $}
%						
%						\tab{\textbf{if} $ \mathsf{DH}(X^*,P,W) \rightarrow 1 $}
%						
%						\tab{$ X^* := X $}
%						
%					   \end{mdframed}
%						\textbf{\textbf{if} $ X^* \notin \pkring $}
%						
%						\tab{$ y \sample \FF_p $}
%						
%						\tab{$  \mathtt{oracle\_queries\_h}[m, \pkring, W]:=y $}
%						
%						{\textbf{else if} $ X^* \in \malkeys$}
%						
%						\tab{$ y \sample \FF_p $}
%						
%						\tab{$  \mathtt{oracle\_queries\_h}[m, \pkring, W]:=y $}
%						
%						{\textbf{else if} $ \mathtt{oracle\_queries\_h}[m, \pkring, W]  = \perp$}
%						
%						\tab{\textbf{return} \textsc{Abort}}
%						
%						\textbf{return $  \mathtt{oracle\_queries\_h}[m, \pkring, W] $}
%						
%				}}	
%				\caption{The simulation of the random oracle $ \hash $ by $ \mathcal{B} $. The removed steps are crossed out. The different steps than Figure \ref{oracle:Hgame} are in the box.}
%				\label{oracle:HbyB}
%			\end{figure}
%		
%			If $ \mathcal{B} $ aborts during the simulation of $ \hash $, it means that $ \mathcal{A} $ obtained $ W $ before the honest party with the public key $ X^* $ runs the evaluation algorithm. Given that $ \mathcal{B} $ runs the evaluation at most $ n_e \leq  q_{h_g} $ times, if abort occurs during the simulation of $ \hash $, the number of evaluations is when the abort occurs at most $ q_{h_g} - 1$. The reason of this, if $ n_e = q_{h_g} $, it means that all  $ \hashG $ queries are used up by $ \mathcal{B} $ for number of  $ n_e = q_{h_g} $ evaluations and so the abort case in the simulation of $ \hash $ cannot occur. Therefore, when  abort occurs in $ \hash $, it means that $ n_e \leq q_{h_g} - 1 $.  In the end, $ \mathcal{B} $ outputs $ \{\lst{P}\}_{P \in \setsym{C}} $ to the OM-GDH game. The success probability of $ \mathcal{B} $ solving the OM-GDH problem is equal to the probability that abort occurs in $ \hash $. So, the probability of abortion during the simulation of $ \hash $ in this game is negligible.
			
			\item This game is the same as the previous game except that if $ \simulator $ receives evaluation and proof $ y, \pi $ for the input $ m, \pkring $, but $ \mathtt{oracle\_queries\_h}[m, \pkring, W] $ where $ W \in \pi $ is not defined and $ \rvrf.\verify(\pkring, m, y, \pi) \rightarrow 1$, $ \simulator $ aborts. If $ \rvrf.\verify(\pkring, m, y, \pi) \rightarrow 1$,  $ y = \hash(m, \pkring,W) $. So, it means that $ \adv $ guessed the output of $ \hash(m, \pkring,W) $ without asking the random oracle $ \hash $. Since the probability that it happens is negligible, this game is indistinguishable from the previous game.
			
%			Simulation verify is indistinguishable because the only way for the environment to learn $ y $ is asking the evaluation of the $ m, \pkring $ from the adversary. The adversary can learn it from the random oracle. This game guarantees that the evaluation is stored in the random oracle $ \hash $'s database 
			The last game is exactly the same as the simulator that we described for $ \fgvrf $. This shows that the simulator $ \simulator $ that we describe for $ \fgvrf $ is indistinguishable from the real protocol. So, the environment $ \env $ cannot distinguish the real-world simulation from the real protocol except with a negligible probability. 
				
		
		\end{enumerate}
	\qed
	\end{proof}	

	\begin{claim}
		The distributions of the outputs of honest and malicious parties in ideal and real worlds are indistinguishable.
	\end{claim}
	$ \simulator $ outputs whatever the honest parties output in the ideal-world. 
	Therefore, the output of the honest parties are identical. While outputting them, it updates the random oracles respectively so that the verification process of these outputs (either $ \rvrf.\verify $ or $ \rvrf.\link\verify $) verifies in the real protocol. The cases where $ \simulator $ aborts happens if corresponding random oracle input is already assigned to another value which can happen with a negligible probability. Therefore, the outputs in the real and ideal world are indistinguishable.
	
\qed
\end{proof}


\section{A Ring VRF Construction with Secret Evaluation} 
	



