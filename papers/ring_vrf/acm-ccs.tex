%%
%% This is file `sample-sigconf.tex',
%% generated with the docstrip utility.
%%
%% The original source files were:
%%
%% samples.dtx  (with options: `sigconf')
%% 
%% IMPORTANT NOTICE:
%% 
%% For the copyright see the source file.
%% 
%% Any modified versions of this file must be renamed
%% with new filenames distinct from sample-sigconf.tex.
%% 
%% For distribution of the original source see the terms
%% for copying and modification in the file samples.dtx.
%% 
%% This generated file may be distributed as long as the
%% original source files, as listed above, are part of the
%% same distribution. (The sources need not necessarily be
%% in the same archive or directory.)
%%
%%
%% Commands for TeXCount
%TC:macro \cite [option:text,text]
%TC:macro \citep [option:text,text]
%TC:macro \citet [option:text,text]
%TC:envir table 0 1
%TC:envir table* 0 1
%TC:envir tabular [ignore] word
%TC:envir displaymath 0 word
%TC:envir math 0 word
%TC:envir comment 0 0
%%
%%
%% The first command in your LaTeX source must be the \documentclass
%% command.
%%
%% For submission and review of your manuscript please change the
%% command to \documentclass[manuscript, screen, review]{acmart}.
%%
%% When submitting camera ready or to TAPS, please change the command
%% to \documentclass[sigconf]{acmart} or whichever template is required
%% for your publication.
%%
%%
\documentclass[sigconf]{acmart}

%%
%% \BibTeX command to typeset BibTeX logo in the docs
\AtBeginDocument{%
  \providecommand\BibTeX{{%
    Bib\TeX}}}

%% Rights management information.  This information is sent to you
%% when you complete the rights form.  These commands have SAMPLE
%% values in them; it is your responsibility as an author to replace
%% the commands and values with those provided to you when you
%% complete the rights form.
\setcopyright{acmcopyright}
\copyrightyear{2018}
\acmYear{2018}
\acmDOI{XXXXXXX.XXXXXXX}

%% These commands are for a PROCEEDINGS abstract or paper.
\acmConference[Conference acronym 'XX]{Make sure to enter the correct
  conference title from your rights confirmation emai}{June 03--05,
  2018}{Woodstock, NY}
%%
%%  Uncomment \acmBooktitle if the title of the proceedings is different
%%  from ``Proceedings of ...''!
%%
%%\acmBooktitle{Woodstock '18: ACM Symposium on Neural Gaze Detection,
%%  June 03--05, 2018, Woodstock, NY}
\acmPrice{15.00}
\acmISBN{978-1-4503-XXXX-X/18/06}


%%
%% Submission ID.
%% Use this when submitting an article to a sponsored event. You'll
%% receive a unique submission ID from the organizers
%% of the event, and this ID should be used as the parameter to this command.
%%\acmSubmissionID{123-A56-BU3}

%%
%% For managing citations, it is recommended to use bibliography
%% files in BibTeX format.
%%
%% You can then either use BibTeX with the ACM-Reference-Format style,
%% or BibLaTeX with the acmnumeric or acmauthoryear sytles, that include
%% support for advanced citation of software artefact from the
%% biblatex-software package, also separately available on CTAN.
%%
%% Look at the sample-*-biblatex.tex files for templates showcasing
%% the biblatex styles.
%%

%%
%% The majority of ACM publications use numbered citations and
%% references.  The command \citestyle{authoryear} switches to the
%% "author year" style.
%%
%% If you are preparing content for an event
%% sponsored by ACM SIGGRAPH, you must use the "author year" style of
%% citations and references.
%% Uncommenting
%% the next command will enable that style.
%%\citestyle{acmauthoryear}

\def\eprint#1#2{#2} % eprint
\def\doublecolumn#1#2{#1}
%%
%% end of the preamble, start of the body of the document source.
% \usepackage{amsthm}
\usepackage{amsfonts}
\usepackage{amsmath}
\usepackage{mathtools}
\usepackage{algorithm}

\usepackage[noend]{algpseudocode}

\usepackage[utf8]{inputenc}
\usepackage{xspace}

\usepackage{url}
\usepackage{hyperref}
% \usepackage[capitalize,nameinlink]{cleveref}

% \usepackage{graphicx}
% \usepackage{xcolor}

% \usepackage{tikz}
% \usetikzlibrary{arrows,chains,matrix,positioning,scopes}

\usepackage{enumitem}
% \setlist[itemize]{leftmargin=*}

\usepackage{etoolbox}
\makeatletter
\patchcmd{\@maketitle}% <cmd>
{\end{center}}% <search>
{\bigskip\small\@date
\end{center}}% <replace>
{}{}% <success><failure>
\makeatother

% Italics should be prefered over underlines..
% \usepackage{ulem}

% Avoid boxes if pssible..
% \usepackage{framed}
% \usepackage{fancybox}
% \usepackage{tcolorbox}
% ..but if required mdframe splits across pages.
% \usepackage[linewidth=1pt]{mdframed}

% TODO: Clean up macros

% \newcommand\doubleplus{+\kern-1.3ex+\kern0.8ex}
\newcommand\doubleplus{\ensuremath{\mathbin{+\mkern-10mu+}}}


%% oracles
\newcommand{\ora}[1]{\ensuremath{\mathcal{O}\mathsf{#1}}\xspace}
\newcommand{\oramsg}[1]{\ensuremath{\mathtt{#1}}\xspace}
%% algorithm
\newcommand{\algo}[1]{{\textsc{#1}}}
%%primitive algo
\newcommand{\primalgo}[1]{{\ensuremath{\mathsf{#1}}}\xspace}
%%primitive
\newcommand{\prim}[1]{{\ensuremath{\mathsf{#1}}}\xspace}
%%set
\newcommand{\setsym}[1]{{\ensuremath{\mathcal{#1}}}}
%%array
\newcommand{\arraysym}[1]{{\ensuremath{\mathsf{#1}}}}

\newcommand\N{\mathbb{N}}
\newcommand\F{\mathbb{F}}
% \newcommand\Gr{\mathbb{G}}


\def\mathperiod{.}
\def\mathcomma{.}



\newcommand*\set[1]{\{ #1 \}} % in text, we don't want {} to grow
\newcommand*\Set[1]{\left\{ #1 \right\}}
\newcommand*\setst[2]{\{ #1 | #2 \}}
\newcommand*\Setst[2]%
        {\left\{\,#1\vphantom{#2} \;\right|\left. #2 \vphantom{#1}\,\right\}}
% ``set such that''; puts in a vertical bar of the right height


\section{Primitives}
\label{sec:lambda}

\def\ecE{{\mathbb{E}}}
\def\grE{{\mathbf{E}}}
\def\genE{E}
\def\genG{G}
\def\genB{K} %{\genE_{\mathrm{bind}}}

\def\ecJ{{\mathbb{J}}}
\def\grJ{{\mathbf{J}}}
\def\genJ{J}

% As our ring VRF is built by composing them, 
We briefly recall the primitives and security assumptions underlying both Chaum-Pederson proofs and pairing based zkSNARKs. 


\subsection{Elliptic curves}

We obey mathematical and cryptographic implementation convention by using additive notation for elliptic curve and multipicative notation for eliptic curve scalar multiplications. 

We always implicitly have a paramater generation procedure $\mathtt{Params}$ that takes a security level $\lambda \in \N$ and returns elliptic curve paramaters including some prime numbers and efficient algorithms for computing elliptic curve operations.  As customary, we treat $\lambda$ and the output of $\mathtt{Params}$ as fixed paramaters, which makes sense because humans run $\mathtt{Params}$ manually in practice. 

As implicit outputs of $\mathtt{Params}$, we work with an elliptic curve $\ecE[\F]$ over some base field $\F$ of (prime) characteristic $q_{\grE}$, along with a distinguished subgroup $\grE \le \ecE[\F]$ of prime order $p_{\grE} \approx 2^{2\lambda}$.  As $\grE$ distinguishes $\ecE[\F]$, we let $h_{\grE}$ denote the cofactor of $\grE$ in $\ecE[\F]$, meaning $\ecE[\F]$ has $h_{\grE} p_{\grE}$ points.
% but abbreviate $h = h_{\grE}$, $p = p_{\grE}$, and $q = q_{\grE}$ when $\grE$ is clear from context.
We write $\grE$ without subscript, and abbreviate $h = h_{\grE}$, $p = p_{\grE}$, and $q = q_{\grE}$, when $\ecE$ is either our uinque pairing friendly curve or else the only curve in view.

We let $H_p : \{0,1\}^* \to \F_p$ or $H_q : \{0,1\}^* \to \F_q$ denote random oracles (RO) with a range $\F_p$ or $\F_q$.  We let $H_\ecE : \{0,1\}^* \to \ecE$ or $H_\grE : \{0,1\}^* \to \grE$ denote a hash-to-curve for $\ecE$ or hash-to-group for $\grE$, which we model as a random oracle too.  We note $H_\grE(x) = h H_\ecE(x)$ always works, although more efficent exist.

\smallskip

Almost all SNARKs like \cite{groth16} or \cite{plonk} work on a pairing friendly elliptic curve $\ecE$ over a field $\F_q$ of characteristic $q \approx 2^{2\lambda}$, which comes equipped with a type III pairing on subgroups of prime order $p \approx 2^{2\lambda}$:  We let $q_1,q_2,q_T$ denote small powers of $q$, and let $\grE_1 \le \ecE[\F_{q_1}]$ and $\grE_2 \le \ecE[\F_{q_2}]$ and $\grE_T \le \F_{q_T}^\times$ denote subgroups all of prime order $p$.  We also let $e : \ecE_1 \times \ecE_2 \to \ecE_T$ denote a type III pairing, meaning a computable bilinear map without known efficiently computable maps between $\ecE_1$ and $\ecE_2$.  Also $q_i = q_{\grE_i}$ for $i=1,2$ in our above notation.  

Any pairing friendly elliptic curve $\ecE$ provides solutions to the decisional Diffie-Hellman problem (DDH).  We do however assume the computational Diffie-Hellman problem (CDH) remains hard in $\ecE$.  We remark that $H_\grE$ being a random oracle plus CDH hardness prevents computable relationships between $H_\grE$ outputs.

% TODO: Pairing assumptions required by Groth16

\smallskip

% We shall require ZCash Sapling style ``Jubjub'' Edwards curves, whose base field characteristic divides of the order of a pairing friendly elliptic curve, thereby making Jubjub base field arithmetic SNARK friendly, and hence Jubjub elliptic curve operations as well \cite{}.

We let $\ecJ$ denote a ZCash Sapling style ``JubJub'' Edwards curve associated to the pairing friendly elliptic curve $\ecE$, meaning $\ecJ$ has base field $\F_p$ whose characteristic $q_{\grJ} = p$ matches the group order $p$ of $\grE_1 \cong \grE_2 \cong \grE_T$.  As in ZCash Sapling, we now prove statements about elliptic curve operations inside $\ecJ$ by proving base field arithmetic in $\F_p$, which our $q_{\grJ} = p$ makes reltively inexpensive inside SNARKs on $\ecE$.

As above, $\grJ \le \ecJ[\F_p]$ has large prime order $p_{\grJ}$ and a small cofactor $h_{\grJ}$.  We always support $4 p_{\grJ} < p$ because if $\ecJ$ is an Edwards curve then $h_{\grJ} \ge 4$ which imposes this by the Hasse bound.

\smallskip

We ask that deserialization prove that putative elements of $\grE$ lie in $\ecE[\F]$ by verifying curve equations, perhaps including twist checks.  We do sometimes require that deserialization checks membership in the prime order subgroup $\grE$ by checking $|\grE| X = 1$ or similar.  Yet, we also sometimes work in $\ecE[\F]$ directly when judicious multiplications by $h_{\grE}$ suffice.  

Anytime $\ecE$ represents a pairing friendly curve then we do ask that deserialization prove elements of $\grE_1$, $\grE_2$, and $\grE_T$ lie inside the correct subgroup of order $p$.  As our SNARKs handle with points in $\ecJ$ directly, we prefer writing $\grJ$ equations in $\ecJ[\F_p]$ and explicitly describe where one clears the cofactor $h_{\grJ}$.  We handled $\grE$ with $\ecE$ not necessarily pairing friendly similarly to $\ecJ$.  We scrape by with only CDH hardness for $\grJ$ for convenience, although DDH winds up hard in $\grJ$.



\subsection{zkSNARKs}

We require details about Groth16 \cite{groth16} becuase our zero-knowledge continuations demand rerandomizing existing zkSNARKs. % which only Groth16 supports.

We thus have an implicit setup procedure $\mathtt{Setup}$ that take our paramaters generated by $\mathtt{Params}$ and produces a structured reference string (SRS).  In practice, our SRS consists of points in $\ecE_1$ and $\ecE_2$ with specific relationships.  We suppress both $\mathtt{Setup}$ and $\mathtt{Params}$ whenever convenient, but $\mathtt{Setup}$ makes an appearance in \S\ref{sec:srs_second_stage}.

TODO: AGM

TODO: Describe Groth16 \cite{groth16} 

































\endinput



BROKEN BOLOW THIS




We fix $J \in \ecJ$ as a generator for public keys.  Any $\KeyGen$ algorithm randomly samples a secret keys $\sk \in \F_q$ and then computes its associate public keys $\pk = \sk J$.  We shall not discuss infrastructure that authorizes public keys.  Yet although our results do not require proof-of-knowledge on $\pk$ per se, we still strongly recommend that back certifications accompany any certificates that authorize $\pk$.

\smallskip






\newcommand{\tab}[1]{\hspace{.03\textwidth}\rlap{#1}}
\newcommand{\tabdbl}[1]{\hspace{.05\textwidth}\rlap{#1}}
\newcommand{\tabdbldbl}[1]{\hspace{.07\textwidth}\rlap{#1}}
\newcommand{\tabdbldbldbl}[1]{\hspace{.19\textwidth}\rlap{#1}}


\newcommand{\Gen}{\ensuremath{\mathsf{Gen}}}

\newcommand{\anonymouskeymap}{\ensuremath{\mathtt{anonymous\_key\_map}}}
\newcommand{\anonymouskeylist}{\mathcal{W}}
\renewcommand{\sim}{\simulator}
\newcounter{FunCond}
\newcommand{\game}[3][]{\operatorname{#2}^{#1}_{#3}(\secpar)}
\newcommand{\transcript}[1]{\langle #1 \rangle}
\newcommand{\eppt}{\pccomplexitystyle{EPPT}}
\newcommand{\pt}{\pccomplexitystyle{PT}}

% \renewcommand{\pcadvstyle}[1]{\ensuremath{\mathsf{#1}}}
% \newcommand{\zdv}{\pcadvstyle{Z}}

% \newcommand{\msg}[1]{\mathsf{#1}}

\newcommand{\simulator}{\ensuremath{\mathsf{Sim}}}
%\newcommand{\minote}[1]{\todo[color=green!30,inline]{\textbf{Michele says:} #1}}

\newcommand{\fvrf}{\mathcal{F}_{\textsf{vrf}}}
\newcommand{\fgvrf}{\mathcal{F}_{\textsf{rvrf}}}
\newcommand{\fcpke}{\mathcal{F}_{\mathsf{CPKE}}}
\newcommand{\pvrf}{\mathsf{\Pi}_{\textsf{rvrf}}}
\newcommand{\svrf}{\simulator_\mathsf{gvrf}}
\newcommand{\fnizk}{\mathcal{F}_{\textsf{nizk}}}
\newcommand{\fkes}{\mathcal{F}_{\textsf{sgke}}}
\newcommand{\fcom}{\ensuremath{\mathcal{F}_{\mathsf{com}}}}
\newcommand{\fsec}{\ensuremath{\mathcal{F}_\mathsf{ED-SMT}}}
\newcommand{\frsc}{\ensuremath{\mathcal{F}_{\mathsf{rSC}}}}
\newcommand{\fsasle}{\ensuremath{\mathcal{F}_{\mathsf{sle}}}}
\newcommand{\finit}{\ensuremath{\mathcal{F}_{\mathsf{init}}}}
\newcommand{\fsig}{\mathcal{F}_{\mathsf{sig}}}
\newcommand{\fros}{\mathcal{F}_{\mathsf{ros}}}
\newcommand{\fzkvrf}{\mathcal{F}_{\mathsf{zkvrf}}}
\newcommand{\fcommit}{\mathcal{F}_{\mathsf{commit}}}
\newcommand{\gclock}{\mathcal{G}_{\mathsf{clock}}}
\newcommand{\fcrs}{\mathcal{F}_{crs}}
\newcommand{\env}{\mathcal{Z}}
\newcommand{\stake}{\mathsf{st}}
\newcommand{\stakeset}{\setsym{ST}}

\newcommand{\sid}{\textsf{sid}}
\newcommand{\pid}{\textsf{pid}}
\newcommand{\user}{\mathsf{P}}
\newcommand{\defeq}{\coloneqq}


\newcommand{\evaluationslist}{\texttt{evaluations}}
\newcommand{\evaluationsecretlist}{\texttt{secrets}}
\newcommand{\vklist}{\texttt{signing\_keys}}
\newcommand{\siglist}{\texttt{signatures}}
\newcommand{\prooflist}{\texttt{proofs}}
\newcommand{\proofzklist}{\texttt{zkproofs}}
\newcommand{\Linklist}{\texttt{links}}
\newcommand{\emptylist}{\emptyset}
\newcommand{\fail}{\mathbf{fail}}
\newcommand{\R}{\mathsf{R}}
\newcommand{\bool}{\textit{bool}}
\newcommand{\lst}{\setsym{L}}
\newcommand{\distribution}{\setsym{D}}

\newcommand{\weak}{\ensuremath{W}}
\newcommand{\inbox}{\ensuremath{\setsym{I}}}
\newcommand{\dqueue}{\ensuremath{\setsym{Q}^\D}}
\newcommand{\wqueue}{\ensuremath{\setsym{Q}^\weak}}
\newcommand{\weaklist}{\ensuremath{\setsym{\weak}}}
\newcommand{\mID}{\ensuremath{\mathsf{mid}}}
\newcommand{\plist}{\ensuremath{\setsym{P}}}
\newcommand{\timeoutlist}{\ensuremath{\setsym{T}}}
\newcommand{\anony}{\ensuremath{\mathfrak{a}}}
\newcommand{\dleqr}{\R_\textsf{dleq}}
\newcommand{\view}{\mathsf{view}} 
\newcommand{\preoutputlist}{\arraysym{pre\-outputs}}



\renewcommand{\msg}{\ensuremath{\mathsf{input}}\xspace}
\renewcommand{\aux}{\ensuremath{\mathsf{ass}}\xspace}


\newcommand{\PedVRF}{\primalgo{PedVRF}} 

\newcommand\baseL{\mathcal{L}}
\newcommand\Lrvrf{\ensuremath{\baseL_{\mathtt{rvrf}}}\xspace}
\newcommand\Leval{\ensuremath{\baseL_{\mathtt{eval}}}\xspace}
\newcommand\Lring{\ensuremath{\baseL_{\mathtt{ring}}}\xspace}
\newcommand\Lfast{\ensuremath{\baseL_{\mathtt{fast}}}\xspace}

\newcommand\baseR{\mathcal{R}}
\newcommand\Reval{\ensuremath{\baseR_{\mathtt{eval}}}\xspace}
\newcommand\Rring{\ensuremath{\baseR_{\mathtt{ring}}}\xspace}
\newcommand\Rfast{\ensuremath{\baseR_{\mathtt{fast}}}\xspace}

\newcommand\hsis{{h'}}
\newcommand\ecEsis{{\mathbb{G}'}}
\newcommand\grEsis{{\mathbf{G}'}}

\newcommand\Lsk{\ensuremath{\baseL_{\mathtt{sk}}}\xspace}
\newcommand\Lpk{\ensuremath{\baseL_{\mathtt{pk}}}\xspace}

\newcommand\rrSNARK{\primalgo{Groth16}\xspace}
\newcommand\rrSNARKweak{\primalgo{Groth16/KZG}\xspace}

\newcommand\pieval{\ensuremath{\pi_{\mathtt{eval}}}\xspace}
\newcommand\piring{\ensuremath{\pi_{\mathtt{ring}}}\xspace}

\newcommand\pifast{\ensuremath{\pi_{\mathtt{fast}}}\xspace}
% \newcommand\pifastdot{\ensuremath{\dot{\pi}_{\mathtt{fast}}}\xspace}
\newcommand\pisk{\ensuremath{\pi_{\mathtt{sk}}}\xspace}
\newcommand\pipk{\ensuremath{\pi_{\mathtt{pk}}}\xspace}

\newcommand{\SpecialG}{\ensuremath{\primalgo{SpecialG}}\xspace}
\newcommand{\Preprove}{\ensuremath{\primalgo{Preprove}}\xspace}
\newcommand{\Reprove}{\ensuremath{\primalgo{Reprove}}\xspace}
\newcommand{\inner}{\mathtt{inner}}

\def\maybestack#1#2{\eprint{ #1, #2 }{
    \begin{aligned}
        &#1, \\
        % \exists \openring \textrm{\ s.t.\ }
        &#2  \\      
    \end{aligned}
}}



\usepackage{tcolorbox}
\tcbset{colback=white}
% \usepackage{todonotes}
\sloppy

\begin{document}

%%
%% The "title" command has an optional parameter,
%% allowing the author to define a "short title" to be used in page headers.

\title{Ring Verifiable Random Functions and Zero-Knowledge Continuations}

%%
%% The "author" command and its associated commands are used to define
%% the authors and their affiliations.
%% Of note is the shared affiliation of the first two authors, and the
%% "authornote" and "authornotemark" commands
%% used to denote shared contribution to the research.
%\author{Ben Trovato}
%\authornote{Both authors contributed equally to this research.}
%\email{trovato@corporation.com}
%\orcid{1234-5678-9012}
%\author{G.K.M. Tobin}
%\authornotemark[1]
%\email{webmaster@marysville-ohio.com}
%\affiliation{%
%  \institution{Institute for Clarity in Documentation}
%  \streetaddress{P.O. Box 1212}
%  \city{Dublin}
%  \state{Ohio}
%  \country{USA}
%  \postcode{43017-6221}
%}
%
%\author{Lars Th{\o}rv{\"a}ld}
%\affiliation{%
%  \institution{The Th{\o}rv{\"a}ld Group}
%  \streetaddress{1 Th{\o}rv{\"a}ld Circle}
%  \city{Hekla}
%  \country{Iceland}}
%\email{larst@affiliation.org}
%
%%
%% By default, the full list of authors will be used in the page
%% headers. Often, this list is too long, and will overlap
%% other information printed in the page headers. This command allows
%% the author to define a more concise list
%% of authors' names for this purpose.



\begin{abstract}
	
\def\eprintsmallskip{\smallskip}{}%
We introduce a new cryptographic primitive,  named
\emph{ring verifiable random function (ring VRF)}. Ring VRF combines properties of VRF  and ring signatures, offering verifiable unique, pseudorandom outputs while ensuring  anonymity of the output and message authentication. We design its security in the universal composability (UC) framework and construct two protocols secure in our model.
We also formalize a new notion of \emph{zero-knowledge (ZK) continuations} allowing for the reusability of proofs by randomizing and enhancing the efficiency of one of our ring VRF schemes. We instantiate this notion with our protocol $ \SpecialG $
which allows a prover to reprove a statement in a constant time and be unlikable to the previous proof(s). 

%ty to sign a message 
%% which enables better anonymous credentials...
%% Anonymized
%%\eprint{Ring VRFs are}{We introduce ring VRFs, which are}
%Ring VRF is a ring signature that proves a correct evaluation
%of a random, while hiding the signer's
%identity within a ring, some set of possible signers. We design a ring VRF protocol which has efficient instantiations with our novel {\em zero-knowledge continuation} technique.
%% \eprint{We propose ring VRFs as a natural fulcrum around which a diverse array of zkSNARK circuits turn, making them an ideal target for optimization and eventually standards.}{}
%We demonstrate a {\em zero-knowledge continuation} technique,
%which works by adjusting a Groth16 trusted setup to hide public inputs
%when rerandomizing the Groth16, ensuring that muliple uses of a proof generated once are unlinkable.  We then build ring VRFs that amortizes
%expensive ring membership proofs across many ring VRF signatures.
%%
%Our ring VRF needs only eight $\mathcal{G}_1$ and two
%$\mathcal{G}_2$ scalar multiplications, making it the only ring signature
%with performance competitive with group signatures.
%
%A ring VRF can be used to obtain a unique pseudo-anonymous identity from a given a list of identities.
%By using a different input for the ring VRF in different contexts, we can generate a pseudonym for each context that is unlinkable between different contexts. 
%We discuss applications that range across the anonymous credential space.

%Ring VRFs produce a unique identity for any given context but remain
%unlinkable between different contexts.  These unlinkable but unique
%pseudonyms provide a far better balance between user privacy and service
%provider or social interests than attribute based credentials like IRMA credentials.
%Ring VRFs also support anonymously rationing or rate limiting resource
%consumption that winds up vastly more flexible and efficient than
%purchases via money-like protocols.

%We define the security of ring VRFs in the universally composable (UC) model and show that our protocol is UC secure.

\end{abstract}

%%
%% The code below is generated by the tool at http://dl.acm.org/ccs.cfm.
%% Please copy and paste the code instead of the example below.
%%
%\begin{CCSXML}
%<ccs2012>
% <concept>
%  <concept_id>10010520.10010553.10010562</concept_id>
%  <concept_desc>Computer systems organization~Embedded systems</concept_desc>
%  <concept_significance>500</concept_significance>
% </concept>
% <concept>
%  <concept_id>10010520.10010575.10010755</concept_id>
%  <concept_desc>Computer systems organization~Redundancy</concept_desc>
%  <concept_significance>300</concept_significance>
% </concept>
% <concept>
%  <concept_id>10010520.10010553.10010554</concept_id>
%  <concept_desc>Computer systems organization~Robotics</concept_desc>
%  <concept_significance>100</concept_significance>
% </concept>
% <concept>
%  <concept_id>10003033.10003083.10003095</concept_id>
%  <concept_desc>Networks~Network reliability</concept_desc>
%  <concept_significance>100</concept_significance>
% </concept>
%</ccs2012>
%\end{CCSXML}

%\ccsdesc[500]{Computer systems organization~Embedded systems}
%\ccsdesc[300]{Computer systems organization~Redundancy}
%\ccsdesc{Computer systems organization~Robotics}
%\ccsdesc[100]{Networks~Network reliability}

%%
%% Keywords. The author(s) should pick words that accurately describe
%% the work being presented. Separate the keywords with commas.
\keywords{VRF, ring signature, zero-knowledge, anonymous credentials}
%% A "teaser" image appears between the author and affiliation
%% information and the body of the document, and typically spans the
%% page.
%\begin{teaserfigure}
%  \includegraphics[width=\textwidth]{sampleteaser}
%  \caption{Seattle Mariners at Spring Training, 2010.}
%  \Description{Enjoying the baseball game from the third-base
%  seats. Ichiro Suzuki preparing to bat.}
%  \label{fig:teaser}
%\end{teaserfigure}

%\received{20 February 2007}
%\received[revised]{12 March 2009}
%\received[accepted]{5 June 2009}

%%
%% This command processes the author and affiliation and title
%% information and builds the first part of the formatted document.
\maketitle

\section{Introduction}

\def\qaudbreak{\eprint{\quad}{\\}}


We introduce an anonymous credential flavor called
ring verifiable random functions (ring VRFs),
in essence ring signatures that anonymize signers but
also prove evaluation of the signers' PRFs.
Ring VRFs provide a better foundation for anonymous credentials
across a range of concerns, including formalization, optimizations,
the nuances of use-cases, and miss-use resistance.

Along with some formalizations, we address three questions within
the unfolding ring VRF story:
What are the cheapest SNARK proofs?  Ones users reuse without reproving.
How can identity be safe for general use?  By revealing nothing except users' uniqueness.
How can ration card issuance be transparent?  By asking users trust a public list, not certificates.

We introduce new privitive, a ring VRF, and prove its security.
To efficiently implement our primitive, we present a novel techique,
the zero-knowldege continuation and prove its security. We discuss applications to identity, rationing and other use cases.

% After briefly introducing ring VRFs, we discuss these three questions,
% which we later elaborate upon in
%  \S\ref{sec:rvrf_cont}, \S\ref{sec:app_identity}, and \S\ref{sec:app_rate_limits} 

% \smallskip 
\paragraph{Ring VRFs:}

A ring signature \cite{ring_accountable,ring_efficient,ring_linkable,ring_noRO,ring_sublinear} proves only that its  signer lies in a ring
of public keys, without revealing which signer  signed the message.
A {\it verifiable random function} (VRF) is a signature that proves
correct evaluation of a PRF defined by the signer's key.
% but nuances exist .
A {\it ring verifiable random function} (ring VRF) is a ring signature, in
that it anonymizes its  signer within a ring,
but also proves correct evaluation of a pseudo-random function (PRF)
defined by the actual signer's key. % (see \S\ref{sec:rvrf_games}).
%
Ring VRF outputs then provide linking proofs between different signatures
if and only if  the signatures have identical inputs, as well as pseudo-randomness.
As the pseudo-random output is uniquely determined by the signed message
and signer's key, we can therefore link signatures by the
same signer if and only if they sign identical messages.
In effect, ring VRFs restrict anonymity similarly to but less than
linkable ring signatures \cite{ring_linkable,ring_linkablee}  do, which makes them multi-use and contextual.

We define the security of a ring VRF scheme in the universally composable (UC)
\cite{canetti1,canetti2} model. Then, we construct our UC-secure ring VRF scheme.
We slightly modify it by preserving its security to build extremely efficient 
ring VRFs by amortizing a zero-knowledge continuation that unlinkably
proves ring membership of a secret key, and  cheaply proving
individual VRF evaluations.

% We discuss the applications to identity in \S\ref{sec:app_identity} and
% to rationing in \S\ref{sec:app_rate_limits}.
% As a highly multi-use primitive, ring VRFs also present a multi-use

% First
% \smallskip 
\paragraph{Zero-knowledge continuations:}

Rerandomizable zkSNARKs like Groth16 \cite{Groth16} admit a
transformation of a valid proof into another valid but unlinkable
proof of the same statement.  However, in practice, rerandomization
never gets deployed because the public inputs without further processing actually 
link different usages, thus breaking privacy.
We formally define zero-knowledge continuations in a way that
 it preserves privacy after rerandomization and then  we demonstrate in \S\ref{sec:rvrf_cont} a simple transformation of
any Groth16 zkSNARK into a {\it zero-knowledge continuation} whose
public inputs involve opaque Pedersen commitments (i.e., hiding commitments), with cheaply
rerandomizable blinding factors and cheaply rerandomizable proofs.
These zero-knowledge continuations then prove validity of the contents
of Pedersen commitments, but can be reused arbitrarily many times,
without linking the usages.
In brief, we adjust the trusted setup of the Groth16 to additionally
produce a blinding factor base for the Groth16 public input, 
along with an absorbing base that cancels out this blinding factor in the
Groth16 verification.
As our public inputs involve opaque Pedersen commitments,
they now require proofs-of-knowledge resentment of to \cite{LegoSNARK}. 
% In essence, this specializes the Groth16 
As recursive SNARKs remain slow,
we expect zero-knowledge continuations via rerandomisations become
 a very viable efficient alternative for zkSNARKs.

% Second
% \smallskip

%Now, we describe briefly the possible applications for ring VRF.
\paragraph{Identity uses:}

An identity system can be based upon ring VRFs in an natural way:
After verifying an identity requesting domain name in TLS,
our user agent signs into the session by returning a ring VRF
signature whose input is the requesting domain name, so their
ring VRF output becomes their unique identity at that domain.
At this point, the requesting domain knows each user represents
distinct ring members, which prevents Sybil behavior, and permits
banning specific users.
At the same time, users' activities remain unlinkable across different
domains.
In essence, ring VRF based credentials, if correctly deployed, only
prevent users being Sybil, but leak nothing more about users.  
%We argue
%this yields diverse legally and ethically straightforward identity usages.

As a problematic contrast, attribute based credential schemes like
IRMA (``I Reveal My Attributes'') credentials \cite{IRMAcredentials}
are being marketed as an online privacy solution, but cannot prevent
users being Sybil unless they first reveal numerous attributes.
Attribute based credentials therefore provide little or no privacy
when used to prevent abuse.
Abuse and Sybil prevention are not merely the most common use cases for
anonymous credentials, but in fact they define the general use cases for
anonymous credentials.
IRMA might improve privacy when used as special purpose  credential 
in narrower situations of course, but overall attribute based credentials
should {\it never} be considered fit for abuse and sybil prevention.
Aside from this, existing offline processes often better protect users'
privacy and human rights than adopting online processes like IRMA.
In particular, there are many proposals by the W3C for attribute based
credential usage in \cite{w3c_vc_use_cases}, but broadly speaking they
all bring matching harmful uses. % https://www.w3.org/TR/vc-use-cases/
As an example, the W3C wants users to be able to easily prove their
employment status, ostensibly so users could open bank accounts purely
online.  Yet, job application sites could similarly demand these same
proofs of current employment, a discriminatory practice.
Average users apply for jobs far more often than they open bank accounts,
so credentials that prove current employment do more harm than good.
An IRMA deployment should prevent this abusive practice by making
verifiers prove some legal authorization to request employment status,
or other attributes, before user agents prove their attributes.
Indeed IRMA deployments need to regulate IRMA verifiers, certainly by
government privacy laws,
but this limits their flexibility and becomes hard internationally.
Ring VRFs avoid these abuse risks by being unlinkable, and thus
yield anonymous credentials which safely avoid legal restrictions.
%{\it Any ethical general purpose identity system should be based
%	upon ring VRFs, not attribute based credentials like IRMA.}
We credit proof-of-personhood parties by Bryan Ford, et al. \cite{pop2008,pop2017}
% https://bford.info/pub/dec/pop-abs/  https://bford.info/pub/net/sybil-abs/
with first espousing the idea that anonymous credentials should produce
contextual unique identifiers, without leaking other attributes.

As a rule, there exist simple VRF variants for all anonymous credentials,
including IRMA \cite{IRMAcredentials} or group signatures \cite{group_sig_survey}.
We focus exclusively upon ring VRFs for brevity, and because
ring VRF's contextual linkability covers the most important use cases.
% and our optimizations make ring VRFs extremely efficient.

% Third
% \smallskip
\paragraph{Rationing uses:}

A rate limiting or rationing system provides users with a stream
of single-use anonymous tokens that each enable consuming some resource.
They are usually constructed
from blind signatures ala \cite{chaum83}, or else
from OPRFs like PrivacyPass \cite{PrivacyPass},
both of which have an $O(n)$ issuance phase.

Ring VRFs yield rate limiting or rationing systems with no issuance phase:
We first place into the ring the public keys for all users permitted to
consume resources, perhaps all legal residents within some country.  
We define single-use tokens to be ring VRF signatures whose VRF input
consists of a resource name, an approximate date, and a bounded counter.
Now merchants reports each anonymous token back to some authority who
enforces rate limits by rejecting duplicate ring VRF outputs
(see \eprint{\S\ref{sec:app_rate_limits}}{\ref{sec:app_short}}).
In other words, our rate limiting authority treats outputs like the
nullifiers in anonymous payment schemes.
Yet, ring VRF nullifiers need only temporarily storage, as eventually one
expires the date in the VRF input.  Asymptotically we thus only need
$O(\mathtt{users})$ storage vs the $O(\mathtt{history})$ storage
required by anonymous payment schemes like ZCash and blind signed tokens.

We further benefit from the ring credential format too,
as opposed to certificate based designs like group signatures:
We expect some fraud whenever deploying purely certificate
based systems, as witnessed by the litany of fraudulent TLS and covid
certificates.  Ring VRFs help mitigate fraudulent certificate concerns
because the ring is a database and can be audited.

Governments may have little choice but to institute
rationing in response to shortages caused by climate change or peak oil.  Ring VRFs help avoid ration card fraud
while also protecting essential privacy. We discuss some of these 
applications in \ref{sec:app_short}.
%
%As an important caveat, ring VRFs need heavier verifiers than single-use
%tokens based on OPRFs \cite{PrivacyPass} or blind signatures, but
%those credentials' heavy issuance phase represents a major adoption hurdle.
%A ring VRF systems issue fresh tokens almost non-interactively merely by
%adjusting allowed VRF input on resource names, dates, and bounds.
%This reduces complexity, simplifies scaling, and increases flexibility.

%In particular, if governments issue ration cards based upon ring VRFs
%then these credentials could safely support other use cases, like
%free tiers in online services or games, and advertiser promotions,
%as well as identity applications like prevention of spam and online abuse. We discuss some of these 
%applications in \S\ref{subsec:app_ration_carts}.

\begin{comment}
	In this, we need authenticated domain separation of products or identity
	consumers in queries to users' ring VRF credentials.  We briefly discuss
	some sensible patterns in \S\ref{subsec:app_ration_carts} below, but
	overall authenticated domain separation resemble TLS certificates except
	simpler in that roots of trust can self authenticate if root keys act as
	domain separators.
\end{comment}


%TODO Our contributions should be listed here: https://crypto.iacr.org/2023/papersubmission.php says that 'The introduction should summarize the contributions of the paper at the level understandable for a non-expert reader.'




\endinput




As a field, anonymous credentials come in myriad flavors,
many of which exist to limits the anonymity provided, ala
attribute based credentials and group signatures. % \cite{group_sig_survey}.
% aka anonymized signatures
%
Ring VRFs by weakening anonymity only contextually provide a safer,
more private, more flexible, more powerful, and more ethical
choice for all everyday anonymous credential use cases.  % needs:  ???



% 
\section{Identity}

% “We can judge our progress by the courage of our questions and the depth of our answers, our willingness to embrace what is true rather than what feels good.” 
% - Carl Sagan

% https://twitter.com/IdentityZack/status/1480631954689216516

% bryan ford https://twitter.com/brynosaurus/status/1460094634567344133

% answer https://twitter.com/valkenburgh/status/1442894421289103361
% https://twitter.com/harryhalpin/status/1443053685219725315
% https://twitter.com/OR13b/status/1442964741022830594
% https://twitter.com/jeffburdges/status/1443539630033362948
% https://twitter.com/Steve_Lockstep/status/1448653579330342916

% https://github.com/dckc/awesome-ocap/issues/17

% https://twitter.com/smdiehl/status/1459825936757493770

% https://twitter.com/edri/status/1483818492646281225

% reply https://twitter3e4tixl4xyajtrzo62zg5vztmjuricljdp2c5kshju4avyoid.onion/matthew_d_green/status/1511437992740761601
%   https://twitter3e4tixl4xyajtrzo62zg5vztmjuricljdp2c5kshju4avyoid.onion/RaulACarrillo/status/1545038906776764416
%   https://twitter3e4tixl4xyajtrzo62zg5vztmjuricljdp2c5kshju4avyoid.onion/rohangrey/status/1511808597742612481
%   https://twitter3e4tixl4xyajtrzo62zg5vztmjuricljdp2c5kshju4avyoid.onion/coloradotravis/status/1545252419671572481
%   https://twitter3e4tixl4xyajtrzo62zg5vztmjuricljdp2c5kshju4avyoid.onion/RaulACarrillo/status/1545038906776764416/retweets/with_comments

% https://twitter3e4tixl4xyajtrzo62zg5vztmjuricljdp2c5kshju4avyoid.onion/PeterSweden7/status/1552729179568865280

% Zeroth law:  A robot may not harm humanity, or, by inaction, allow humanity to come to harm.
% First law:  A robot may not injure a human being or, through inaction, allow a human being to come to harm.


An identity system must not harm humanity or its human users, to do otherwise is clearly unethical.  

Identity systems for human users have three participants, an identity provider, an identity consumer, and the user being identified.  There exist two methods by which ethical identity systems avoid harming users, either 
\begin{itemize}
\item special identity systems enforce that identity consumers owe users some legal duty that prevents miss-using the user's details, or else
\item general identity systems merely constrain user activity, often only rate limiting, but avoid providing identity consumers with any user details.
\end{itemize}
In other words, identity consumers should always first prove to the identity provider that they owe the user a legal duty appropriate to the details being revealed by the identity provider.

\subsection{Legal duties}

In this paper, we discuss only cryptographic protocols for general identity systems that avoid legal entanglements by only proving user uniqueness and not providing user details.  We first in this section briefly discuss wider examples that help motivate this problem by clarifying the legal and ethical complexities that arise when revealing user details.

As an unethical example, our largest advertising companies like Google and Facebook track private users using OAuth \cite{oauth}, with the intent to waste users time with increased advertising engagement, manipulate public opinion, ensnare users into unnecessary purchases, often by harming users' psyche, and accumulate personal data users might otherwise wish kept hidden.

As an only moderately harmful example, websites often prevent abuse by demanding commenters identify themselves by email address, which creates moral hazards and should expose the website operators to legal risks.

As beneficial identity examples, financial institutions act as an identity provider for their own identity consumer logic by issuing login credentials, but then owe their customers some fiduciary duty and strongly discourage using the same login credentials elsewhere.  

As a more nuanced example, an employer identifies employees to a personnel management service by way of an external OAuth service, but the employer has some legal relationship with the personnel management service, the OAuth service, and the employee, so any resulting harms rest upon the employer-employee relationship.  

We think Google Single Sign-on or Facebook Connect cannot play the role of OAuth service even in this employer-employee example, and indeed cannot ever be used ethically, because they aggressively track the employee outside the employer-employee relationship.  At the same time, an employees' Github account might or might not serve this role depending upon the specific employee and how they use Github outside work.  

... passports or medical ...

\subsection{Unlinkable identity}

We now lay aside such identity systems that represent a distinguished purpose tied to onerous three-way legal relationships between the parties.  Instead we turn our attention towards the range of identity systems that avoid providing any user details.  

At present, CAPTCHAs provide a popular defense against automated abuse.  There also exist cryptographic tools that amplify defenses against automated abuse, like blind signatures or verifiable oblivious pseudo-random functions (VOPRFs), as used in Privacy Pass \cite{privacypass}.  These dispense single-use tokens within some limits imposed by other identity sources, rate limits, payments, or CAPTCHAs.  

We think single-use tools like CAPTCHAs, blind signatures, and VOPRFs adequately deter abuse in most use cases.  Yet, there also exist situations where abusers cannot be dissuaded by solving another CAPTCHAs or spending another token, like when abuse takes a personal character, or due to a larger profit motive.  

In such harder cases, we still need an anonymous credential so that identity consumers and providers cannot collude to track users, but identity consumers banning problematic users seemingly demands that users have different stable identities with each distinct identity consumer.  
To our knowledge, this identity formulation originates with proof-of-personhood parties \cite{pop2008,pop2017}.
% https://bford.info/pub/dec/pop-abs/
% https://bford.info/pub/net/sybil-abs/

We expect stable identities arise from multi-use anonymous credentials, like group signatures or ring signatures.  In group signatures, an identity provider holds a group manger secret key, with which they both issues credentials and deanonymize users.  We only want identity consumers to recognize returning users, making the deanonymization operation unacceptable.  

Ring signatures have classically given signers' control over their anonymity set aka ``ring'', which turns out mostly useless in practice.  Instead, realistic ring signatures like Zcash's circuits \cite{zcash_prorocol} have a shared public commitment to their ``ring'', so then users need only an opening for their own public key's presence in the ring. 



% sharing economy 
% business-to-business 



%   We think identity consumers should avoid imposing unnecessary constraints upon users and that rate limiting tools usually suffice.  Yet, there exist identity consumers who depend upon stronger Sybil defenses or an ability to ban problematic users.   


\section{Protocol overview}
\label{sec:overview}

% We define ring VRFs in \S\ref{sec:rvrf_games} and \S\ref{sec:rvrf_uc_fun} below, but
Ring VRFs are firstly ring signatures broadly interpreted, in that they
prove an involved public key lies inside some commitment \comring to
the plausible signer set, known as the ring.
Anyone could compute \comring from this set of public keys.
%
At the same time, ring VRFs prove correct output of a PRF keyed by
the signer's actual secret key, and evaluated on a supplied message \msg,
which then links ring VRF signatures on the same \msg.

\begin{definition}
A {\em ring verifiable random function with auxiliary data} (rVRF-AD)
consists of several algorithms:
\begin{itemize}
	\item $\rVRF.\KeyGen$ and returns a public key \pk and a secret key \sk, which one typically instantiates via come commitment scheme. 
	%
	% \item $\rVRF.\CheckRing : \ring \mapsto \comring$ takes a set \ctx of public keys and returns a public key set commitment \comring.
	\item $\rVRF.\CommitRing : (\ring,\pk) \mapsto (\comring,\openring)$ takes a set \ring of public keys, and returns a public key commitment \comring.  It optionally takes a public key \pk too, and then returns an opening \openring.
	\item $\rVRF.\OpenRing : (\comring,\openring) \mapsto \pk$ returns a public key \pk, provided \openring correctly opens the ring commitment \comring, or failure $\perp$ otherwise.
	%
	\item $\rVRF.\Eval : (\sk,\msg) \mapsto \Out$  takes a secret key \sk and an input $\msg$, and then returns a VRF output $\Out$.
	\item $\rVRF.\rSign : (\sk,\openring,\msg,\aux) \mapsto \sigma$ takes a secret key \sk, a ring opening \openring, an input \msg, and auxiliary data \aux, and then returns a ring VRF signature $\sigma$.
	\item $\rVRF.\rVerify$ takes $(\comring,\msg,\aux,\sigma)$ for a ring commitment \comring, an input \msg, and auxiliary data \aux, and then returns either an output $\Out$ or else failure $\perp$.
\end{itemize}
\end{definition}

We could instantiate a ring VRF via NIZKs like

$$ \pi_0 = \NIZK \Setst{ \Out, \msg, \comring }{
    \exists \sk, \pk, \openring \textrm{\ s.t.\ } 
    \begin{aligned}
        (\pk,\sk) &\leftarrow \KeyGen  \textrm{ and } \\
        \pk &= \OpenRing(\comring,\openring)  \textrm{ and } \\
        \Out &= \PRF(\sk,\msg)
    \end{aligned}
} $$

As a rule, all ring signatures work via zero-knowledge execution of
some such \OpenRing, which invariably winds up being extremely heavy.
In \S\ref{sec:rvrf_cont}, we resolve this major performance
obstacle by introducing {\em zero-knowledge continuation},
a new zkSNARK variant designed for composition and build from Groth16.
Our zero-knowledge continuation to split $\pi_0$ into first a Groth16
$\piring$ which runs \OpenRing and permits reuse via rerandomization
ala \cite[Theorem 3, Appendix C, pp. 31]{RandomizationGroth16},
and then a separate a PRF evaluation NIZK $\pieval$.

$$ \piring = \NIZK \Setst{ \compk, \comring }{
    \exists \openring \textrm{\ s.t.\ } 
    \pk = \OpenRing(\comring,\openring) \textrm{ and } \cdots
} $$

$$ \pieval = \NIZK \Setst{ \Out, \msg, \compk }{
    \exists \sk \textrm{\ s.t.\ }
    (\pk,\sk) \leftarrow \KeyGen \textrm{ and }
    \Out = \PRF(\sk,\msg) \textrm{ and } \cdots
} $$

Here wur zero-knowledge continuation passes \pk from \piring to \pieval
using a Pedersen commitment \compk to \pk which hides \pk, and
binds \pk as being equal in \piring and \pieval, but permits cheap
 reblinding of \compk without reproving \piring.
As a consequence, our \compk cannot itself be a true Groth16 public input,
or even exist inside the Groth16 inside \piring.

Instead in \S\ref{sec:rvrf_cont}, we build zero-knowledge continuations
by expanding a Groth16 trusted setup with independent blinding factors
for Pedersen commitments, and providing proofs-of-knowledge when used,
so that the zero-knowledge continuation's own ``public inputs'' become
cheaply malleable in exactly the desired way.
We then exploit the zero-knowledge continuation by wiring these
Pedersen commitments into another NIZK.
%
In \S\ref{sec:pederson_vrf},
we introduce an extremely efficient instantiation for $\pieval$, which
also provides the required proof-of-knowledge for \compk.

It's clear our NIZKs $\piring$ and $\pieval$ above look under specified
in how they handle the Pedersen commitment \compk.  Among other issues
they never explain the proof-of-knowledge required for \compk.
As a rule, we found typical NIZK notation cannot distinguish the Groth16
itself from these Pedersen commitments, and thus cannot describe the
wiring of zero-knowledge continuations properly.
We therefore describe only the base Groth16 in NIZK notation, and then
separately explain how the zero-knowledge continuation wires up its
Pedersen commitments.



\endinput 






















\smallskip

We explain this zero-knowledge continuation $\piring$ in detail
in \S\ref{sec:rvrf_cont}.  Yet first in \S\ref{sec:pederson_vrf}
we introduce an extremely efficient instantiation for $\pieval$, which
also provides the required proof-of-knowledge for \compk.



\endinput 

We shall discuss several variations on $\piring$, as alterations impact
deployment dramatically.  Yet we know interesting variations on $\pieval$
as well, including schemes with post-quantum anonymity, but not post-quantum soundness.

\endinput 

  As an example, if one employs hash functions for \CommitKey
and \PRF, and certain zkSNARKs for $\pieval$, then one obtains post-quantum
anonymity, although not post-quantum soundness.



\section{Background}
\label{sec:background}

\def\secparam{\ensuremath{\lambda}\xspace}

\def\ecE{{\mathbb{E}}}
\def\grE{{\mathbf{E}}}
\def\genE{E}
\def\genG{G}
\def\genB{K} %{\genE_{\mathrm{bind}}}

\def\ecJ{{\mathbb{J}}}
\def\grJ{{\mathbf{J}}}
\def\genJ{J}

% As our ring VRF is built by composing them, 
We briefly recall the primitives and security assumptions underlying
both Chaum-Pedersen DLEQ proofs and pairing based zkSNARKs. 


\subsection{Elliptic curves}

We obey mathematical and cryptographic implementation convention by using additive notation for elliptic curve and multipicative notation for eliptic curve scalar multiplications. 

We always implicitly have a paramater generation procedure $\mathtt{Params}$ that takes a security level $\secparam \in \N$ and returns elliptic curve paramaters including some prime numbers and efficient algorithms for computing elliptic curve operations.  As customary, we treat $\secparam$ and the output of $\mathtt{Params}$ as fixed paramaters, which makes sense because humans run $\mathtt{Params}$ manually in practice. 

As implicit outputs of $\mathtt{Params}$, we work with an elliptic curve $\ecE[\F]$ over some base field $\F$ of (prime) characteristic $q_{\grE}$, along with a distinguished subgroup $\grE \le \ecE[\F]$ of prime order $p_{\grE} \approx 2^{2\secparam}$.  As $\grE$ distinguishes $\ecE[\F]$, we let $h_{\grE}$ denote the cofactor of $\grE$ in $\ecE[\F]$, meaning $\ecE[\F]$ has $h_{\grE} p_{\grE}$ points.
% but abbreviate $h = h_{\grE}$, $p = p_{\grE}$, and $q = q_{\grE}$ when $\grE$ is clear from context.
We write $\grE$ without subscript, and abbreviate $h = h_{\grE}$, $p = p_{\grE}$, and $q = q_{\grE}$, when $\ecE$ is either our uinque pairing friendly curve or else the only curve in view.

We let $H_p : \{0,1\}^* \to \F_p$ or $H_q : \{0,1\}^* \to \F_q$ denote random oracles (RO) with a range $\F_p$ or $\F_q$.  We let $H_\ecE : \{0,1\}^* \to \ecE$ or $H_\grE : \{0,1\}^* \to \grE$ denote a hash-to-curve for $\ecE$ or hash-to-group for $\grE$, which we model as a random oracles too.  We note $H_\grE(x) = h H_\ecE(x)$ always works, although more efficent exist.

\smallskip

Almost all SNARKs like \cite{Groth16} or \cite{plonk} employ a pairing friendly elliptic curve $\ecE$ over a field $\F_q$ of characteristic $q \approx 2^{2\secparam}$, which comes equipped with a type III pairing on subgroups of prime order $p \approx 2^{2\secparam}$:  We let $q_1,q_2,q_T$ denote small powers of $q$, and let $\grE_1 \le \ecE[\F_{q_1}]$ and $\grE_2 \le \ecE[\F_{q_2}]$ and $\grE_T \le \F_{q_T}^\times$ denote subgroups all of prime order $p$.  We also let $e : \grE_1 \times \grE_2 \to \grE_T$ denote a type III pairing, meaning a computable bilinear map without known efficiently computable maps between $\grE_1$ and $\grE_2$.  Also $q_i = q_{\grE_i}$ for $i=1,2$ in our above notation.  

Any pairing friendly elliptic curve $\ecE$ provides solutions to the decisional Diffie-Hellman problem (DDH).  We do however assume the computational Diffie-Hellman problem (CDH) remains hard in $\ecE$.  We remark that $H_\grE$ being a random oracle plus CDH hardness prevents computable relationships between $H_\grE$ outputs.

% TODO: Pairing assumptions required by Groth16

\smallskip

% We shall require ZCash Sapling style ``Jubjub'' Edwards curves, whose base field characteristic divides of the order of a pairing friendly elliptic curve, thereby making Jubjub base field arithmetic SNARK friendly, and hence Jubjub elliptic curve operations as well \cite{}.

We let $\ecJ$ denote a ZCash Sapling style ``JubJub'' Edwards curve associated to the pairing friendly elliptic curve $\ecE$, meaning $\ecJ$ has base field $\F_p$ whose characteristic $q_{\grJ} = p$ matches the group order $p$ of $\grE_1 \cong \grE_2 \cong \grE_T$.  As in ZCash Sapling, we now prove statements about elliptic curve operations inside $\ecJ$ by proving base field arithmetic in $\F_p$, which our $q_{\grJ} = p$ makes relatively inexpensive inside SNARKs on $\ecE$.

As above, $\grJ \le \ecJ[\F_p]$ has large prime order $p_{\grJ}$ and a small cofactor $h_{\grJ}$.  We always support $4 p_{\grJ} < p$ because if $\ecJ$ is an Edwards curve then $h_{\grJ} \ge 4$ which imposes this by the Hasse bound.

\smallskip

We ask that deserialization prove that putative elements of $\grE$ lie in
$\ecE[\F]$ by verifying curve equations, perhaps including twist checks.

Anytime $\ecE$ represents a pairing friendly curve then we ask that
deserialization prove elements of $\grE_1$, $\grE_2$, and $\grE_T$
lie inside the correct subgroup of order $p$,
 which typically requires checking whether $|\grE| X = 1$ or similar.
As our SNARKs works with points in $\ecJ$ directly, we conversely
prefer writing $\grJ$ equations in $\ecJ[\F_p]$ and explicitly describe
where one clears the cofactor $h_{\grJ}$.  We handled $\grE$ withr
$\ecE$ not necessarily pairing friendly similarly to $\ecJ$.
We scrape by with only CDH hardness for $\grJ$ for convenience,
although DDH winds up hard in $\grJ$.


\subsection{Zero-knowledge proofs}

\newcommand\rel{\ensuremath{\mathcal{R}}\xspace}
\newcommand\lang{\ensuremath{\mathcal{L}}\xspace}

% refs.
% https://people.csail.mit.edu/silvio/Selected%20Scientific%20Papers/Zero%20Knowledge/Noninteractive_Zero-Knowkedge.pdf
%   Alright but kinda poorly phrases
% https://inst.eecs.berkeley.edu/~cs276/fa20/notes/Multiple%20NIZK%20from%20general%20assumptions.pdf
%   Addresses the ZK definitions better
% 

We let \rel denote a polynomial time decidable relation, so the language
 $\lang = \{x \mid \exists \omega (\omega,x) \in \rel \}$ lies in NP.
All non-interactive zero-knowledge proof systems have some setup procedure $\mathtt{Setup}$ that takes our parameters generated by $\mathtt{Params}$ and some ``circuit'' description of \rel, and produces a structured reference string (SRS).

A non-interactive proof system for $\rel$ consists of \Prove and \Verify PPT algorithms
\begin{itemize}
%\item $\NIZK.\setup(\rel) \rightarrow (crs, \tau)$ ---- Given the relation $ \rel $ it outputs a common reference string $ crs $ and a trapdoor $ \tau $ for $ \rel $.
\item $\NIZK_\rel.\Prove(\omega, x) \mapsto \pi$ creates a proof $\pi$ for a witness and statement pair $(\omega,x) \in \rel$.
\item $\NIZK_\rel.\Verify(x, \pi)$ returns either true of false, depending upon whether $\pi$  proves $x$.
\end{itemize}	
which satisfy the following completeness, zero-knowledge, and knowledge soundness definitions.

\begin{definition}\label{def:nizk_completeness}
We say $\NIZK_\rel$ is {\em complete} if $\Verify(x, \Prove(\omega,x)$ succeeds for all $(\omega,x) \in \rel$.  % with high probability
\end{definition}

\def\advV{\ensuremath{V^*}\xspace} % Why not use \adv here?

\begin{definition}\label{def:nizk_zero_knowledge}
We say $\NIZK_\rel$ is {\em zero-knowledge} if
there exists a PPT simulator $\NIZK_\rel.\Simulate(x) \mapsto \pi$
that outputs proofs for statement $x \in L$ alone, which are
computationally indistinguishable from legitimate proofs by \Prove,
i.e.\ any non-uniform PPT adversary \advV cannot distinguish pairs $(x,\pi)$
generated by \Simulate or by \Prove except with odds negligible in \secparam
(see \cite[Def. 9, \S A, pap. 29]{RandomizationGroth16}). %  or ...
\end{definition}

\def\advP{\ensuremath{P^*}\xspace} % Why not use \adv here?

\begin{definition}\label{def:nizk_knowledge_sound}
We say $\NIZK_\rel$ is {\em (white-box) knowledge sound} if
for any non-uniform PPT adversary \adv who outputs a statement $x \in \lang$ and proof $\pi$
there exists a PPT extractor algorithm $\Extract$ that white-box observes $\advP$ and
if $\Verify(x,\pi)$ holds then $\Extract$ returns an $\omega$ for which $(\omega,x) \in \rel$
(see \cite[Def. 7, \S A, pap. 29]{RandomizationGroth16}).
\end{definition}

Our zero-knowledge continuations in \S\ref{sec:rvrf_cont} demand
rerandomizing existing zkSNARKs, which only Groth16 supports \cite{Groth16}.
We therefore introduce some details of Groth16 \cite{Groth16} there,
when we tamper with Groth16's SRS and $\mathtt{Setup}$ to create zero-knowledge continuations. 
% TODO: Do we describe Groth16 \cite{Groth16} enough?

% In this, we exploit several arguments given by \cite{RandomizationGroth16},
% but for now we recall that \cite{RandomizationGroth16} proves that Groth16
% satisfies: % white-box weak simulation-extractablity .
%
% \begin{definition}\label{def:nizk_weak_simulation_extractable}
% We say $\NIZK_\rel$ is {\em white-box weak simulation-extractable} if
% for any non-uniform PPT adversary \advP with oracle access to \Simulate
% who outputs a statement $x \in \lang$ and proof $\pi$,
% there exists a PPT extractor algorithm $\Extract$ that white-box observes $\advP$ and
% if \advP never queried $x$ and $\Verify(x,\pi)$ holds
% then $\Extract$ returns an $\omega$ for which $(\omega,x) \in \rel$
% (see \cite[Def. 7, \S 2.3, pap. 29]{RandomizationGroth16}).
% \end{definition}

TODO: AGM and Groth16 here?


\subsection{Universal Composable (UC) Model}

TODO: Chat on why UC is here?

A protocol $ \phi $ in the UC model is an execution between distributed interactive Turing machines (ITM). Each ITM has a storage to collect the incoming messages from other ITMs, adversary \adv or the environment $ \env $. $ \env $ is an entity to represent the external world outside of the protocol execution.  The environment $ \env $ initiates ITM instances (ITIs) and the adversary \adv with arbitrary inputs and then terminates them to collect the outputs.
% An ITM that is initiated by $ \env $ is called ITM instance (ITI). 
We identify an ITI with its session identity $ \sid $ and its ITM's identifier $ \pid $. In this paper, when we call an entity as a party in the UC model we mean an ITI with the identifier $ (\sid, \pid) $.

We define the ideal world where there exists an ideal functionality $ \mathcal{F} $ and the real world where a protocol $ \phi $ is run as follows:

\paragraph{Real world:} $ \env $ initiates ITMs and \adv to run the protocol instance with some input $ z \in \{0,1\}^* $  and a security parameter $ \secparam $. After $ \env $ terminates the protocol instance, we denote the output of the real world by the random variable $ \mathsf{EXEC}(\secparam, z)_{\phi, \adv, \env} \in \{0,1\} $. Let $ \mathsf{EXEC}_{\phi, \adv, \env} $ denote the ensemble $ \{\mathsf{EXEC}(\secparam, z)_{\phi, \adv, \env} \}_{z \in \{0,1\}^*} $.

\paragraph{Ideal world:} $ \env $ initiates ITMs and a simulator $ \sim $ to contact with the ideal functionality $ \mathcal{F} $ with some input $ z \in \{0,1\}^* $  and a security parameter $ \secparam $. $ \mathcal{F} $ is trusted meaning that it cannot be corrupted.
$ \sim $ forwards all messages forwarded by $ \env $ to $ \mathcal{F} $. The output of execution with $ \mathcal{F} $ is denoted by a random variable $ \mathsf{EXEC}(\secparam, z)_{\mathcal{F},\sim, \env} \in \{0,1\}$.  Let $ \mathsf{EXEC}_{\mathcal{F},\sim, \env} $ denote the ensemble $ \{\mathsf{EXEC}(\secparam, z)_{\mathcal{F}, \sim, \env} \}_{z \in \{0,1\}^*} $.

TODO: \secparam should likely be implicit, especially since it appears in both worlds.

\begin{definition}[UC-Security of $ \phi $] \label{def:uc}
Given a real world protocol $ \phi $ and an ideal functionality $ \mathcal{F} $ for the protocol $ \phi $, we call that $ \phi $ is UC-secure if $ \phi $ UC-realizes $ \mathcal{F} $ if for all PPT adversaries \adv, there exists a simulator $ \sim  $ such that for any environment $ \env $,
 $\mathsf{EXEC}_{\phi, \adv, \env}$ indistinguishable from $\mathsf{EXEC}_{\mathcal{F},\sim, \env}$
\end{definition}

TODO: if ... if makes no sense.  These definitions need much clearer explanation, or more likely citations to places with clear explanations. 

\begin{definition}[UC-Security of $ \phi $ in the hybrid world]
Given a real world protocol $ \phi $ which runs some (polynomially many) functionalities $ \{\mathcal{F}_1, \mathcal{F}_2, \ldots, \mathcal{F}_k\} $ in the ideal world and an ideal functionality $ \mathcal{F} $ for the protocol $ \phi $, we call that $ \phi $ is UC-secure in the hybrid model $ \{\mathcal{F}_1, \mathcal{F}_2, \ldots, \mathcal{F}_k\} $ if $ \phi $ UC-realizes $ \mathcal{F} $ if for all PPT adversaries \adv, there exists a simulator $ \sim  $ such that for any environment $ \env $,
 $\mathsf{EXEC}_{\phi, \adv, \env}$ is indistinguishable from $\mathsf{EXEC}_{\mathcal{F},\sim, \env}$.
\end{definition}

% REMARKS:  Removed excessive notation $\approx$.














\endinput



BROKEN BOLOW THIS




We fix $J \in \ecJ$ as a generator for public keys.  Any $\KeyGen$ algorithm randomly samples a secret keys $\sk \in \F_q$ and then computes its associate public keys $\pk = \sk J$.  We shall not discuss infrastructure that authorizes public keys.  Yet although our results do not require proof-of-knowledge on $\pk$ per se, we still strongly recommend that back certifications accompany any certificates that authorize $\pk$.

\smallskip





%
\section{VRF-AD security}
\label{sec:games}

We say a VRF-AD-KC denoted \VRF is {\em secure} if it satisfies
 correctness, uniqueness, and pseudo-randomness as defined below,
 as well as being existentially unforgeable as a signature on $(\msg,\aux)$.
%
We caution that VRF security remain subtle, in part due to
signer and forger each being adversarial in some security properties.
%
% At a high level however VRF security assumptions boil down to translating the PRF definition into the public key setting.
% TODO: What of the above two lines?  Merge?

% We follow \cite{agg_dkg} by distinguishing an algorithm $\VRF.\Eval$,
%  instead of defining it by the equality in correctness,
% which simplifies requiring that verifying honest signatures gives a well-defined function.
% $\VRF.\Eval$ always has more optimized instantiations anyways.

We demand unforgability on $(\msg,\aux)$ because alone
the usual VRF conditions only yield unforgeability for \msg.

\begin{definition}\label{def:vrf_sign_oracle}
We let \ora{Sign} denote a CMA oracle, which creates and stores
a key pair $(\pk,\sk) \leftarrow \KeyGen$, returning \pk, and
thereafter answers oracle calls $\ora{Sign}(\msg,\aux)$ by 
logging $(\msg,\aux)$ and returning $\Sign(\sk,\msg,\aux)$.
\end{definition}

\begin{definition}
We say a VRF-AD satisfies {\em existential unforgeability (EUF-CMA-KC)} if
any PPT adversary \adv has only a negligible advantage in $\secparam$
in the usual chosen-message game adapted to key commitments:
\begin{itemize}
  \item \adv receives $\pk$ from \ora{Sign}, % of Definition \ref{def:vrf_sign_oracle}
  repeatedly queries \ora{Sign},
  and finally produces $\pk,\msg,\aux,\sigma$.
  \item \adv wins if $\Verify(\pk,\msg,\aux,\sigma)$ succeeds, and
  \adv never queried $\ora{Sign}(\msg,\aux)$.
\end{itemize}
\end{definition}

% TODO: Any chat here?

\begin{definition}
We say a VRF-AD satisfies {\em VRF correctness} if
 $\Out = \Verify(\pk,\msg,\aux,\Sign(\sk,\msg,\aux))$ succeeds
whenever $(\pk,\sk) \leftarrow \KeyGen$, and
$\Eval : (\sk,\msg) \mapsto \out$ is a well-defined function.
\end{definition}
% TODO: Is the second condition supurfluous?

We recast the uniqueness as VRFs being well-defined functions of
their public key too, at least up to cryptographic assumptions,
but our definition is clearly equivalent to uniqueness given in
\cite[Def. 2 \S3.2, pp. 4]{vrf_micali} or \cite[Def. 3, pp. 8]{agg_dgk}.

\begin{definition}
We say a VRF-AD satisfies {\em uniqueness} if
if anytime some PPT adversary \adv produces $\msg$, $\pk$, and $\aux_i$, $\sigma_i$ for $i=1,2$, then
$\Verify(\pk,\msg,\aux_1,\sigma_1) = \Verify(\pk,\msg,\aux_2,\sigma_2)$
unless either $\Verify$ returns failure, except with odds negligible in $\secparam$.
\end{definition}

\begin{definition}
We say a VRF-AD satisfies {\em strong uniqueness} if
there exists a (not efficiently computable) function
 $F : (\msg,\pk) \mapsto \Out$ such that
anytime some PPT adversary \adv produces $\msg$, $\pk$, $\aux$, and $\sigma$
then $\Verify(\pk,\msg,\aux,\sigma) \in \{ F(\msg,\pk), \perp \}$
except with odds negligible in $\secparam$.
\end{definition}
% TODO: Keep?

We say VRFs are public key analogs of PRFs, but actually this PRF analogy
fails in the ``residual pseudo-randomness'' definitions by
Micali, et al. \cite[Def. VRF (3) \S3.2, pp. 4]{vrf_micali},
 which employs \ora{Sign} in EUF-CMA-like games,
 but says nothing for adversarially generated keys.

\begin{definition}
We say a VRF-AD-KC satisfies {\em public keyed} or {\em residual pseudo-randomness} if 
any PPT adversary \adv has only a negligible advantage in $\secparam$
in this chosen-message game:
\begin{itemize}
	\item[]
	\adv receives $\pk$ from \ora{Sign} of Definition \ref{def:vrf_sign_oracle},
	repeatedly queries \ora{Sign}, and produces $\msg,\aux$.
	If \adv never queried $\ora{Sign}(\msg,\cdot)$ then
	\adv wins by distinguishing $\msg \mapsto \Eval(\sk,\msg)$ from a random.
\end{itemize}
\end{definition}

In \cite{praos}, there exists a UC functionality which captures the
desired PRF analogy, but brings unnecessary restrictions.

We know a function family $\{ F_\msg : \pk \mapsto F(\msg,\pk) \}$ exists
by strong uniqueness, although not efficiently computable, so intuitively
our VRF-AD is {\em pseudo-random} if an adversary cannot distinguish
$F_\msg$ from a random function.
% TODO: Keep?

\bigskip

MISTAKES BELOW THIS POINT

\bigskip 

As a formalization, we provide a black-box game-based definition which
treats \msg as the PRF key, and handles adversarially supplied keys as
PRF inputs by not necessarily terminating.

\begin{definition}
We say a VRF-AD-KC satisfies {\em message keyed pseudo-randomness} if 
any PPT adversary \adv for whom the following black-box game always
terminates has only a negligible advantage in $\secparam$ of winning.
\begin{itemize}
	\item[]
	Sample a random \msg, a random function $\rho$ with the same range as \Eval, and a bit $b$.
	\adv queries \ora{Verify} by providing both a public key \pk and
	a PPT (black-box) secret key algorithm $f_\sk : () \mapsto (\aux,\sigma)$
	such that repeatedly trying $\Out \leftarrow \Verify(\pk,\msg,f_\sk(\msg))$
	eventually succeeds.
	\ora{Verify} always returns \Out and $\rho(\pk)$ but with their order depending upon $b$.
	\adv wins by guessing $b$, aka by distinguish \Verify from $\rho$.
\end{itemize}
\end{definition}

There are also verifiable unpredictable function (VUF), which replace
pseudo-randomness by the weaker {\em unpredictability} definition from
\cite[Def. VUF (3) \S3.2, pp. 5]{vrf_micali} or \cite[Def. 4, pp. 8]{agg_dgk}.
Interestingly VUFs often suffice threshold security assumptions \cite{agg_dkg}.

\begin{definition}
We say a VRF-AD-KC satisfies {\em residual unpredictability} if 
any PPT adversary \adv has only a negligible advantage in $\secparam$
in this chosen-message game:
\begin{itemize}
	\item[]
	\adv receives $\pk$ from \ora{Sign} of Definition \ref{def:vrf_sign_oracle},
    repeatedly queries \ora{Sign}, and produces $\msg,\aux$.
    If \adv never queried $\ora{Sign}(\msg,\cdot)$ then
    \adv wins by guessing $\Eval(\sk,\msg)$ for an unqueried \msg.
\end{itemize}
\end{definition}

Also, if $H'(\cdot,k)$ is a PRF then \cite[Proposition 1]{vrf_micali}
shows computing $\Out = H'(\Verify(\cdots), \msg)$ transforms
 residual unpredictability into a residual pseudo-randomness.
As $H'$ is cheap, we conclude implementers should prefer VRFs over more subtle VUFs.

\begin{definition}
We say a VRF-AD-KC satisfies {\em message keyed unpredictability} if 
any PPT adversary \adv for whom the following black-box game always
terminates has only a negligible advantage in $\secparam$ of winning.
\begin{itemize}
	\item[]
	Sample a random \msg.
	\adv queries \ora{Verify} by providing both a public key \pk and
	a PPT (black-box) secret key algorithm $f_\sk : () \mapsto (\aux,\sigma)$ such that
	repeatedly trying $\Out \leftarrow \Verify(\pk,\msg,f_\sk(\msg))$ eventually succeeds.
	\ora{Verify} always returns \Out.
	\adv wins by correctly guessing $\Out = F(\msg,\pk)$ for an unqueried \pk. 
\end{itemize}
\end{definition}

TODO: Justify?

TODO: Relationships?  


\subsection{Confusion}
% \smallskip

Although \cite[\S3.2 $\fvrf$]{praos} handles pseudo-randomness better than \cite{vrf_micali},
they formalize VRFs with detached outputs via the two algorithms:
% \begin{itemize}
% \item
$\VRF.\primalgo{EvalProve}(\sk,\msg,\aux) \mapsto (\Out,\sigma)$, in which $\sigma = \VRF.\Sign(\sk,\msg,\aux)$ and $\Out = \VRF.\Eval(\sk,\msg)$, and
% \item
$\VRF.\primalgo{VerifyProof}(\pk,\msg,\aux,\Out,\sigma)$ which returns true only if $\VRF.\Verify(\pk,\msg,\aux,\sigma)$ returns $\Out$.
% \end{itemize}
We strongly prefer the \Sign and \Verify formulation from \cite{agg_dkg}
primarily because the \primalgo{EvalProve}, and \primalgo{VerifyProof}
formulation causes implementation and deployment mistakes:

EC VRF signatures have the form $\sigma = (\PreOut,\pi)$ in which some
inner proof $\pi$ proves correctness of some associated VUF output $\PreOut$. % aka ``pre-output''.  % ``pre-pseudo-random''
It follows $\VRF.\Eval$ never corresponds to $\PreOut$, but if one describes
protocols with an \primalgo{EvalProve} formulation then exposing $\PreOut$
invariably confuses developers into miss-using $\PreOut$ as the output.
% In other words, actual code never corresponds to an \primalgo{EvalProve} and \primalgo{VerifyProof} formulation.

The ``pre-output'' $\PreOut$ preserves algebraic relationships between
secret keys, so protocols described by the \primalgo{EvalProve} formulation
have implementations with broken pseudo-randomness, and perhaps
 related key vulnerabilities and mishandled cofactors.
% We need $\PreOut$ to be exposed by implementations so ...
We avoided the VUF formalism taken by \cite{agg_dkg} in part because
 VUFs obfuscate this difficulty to developers.

As a caveat, there exist UC formalisms that appear simpler with
the \primalgo{EvalProve} and \primalgo{VerifyProof} formulation, like in \cite{praos}.
We therefore propose that VRFs and protocols using VRFs should always be
described using the the \Sign and \Verify formulation, which provides
implementers with a sensible description, but then if needed adopt
 \primalgo{EvalProve} and \primalgo{VerifyProof} only inside the UC formulation itself.
We feel imposing this mental translation upon paper authors and reviewers
 beats imposing the reverse upon developers with less cryptographic knowledge.



\endinput 



\smallskip

There exist VUFs like RSA-FDH or BLS signatures that lack auxiliary data
% There even exist bespoke VRFs that relax correctness to some non-trivial
% relation on the space of secret keys and messages,
%  seemingly including some Rabin variants. 
Yet, these all suffer from either large signature sizes (RSA) or
 slow verification (BLS).
%  VRFs like single-layer XMSS, .

Instead, one prefers instantiating VRFs similarly to
 \cite{nsec5} or \cite{VXEd25519} using Chaum-Pedersen DLEQ proofs \cite{CP92} % Or should it be CP93 ??
 because they provide both small signatures and fast verification.
In these, our auxiliary data \aux can be verified for free,
by binding \aux into the challenge hash, like a Schnorr signature.
VRF protocols could often reduce bandwidth and verifier time this way,
 but some like Sassafras depend upon \aux. 





\endinput % no UC VRF discussion here




 





\section{Security Model  of Ring VRF}
In this section, we define a ring VRF scheme in the UC framework, covering both real-world and ideal-world executions.

\begin{definition}[Ring VRF] \label{def:ringVRF}	It  is defined with public parameters $ pp $ generated by \eprint{a setup algorithm}{} $ \rVRF.\Setup(1^\secparam) $ and with the following  PPT algorithms. All algorithms below include $ pp $ as part of their input, although it may not always be explicitly stated.
	\begin{itemize}
		\item $ \rVRF.\KeyGen(pp) \rightarrow (\sk,\pk)$: It generates a secret key and public key pair $ (\sk,\pk) $ given input $ pp $.
		\item $ \rVRF.\Eval(\sk_i, \msg) \rightarrow \Out$: It is a deterministic algorithm that outputs an evaluation value $ \Out \in \setsym{S}_{eval}$ given  $ \sk_i $ and an input $ \msg $. \eprint{Here, }{}$ \setsym{S}_{eval} \in pp$ and is the domain  of  evaluation values.
	\end{itemize}
	The following algorithms need an input $ \ring = \set{\pk_1, \pk_2, \ldots, \pk_n}$\eprint{ that we call ring}{}:
	\begin{itemize}
		\item $ \rVRF.\CommitRing(\ring, \pk_i)  \rightarrow (\comring, \openring)$: It  outputs a commitment of $ \ring $ with the opening $ \openring $ given input  $ \ring $ and $ \pk \in \ring $.
		\item $ \rVRF.\OpenRing(\comring,\openring) \rightarrow \pk $: It  outputs a public key $ \pk  $ given commitment $ \comring $ and an opening $ \openring $ of $\comring$ to $\pk$.
		\item $ \rVRF.\Sign(\sk_i, \comring,\openring, \msg, \aux)\rightarrow \sigma$: It  outputs a  signature  $\sigma $  of  $ \msg, \aux \in \{0,1\}^*$ given $ \sk_i, \openring $  and $ \comring $ 
		\item $ \rVRF.\Verify(\comring,\msg,\aux,\sigma) \rightarrow  (b, \Out)$: It is a deterministic  algorithm that outputs  $ b \in \{0,1\} $ and $ \Out \in\setsym{S}_{eval}\cup \{\perp\} $. $ b =1 $ means $ \sigma $ and $ \Out $ are verified.
	\end{itemize}
	
\end{definition}


We note that $ \rVRF.\CommitRing $ and $  \rVRF.\OpenRing $ are optional algorithms of a ring VRF scheme. If they are not defined, we should let $ \comring = \ring $ and $ \openring  = \pk$. $ \rVRF.\CommitRing $ and $  \rVRF.\OpenRing $ are useful for a succinct verification process in the case of a large ring.

We summary the  security properties for $ \rVRF $  informally as follows: 
\begin{itemize}
	\item \emph{correctness}; when an honest signer with key $ (\sk_i,\pk_i) $ outputs $ \sigma $ by running $ \rVRF.\Sign(\sk_i, \comring,\openring, \msg, \aux) $, $ \rVRF.\Verify(\comring,\msg,\aux,\sigma)  $ must output $ 1,\Out = \rVRF.\Eval(\sk_i, \msg) $ given  $ \rVRF.\OpenRing(\comring,\openring) \rightarrow \pk_i \in \ring$. Indeed, while verifying the ring VRF signature, a verifier verifies that $ \aux $ is signed by one of the keys is in the ring and also verifies that  $ \Out $ is the evaluation value of $ \msg $ generated with the same key. 
	\item \emph{randomness}; $ \Out $ is random and independent from the input and the  key.
	%\item \emph{determinism} meaning that $ \rVRF.\Eval $ is deterministic,
	\item   \emph{anonymity} meaning that the output of $ \rVRF.\Sign $ does not leak any information about the key of its signer except that the key is in the ring.
	\item \emph{unforgeability}; an adversary should not be able to forge a ring VRF signature 
	\item  \emph{uniqueness}; the number of verified evaluation values should not be more than the number of the keys in the ring.
\end{itemize}



%We could define all these security properties formally in the standard model but this may cause composability issues when the ring VRF protocol is composed with other protocols. Considering the applications of ring VRF protocol, we want to achieve stronger security guarantees in different environments. Therefore, we define the security of a ring VRF scheme in the UC model in Section \ref{subsec:uc_model}. 

%One can consider a ring VRF scheme is a combination of a VRF scheme and a ring signature scheme where $ \rVRF.\Eval $ is similar to $ \Eval $ algorithm of a VRF scheme and $ \rVRF.\Sign $ is similar to $ \Sign $ algorithm of a ring signature scheme. The  subtle difference is in $ \rVRF.\Verify $ that works  similar to both $ \Verify $ of ring signature and VRF schemes.  $ \rVRF.\Verify $ does not need the signer's public key to verify a ring VRF signature as in a ring signature scheme but it outputs the signer's evaluation value for every verified signature. 
%If the evaluation value is generated independent from the signer's key, $ \rVRF.\Verify(\ring,\msg,\aux,\sigma) $ does not reveal any identity of the signer except that the signer's key is in the ring.


We remark that the output of $\rVRF.\Eval$ is independent of any specific ring.  Consequently, the verification of two signatures for a given input using different rings results in the same evaluation value.  This property allows a party to disclose their identity as needed.  For instance, suppose $\Out \leftarrow \rVRF.\Eval(\sk_i, \msg)$ is verified via a ring VRF signature $\sigma$ with a ring containing $\pk_i$.  Later, if the corresponding party wishes to affirm that $\Out$ was generated using their key, they simply need to sign the same input with a ring which consists of only their key i.e., $ \ring = \{\pk_i\} $.



%Therefore, if a signer $ \user $ with a public key $ \pk $ signs a message $ \msg $ for $ \ring $ where $ \pk \in \ring $ and obtains $ \sigma_1 $ at some point  and later wants to reveal its identity i.e.. $ \sigma_1 $ is signed by $ \user $ with $ \pk $, $ \user $ should resign $ \msg $ with another ring $ \{\pk\} $ consisting on \emph{only} its public key and obtain another signature $ \sigma_2 $. Since the verification of $ \sigma_1 $ and $ \sigma_2 $ with $ \ring $ and $ \{\pk\} $, respectively, outputs the same evaluation value, the verifier can be convinced that $ \user $ signed the both signatures. Hence, a ring VRF scheme can link the identities of the signers if it is necessary.




\paragraph{The ring VRF in the ideal world:} We introduce a ring VRF functionality $ \fgvrf $ to model execution of a ring VRF protocol in the ideal world. In other words, we define a ring VRF protocol in the case of having a trusted entity $ \fgvrf $. There are many straightforward ways of defining a ring VRF protocol in the ideal world satisfying the desired security properties. However, defining simple and intuitive functionality while being as expressive and realizable in the real world execution is usually at odds \cite{canetti1}. Therefore, we have a lengthy $ \fgvrf $ (See Figure \ref{f:gvrf}) which satisfies the security properties that we expect from a ring VRF scheme and at the same time as faithful to the reality as possible. For the sake of clarity and accessibility, we split each execution part of $ \fgvrf $ while we introduce our functionality. The composition of all parts is in Figure \ref{f:gvrf}. We first describe how $  \fgvrf $ works and then show which security properties it achieves.


$ \fgvrf $ has tables to store the data generated from the requests from honest parties and the adversary $ \simulator $. The table $ \vklist $  keeps the keys of parties. The other table  $ \anonymouskeymap $ stores an anonymous key that corresponds to an input  of a party with a key $ \pk $. We note that the real execution of a ring VRF (Definition \ref{def:ringVRF}) does not have a concept of an anonymous key but $ \fgvrf $ needs this internally to execute the verification of a ring signature. Related to anonymous keys, $ \fgvrf $ also stores  all  malicious anonymous keys in a table $ \anonymouskeylist $. Finally, $ \fgvrf $ stores the evaluations values of all parties in $ \evaluationslist $. In a nutshell,  given $ \pk $
and $ \msg $, $ \fgvrf $  generates an anonymous key $ W $ as explained below and  sets $ \anonymouskeymap[\msg,W]  $ to $ \pk $. Then, it generates an evaluation value $ \Out $ as explained below and sets $ \evaluationslist[\msg,W]  $ to $ \Out $. In short, given honestly generated secret, public key pair $ (\sk,\pk) $ in the real world, the algorithm
$ \rVRF.\Eval(\sk,\msg) $  that outputs evaluation value corresponds to generating an anonymous key $ W $ for $ \pk, \msg $ and obtaining the evaluation value stored in $ \evaluationslist[\msg,W] $ in the ideal world. The necessity and usage of all these tables and anonymous keys will be more clear while we explain $ \fgvrf $ in detail. $ \fgvrf $ consists of the following execution parts.



\paragraph{Key Generation:}  When an honest party requests  a key, $ \fgvrf $ obtains a key pair $ (\sk, \pk) $ from $ \simulator $. \eprint{$ \fgvrf $ stores them if they have not been recorded. If it is the case, }{}$ \fgvrf $ gives only $ \pk $ to the honest party. $ \fgvrf $ will later use $ \sk $ during signature generation. One can imagine $ \sk $ as a secret key and $ \pk $ as a public key but retrieving $ \sk $ from $ \simulator $ poses no issue in the ideal model. This is due to the fact that each evaluation value is randomly sampled, and a signature generated by an honest party can be considered valid if and only if they request it, as guaranteed by the verification process of $\fgvrf$.



\begin{tcolorbox}[left=2pt,right=2pt]
	\eprint{}{\scriptsize}
	\textbf{[Key Generation.]} upon receiving a message $(\oramsg{keygen}, \sid)$ from  $\user_i$, send $(\oramsg{keygen}, \sid, \user_i)$ to the simulator $\simulator$.
	Upon receiving a message $(\oramsg{verificationkey}, \sid, \sk,\pk)$ from $\simulator$, verify that $\sk $ or $\pk$ has not been recorded before for $ \sid $ in $ \vklist $. If it is the case, store  the value $\sk,\pk$ in the table $\vklist$ under $\user_i$ and return $(\oramsg{verificationkey}, \sid, \pk)$ to $ \user_i$.
\end{tcolorbox}


\paragraph{Honest Ring VRF Signature and Evaluation:} This part of $ \fgvrf $ functions for honest parties who evaluate an input $ \in $ and sign a message $ \aux $ and $ \msg $. An honest party $ \user_i $  provides to $ \fgvrf $ a ring, its own public key $ \pk_i $, $ \aux $ and  $ \msg $ to be  evaluated. Afterwards, $ \fgvrf $ generates the evaluation value of $ \msg $ and $ \pk_i $ and signs $ \msg $ and $ \aux $ for a given $ \ring $ if $ \pk_i \in \ring $. The evaluation for honest parties works as follows: If $ \fgvrf $ did not select any anonymous key for $ \msg $ and $ \pk_i $ before, it samples randomly an anonymous key $ W $ and samples randomly the evaluation value $ \Out $. The ring signature generation works as follows:  $ \fgvrf $ runs a PPT algorithm $ \Gen_{sign}(\ring, \sk,\pk,\aux,\msg) $ where $ (\sk,\pk) \in \vklist $ and obtains a signature $ \sigma $. It records $  [\msg,\aux, W, \ring,\sigma, 1]  $ for verification. Here, $ 1 $ indicates that $ \sigma  $ is a valid ring signature of $ \msg $ and $ \aux $ generated for $ \ring $ with the anonymous key $ W $.

\begin{tcolorbox}[left=2pt,right=2pt]
	\eprint{}{\scriptsize}
	\textbf{[Honest Ring VRF Signature and Evaluation.]} upon receiving a message $(\oramsg{sign}, \sid, \ring, \pk_i,\aux, \msg)$ from $\user_i$, verify that $\pk_i \in \ring$ and that there exists a public key $\pk_i$ associated to $\user_i$ in $ \vklist $. If it is not the case, just ignore the request. 	
	If there exists no $ W' $ such that $ \anonymouskeymap[\msg,W'] =  \pk_i $, let $ W \leftsample \setsym{S}_W $ and let $\Out \leftsample \setsym{S}_{eval}$. Set $ \anonymouskeymap[\msg,W] = \pk_i $ and set $ \evaluationslist[\msg, W] = \Out$.
	In any case (except ignoring), obtain $ W, \Out$ where $ \anonymouskeymap[\msg,W] =\pk_i $, $ \evaluationslist[\msg, W] = \Out$ and  $ (\sk, \pk) $ is in $\vklist $. Then run  $ \Gen_{sign}(\ring,\sk,\pk,\aux,\msg) \rightarrow \sigma $.
	%Verify that $ [\msg,\aux, W,\ring, \sigma, 0] $ is not recorded. If it is recorded, abort. Otherwise,
	Let $ \sigma = (\sigma,W)$ and record $ [\msg,\aux, W, \ring,\sigma, 1] $. Return $(\oramsg{signature}, \sid, \ring,W,\aux,\msg, \Out, \sigma)$ to $\user_i$.
\end{tcolorbox}






\paragraph{Malicious Ring VRF Evaluation:} This part is designed for $ \simulator $ to evaluate an input $ \msg $ with an anonymous key. For this,  it  provides to $ \fgvrf $  $ \msg $, a malicious key $ \pk $ and an anonymous key $ W $.  Then, $ \fgvrf $ evaluates  $ \msg $ with $ \pk $ if an anonymous key $ W' \neq W$  is not assigned to $ \msg $ and $ \pk $ before.  If it is the case, it returns the randomly selected evaluation value stored  in $ \evaluationslist[\msg, W] $. The reason of conditioning on a unique anonymous key for $ \msg $ and $ \pk $ is to prevent $ \simulator $ to obtain more than one evaluation values for $ \msg $ and $ \pk $. This is necessary for the uniqueness property.
We remark that it is possible  for $ \simulator $ to  obtain the same evaluation value of $ \msg $ with two different malicious  keys \eprint{$ \pk_i, \pk_j $}{} by sending $ (\oramsg{eval}, \sid, \pk_i, W, \msg) $ and $(\oramsg{eval}, \sid, \pk_j, W, \msg)$. However, this does not break the uniqueness.

\begin{tcolorbox}[left=2pt,right=2pt]
	\eprint{}{\scriptsize}
	\textbf{[Malicious Ring VRF Evaluation.]}  upon receiving a message $(\oramsg{eval}, \sid, \pk_i, W, \msg)$ from $\sim$, if $ \pk_i $ is recorded under an honest party's identity or if there exists $ W'\neq W $ where $ \anonymouskeymap[\msg,W'] = \pk_i $, ignore the request.
	Otherwise, record in the table $\vklist$ the value $(\perp,\pk_i)$ under $\simulator$ if $ (.,\pk_i) $ is not in $ \vklist $.
	If  $\anonymouskeymap[\msg,W]  $ is not defined before, set $ \anonymouskeymap[\msg,W] = \pk_i $ and let   $\Out \leftsample \setsym{S}_{eval}$ and set $ \evaluationslist[\msg, W] = \Out$.
	In any case (except ignoring), obtain $ \Out = \evaluationslist[\msg, W] $ and return $(\oramsg{evaluated}, \sid,  \msg, \pk_i,W, \Out)$ to $ \user_i $.
\end{tcolorbox}

We remark that if $ \simulator $ provides an anonymous key $ W $ of any honest party during  the evaluation process,  $ \simulator $ can learn the  evaluation of  $ \msg $ for this honest party without needing to know who is this party. For this, it just needs to send the message $ (\oramsg{eval}, \sid, \pk_i,W,\msg) $ where $ \pk_i $ is any  verification key. In such a case, $ \fgvrf $  returns immediately $ \evaluationslist[\msg, W] $ without checking whether $ \anonymouskeymap[\msg,W] = \pk_i $. So if   $ \anonymouskeymap[\msg,W]  $ belongs to an honest party, $ \simulator $ learns the evaluation value of some honest party but does not who they are. We note that this leakage does not contradict the desired security properties and helps us to prove our ring VRF protocols realizes $ \fgvrf $.  

\paragraph{Requests of  Signatures:} If $ \simulator $ provides $ W, \aux, \msg$, $ \simulator $  obtains all valid and stored ring signatures of $ \msg $ and $ \aux $ generated with an anonymous key $ W $.  

\begin{tcolorbox}[left=2pt,right=2pt]
	\eprint{}{\scriptsize}
	\textbf{[Malicious Requests of  Signatures.]} upon receiving a message $ (\oramsg{signs}, \sid, W, \aux,\msg) $ from $ \simulator $, obtain all existing valid signatures $ \sigma $ such that $ [\msg, \aux,W,.,\sigma, 1] $ is recorded and add them in a list $ \lst_{\sigma} $. 	Return $ (\oramsg{signs}, \sid, W,\aux,\msg, \lst_{\sigma})  $ to $ \simulator $.
\end{tcolorbox}




\paragraph{Ring VRF Verification:} This part of $ \fgvrf $ is to check whether  $ \sigma $ signs $ \msg $ and $ \aux $ for $ \ring $ with anonymous key $ W $. This part corresponds to $ \rVRF.\Verify $ in the real world ring VRF protocol. Therefore,
$ \fgvrf $ first checks various conditions to decide if the signature is valid. If the signature is verified, $ \fgvrf $ outputs $ b = 1 $ and $ \Out = \evaluationslist[\msg, W] $. Otherwise, it outputs $ b = 0 $ and $ \Out = \perp $. 


For the verification of the signature, $ \fgvrf $  first checks its records to see whether this signature is verified or unverified in its records i.e., checks  whether $ [\msg,\aux,W,\ring,\sigma, b'] $ is recorded (See  C\ref{cond-main:consistency}). If it is recorded, $ \fgvrf $ lets  $ b = b' $ to be consistent. Otherwise, it checks whether $ W $ is an anonymous key of an honest party generated for $ \msg $ (See C\ref{cond-main:differentsignature}). If it is the case, $ \fgvrf $ checks  its records whether this honest party requested signing  $ \msg $ and $ \aux $ for $ \ring $.  If there exists such record i.e., $ [\msg, \aux, W, \ring, ., 1] $, it stores the new signature $ \sigma $ as a valid signature in its records and lets $ b = 1 $. We remark that $ \simulator $ can create arbitrary verified signatures that sign any $ \msg $ and $ \aux $ for $ \ring $ with $ W $ once the honest party owning $ W $ has requested signing $ \msg $ and $ \aux $ for $ \ring $. This does not break the forgeability property because the honest party has already signed for it. 
If none of the above conditions (C\ref{cond-main:consistency} and C\ref{cond-main:differentsignature}) holds, it means that $ \sigma $ could be a signature generated for a malicious party. Therefore, $ \fgvrf $ asks about it to $ \simulator $ and $ \simulator $ replies with  a public key $ \pk_\simulator $ and an indicator $ b_\simulator $ showing that $ \sigma  $ is valid or invalid. Then, $ \fgvrf $ checks various conditions to prevent $ \simulator $ forging and violating the uniqueness. To prevent forging, it lets  directly $ b = 0 $, if $ \pk_\simulator $ is a key of an honest party. If $ \pk_\simulator $ is not an honest key, then $ \fgvrf $ checks its table $ \anonymouskeylist[\msg,\ring] $ which stores the anonymous keys of valid malicious signatures of $ \msg $ for $ \ring $. If the number of anonymous keys in $ \anonymouskeylist[\msg,\ring] $ is greater than or equal to the number of malicious keys in $ \ring $, then $ \fgvrf $ invalidates $ \sigma $ by letting $ b = 0 $. This condition guarantees  uniqueness meaning that the number of verifying evaluation values that $ \sim $ can generate for $ \msg $ with $ \ring $ is at most the  number of malicious keys in $ \ring $. If the number of malicious anonymous keys of valid signatures does not exceed the number of malicious keys in $ \ring $, then $ \fgvrf $ checks whether $ W $ is a unique anonymous key assigned to $ \msg,\pk_\simulator $ as in the ``Malicious Ring VRF Evaluation''. If $ W $ is unique then $ \fgvrf $ lets $ b = b_\simulator $.

After deciding $ b $, $ \fgvrf $ records it as $ [\msg, \aux, W,\ring,\sigma,b] $ to be able to reply with the same $ b $ for the same verification query later. If $ b = 1 $, $ \fgvrf $ returns $ \evaluationslist[\msg, W] $ as well.

\begin{tcolorbox}[left=2pt,right=2pt]
	\eprint{}{\scriptsize}
	\textbf{[Ring VRF Verification.]} upon receiving a message $(\oramsg{verify}, \sid, \ring,W, \aux, \msg, \sigma)$ from a party, do the following: 
	% \begin{list}[label={{C}}{{\arabic*}}, start = 1]
		% https://texblog.net/help/latex/ltx-260.html
		
		\begin{list}{\hspace*{1pt} C\arabic{FunCond}}{\usecounter{FunCond}\setlength\leftmargin{0.15in}}
			\item If there exits a record $ [\msg,\aux,W,\ring,\sigma, b'] $, set $ b = b' $. 
			%(This condition guarantees the completeness and consistency.)
			%					\item Else if $ \pk  $ is an honest verification key where $ \anonymouskeymap[W] = (.,., \pk) $ and there exists no record $ [m, \ring, W, \sigma', 1] $ for any $ \sigma' $, then let $ b= 0  $.
			%					(This condition guarantees unforgeability meaning that if an honest party never signs a message $ m $ for a ring $ \ring $, then the verification fails.)\label{cond-main:forgery}
			
			%\item Else if there exists a record  such as $ [m,W,\ring,\sigma, b'] $, set $ b = b' $. (This condition guarantees consistency meaning that all identical verification requests will output the same $ b $) 
			\label{cond-main:consistency}
			\item Else if $ \anonymouskeymap[\msg,W]  $ is an honest verification key and  there exists a record $ [\msg,\aux, W,\ring, \sigma', 1] $ for any $ \sigma' $, then let $ b=1 $ and record $ [\msg,\aux, W,\ring,\sigma, 1] $. 
			%(This condition guarantees that if $ \msg $ is signed by an honest party for the ring $ \ring $ at some point, then the signature is $ \sigma' \neq \sigma $ which is generated by the adversary is valid) 
			\label{cond-main:differentsignature}
			
			\item \label{cond-main:malicioussignature}Else relay the message $(\oramsg{verify}, \sid, \ring,W,\aux, \msg, \sigma)$ to $ \simulator $ and receive back the message $(\oramsg{verified}, \sid, \ring,W,\aux, \msg, \sigma, b_{\simulator}, \pk_\simulator)$.  Then check the following:
			
			\begin{enumerate}
				\item If   $ \pk_\simulator $ is an honest verification key, set $ b = 0 $. 
				% (This condition guarantees unforgeability meaning that if an honest party never signs a message $ \msg$ for a ring $ \ring $)
				\label{cond-main:forgery}
				
				\item Else if $ W \notin \anonymouskeylist[\msg,\ring] $ and $ |\anonymouskeylist[\msg, \ring]| \geq |\ring_{mal}| $ where $ \ring_{mal} $ is a set of malicious keys in $ \ring $, set $ b = 0 $.
				%(This condition guarantees  uniqueness meaning that the number of verifying outputs that $ \sim $ can generate for $ \msg, \ring $ is at most the  number of malicious keys in $ \ring $.)
				\label{cond-main:uniqueness}.
				%\item \label{cond-main:forgerymalicious}Else if there exists $ \anonymouskeymap[W] = (m', \ring',.)  $ where $ (m', \ring') \neq (m, \ring) $ or $ \counter[m, \ring] > |\ring_m| $ where $ \ring_m $ is a set of keys in $ \ring $ which are not honest or $ b_{\simulator} = 0 $ or $ \pk_\simulator $ belongs to an honest party, set $ b = 0 $ and record $ [m, \ring,W,\sigma, 0] $. (This condition guarantees that if $ W $ is an anonymous key of a different message and ring or the number of anonymous keys of malicious parties in $ \ring $ is more than their number or     $ \simulator $ does not verify $ \sigma $, then the verification fails.)
				
				\item Else if there exists $ W' \neq W $ where  $ \anonymouskeymap[\msg,W'] = \pk_\simulator $, set $ b = 0 $. \label{cond-main:differentWforsamepk} 
				%(This condition guarantees that there exists a unique anonymous key for each $ (\msg, \pk_\simulator) $)
				\item Else set $ b = b_\sim$. \label{cond-main:simulatorbit}
			\end{enumerate}		
			
		\end{list}
		In the end,  record $ [\msg,\aux,W,\ring,\sigma, 0] $ if it is not stored. If $ b = 0 $, let $\Out = \perp $. Otherwise,   do the following:
		\begin{itemize}
			\item if $ W \notin \anonymouskeylist[\msg,\ring] $, add $ W $ to $ \anonymouskeylist[\msg,\ring]  $.
			%\item if $ \pk_\simulator $ is not recorded, record it in $ \vklist $ under $ \simulator $.
			\item if $ \evaluationslist[\msg,W] $ is not defined, sample $ y \leftsample \setsym{S}_{eval}$. Then, set $ \anonymouskeymap[\msg,W]  = \pk_\simulator$ and $ \evaluationslist[\msg, W] = \Out$.
			\item otherwise, set $ \Out = \evaluationslist[\msg, W]$. 	
		\end{itemize}
		Finally, output $(\oramsg{verified}, \sid, \ring,W, \aux,\msg, \sigma, \Out, b)$ to the party.
		
	\end{tcolorbox}
	
	In the real-world ring VRF, the verification algorithm outputs the corresponding evaluation value of the signer. Therefore, $ \fgvrf $  outputs the signer's evaluation value if the signature is verified. However, it achieves this together with the anonymous key which is not defined in the ring VRF in the real world.  If $ \fgvrf $ did not define an anonymous key for  each signature, then there would be no way that $ \fgvrf $ determines the signer's key and outputs the evaluation value because $ \sigma $ does not need to be unique for each key. Therefore, $ \fgvrf $ maps a random and independent anonymous key to each $ \msg $ and $ \pk $ so that this key behaves as if it is the verification key of the signature. Since it  is  random and independent from $ \msg $ and $ \pk $, it does not leak any information about the party during the verification but it still allows $ \fgvrf $ to distinguish the signer.
	
	We remark that when $ \fgvrf $ is in C\ref{cond-main:malicioussignature}, it does not check whether the provided public key $ \pk_\simulator $ is in the ring. This allows $\simulator$ to generate a signature of $\msg$ for $\ring$ that is signed by $\pk_\simulator$, even if $\pk_\simulator$ is not necessarily a part of $\ring$. However, it does not break any security properties that we aim for a ring VRF scheme as it can be seen in the analysis of $ \fgvrf $ below.
	% Instead checking whether the key is in the ring, it checks
	%number of anonymous keys of malicious verified signatures for $ \ring $ and $ \msg $ to make sure that the verified malicious signatures of $ \msg $ for $ \ring $ do not output evaluation values more than malicious keys in $ \ring $ to preserve the uniqueness. 
	
	\paragraph{Corruption:} $ \simulator $ can corrupt any honest party at any time.  \eprint{So, $ \fgvrf $ provides security against an adaptive adversary.}{}
	
	\begin{tcolorbox}[left=2pt,right=2pt]
		\eprint{}{\scriptsize}
		\textbf{[Corruption:] } 
		upon receiving $ (\oramsg{corrupt}, \sid, \user_i) $ from $ \simulator $, remove $ (x_i,\pk_i) $ from $ \vklist[\user_i] $ and store them to $ \vklist $ under $ \sim $. Return $ (\oramsg{corrupted}, \sid,\user_i) $.
	\end{tcolorbox}
	%In $\fgvrf$, we suppress
	%associated data and ring commitment details to make our UC functionality
	%more accessible, meaning our ring commitment is simply the full ring. We introduce more variations of ring VRF functionalities in Appendix \ref{sec:morefuncs} with additional security properties.
	%Here are several important remarks that help elucidate $\fgvrf$ in Figure \ref{f:gvrf}:
	
	
	%
	%\begin{enumerate}[label={{R-} }{{\arabic*}}, start = 1]
	%	\item Each party is distinguished by a unique verification key which is given by the simulator. Verification keys have the identifier role of  the signatures and outputs rather than  influencing the value of them. Therefore, there exists no secret key as in the real world protocol. \eprint{We note that we need to define verification keys in $ \fgvrf $ because the real world protocol $ \rVRF $ defines a verification key (or public key) for each party.}{}
	%	
	%	\item In ring VRF, the verification algorithm outputs the corresponding evaluation value of the verified signature. Therefore, $ \fgvrf $  outputs the corresponding output during the signature verification if the signature is verified. However, it achieves this together with the anonymous key which is not defined in the ring VRF in the real world.  If $ \fgvrf $ did not define an anonymous key of each signature, then there would be no way that $ \fgvrf $ determines the actual verification key of the signature $ \sigma $ and outputs the evaluation value because $ \sigma $ does not need to be associated with the signer's key. Therefore, $ \fgvrf $ maps a random anonymous key to each $ \msg $ and $ \pk $ so that this key behaves as if it is the verification key of the signature. Since it  is  random and independent from $ \msg $ and $ \pk $, it does not leak any information about the signer during the verification but it still allows $ \fgvrf $ to distinguish the signer.
	%	
	%	\item $ \fgvrf $ does not have a separate signing protocol for malicious parties as honest parties because they can generate it as they want. If they generate a signature, it is added to the $ \fgvrf $'s records as valid or invalid when an honest party sends a verification message of it.  Its validity depends on $ \simulator $ as it can be seen in  the condition C\ref{cond-main:malicioussignature} in Figure \ref{f:gvrf}. 
	%	
	%	\item Once $ \simulator $ obtains an anonymous key $ W $ of a message $ \msg $ generated for an honest party with a key $ \pk $, we let $ \simulator $ learn the  evaluation of  $ \msg $ with $ \pk $ without knowing the $ \pk $. $ \simulator $ can do this via malicious ring VRF evaluation i.e., send the message $ (\oramsg{eval}, \sid, \pk_i,W,\msg) $ where $ \pk_i $ is a malicious verification key. Here, if $ W $ is an anonymous key of $ \msg, \pk  $, $ \fgvrf $ returns $ \evaluationslist[\msg, W] $ even if $ \pk \neq \pk_i $. 
	%	%$ \fgvrf $ returns it independent from which verification key given in $ \simulator $'s message. 
	%	
	%	\item Once $ \simulator $ obtains an anonymous key of a message $ \msg $ generated for an honest party, it can learn all valid signatures generated by $ W $ for a ring $ \ring $ and $ \msg $ via malicious requests of signatures.
	%	
	%	\item Each honest party's public key $ \pk $ is associated with a unique key $ x $ which is only used to generate honest signatures by $ \Gen_{sign} $. It is never shared with honest parties. It corresponds to the secret key of $ \pk $ in the real protocol in our instantiation of $ \Gen_{sign} $ (Algorithm \ref{alg:gensign}). Since an honest signature can be only generated by honest parties (showed below), even if $ x $ is a secret key, it does not help $ \simulator $ to generate forgeries in the ideal world.
	%	
	%\end{enumerate}
	% The functionality lets parties generate a key (Key Generation), evaluate a message with the party's key (Ring VRF Evaluation), sign a message by one of the keys (Ring VRF signature) and verify the signature and obtain the evaluation output without knowing the key used for the signature and evaluation (Ring VRF Verification). 
	%
	%%We also define linking procedures in $ \fgvrf $ to link a signature with its associated key. So, if a party wants to reveal its identity at some point, it can use the linking process to show that the evaluation is executed with its key (Linking Signature). Later on, anyone can verify the linking signature (Linking Verification).
	%
	%In a nutshell, the functionality $\fgvrf$, when given as input a message $m$ and a key set $\ring$ (that we call a ring) of party, allows to create $n$ possible different outputs pseudo-random that appear independent from the inputs. The output can be verified to have been computed correctly by one of the participants in $\ring$ without revealing who they are. At a later stage, the author of the ring VRF output can prove that the output was generated by him and no other participant could have done so.
	
	
	This is the end of description $ \fgvrf $.  It is not immediately evident which security properties our functionality provides. Therefore, we  now proceed to analyse these properties. Throughout our analysis,  the evaluation value of $ (\msg, \pk_i) $ refers to $ \evaluationslist[\msg,W] $ where $ \anonymouskeymap[\msg,W] = \pk_i $.
	

	\paragraph{Randomness:} $ \fgvrf $  satisfies the following randomness property: The evaluation value of $ (\msg, \pk_i) $ is independently and randomly selected for all honest keys $\pk_i$.
	Likewise, the evaluation value of pairs ${(\msg, \pk_i)}$ with an anonymous key $W$ provided by $\simulator$ is also randomly selected independently for all malicious keys $\pk_i$.  We remark that since $ \simulator $ can provide the same anonymous key for different public keys for the same input $ \msg $, we consider the randomness of an evaluation value that is generated for all pairs $ \{(\msg, \pk_i)\} $ sharing the same anonymous key in the case of malicious evaluations.
	
	%Given that the evaluation of $  m, \pk  $ for any verification key $ \pk $ and for any message $ m $ has never been given to $ \simulator $, the probability that $ \simulator $ guesses the evaluation of $ m, \pk_i $ is $ \frac{1}{2^{\ell_\rvrf}} $, given that $  $
	
	%Evaluation of $ (\msg, \pk_i) $ where $ \pk_i $ is an honest key is generated by first assigning a random anonymous key $ W $ to it and then assigning a random evaluation value $ y $ to $ (\msg, W)$. So, honest evaluations are always random and independent from $ (\msg,\pk_i) $. Malicious evaluation value of pairs $ \{(\msg, \pk_i)\} $ with the same anonymous $ W $ is $ \evaluationslist[\msg,W] $ which is sampled randomly and independently from $ \{(\msg,\pk_i)\} $ by $ \fgvrf $.
	
	
	
	
	\paragraph{Determinism:} $ \fgvrf $  satisfies that the evaluation value  of $ (\msg, \pk_i) $, once it has been evaluated, is unique and  cannot be changed. 
	
	%$ \fgvrf $ satisfies determinism because it checks whether $ (\msg, \pk_i) $ is evaluated before every time that it needs it. The evaluation of $ (\msg, \pk_i) $ could be changed by changing the anonymous key of $ (\msg, \pk_i)  $ but the anonymous keys cannot be changed similarly once it is set. 
	The reason of it is that once an anonymous key $ W $ is assigned to $ (\msg, \pk_i) $, it cannot be updated. Therefore, when this happen, $ \evaluationslist[\msg, W] $ is fixed leading to output always the same evaluation value.
	
	\paragraph{Unforgeability:}  If an honest party with a public key $ \pk $ never signs an input $ \msg $  and an associated data $ \aux $ for a $ \ring $, then no other party can generate a  forgery of $ \msg $ and $ \aux $ for $ \ring $ signed by $ \pk $. Formally, if an honest party with $ \pk $ never sends a message $(\oramsg{sign}, \sid, \ring, \pk,\aux, \msg)$ for some $ \ring, \msg, \aux $, then  no party can create a record $ [\msg, \aux,W,\ring, ., 1] $ in $ \fgvrf $  where $ \anonymouskeymap[\msg,\pk] = W $.
	
	%We need verify that $ \sim $ cannot generate a signature $ \sigma $ that signs a message $ m $ for a ring $ \ring $ by an honest party's key $ \pk $. In other words, we need to verify that if $ \fgvrf $ never received a message $ (\oramsg{sign}, \sid,\ring,\pk,m) $ from an honest party $ \user $ with the key $ \pk $, $ \fgvrf $ cannot have a record $ [m, W, \ring, \sigma, 1] $ such that $ \anonymouskeymap[m,W]  = \pk$ (meaning that $ \sigma $ is a valid signature generated by the honest key $ \pk $). 
	To analyse this, we need to check the places where $ \fgvrf $ records a valid signature for an honest party. The first place is during the process of honest ring VRF signature and evaluation. Here, $ \fgvrf $ records a valid signature if an honest party having a key $ \pk $ sends a message $ (\oramsg{sign}, \sid,\ring,\pk,\aux,\msg) $ to $ \fgvrf $. Therefore, 	$ \sim $ cannot create a forgery there.
	The other place is during the verification process. $ \fgvrf $ creates a valid signature record in C\ref{cond-main:differentsignature} if the corresponding honest party has already signed for $ \msg, \aux $ for $ \ring $. So, forgery is not possible  in C\ref{cond-main:differentsignature} as well. It also creates a valid signature record in C\ref{cond-main:malicioussignature}. However, $ \fgvrf $ never records a valid signature  for an honest party here because it forbids it by C3-\ref{cond-main:forgery}.
	
	
	%$ \sim $ cannot create a forgery by sending a message $ (\oramsg{sign}, \sid,\ring,\pk,\aux,\msg) $ to $ \fgvrf $ because $ \fgvrf $ checks whether the sender's key is $ \pk $ to generate a signature. Another way for $ \sim $ to create a forgery is by sending an honest key $ \pk_\simulator $  in C\ref{cond-main:malicioussignature}. However, this is also not allowed by $ \fgvrf $ in condition C3-\ref{cond-main:forgery}.\eprint{ So, the only way that $ \fgvrf $ has a record $ [\msg, \aux,W,\ring, \sigma, 1] $ where $ \anonymouskeymap[\msg,\pk] = W $ is receiving a message $ (\oramsg{sign}, \sid,\ring,\pk,\aux,\msg)  $ from the honest party $\user  $ with key $ \pk $. Therefore, forgery is not possible in $ \fgvrf $.}{}
	
	\paragraph{Uniqueness:} An evaluation value $ \Out $ for an input $ \msg $  is verified with $ \ring $, if there exists a signature $ \sigma $ such that $ \fgvrf$ returns $ (\Out, 1)$ for a query $ (\oramsg{verify}, \sid, \ring,W,\aux,\msg,\sigma)$ for some anonymous key $ W $ and message $\aux $. The uniqueness property guarantees that the number of verified evaluation values of an input $ \msg $ with $ \ring $ is not more than $ |\ring| $. $ \fgvrf $ satisfies uniqueness:
	
	If $ \fgvrf $ outputs $ (1,\Out) $ for a query $ (\oramsg{verify}, \sid, \ring,W,.,\msg,\sigma)$, it means that there exists a record $ [\msg, ., W,\ring, \sigma,1] $ and $ \Out = \evaluationslist[\msg, W] $, $ \anonymouskeymap[\msg, W]  = \pk$.  If $ \pk $ is an honest key, then it means that $ \pk \in \ring $ because $ \fgvrf $ generates a signature for an honest party with a key if $ \pk \in \ring $. Now,  let's assume $ \fgvrf $ does not satisfy uniqueness i.e.,
	there exist $ t$ different verified evaluation values $ \setsym{O} = \{\Out_1, \Out_2, \ldots, \Out_t\} $ of an input $ \msg $ with $ \ring $ where $ |\ring| = t-1 $. This implies that for each $ \Out_i \in \setsym{O} $, there exists a  record $ [\msg, ., W_i,\ring,\sigma_i,1] $ such that  $\evaluationslist[\msg,W_i] = \Out_i $ where $ \anonymouskeymap[\msg,W_i] = \pk_i $ and $ W_i \neq W_j $ for all $ i,j \in [1,t] $. Since $ \fgvrf $ makes sure that there cannot be two different anonymous keys mapping to same $ (\msg, \pk) $,  $ \pk_i \neq \pk_j $ for all $ i \neq j \in [1,t] $.
	If $ \pk_i $ is an honest key, it means that $ \sigma_i $ is not a forgery so $ \pk_i \in \ring $. Therefore, each honest evaluation value  in $ \setsym{O} $ maps to one honest public key in $ \ring $ meaning that honest evaluation values in $ \setsym{O} $ is at most $ |\ring \setminus \ring_{mal}| = n_h $. If $ \pk_i $ is not an honest  key, $ W_i \in \anonymouskeylist[\msg,\ring] $ since $ \fgvrf $ adds $ W_i $ to $ \anonymouskeylist[\msg, \ring] $ whenever it creates such record for a malicious signature. $ \fgvrf $ makes sure that in the condition C3-\ref{cond-main:uniqueness} that $ \anonymouskeylist[\msg,\ring] \leq |\ring_{mal}| = n_m$. Therefore, $ t \leq n_h + n_m = |\ring| $ which is a contradiction.
	
	%We note that $ \simulator $ can generate a valid ring signature $ \sigma $ that signs $ \msg $ with a malicious key $ \pk $ for $ \ring $ where $ \pk \notin \ring $ i.e.,  $ \fgvrf $  can have a record for a malicious signature $ \sigma $ such that $ [\msg, ., W_i,\ring,\sigma,1] $ and $ \anonymouskeymap[\msg,W]  = \pk \notin \ring$. However, it cannot create signatures of $ \msg $ for $ \ring $ which $ \fgvrf $ verifies and outputs more than  $ |\ring_{mal}| $  different evaluation values.  
	
	
	
	%
	% Clearly, in this case, $ W_i \neq W_j $ for all $ i \neq j $ since $ y_i \neq y_j $ by our assumption. The number of verified honest signatures in $ \setsym{S} $ cannot be more than number of honest verification keys in $ \ring $ because $ \fgvrf $ generates a signature for an honest party if the honest key is in the ring. Since each $ \msg,\pk $ is mapped to one unique evaluation value,
	%
	%Also, it generates a unique anonymous key for each pair $ \msg,\pk $. So, if an honest signature for $ \msg,\ring $ is verified it means that the honest verification key $ \pk $ is in $ \ring $ and $ (\msg,\pk) $ has a unique anonymous key so that unique evaluation output. Therefore, the number of anonymous keys (so that evaluation outputs) for the honest signatures is at most number of honest keys in $ \ring $. Since we know that once anonymous key $ W_i $ is set for $ m, \pk_i $, it cannot be changed. This means that there exist at least $ |\ring_{mal}| + 1 $ signatures in $ \setsym{S} $. When $ \fgvrf $ verifies a malicious signature, it checks in the condition C3-\ref{cond-main:uniqueness} how many malicious anonymous keys generated for malicious signatures of $ m, \ring $ so far. If it is more that $ |\ring_{mal}| $, $ \fgvrf $ does not verify it so that it does not output an evaluation value during such signature verification. Therefore, the simulator can generate at most $ |\ring_{mal}| $ anonymous keys for verified signatures for $ \ring $. This implies that the number of verified outputs of malicious parties   is $ |\ring_{mal}| $. 
	
	\eprint{\paragraph{Robustness:} $ \simulator $ cannot prevent an honest party to evaluate, sign or verify.
		The only place that $ \fgvrf $ does not respond any query is when it aborts. It happens when it selects an honest anonymous key which already existed. This happens in negligible probability in $ \secparam $. }{}
	
	
	\paragraph{Anonymity:} We expect from an anonymous $ \fgvrf $  to adhere to the condition that an honest signature $\sigma$ generated for an input $\msg$ with $ \Gen_{sign} $ along with its associated anonymous key $W$ should not give any information regarding the honest party's key, except for the fact that it is a member of $ \ring $.  However, this condition should hold unless $\msg$ has been signed by the same party for any other  ring. In such a case,  since both signatures includes $ W $, the anonymity may be compromised  i.e.,  $ \simulator $  learns the party's key is in the intersection of $ \ring $ and $ \ring' $. We note that this design choice is intentional, as it provides parties with the flexibility to reveal their identity when necessary.
		
	It is evident that anonymous keys do not give any information related to honest party's key as they are randomly sampled by $ \fgvrf $. However, this cannot be conclusively asserted for the signatures, because it depends on the specification of $ \Gen_{sign} $. Therefore, we introduce an anonymity definition (See Definition \ref{def:anonymity}) for $ \Gen_{sign} $ and establish that $ \fgvrf $ is anonymous if $ \Gen_{sign} $ is anonymous according to this definition.
	

	
%	\begin{definition}[Anonymity]\label{def:anonymity} We define the anonymity game against a special environment $ \mathcal{D} $ which plays the following anonymity game.
%	We define the anonymity game between a challenger and $ \mathcal{D} $.  $\mathcal{D}$ accesses a signing oracle $ \mathcal{O}_{Sign} $ and $ \fgvrf $ simulated by the challenger as described in Figure \ref{f:gvrf}. 
%		\begin{itemize}
%			\item Given the input $ '\mathsf{keygen}' $, $\mathcal{O}_{Sign} $ sends $ (\oramsg{keygen, \sid}) $ to the challenger and obtains a verification key $ \pk $. Then, it stores $ \pk $ to a list $ \mathcal{K} $ and outputs $ \pk $.
%			\item Given the input $ '(\pk,\ring,\aux,\msg)' $, $ \mathcal{O}_{Sign} $ sends $ (\oramsg{sign}, \sid,\ring,\pk, \aux, \msg) $ to the challenger and receives $ (\oramsg{signature}, \sid, \ring, W,\aux, \msg, y, \sigma) $ if $ \pk \in \ring $.  Then $ \mathcal{O}_{Sign} $ stores $ \msg $ to a list $ \arraysym{signed}[\pk]  $.
%			% and $ W  $ to a list $ \anonymouskeylist $. 
%			It outputs $ (\sigma,W) $. Otherwise, it outputs $ \perp $.
%		\end{itemize}
%		At some point,	
%		$ \mathcal{D} $ sends $ (\ring, \pk_0, \pk_1, \msg,\aux)$ to  the challenger where $ \pk_0, \pk_1 \in \ring $, $ \msg  \notin \arraysym{signed}[\pk_0]$ and $ \msg  \notin \arraysym{signed}[\pk_1] $\footnote{The challenger needs to check this because if $ \msg $ is signed before by one of $ \pk_0,\pk_1 $, then the anonymity is broken trivially because verification of any signature of $ \msg $ for different rings outputs the same evaluation value}.  Challenger lets $ b \leftarrow_r\bin$. Then it gives the input $ (\pk_b, \ring, \msg,\aux) $ to $ \ora{Sign} $ and receives either $ \perp $ or $(\sigma,W)$. If it is $ (\sigma,W) $, it sends $ (\sigma,W) $ to $ \mathcal{D} $ as a challenge.
%		If $ \mathcal{D} $ sends $ '(\pk,\ring,\aux,\msg)' $ to $ \mathcal{O}_{Sign} $ where $ \pk = \pk_0 $ or $ \pk = \pk_1 $, it loses the game. 
%		During the game if $ \mathcal{D} $ outputs $ b' = b $, $ \mathcal{D} $ wins.
%		
%		$ \fgvrf $ satisfies anonymity, if any PPT distinguisher $ \mathcal{D} $ has a negligible advantage in $ \secparam $ to win the anonymity game defined as follows:
%		
%	\end{definition}
	
	\begin{definition}[Anonymity of $ \Gen_{sign} $]\label{def:anonymity}
		We define an anonymity game between a PPT distinguisher $ \mathcal{D} $ and a challenger.
		In the game, $\mathcal{D}$ sends the query  $ (\oramsg{challenge}, \ring, (\sk_0,\pk_0), (\sk_1, \pk_1), \msg,\aux)$. Then the challenger checks if  $ \pk_0, \pk_1 \in \ring $. If it is the case, the challenger samples randomly  $ b\in \{0,1\}  $ and runs $ \Gen_{sign}(\ring, \sk_b, \pk_b,\aux, \msg) \rightarrow \sigma_b $. It gives $ \sigma_b $ as a challenge to $ \mathcal{D} $. In the end of the game, if $ \mathcal{D} $ outputs $ b' = b $, then wins the game.
			
		We say that $ \Gen_{sign} $ is anonymous  if any PPT distinguisher $ \mathcal{D} $ has a negligible advantage in $ \secparam $ to win the anonymity game.
		%TODO Specify the sec param for the functionality
	\end{definition}
	
	%	We remark that $ \mathcal{D} $ generates keys of honest parties and forwards them via dummy adversaries in the ideal model. So, $ \Gen_{sign} $ of $ \fgvrf $ should be defined in a way that it preserves the anonymity even if $ \mathcal{D} $ generates the keys.
	
	
	%
	%to make $ \fgvrf $ set $ \anonymouskeymap[m,W] $ with an honest key for any $ m,W $.  Let's see whether this is possible. The places that $ \fgvrf $ can set $ \anonymouskeymap[m,W] = \pk $  is during ``malicious ring VRF evaluation", ``honest ring VRF signature and evaluation" and ``verification". Clearly, this event cannot happen via malicious ring VRF evaluation  because $ \pk $ is an honest key. It cannot happen via  ``honest ring VRF signature and evaluation''  because we assume that the party $ \user $ with the key $ \pk $ never asked for it. It cannot happen via ``verification'' because \ref{cond:forgery} prevents $ \simulator $ to generate an anonymous key for $ \pk $.      This shows us that $ \simulator $ cannot generate a signature.
	
	%
	% to generate a valid signature is via verification i.e., when a party sends a message  $ (\oramsg{verify}, \sid,\ring,W,m, \sigma) $ to $ \fgvrf $.  During the verification, if $ \fgvrf $ is in \ref{cond:differentsignature} and \ref{cond:simulatorbit} then the validity of the signature is decided by $ \simulator $. If $ \fgvrf $ is in \ref{cond:simulatorbit}, it means that there exists no $ \anonymouskeymap[m,W] = \pk \in \mathcal{P}_H $ because if it existed, $ \fgvrf $ would be in \ref{cond:differentsignature}. Therefore, the signature verified in \ref{cond:simulatorbit} cannot be a signature of an honest party's key.  This means that $ \simulator $ cannot generate a forgery via \ref{cond:simulatorbit}. So, the only left way for $ \simulator $ to generate a forgery is via \ref{cond:differentsignature}.
	%If $ \fgvrf $ is in \ref{cond:differentsignature}, then $ \anonymouskeymap[m,W] $ belongs to an honest party and another signature $ \sigma' \neq \sigma $ has already been stored as valid for $ W, \ring, m $.  If $ \sigma' $ is not generated by this honest party, then it means that $ \simulator $ forges. Let's see whether this is possible. If there exists a record $ [m, W, \ring, \sigma',1] $ , it means that $ \anonymouskeymap[m,W] = \pk \in \mathcal{P}_H$ exists. The places that $ \fgvrf $ can set $ \anonymouskeymap[m,W] = \pk $  is during ``malicious ring VRF evaluation", ``honest ring VRF signature and evaluation" and ``verification". Clearly, this event cannot happen via malicious ring VRF evaluation  because $ \pk $ is an honest key. It cannot happen via  ``honest ring VRF signature and evaluation''  because we assume that the party $ \user $ with the key $ \pk $ never asked for it. It cannot happen via ``verification'' because \ref{cond:forgery} prevents $ \simulator $ to generate an anonymous key for $ \pk $.      This shows us that $ \simulator $ cannot generate a signature via \ref{cond:differentsignature}.
	
	
	
	%Now, we verify that $ \fgvrf $ satisfies these properties. During our analysis, when we say that a message $ m $ signed by an anonymous key $ W $ we mean that $ [m,W,.,.,1] $ is recorded. We say that the signature is honest if $ \anonymouskeymap[m,W] = \pk $ is an honest party's key.
	%
	%
	%
	%\paragraph{\textbf{Uniqueness:}}
	%
	%\paragraph{\textbf{Robustness:}} We check whether $ \simulator$ can prevent an honest party signing and verifying. $ \fgvrf $ does not abort during the verification so an honest party can verify all signatures. $ \fgvrf $ aborts during the honest signing process if $ \gen_{sign}(m, \ring) $ generates a signature which was invalidated before i.e., there exists a record $ [m, W, \ring, \sigma,0] $.
	%
	%
	%
	%
	%\begin{enumerate}[label={{R-} }{{\arabic*}}, start = 1]
	%
	%%\item The ring VRF signature does not need to be random but it must be \emph{unique}  for its ring and the message. The reason of it to have a mapping from a ring VRF signature to its evaluation output. The map is necessary for $ \fgvrf $ to output the corresponding evaluation value for the signature during the verification process i.e, $ [m, \ring, \sigma] \rightarrow \pk, \evaluationsecretlist[m, \ring][\pk] \rightarrow y $.
	%\item In classical VRF, a VRF $ F $ is a deterministic function which maps a message and a public key to a random output. While in ring VRF, a message, a public key and a ring map to a random value, the verification algorithm of a ring VRF does not take the key as an input because it should be hidden. Therefore, the verification should be executed without the public key.  So, the functionality $ \fgvrf $ needs to find a way to verify the ring VRF output of a message, a public key and a ring map without knowing the public key. Because of this, $ \fgvrf $ generates an anonymized key $ W $ for each evaluation so that a message $ m $  and $ W $ maps to the random output. One can imagine this  as if a VRF output is generated with the input message $ m $ and the key $ W $ as in classical VRF i.e.,  $ F(m, W) $. 
	%
	%\item  If an honest party signs a message for a ring and obtains a signature, $ \fgvrf $ allows the simulator to generate another signature in \ref{cond-main:differentsignature} if the simulator wants. We remark that this is not a security issue because an honest party has already committed to sign the message.  A similar condition  exists in the EUF-CMA secure signature functionality $ \fsig $ \cite{canettiFsig}.
	%
	%\item \ref{cond-main:simulatorbit} of the ring VRF verification process covers the case where the adversary decides whether accepting the signature generated for its key if  it could be a valid signature for the ring i.e., the malicious key is in the ring and the anonymous key in the verification request is unique.
	%
	%\item The linking signature and the linking verification works similar to the EUF-CMA secure signature functionality $ \fsig $ \cite{canettiFsig}.
	
	
	%\end{enumerate}
	
	
	%\begin{definition}[Anonymous $ \fgvrf $]\label{def:anonymity}
	%	We call that $ \fgvrf $ is anonymous if the outputs of $ \gen_{sign} $ and $ \gen_W $ are pseudo-random.
	%	%TODO define this more formally.
	%\end{definition}
	
	
	%
	%Below, we define the real-world execution of a ring VRF.
	%\begin{definition}[Ring-VRF (rVRF)]\label{def:ringvrf}
	%	%TODO ADD anonymous key here
	%	A ring VRF is a VRF with a  function $ F(.):\{0,1\}^* \rightarrow \{0,1\}^{\ell_\rvrf} $ and with the following algorithms:
	%	
	%	\begin{itemize}
		%		\item $ \rvrf.\keygen(1^\kappa) \rightarrow (\skrvrf,\pk)$ where $ \kappa $ is the security parameter,
		%	\end{itemize}
	%	Given list of public keys $ \ring = \set{\pk_1, \pk_2, \ldots, \pk_n}$, a message $ m \in \{0,1\}^{\ell_m} $
	%	\begin{itemize}
		%		\item $ \rvrf.\eval(\skrvrf_i, \ring, m) \rightarrow y$
		%		\item $ \rvrf.\sign(\skrvrf_i, \ring, m)\rightarrow (\sigma,W) $ where  $\sigma $ is a signature of the message $ m $ signed by $ \skrvrf_i, \ring $ and $ W $ is an anonymous key.
		%		\item $ \rvrf.\verify(\ring,W, m,\sigma) \rightarrow  (b, y)$ where $ b \in \{0,1\} $ and $ y \in \{0,1\}^{\ell_\rvrf}\cup \{\perp\} $. $ b =1 $ means verified and $ b = 0 $ means not verified.
		%%		\item $ \rvrf.\link(\skrvrf_i, \ring,W,m, \sigma) \rightarrow \hat\sigma $ where  $ \hat\sigma $ is a signature that links signer of the ring signature $ \sigma $. 
		%%		\item $ \rvrf.\link\verify( \pk_i,\ring,W, m, \sigma, \hat\sigma)\rightarrow b$ where $ b \in \{0,1\} $. $ b =1 $ means verified and $ b = 0 $ means not verified.
		%	\end{itemize}
	%	
	%\end{definition}

%
\section{Ring VRF construction}% {Ring VRFs from the Pedersen VRF}
\label{sec:pederson_vrf}

In the following we instantiate our ring VRF construction with an efficient evaluation proof, which
we call the Pedersen VRF and denote \PedVRF.
\PedVRF instantiates the NIZK for the relation $\Reval$ introduced in a general form 
in \S\ref{sec:overview}. In this section we focus upon Pedersen VRF and the relations 
describing a SNARK for ring membership; we discuss the zero-knowledge continuation
that makes the overall ring VRF efficient in the next section.

Our ring VRF construction works with public parameters $ \pprvrf = (\crsrvrf, p,\grE, \genG,\genB, \setsym{S}_{eval}  = \F_p)$ generated by $ \rVRF.\Setup(1^\secparam) $. Here, $ p $ is a prime number and the order of the group $ \grE $ which has generators $ \genG,\genB $.  $ \crsrvrf $ is a common reference string generated by $ \NIZK_{\Rring}.\Setup(1^\secparam) $. Our ring VRF construction deploys random oracles $H_p,H: \{0,1\}^* \to \F_p$, $H_{\grE}: \{0,1\}^* \to \grE$ and $ H_{\ring} $ for constructing a Merkle tree.

%We refer readers to \S\ref{sec:ec_background} for notation,
% like our curves and hash functions.
%In particular miss-use resistance dictates \PedVRF be instantiated with
%two elliptic curves: $\ecE$ (or $\ecE_1$) handles key commitments
% build with two independent base points $\genG$ and $\genB$,
%We hash-to a ``sister'' Edwards curve $\ecEsis$ with a subgroup $\grEsis$
% of the same order $p$ as $\grE$.
%In practice $\grEsis$ has cofactor $\hsis$ divisible by 4, while
% $\grE$ might have effective cofactor 1 if deserialization enforces subgroup checks.
%Any readers only interested in theoretical security arguments
%should assume $\ecEsis = \ecE$ and $\hsis = 1 = h$,
% while implementers should read more carefully.
We first describe Pedersen VRF before giving our ring VRF construction.
\paragraph{Pedersen VRF:} 
We construct \PedVRF similarly to 
\cite{nsec5,VXEd25519,draft-irtf-cfrg-vrf-10},
except we replace the public key by a Pedersen commitment
$\sk \, \genG + \openpk \, \genB$ to the secret key \sk.
We do not expose a public key from \KeyGen, nor inject the public key in \Eval.

\begin{itemize}
	\item $\PedVRF.\KeyGen: \pprvrf \mapsto \sk$  where $\sk \leftsample \F_p$. % and $\pk = \sk \, \genG$.
	\item $\PedVRF.\Eval : (\sk,\msg) \mapsto H(\msg, \PreOut)$ where $\PreOut = \sk \, H_{\grE}(\msg)$.
\end{itemize}
\noindent We add an algorithm to obtain a Pedersen commitment to the secret key \sk.
\begin{itemize}
	\item $\PedVRF.\CommitKey: \sk \mapsto (b, \compk)$ \,
	where  $\openpk \leftsample \F_p$ is a blinding factor
	and $\compk = \sk \, \genG + \openpk \, \genB$ is a Pedersen commitment.
	% \item $\PedVRF.\OpenKey(\compk,\openpk)$ \,
	% returns $\pk = \compk - \openpk \, \genB$.
	% \item $\PedVRF.\OpenKey(\sk,\openpk)$ \,
	% returns $\compk = \sk \, \genG + \openpk \, \genB$.
\end{itemize}
%We do not expose an opening algorithm here because opening occurs inside
%our zero knowledge continuation,
%as described in $\Rring$ and \S\ref{sec:rvrf_cont} blow.

Our \Sign and \Verify algorithms of \PedVRF correspond to
the \Prove algorithm and verification procedure of a Chaum-Pedersen DLEQ proof
for relation $\mathcal{R}_{eval}$ (see below),
instantiated by a Fiat-Shamir transform of a sigma protocol.

$$ \mathcal{R}_{eval} = \Setst{
	\begin{aligned}
		& (\compk,\PreOut,\In) ; \\ 
		& \,\, (\sk, \openpk) 
	\end{aligned}
}{
	\begin{aligned}
		& \compk = \sk\,\genG + \openpk\,\genB, \\
		& \,\, \PreOut = \sk \, H_{\grE}(\msg) 
	\end{aligned}
}  \mathperiod \label{rel:commit} 
$$




\begin{itemize}
	\item $\PedVRF.\Sign : (\sk,\openpk,\msg,\aux) \mapsto \sigma$ \,.
	% takes a secret key \sk and blinding factor \openpk, an input $\msg$, and associated data \aux, and then performs
	% $ \NIZK_{\mathcal{R}_{eval}}.\Prove(((\genG, \genB,\grE,\compk,\PreOut,\In); (\sk, \openpk))) $
	% $ \sigma = (\PreOut,\compk,\pi_{eval}) $
	Compute $\PreOut := \sk \, H_{\grE}(\msg)$ and $ \compk = \sk \genG + \openpk\genB $. Then, run $\NIZK_{\rel_{eval}}.\Prove(\compk, \PreOut, \msg;\sk,\openpk,\msg)  $ which generates a Chaum-Pedersen DLEQ proof
	for relation $\mathcal{R}_{eval}$ i.e.,
	let $r_1,r_2 \leftsample \F_p$
	and compute $R = r_1 \genG + r_2 \genB, R_m = r_1 H_{\grE}(\msg) $ and $c = H_p(\aux,\msg,\compk,\PreOut,R,R_m)$, finally compute $s_1 = r_1 + c \, \sk$ and $s_2 = r_2 + c \, \openpk$ and let $ \pi = (R,R_m,s_1,s_2) $.
	Return the signature $\sigma = (\PreOut,\pi) $.
	% and return the signature $\sigma = (\PreOut,c,s_1,s_2)$.
	
	\item $\PedVRF.\Verify : (\compk,\msg,\aux,\sigma) \mapsto {\color{red}\Out \,\, \lor \perp}$ \,.
	% $ \NIZK_{\mathcal{R}_{eval}}.\Verify(\genG, \genB,\grE,\compk,\PreOut,\In, \pi_{eval} ) $
	Parse $\sigma = (\PreOut, R,R_m,s_1,s_2) $ and compute 
	$c = H_p(\aux,\msg,\compk,\PreOut,R,R_m)$.
	If $ R = s_1 \genG + s_2 \genB - c \, \compk$ and
	and $ R_m = s_1 H_{\grE}(\msg) - c \, \PreOut$,
	% First parse $\sigma = (\PreOut, \pi_{eval} =(c,s_1,s_2))$,
	% recompute $\In := H_{\grEsis}(\msg)$, 
	% $R = s_1 \genG + s_2 \genB - c \, \compk$, and
	% $R_m = s_1 \In - c \, \PreOut$.
	% Finally if $c = H_p(\aux,\msg,\compk,\PreOut,R,R_m)$ holds,
	then return $H(\msg,\PreOut)$, which equals $\PedVRF.\Eval(\sk,\msg)$,
	or return failure $\perp$ otherwise.
\end{itemize}




% DESCRIPTION WITH COFACTOR
%\begin{itemize}
%    \item $\PedVRF.\KeyGen$ \, returns $\sk \leftsample \F_p$. % and $\pk = \sk \, \genG$.
%    \item $\PedVRF.\Eval : (\sk,\msg) \mapsto H'(\msg, \hsis\,\PreOut)$ where $\PreOut = \sk \, H_{\grEsis}(\msg)$.
%\end{itemize}
%\noindent We instead add an algorithm to obtain a Pedersen commitment to the secret key \sk.
%\begin{itemize}
%    \item $\PedVRF.\CommitKey(\sk)$ \,
%    returns a blinding factor $\openpk \leftsample \F_p$
%    and a commitment $\compk = \sk \, \genG + \openpk \, \genB$.
%    % \item $\PedVRF.\OpenKey(\compk,\openpk)$ \,
%    % returns $\pk = \compk - \openpk \, \genB$.
%    % \item $\PedVRF.\OpenKey(\sk,\openpk)$ \,
%    % returns $\compk = \sk \, \genG + \openpk \, \genB$.
%\end{itemize}
%We do not expose an opening algorithm here because opening occurs inside
%our zero knowledge continuation,
% as described in $\Rring$ and \S\ref{sec:rvrf_cont} below.
%
%Our \Sign and \Verify algorithms of \PedVRF correspond to
%the \Prove and \Verify algorithms of a Chaum-Pedersen DLEQ proof
% for relation $\mathcal{R}_{eval}$,
%instantiated by a Fiat-Shamir transform of a sigma protocol.
%$$ \mathcal{R}_{eval} = \Setst{
	%  \begin{aligned}
		%    & (\compk,\PreOut,\In) ; \\ 
		%    & \,\, (\sk, \openpk) 
		%  \end{aligned}
	%}{
	%  \begin{aligned}
		%    & \compk = \sk\,\genG + \openpk\,\genB, \\
		%    & \,\, \PreOut = \sk\,\In 
		%  \end{aligned}
	%}  \mathperiod \label{rel:commit} 
%$$
%%
%\begin{itemize}
%	\item $\PedVRF.\Sign : (\sk,\openpk,\msg,\aux) \mapsto \sigma$ \,
%	% takes a secret key \sk and blinding factor \openpk, an input $\msg$, and associated data \aux, and then performs
%    % $ \NIZK_{\mathcal{R}_{eval}}.\Prove(((\genG, \genB,\grE,\compk,\PreOut,\In); (\sk, \openpk))) $
%    % $ \sigma = (\PreOut,\compk,\pi_{eval}) $
%	First compute $\In := H_{\grEsis}(\msg)$ and $\PreOut := \sk \, \In$ and \compk.
%	Next sample random $r_1,r_2 \leftsample \F_p$
%	to compute $R = r_1 \genG + r_2 \genB$ and $R_m = r_1 \In$.
%	Compute the challenge $c = H_p(\aux,\msg,\compk,\PreOut,R,R_m)$.
%	Finally compute $s_1 = r_1 + c \, \sk$ and $s_2 = r_2 + c \, \openpk$.
%	and return the signature $\sigma = (\PreOut,R,R_m,s_1,s_2)$.
%	% and return the signature $\sigma = (\PreOut,c,s_1,s_2)$.
%
%	\item $\PedVRF.\Verify : (\compk,\msg,\aux,\sigma) \mapsto \Out \,\, \lor \perp$ \,
%	% $ \NIZK_{\mathcal{R}_{eval}}.\Verify(\genG, \genB,\grE,\compk,\PreOut,\In, \pi_{eval} ) $
%	First parse $\sigma = (\PreOut, \pi_{eval} = (R,R_m,s_1,s_2))$,
%	recomputes $\In := H_{\grEsis}(\msg)$ and 
%	$c = H_p(\aux,\msg,\compk,\PreOut,R,R_m)$.
%    Finally if $h \, R = h \, (s_1 \genG + s_2 \genB - c \, \compk)$ and
%    and $h \, R_m = h \, (s_1 \In - c \, \PreOut)$ both hold,
%	% First parse $\sigma = (\PreOut, \pi_{eval} =(c,s_1,s_2))$,
%	% recompute $\In := H_{\grEsis}(\msg)$, 
%    % $R = s_1 \genG + s_2 \genB - c \, \compk$, and
%    % $R_m = s_1 \In - c \, \PreOut$.
%    % Finally if $c = H_p(\aux,\msg,\compk,\PreOut,R,R_m)$ holds,
%    then return $H(\msg, \hsis\,\PreOut)$, which equals $\PedVRF.\Eval(\sk,\msg)$,
%    or return failure $\perp$ otherwise.
%\end{itemize}



%\noindent We described the deterministically batchable flavor analogous
%to \cite{HdVBatchEd25519} because $s_2$ makes our signature large enough
%that half-aggregation makes sense, unlike EC VRF. %NOT CLEAR SENTENCE
We remark that \PedVRF becomes almost EC VRF if
we demand $\openpk = r_2 = 0$ in \Sign.
%but our public key handling in \PedVRF breaks VRF definitions somewhat. %NOT CLEAR SENTENCE

\paragraph{The Ring VRF Construction:} We now describe how our ring VRF construction works as a combination of $ \PedVRF $, $ \NIZK_{\Rring} $ (for a relation 
$\Rring$ instantiated below) and a commitment scheme $ \mathsf{Com} $.

\begin{itemize}
	\item $\rVRF.\KeyGen$ \, returns as secret key $\sk,r \leftsample \F_p$ and $ \pk $ as public key where $ \pk = \mathsf{Com}.\mathsf{Commit}(\sk,r)  $. We note that $ \pk  $ can be alternatively defined as $ \pk = \sk \genG $ according to the SNARK used for $ \Rring $. In this case, we would not have $ r $ as a part of the secret key.  We provide one optimal public key design in \S\ref{subsec:rvrf_faster} for our SNARK used for $ \Rring $. 
	
	\item $\rVRF.\Eval((\sk,r), \msg) $ runs $\PedVRF.\Eval(\sk,\msg)$. We remark that the evaluation value is generated with \emph{only} the first part of the secret key which is $ \sk $.
	
	\item $ \rVRF.\CommitRing: (\ring,\pk) \mapsto (\comring,\openring)$ compute a Merkle tree root $\comring  $ using random oracle $ H_\ring $ considering the elements of $ \ring $ as leaves and generating a  Merkle tree path $ \openring $ that verifies $ \pk \in \ring $.
	
	\item $ \rVRF.\OpenRing: (\comring, \openring) \mapsto \pk $ verify that the root computed via Merkle tree path $ \openring $ is  $ \comring $. If it is the case, output $ \pk \in \openring $. Otherwise, output $ \perp $. 
	
	We choose the ring commitment scheme so the $\rVRF.\OpenRing$ invocation
	is relatively SNARK friendly in our ring membership relation $ \Rring $. We note that an alternative ring commitment scheme may be used 
	where $ \comring = \ring $ and $ \openring = \pk $, however as we will see below, it incurs a high cost of $O(|\ring|)$ for the ring VRF verifier. 
		
\end{itemize}

\noindent The \Sign and \Verify for  our \rVRF are a combination of \Sign and \Verify from \PedVRF and
\Prove and \Verify from $\NIZK_{\Rring}$, as follows:
\def\tmpaux{\aux \doubleplus \piring \doubleplus \comring}
\def\tmpeprintaux{\eprint{\aux'}{\tmpaux}}
\def\tmpindent{\hspace*{5pt}}
\begin{itemize}
	
	\item $\rVRF.\rSign : ((\sk,r),\comring, \openring,\msg,\aux) \mapsto \rho$
	returns a ring VRF signature $\rho = (\compk,\piring,\sigma, \comring)$
	if \openring is a correct opening of \comring.  In this, $(\openpk,\compk) \leftarrow \PedVRF.\CommitKey(\sk)$,  $\piring \leftarrow \NIZK_{\Rring}.\Prove((\compk,\comring); \openpk,\openring, \pk,\sk,r)$ where  $\aux' \leftarrow \tmpaux$,  $\sigma \leftarrow \PedVRF.\Sign(\sk,\openpk,\msg, \aux')$.  We instantiate  $ \Rring 
	$\footnote{We note that if $ \pk = \sk\genG $ then $ \Rring $ does not need $ \sk,r $ as a part of the witness. In this case we need to replace the last two conditions by $ \compk = \pk + \openpk \genB $.} with

	
	$$ \Rring = \Setst{ (\compk,\comring ;\openpk,\openring,\sk,r) }{
		\begin{aligned}
			&	\pk = \OpenRing(\comring,\openring), \\
			& 	\sk = \mathsf{Com}.\mathsf{Open}(\pk,r), \\
			& 	\compk = \sk \genG + \openpk \genB
		\end{aligned}	
	}$$
	\item $\rVRF.\rVerify : (\comring,\msg,\aux,\rho) \mapsto {\color{red} \Out \,\, \lor \perp}$ \, 
	parses $\rho$ as $(\compk,\piring,\sigma)$,  sets $\aux' \leftarrow \tmpaux$ and runs $\NIZK_{\Rring}.\Verify((\compk,\comring); \piring)$.
	If it fails, returns $ (0,\perp) $. Otherwise, returns $\PedVRF.\Verify(\compk,\msg, \aux', \sigma)$.
\end{itemize}


% BEGIN TODO: Oana

% % Although \PedVRF itself exhibits surprising properties, our gestalt 
% \rVRF satisfies sensible security definitions:
% Pseudo-randomness holds by reduction to singleton rings.
% Ring uniqueness, ring unforgeability, and ring anonymity resemble security
% arguments for other ring signatures built from SNARKs.

% \begin{proposition}\label{prop:rvrf_games}
	% $\rVRF$ satisfies ring uniqueness, ring unforgeability, and ring anonymity.
	% \end{proposition}

% END TODO: Oana

Appendix \ref{ap:ucproof} proves our ring VRF construction realizes $ \fgvrf $ in Figure \ref{f:gvrf}. Intuitively, the randomness and the determinism of $ \rVRF.\Eval $ come from the random oracles $ H' $ and $ H_{\grE'} $.  The anonymity of our ring VRF signature comes from the perfect hiding property of Pedersen commitment, the zero-knowledge property of $ \NIZK_{\mathcal{R}_{ring}} $ (Lemma \ref{lem:anonymity}) and the difficulty of DDH in  $ \grE $ (Lemma \ref{lem:honestoutput}) so that $ \PreOut $ is indistinguishable from a random element in $ \grE $. The unforgeability and uniqueness come from the fact that CDH is hard in $ \grE $ (Lemma \ref{lem:simulation-ind}), i.e., for unforgeability,  one cannot commit an honest party's secret key without breaking the CDH problem and for the uniqueness,  if one can obtain $ \PedVRF $ signatures such that $ \sigma_1 = (\PreOut_1, \pi_{\PedVRF}) $ and $ \sigma_2 = (\PreOut_2, \pi'_{\PedVRF}) $ where  $ \PreOut_1 \neq \PreOut_2 $  and verified by \compk for the message \msg, then we break a CDH problem in $ \grE $.

\begin{theorem}\label{thm:rvrfmain}
	$ \rVRF $  over the group structure $ (\grE,p,\genG,\genB) $ realizes $ \fgvrf $ in Figure \ref{f:gvrf} in the random oracle model assuming that $ \NIZK_{\mathcal{R}_{eval}} $ and $ \NIZK_{\mathcal{R}_{ring}}$ are zero-knowledge and knowledge sound, the decisional Diffie-Hellman (DDH) problem are hard in $ \grE  $ and the commitment scheme $ \mathsf{Com} $ is binding and perfectly hiding. 
\end{theorem}
%TODO: Proof Sketch
% NOTE:  Is this redundant after the above paragraph?
% The security proof of Theorem \ref{thm:rvrfmain} is in Appendix \ref{ap:ucproof}.



\endinput

\section{Zero-knowledge Continuations}
\label{sec:rvrf_cont}

\noindent In the following, we describe a NIZK for a relation $\rel$ where
$$\rel = \{(\bary, \barz; \barx, \baromega_1, \baromega_2):  (\bary, \barx; \baromega_1) \in \relone, (\barz, \barx; \baromega_2) \in \reltwo \},$$
and $\relone$, $\reltwo$ are some NP relations. Our NIZK is designed to efficiently re-prove membership for relation $\relone$
via a new technique which we call \emph{zero-knowledge continuation}. In practice, using a NIZK that is a zero-knowledge continuation 
ensures one essentially needs to create only once an otherwise expensive proof for $\relone$ which can later be 
re-used multiple times (just after inexpensive re-randomisations) while preserving knowledge soundness and zero-knowledge. 
Below, we formally define zero-knowledge continuation. In section~\ref{sec:rvrf_groth16} we instantiate it via a \emph{special(ized) 
Groth16} or \SpecialG, and finally, in section~\ref{subsec:rvrf_faster} we use it to build a ring VRF with fast amortised prover time. \\

\noindent In addition, the anonymity property of our ring VRF demands we not only finalise multiple times a component of the zero-knowledge 
continuation and but also each time the result remains unlinkable to previous finalisations, meaning our ring VRF stays zero-knowledge 
even with a continuation component being reused. We formalise such a more general zero-knowledge property in 
section~\ref{sec:rvrf_groth16} and give an instantiation of our NIZK fulfilling such a property in section~\ref{subsec:rvrf_faster}. 
%Moreover, the anonymity property of a ring VRF demands we finalise the amortized ``continuation'' multiple
%times, with each time being unlinkable to the others, meaning our rVRF
%stays zero-knowledge even with the continuation being reused.


%\begin{definition}[ZK Continuations] A zero-knowledge continuation $\SpecialG_\rel$ consists of four algorithms 
%($\SpecialG_\rel.\Preprove$, $\SpecialG_\rel.\Reprove$, $\ldots$, $\SpecialG_\rel.\Verify$) such that:
%\begin{itemize}
%\item $\SpecialG_\rel.\Preprove : (\bar{y}, \bar{x}; \baromega_1) \mapsto (X,\pi)$ \,
%constructs a commitment $X$ and a proof $\pi$ for relation $\rel$ from a vector 
%of inputs $\bar{y}$ (called \em{transparent}), a vector of inputs $\bar{x}$ (called \em{opaque}), and witnesses $\baromega_1$.
%\item $\SpecialG_\rel.\Reprove : (X,\pi) \mapsto ((X',\pi'); b)$ \,
%finalises the commitment $X'$ and proof $\pi'$ and returns an opening $b$ for the commitment. 
%\item $\SpecialG_\rel.\Verify(\bar{y}; (X',\pi') )$ \, 
%verifies the 
%\end{itemize}
%%TO DO: add an algorithm to the $\SpecialG_\rel$ such that: Our \Verify needs a separate proof-of-knowledge for $X'$, 
%%the production of which requires knowledge of $\bar{x}$, and occurs in parallel to \Reprove.

%We define (white-box) knowledge soundness for zero-knowledge continuations
%exactly like for zero-knowledge proofs, but with the composition 
%$\Prove : (\bar{y}, \bar{x}; \baromega_1) \mapsto \Reprove(\Preprove(\bar{y}, \bar{x}; \baromega_1))[0]$
%as well as this additional proof-of-knowledge.
%Zero-knowledge however should hold even if \Reprove gets invoked multiple
%times upon the same \Preprove results, again even with the additional proof-of-knowledge.
%\end{definition} 

\begin{definition}[ZK Continuations]
\label{def:zk_cont}
 A zero-knowledge continuation $\ZKCont$ for a relation $\relone$ with 
input $(\bary, \barx)$ and witness $\baromega_1$ is a tuple of efficient algorithms 
($\ZKCont.\Setup$, $\ZKCont.\Gen$, $\ZKCont.\Preprove$, $\ZKCont.\Reprove$, $\ZKCont.\VerifyCom$, $\ZKCont.\Verify$, $\ZKCont.\Sim$) 
such that for implicit security parameter $\lambda$,
\begin{itemize}

\item $\ZKCont.\Setup: (1^{\lambda}) \mapsto (\crs, \tw)$ a setup algorithm that on input the security parameter 
outputs a common reference string $\crs$ and a trapdoor $\tw$,

\item $\ZKCont.\Gen: (\crs, \relone) \mapsto (\pp, \crspk, \crsvk)$ \, 
outputs a list $\pp$ of public parameters and a pair of proving key $\crspk$ and verification key $\crsvk$, 

\item $\ZKCont.\Preprove: (\crspk, \bar{y}, \bar{x}, \baromega_1, \relone) \mapsto (X, \pi, b)$ \,
constructs commitment $X$ from a vector of inputs $\bar{x}$ (called \emph{opaque}) and 
constructs proof $\pi$ from vector of inputs 
$\bar{y}$ (called \emph{transparent}), from $\bar{x}$ and vector of witnesses $\baromega_1$, and 
also outputs $b$ as the opening for $X$,

\item $\ZKCont.\Reprove: (\crspk, X, \pi, b, \relone) \mapsto (X', \pi', b')$ \,
finalises commitment $X'$ and proof $\pi'$ and returns an opening $b'$ for the commitment, 

\item $\ZKCont.\VerifyCom: (\pp, X, \barx, b) \mapsto 0/1$ \, 
verifies that indeed $X$ is a commitment to $\barx$ with opening (e.g., randomness) $b$ and 
outputs 1 if indeed that is the case and 0 otherwise,
 
\item $\ZKCont.\Verify: (\crsvk, \bar{y}, X', \pi', \relone) \mapsto 0/1$ \, outputs $1$ in case it accepts and $0$ otherwise,

\item $\ZKCont.\Sim: (\tw, \bary, \relone) \mapsto (\pi', X')$ takes as input a simulation trapdoor $\tw$ and statement $(\bary, \barx)$ and returns 
arguments $\pi'$ and $X'$,
\end{itemize}
and satisfies perfect completeness for $\Preprove$ and for $\Reprove$,  knowledge soundness and zero-knowledge as defined below:\\
%TO DO: Re-write this: We define (white-box) knowledge soundness for zero-knowledge continuations
%exactly like for zero-knowledge proofs, but with the composition 
%$\Prove : (\bar{y}, \bar{x}; \baromega_1) \mapsto \Reprove(\Preprove(\bar{y}, \bar{x}; \baromega_1))[0]$
%as well as this additional proof-of-knowledge.
%Zero-knowledge however should hold even if \Reprove gets invoked multiple
%times upon the same \Preprove results,
%again even with the additional proof-of-knowledge.
\noindent \textbf{Perfect Completeness for $\Preprove$} For every $(\bary, \barx; \baromega_1) \in \relone$ it holds:
\begin{align*}
\mathit{Pr} (& \ZKCont.\Verify(\crsvk, \bar{y}, X, \pi, \relone) = 1 \ \wedge \ \ZKCont.\VerifyCom (\pp, X, \barx, b) = 1\  | \ \\ 
                   & (\crs, \pp) \leftarrow \ZKCont.\Setup (1^{\lambda}), (\crspk, \crsvk) \leftarrow \ZKCont.\Gen(\crs, \relone), \\ 
                   & (X, \pi, b) \leftarrow \ZKCont.\Preprove(\crspk, \bar{y}, \bar{x}, \baromega_1, \relone)) = 1
\end{align*}

\noindent \textbf{Perfect Completeness for $\Reprove$} For every efficient adversary $A$ it holds: 
\begin{align*}
\mathit{Pr} (& (\ZKCont.\Verify(\crsvk, \bar{y}, X, \pi, \relone) = 1  = >  \ZKCont.\Verify(\crsvk, \bar{y}, X', \pi', \relone) = 1)  \ \wedge \  \\
                   & \wedge \ (\ZKCont.\VerifyCom (\pp, X, \barx, b) = 1 => \ZKCont.\VerifyCom (\pp, X', \barx, b') = 1) \ | \\
                   & (\crs, \pp) \leftarrow \ZKCont.\Setup (1^{\lambda}), (\crspk, \crsvk) \leftarrow \ZKCont.\Gen(\crs, \relone), \\ 
                   & (\bary, \barx, X, \pi, b) \leftarrow A(\crs,\pp, \relone) \\
                   & (X', \pi', b') \leftarrow \ZKCont.\Reprove(\crspk, X, \pi, b, \relone)) = 1
\end{align*}
\noindent \textbf{Knowledge Soundness} For every benign auxiliary input $\realaux$ (as per~\cite{bening_auxiliary}) and 
every non-uniform efficient adversary $A$, there exists efficient non-uniform extractor $E$  
\begin{align*}
\mathit{Pr} (& (\ZKCont.\Verify(\crsvk, \bar{y}, X, \pi, \relone) = 1) \ \wedge \ (\ZKCont.\VerifyCom(\pp, X, \bar{x}, b) = 1) \ \wedge\ \\
                   & \wedge \ ( (\bary, \barx; \baromega_1) \notin \relone) \ | \ (\crs, \pp) \leftarrow \ZKCont.\Setup (1^{\lambda}), \\
                   & (\bary, \barx, X, \pi, b; \baromega_1 ) \leftarrow A || E (\crs, \realaux, \relone)) = \negl (\lambda),
\end{align*}
\noindent %where $\ZKCont.\Preprove_{|X}(\crspk, \bar{y}, \bar{x}, \baromega_1, \relone; b)$ means running the part of algorithm 
%$\ZKCont.\Preprove$ that computes and outputs $X$ with its regular inputs and using $b$ when randomness is required; 
where by $(\mathit{output_{A}};\mathit{output_{B}}) \leftarrow A || B(\mathit{input})$ we denote algorithms $A$, $B$  running on the same 
$\mathit{input}$ and $B$ having access to the random coins of $A$. \\

\noindent \textbf{Perfect Zero-knowledge w.r.t. $\relone$} For all $\lambda \in \mathbb{N}$, for every benign auxiliary input $\realaux$, 
for all  $(\bary, \barx; \baromega_1) \in \relone$, for all $X$, for all $\pi$, for all $b$, for every adversary $A$ it holds:
\begin{align*}
\mathit{Pr}(& A(\crs, \realaux, \pi', X', \relone) = 1 \ | \ (\crs, \pp) \leftarrow \ZKCont.\Setup (1^{\lambda}), \\
                  & (\crspk, \crsvk) \leftarrow \ZKCont.\Gen(\crs, \relone), \\ 
                  & (\pi', X', \_) \leftarrow \ZKCont.\Reprove (\crspk, X, \pi, b, \relone), \\
                  &  \ZKCont.\Verify(\crsvk, \bary, X, \pi, \relone) = 1) =  \\
= \mathit{Pr}(& A(\crs, \realaux, \pi', X', \relone) = 1 \ | \ (\crs, \pp) \leftarrow \ZKCont.\Setup (1^{\lambda}), \\ 
                     & (\crspk, \crsvk) \leftarrow \ZKCont.\Gen(\crs, \relone), (\pi', X') \leftarrow \ZKCont.\Sim(\tw, \bary, \relone) \\ 
                     &  \ZKCont.\Verify(\crsvk, \bary, X, \pi, \relone) = 1)
\end{align*}
 
\end{definition} 

% $$ \Lring = \Setst{ \compk, \comring }{
%  \exists \openpk,\openring \textrm{\ s.t.\ } 
%  \genfrac{}{}{0pt}{}{\PedVRF.\OpenKey(\compk,\openpk) \quad}{\,\, = \rVRF.\OpenRing(\comring,\openring)}
% } \mathperiod $$

% \smallskip
\subsection{Specialised Groth16}
\label{sec:rvrf_groth16}

Below we instantiate our zero-knowledge continuation notion with a scheme based on Groth16~\cite{Groth16} SNARK;
hence, we call our instantiation \emph{specialised Groth16} or \emph{$\mathsf{SpecialG}$}. In order to do that, we need a 
reminder of the definition of Quadratic Arithmetic Program (QAP)~\cite{LegoSNARK}, ~\cite{GGPR13}.

\begin{definition}[QAP] 
\label{def:QAP}
A Quadratic Arithmetic Program (QAP) $\cQ = (\cA, \cB, \cC, t(X))$ of size $m$ 
and degree $d$ over a finite field $\F_q$ is defined by three sets of polynomials $\cA = \{a_i(X)\}_{i=0}^m$, 
$\cB = \{b_i(X)\}_{i=0}^m$, $\cC = \{c_i(X)\}_{i=0}^m$ of degree less than $d-1$ and a target degree $d$ polynomial $t(X)$. Given 
$\cQ$ we define $\relRQ$ as the set of pairs $((\bary, \barx); \baromega) \in \F_q^{l} \times \F_q^{n-l} \times \F_q^{m-n}$ for which it 
holds that there exist a polynomial $h(X)$ of degree at most $d-2$ such that:
$$(\sum_{k=0}^m v_k \cdot a_k(X)) \cdot (\sum_{k=0}^m v_k \cdot b_k(X)) = (\sum_{k=0}^m v_k \cdot c_k(X)) + h(X)t(X) \ \ (\ast)$$ 
where $\barv = (v_0, \ldots, v_m) = (1, x_1, \ldots, x_n, w_1, \ldots w_{m-n})$ and $\bary = (x_1, \ldots, x_l)$ and 
$\barx = (x_{l+1}, \ldots, x_n)$ and $\baromega = (w_1, \ldots, w_{m-n})$. 
\end{definition}

\noindent Given notation provided in section~\ref{sec:background}, we introduce
%Let $\mathbb{F}_q$ be a prime field, 
%let $G_1$, $G_2$, $G_T$ be as defined in section~\ref{??}, let $e$, $g$, $h$ be defined as $\ldots$. Let $t(X)$ and
%$\{u_i(X),v_i(X),w_i(X)\}_{i=0}^m$ be polynomials in $\F_q[X]$, let $\ldots$ be $\ldots$ such that there exists $h(X) \in \F_q[X]$ with
% $$ \sum_{i=0}^m a_iu_i(X) \cdot  \sum_{i=0}^m a_iv_i(X)  = \sum_{i=0}^m a_iw_i(X) + h(X)t(X)  \ (\ast)$$
%Then let $\relone = \{ (;) \ | \ (;)  (\ast) \}$

\begin{definition}[Specialised Groth16 ($\SpecialG$)]
\label{insta:sg16} 
Specialised Groth16 for relation $\relRQ$ is the following instantiation of the zero-knowledge continuation notion from Definition~\ref{def:zk_cont}:
\begin{itemize}
\item $\SpecialG.\Setup: (1^{\lambda}) \mapsto (\crs, \tw)$. \\ 
\noindent Let $\alpha, \beta, \gamma, \delta, \tau, \eta  \xleftarrow{\$} \F_q^{*}$. Let $\tw = (\alpha, \beta, \gamma, \delta, \tau, \eta)$. \\ 
Let $\crs = ([\barsig_1]_1, [\barsig_2]_2)$ where 
\begin{align*}
\barsig_1 = (&\alpha, \beta, \delta, \{\tau_i\}_{i=0}^{d-1}, \left\{\frac{\beta a_i(\tau)+ \alpha b_i(\tau)+ c_i(\tau)}{\gamma}\right\}_{i=1}^n,  
\frac{\eta}{\gamma}, \\ 
&\left\{\frac{\beta a_i(\tau)+ \alpha b_i(\tau)+ c_i(\tau)}{\delta} \right\}_{i=n+1}^m, \left\{\frac{1}{\delta}\sigma^it(\sigma) \right\}_{i=0}^{d-2}, 
\frac{\eta}{\delta}), \\
\barsig_2 = (&\beta, \gamma, \delta, \{\tau^i\}_{i=0}^{d-1}). 
\end{align*} 

\noindent Moreover, for simplicity and later use, we call $\Kgamma = \left[\frac{\eta}{\gamma}\right]_1$  and $\Kdelta = \left[ \frac{\eta}{\delta}\right] _1$.

\item $\SpecialG.\Gen: (\crs, \relRQ) \mapsto (\pp, \crspk, \crsvk)$ where \\
$\crspk = \left([\barsig_1]_1, [\beta]_2, [\delta]_2, \left\{[\tau^i]_2\right\}_{i=0}^{d-1}\right)$ and \\ 
$\crsvk = \left([\alpha]_1, \left\{ \left[ \frac{\beta a_i(\tau)+ \alpha b_i(\tau)+ c_i(\tau)}{\gamma} \right]_1 \right\}_{i=1}^{l}, 
[\beta]_2, [\gamma]_2, [\delta]_2\right)$ and \\ 
$\pp = \left( \left \{ \left[ \frac{\beta a_i(\tau)+ \alpha b_i(\tau)+ c_i(\tau)}{\gamma} \right]_1 \right \}_{i=l+1}^{n}, \left[ \frac{\eta}{\gamma} \right]_1 \right)$.


\item $\SpecialG.\Preprove: (\crspk, \bar{y}, \bar{x}, \baromega_1, \relRQ) \mapsto (X, \pi, b)$ such that \\
\begin{align*}
&b = 0; r, s\xleftarrow{\$} \F_p; X = \sum_{i=l+1}^{n} v_i\left[ \frac{\beta a_i(\tau)+ \alpha b_i(\tau)+ c_i(\tau)}{\gamma} \right]_1; \pi = ([A]_1, [B]_2, [C]_1); \\
&A = \alpha + \sum_{i=0}^{m} v_i \cdot a_i(\tau) + r\delta; B = \beta + \sum_{i=0}^{m} v_i \cdot b_i(\tau) + s\delta; \\ 
&C = \frac{\sum_{i=n+1}^{m} v_i \beta a_i(\tau)+ \alpha b_i(\tau)+ c_i(\tau) + h(\tau)t(\tau)}{\delta}   + As + Br - rs\delta, 
\end{align*}
and where $\bary = (x_1, \ldots, x_l)$, $\barx = (x_{l+1}, \ldots, x_n)$, $\baromega = (w_1, \ldots, w_{m-n})$, \\
$\barv = (1, x_1, \ldots, x_n, w_1, \ldots, w_{m-n})$ (same as in Definition~\ref{def:QAP}).


\item $\SpecialG.\Reprove: (\crspk, X, \pi, b, \relRQ) \mapsto (X', \pi', b')$  such that
\begin{align*}
&b ', r_1, r_2  \xleftarrow{\$} \F_p, X' = X + (b'- b) \Kgamma, \pi' = (U', V', W'), \\
&U' = \frac{1}{r_1} U, V' = r_1 V + r_1r_2[\delta]_2, W' = W + r_2U  - (b' - b) \Kdelta \mathperiod
\end{align*}
\noindent where $\pi = (U, V, W)$.
 
\item $\SpecialG.\VerifyCom: (\pp, X, \barx, b) \mapsto 0/1$ where the output is 1 iff the following holds
$$X = \sum_{i=l+1}^{n} x_i\left[ \frac{\beta a_i(\tau)+ \alpha b_i(\tau)+ c_i(\tau)}{\gamma} \right]_1  + b \Kgamma,$$
where $\barx = (x_{l+1}, \ldots, x_n)$, $ 0 \leq l \leq n-1$. 
\item $\ZKCont.\Verify: (\crsvk, \bar{y}, X', \pi', \relRQ) \mapsto 0/1$ where the output is 1 iff the following holds 
$$e(U',V') = e([\alpha]_1, [\beta]_2) \cdot e(X' + Y, [\gamma]_2) \cdot e(W', [\delta]_2),$$
where $\pi' = (U', V', W')$, $Y = \sum_{i=1}^{l} x_i\left[ \frac{\beta a_i(\tau)+ \alpha b_i(\tau)+ c_i(\tau)}{\gamma} \right]_1$ 
and $\bary = (x_1, \ldots, x_l)$.

\item $\SpecialG.\Sim: (\tw, \bary, \relRQ) \mapsto (\pi', X')$ where $u, A', B' \xleftarrow {\$} \F_p$ and let \\
$\pi' = ([A']_1, [B']_2, [C']_1)$ where $C' = \frac{A'B' - \alpha \beta - \sum_{i=1}^{l} x_i (\beta a_i(\tau)+ \alpha b_i(\tau)+ c_i(\tau))- u}{\delta}  $ and, 
by definition $\bary = (x_1, \ldots, x_l)$. Note that $\pi$ is a simulated proof for transparent input $\bary$ and commitment $X' = [u]_1$.
\end{itemize} 
\end{definition}

\noindent \paragraph{Notes:} First, the trusted setup required by \SpecialG is 
an extension of that required by original Groth16~\cite{Groth16} by two additional 
group elements $\Kgamma = [\frac{\eta}{\gamma}]_1$ and $\Kdelta = [\frac{\eta}{\delta}]_1$. 
An identical trusted setup to that used by \SpecialG was used in LegoSNARK~\cite[Fig.~22]{LegoSNARK} which defines 
a commit-carrying SNARK based on Groth16. Second, our $\SpecialG.\Reprove$ algorithm uses a Groth16 re-randomisation 
technique for the proof (see~\cite[Fig.~1]{RandomizationGroth16} or LegoSNARK~\cite[Fig.~22]{LegoSNARK}), 
but, in addition, $\SpecialG.\Reprove$ also re-randomises $X$ which is a commitment to a slice of the public input; moreover, in terms of security 
properties, we appropriately define the zero-knowledge for zk continuations such that even after iteratively applying 
$\SpecialG.\Reprove$ zero-knowledge property is preserved for both the witness as well as the public input committed to in $X$.  \\

%{\color{red}Note that, if a trusted setup is used (for example the one described in~\cite{subversion_zk}) such that there 
%exist a public and efficient procedure for verifying it, then, by extending it with $\Kgamma$ and $\Kdelta$ (which is our 
%extension of the standard Groth16 trusted setup), the resulting setup remains publicly verifiable (i.e., by using the additional 
%verification equation $e(\Kgamma, [\gamma]_2) = e(\Kdelta, [\delta]_2)$), and, hence, according to~\cite{subversion_zk}, 
%subversion zero-knowledge. }
%From page 3 of ~\cite{subversion_zk}"We change Groth?s zk-SNARK by adding extra elements to the CRS so that the CRS will become 
%publicly verifiable; this minimal step (clearly, some public verifiability of the CRS is needed in the case the CRS generator 
%cannot be trusted) will be sufficient to obtain Sub-ZK. However, choosing which elements to add to the CRS is not straightforward 
%since the zk-SNARK must remain knowledge-sound even given enlarged CRS; adding too many (or just ?wrong?) elements to the 
%CRS can break the knowledge-soundness."

\noindent Finally, we are ready to prove the following result:

\begin{theorem}
\label{sec_specialg}
Let $\relRQ$ be such that $\{a_k(X)\}_{k=0}^n$ are linearly independent polynomials. Then, in the 
AGM~\cite{Fuchs_AGM}, $\SpecialG$ is a zero-knowledge continuation as per definition~\ref{def:zk_cont}. 
\end{theorem}
\begin{proof} It is straightforward to prove that $\SpecialG$ has completeness for $\Preprove$ and for $\Reprove$. \\

\noindent We prove knowledge-soundness (KS) an in Definition~\ref{def:zk_cont} by first arguing $\SpecialG$ is a 
commit-carrying SNARK with double binding (cc-SNARK with double binding) as per Definition 3.4~\cite{LegoSNARK}. 
We use the fact that $\ccgroth$ as defined by the NILP detailed in Fig.22, Appendix H.5~\cite{LegoSNARK} satisfies that latter definition. Moreover, 
$\SpecialG$'s $\Setup$ together with $\Gen$ and $\ccgroth$'s $\mathit{KeyGen}$ are the same procedure. Also $\SpecialG$ 
and $\ccgroth$ share the same verification algorithm. Hence, translating the notation appropriately, $\SpecialG$ also satisfies KS 
for a cc-SNARK with double binding. \\

% we first argue that We prove knowledge-soundness by reducing it to the knowledge-soundness property of the Groth16 
%commit-carrying with double binding scheme (for short $\ccgroth$). This knowledge-soundness property has been formalised in 
%Definition 3.4~\cite{LegoSNARK} and the NILP corresponding to $\ccgroth$ has been detailed in Fig.22, Appendix H.5~\cite{LegoSNARK}. 
%Indeed, $\SpecialG$'s $\Setup$ together with $\Gen$ and $\ccgroth$'s $\mathit{KeyGen}$ are the same procedure. Moreover $\SpecialG$ 
%and $\ccgroth$ share the same verification algorithm. Since $\ccgroth$ satisfies the definition of a cc-SNARK with double binding, translating 
%the notation appropriately, $\SpecialG$ also satisfies the knowledge soundness properties for a cc-SNARK with double binding. 
\noindent Let $A_{\SpecialG}$ be an adversary for KS in Definition~\ref{def:zk_cont} and 
define adversary $A_{\ccgroth}$ for KS  in Definition 3.4~\cite{LegoSNARK}:
\begin{align*}
&\mathit{If} \ (\bary, \barx, X, \pi, b) \leftarrow A_{\SpecialG} (\crs, \pp, \realaux, \relRQ)\  (1)\ \mathit{then} \ (\bary, X, \pi) \leftarrow A_{\ccgroth} (\relRQ, \crs, \realaux).
\end{align*}

\noindent Given extractor $E_{\ccgroth}$ for $A_{\ccgroth}$ fulfilling Definition 3.4~\cite{LegoSNARK}, 
construct extractor $E_{\ccgroth}$ for $A_{\ccgroth}$
\begin{align*}
&\mathit{If} \ (\barx^*, b^*, \baromega^*) \leftarrow E_{\ccgroth} (\relRQ, \crs, \realaux)\ (2)\ \mathit{then} \ \baromega^* \leftarrow E_{\SpecialG} (\relRQ, \crs, \realaux).
\end{align*}
We show $E_{\SpecialG}$ fulfils Definition~\ref{def:zk_cont} for $A_{\SpecialG}$. Assume by contradiction that is not the case. 
This implies there exist auxiliary input $\realaux$ such that each: 
\begin{align*}
\ZKCont.\Verify(\crsvk, \bary, X, \pi, &\relRQ) =1 \ (10)\ ; \ \ZKCont.\VerifyCom(\pp, X, \barx, b) =1 \ (20) \\
& (\bary, \barx; \baromega) \notin \relRQ  \ \ (30) 
\end{align*}
hold with non-negligible probability. Since $(20)$ holds with non-negligible probability and verification is identical for 
$\SpecialG$ and $\ccgroth$, and since $E_{\ccgroth}$ is an extractor for $A_{\ccgroth}$ as per Definition 3.4~\cite{LegoSNARK},
 then each of the two events 
\begin{align*}
\mathit{VerCommit^*}(\pp, X, \barx^*, b^*) =1 \ (40) \ ; \ (\bary, \barx^*; \baromega^*) \in  \relRQ \ (50)
\end{align*}
holds with overwhelming probability. Since $(20)$ holds with non-negligible probability and $(40)$ holds with overwhelming probability and 
together with (ii) from Definition 3.4~\cite{LegoSNARK} we obtain that $\barx^* = \barx$. Since $(50)$ holds with overwhelming probability, it implies 
$(\bary, \barx; \baromega^*) \in \relRQ $ with overwhelming probability which contradicts our assumption, so our claim that $\SpecialG$ does not have 
KS as per Definition~\ref{def:zk_cont} is false. \\

\noindent Finally, regarding zero-knowledge, it is clear that if $\pi' = (U', V', W')$ is part of the output of $\SpecialG.\Reprove$, 
then $U'$ and $V'$ are uniformly distributed as group elements in their respective groups. This holds, as long as the 
input to $\SpecialG.\Reprove$ is a verifying proof, even when the proof was maliciously generated. Hence, it is easy to check  
that the output $\pi'$ of $\SpecialG.\Sim$ is identically distributed to a proof $\pi'$ output by $\SpecialG.\Reprove$ so the perfect 
zero-knowledge property holds for $\SpecialG$. 
\end{proof}

\subsection{Putting Together a NIZK and a $\ZKCont$  for Proving $\rel$} \label{sec:nizkR}

%TODO Check consistency with the preliminaries section
%$$\rel = \{(\bary, \barz; \barx, \baromega_1, \baromega_2):  (\bary, \barx; \baromega_1) \in \relone, (\barz, \barx; \baromega_2) \in \reltwo \},$$
Let $\ZKCont_{\relone}$ be a zk continuation for $\relone$ (from preamble of this section) with public parameter $ \pp $ and 
let $\nizktwo$ be a NIZK for $\reltwo'$ defined as
$$\reltwo' = \{(X, \barz; \barx, b, \baromega_2): \ZKCont_{\relone}.\VerifyCom(\pp, X, \barx, b) =1\ 
\wedge \ (\barz, \barx; \baromega_2) \in \reltwo \},$$ 
with $\reltwo$ from preamble of Section~\ref{sec:rvrf_cont}. Then we define the system $\nizkR$ for relation $\mathcal{R}$ 
from the preamble of this section as:
\begin{itemize}
\item $\nizkR.\Setup(1^{\lambda})\rightarrow (\crsR = (\crs,\crstwo), \twR = (\tw, \twtwo), \pp_\rel = \pp)$: Here,
$(\crs, \tw, \pp) \leftarrow \ZKCont_{\relone}.\Setup(1^{\lambda}, \relone)$, $(\crstwo, \twtwo) \leftarrow \nizktwo.\Setup(1^{\lambda})$.

%\item $\nizkR.\Gen: (\crsR) \mapsto (\ppR = \pp, \crspkR = (\crspk,\crspktwo), \crsvkR = (\crsvk,\crsvktwo))$ where 
%$(\crspk, \crsvk, \pp) \leftarrow \ZKCont_{\relone}.\Gen(\crs, \relone)$, \\ $(\crspktwo, \crsvktwo) \leftarrow \nizktwo.\Gen(\crstwo)$ 

\item $\nizkR.\Prove(\crsR,\bary, \barz; \barx, \baromega_1, \baromega_2 ) \rightarrow (\pi_1, \pi_2, X)$: Here 
$(X', \pi'_1, b')$ is generated by $\ZKCont_{\relone}.\Preprove(\crs, \bar{y}, \bar{x}, \baromega_1)$ and 
$(X, \pi_1, b)$ generated by  $\ZKCont_{\relone}.\Reprove(\crs, X', \pi_1', b')$ and $ \pi_2 $ is generated by
$ \nizktwo.\Prove(\crstwo, X, \barz; \barx, b, \baromega_2)$. 

\item $\nizkR.\Verify(\crsR,(\bary, \barz), (\pi_1, \pi_2, X)) \rightarrow 0/1$: It outputs $1$ iff 
$\ZKCont_{\relone}.\Verify(\crs, \bar{y}, X, \pi_1) = 1 $ and $ \nizktwo.\Verify(\crstwo, X, \barz, \pi_2) =1.$

\item $\nizkR.\Sim: (\twR, \bary, \barz) \mapsto (\pi_1, \pi_2, X)$ where 
$(\pi_1, X) \leftarrow \ZKCont_{\relone}.\Sim(\tw, \bary)$, $\pi_2 \leftarrow \nizktwo.\Sim(\twtwo, X, \barz)$.
 \end{itemize}
%\vspace{-0.5cm}

 
\begin{theorem}
	If $\ZKCont_{\relone}$ is a zk continuation for $\relone$ and $\nizktwo$ is a NIZK for $\reltwo'$ for some appropriately chosen public parameters $\pp$, 
	then the $\nizkR$ construction described above is a NIZK for $\rel$.
\end{theorem}

\textit{Proof sketch:} The correctness, knowledge soundness and zk properties of $ \nizkR $ comes from the same properties of $ \ZKCont_{\relone} $ and $ \nizktwo $. See Appendix \ref{ap:nizkR} for the proof.

\subsection{Our Ring VRF Construction based on SpecialG}

\label{subsec:rvrf_faster}
We modify slightly the signing and verification algorithms of our ring VRF construction in Section \ref{sec:pederson_vrf} by deploying $ \SpecialG $. 
Our modified construction enjoys the rerandomization properties of $ \SpecialG $ which lets a signer sign a 
different $ \msg $ for the same ring without running the prove algorithm of a NIZK related to showing the key is in the ring.  
\begin{itemize}
	\item $ \rVRF.\Setup(1^\secparam)   $ outputs $ \pprvrf = (crs, \pp = (p, \grE,\genG, \genB), \F_p) $ where $ \pp $ is the output of $ \SpecialG.\Setup(1^\secparam, \relinner) $ and $\relinner$ is defined below.
	\item $\rVRF.\rSign : ((\sk,r),\comring, \openring,\msg,\aux) \mapsto \sigma$ generates $ \PreOut = \sk H_\grE(\msg) $, lets $ \Out = H(\msg, \PreOut) $ and then runs $ \NIZK_{\relrvrf}.\Prove(\comring,\PreOut, \msg, \Out; \sk,r,\openring)\rightarrow \pi_{rvrf}$ where
	
	$$ \relrvrf = \Setst{
		\begin{aligned}
			&\comring, (\PreOut,\msg, \Out); \\
			&\sk, \openring
		\end{aligned}
	}{
		\begin{aligned}
			& (\comring,\sk;r,\openring) \in \relinner \wedge \\
			& (\PreOut, \msg, \Out; \sk) \in \relout
	\end{aligned}	}$$   
	
	and 
	
	$$ \relinner = \Setst{ (\comring, \sk ; r,\openring) }{
		\begin{aligned}
			&	\pk = \OpenRing(\comring,\openring) \wedge \\
			& 	\sk = \mathsf{Com}.\mathsf{Open}(\pk,r) 
		\end{aligned}	
	},$$
	
	$$\relout = \{(\PreOut, \msg,\Out; \sk): \PreOut = \sk H_\grE(\msg) \wedge \Out = H(\msg,\PreOut)\}$$
	
	
	We instantiate $ \NIZK_{\relrvrf}.\Prove(\comring,\PreOut, \msg, \Out; \sk,r,\openring) $ as described in Section \ref{sec:nizkR} where $ \relone = \relinner $ and $ \reltwo'(pp) = \rel_{eval} $ and $\baseR = \relrvrf$ . It works as follows: It runs $ \SpecialG.\Preprove(crs, \comring,\sk,(r,\openring),\relinner) $ and obtains $ \compk', \pi' , \openpk' = 0$. Then, it runs $ \SpecialG.\Reprove(crs, X',\pi',\openpk',\relinner) $ and obtains $ (\compk, \pi_1, \openpk) $. Finally, it lets $ \PreOut = \sk H_\grE(\msg) $ and runs $ \NIZK_{\rel_{eval}}.\Prove(\compk, \PreOut, \msg;\sk,\openpk,\msg) \rightarrow \pi_2 $ as described in Section $ \ref{sec:pederson_vrf} $ with $\aux' = \tmpaux$ .	
	In the end, it returns the ring signature $ \sigma = (\PreOut,  \pi_1,\pi_2, \compk) $.
	
	\item  $\rVRF.\rVerify : (\comring,\msg,\aux,\sigma) \mapsto \Out \,\, \lor \perp$ \,
	it parses $\sigma$ as $\PreOut,  \pi_1,\pi_2, \compk$ and runs  $\NIZK_{\relrvrf}.\Verify(\comring, \PreOut, \msg, \Out; \pi_{rvrf})$ i.e., runs $ \SpecialG.\Verify(crs, \comring, \compk,\pi_1,\relinner) $ and $ \NIZK_{\rel_{eval}}.\Verify((\compk, (\PreOut, \msg, \Out)), \pi_2) $. If all verify, it outputs $ \Out = H(\msg,\PreOut) $. Otherwise, it returns 0.
\end{itemize}

\begin{theorem}\label{thm:rvrfspecial}
	Our specialized $ \rVRF $  over the group structure $ (\grE,p,\genG,\genB) $ realizes $ \fgvrf $ in Figure \ref{f:gvrf} in the random oracle model assuming that $ \NIZK_{\relrvrf} $ is zero-knowledge and knowledge sound, the decisional Diffie-Hellman (DDH) problem are hard in $ \grE  $ and the commitment scheme $ \mathsf{Com} $ is binding and perfectly hiding. 
\end{theorem}
%TODO proof sketch

We now present an appropriate $ \mathsf{Com}.\mathsf{Commit}(\sk) $ algorithm that together with $ \SpecialG $ efficiently instantiate the NIZK for $ \Rring^{\mathtt{inner}} $. This works on the Jubjub curve $\ecJ$ which contains a large subgroup $\grJ$ of prime order $p_\grJ$. Here, $p_\grJ < p$ where $ p $ is the order of $\grE$ used in our ring VRF construction\footnote{This condition can be satisfied if $\ecJ$ is an Edwards curve with a cofactor.}. We let $\genJ_0,\genJ_1,\genJ_2 \in \grJ$ be independent generators. We also fix a parameter $ \kappa $ where $(\log_2 p)/2 < \kappa < \log_2 p_\grJ$. $ \mathsf{Com}.\mathsf{Commit}(\sk) $ first samples $\sk_1,\sk_2 \in 2^\kappa$  where $\sk = \sk_0 + \sk_1 \, 2^{\lambda} \mod p$ and samples a blinding factor $d \leftsample \F_{p_\grJ} $. In the end, it outputs $ \sk_0, \sk_1,d $ as an opening and the commitment $\pk=\sk_0\, \genJ_0 + \sk_1\, \genJ_1 + d \genJ_2$ as a public key of our ring VRF construction. This commitment scheme is binding and perfectly hiding as our ring VRF construction requires because $ \pk $ is, in fact, a Pedersen commitment. Indeed, $\pk$ is a Pedersen commitment to $\sk$ because we can represent $ \sk = \sk_0\, \genJ_0 + \sk_1 \mod p$ since we have selected $ \kappa $ accordingly.

%TODO MAKE THIS CONSISTENT WITH THE DESCRIPTION SINCE WE DEFINE EVERYTHING ON ONE GROUP
\paragraph{Efficiency:} If we have a $\SpecialG$ proof for $\relinner$ for our $\pk$ in a ring defined by $\comring$, 
to generate a ring VRF proof for the same ring, we need to run $\SpecialG.\Reprove$ and $\NIZK_{\mathcal{R}_{eval}}.\Prove$. $\NIZK_{\mathcal{R}_{eval}}.\Prove$ 
requires two scalar multiplications on $\grE_1$
and two on the same or faster $\grE'$,
so together with $\SpecialG.\Reprove$ costing four scalar multiplications
on $\grE_1$ and two on $\grE_2$, our amortised prover time
runs faster than 12 scalar multiplications on typical $\grE_1$ curves. 
We expect the three pairings dominate verifier time, but
verifiers also need five scalar multiplications on $\grE_1$.

%TODO WHAT DOES IT MEAN
Importantly, our fast ring VRF's amortised prover time now rivals
group signature schemes' performance \cite{group_sig_survey}.
We hope this ends the temptation to deploy group signature like
 constructions where the deanonymisation vectors matter. 


\begin{comment}
% TODO \PedVRF.\OpenKey(\compk,\openpk)

\def\longeq{=\mathrel{\mkern-10mu}=}% {=\joinrel=} % https://tex.stackexchange.com/questions/35404/is-there-a-wider-equal-sign
We describe a much faster choice \pifast for \piring with
opaque inputs $x_1 \longeq \sk$ and transparent inputs $y_1 \longeq \comring$
 so that taking
 $\genG \longeq \chi_1$, $\genB \longeq \genB_\gamma$, and $\openpk \longeq b$
in \PedVRF yields an incredibly fast amortised ring VRF prover.
Also \PedVRF itself proves knowledge of $X' =  \sk\, \chi_1 + b \genB_\gamma $,
 as required by $\SpecialG.\Verify$.
% $$ X' + Y = \comring\, \Upsilon_1 + \sk\, \chi_1 + b \genB_\gamma $$

A priori, we do not know $\chi_1$ during the trusted setup for $\pifast$,
which prevents computing $\pk = \sk\, \chi_1$ inside $\pifast$.
Instead, we propose $\ring$ contain commitments to $\sk$ over
some Jubjub curve $\ecJ$, while $\sk \in \F_p$ remains a scale for $\grJ$.

We know the large subgroup $\grJ$ of $\ecJ$ typically has smaller prime
order $p_\grJ$ than $\grE$, itself due to $\ecJ$ being an Edwards curve.
%
We thus choose $\sk_0,\sk_1 < p_\grJ$ with at least $\lambda$ bits
so that
 $\PedVRF.\sk = \sk_0 + \sk_1 \, 2^{\lambda} \mod p_\grE$
becomes our secret key.
\footnote{If $\lambda \approx 128$ then $p, p_\grJ > 2^{2\lambda-3}$.}
Our $\rVRF.\KeyGen$ \eprint{returns}{shall now return}
a secret key of the form $\rVRF.\sk = (\sk_0,\sk_1,d)$
 with $d \leftsample \F_{p_\grJ}$ and
a public key of the form
 $\rVRF.\pk = \sk_0\, \genJ_0 + \sk_1\, \genJ_1 + d \genJ_2$,
for some independent $\genJ_0,\genJ_1,\genJ_2 \in \grJ$. % (see \S\ref{subsec:AML_KYC}).
\footnote{Interestingly we avoid range proofs for $\sk_1$ and $\sk_2$ by this independence.}
We thus have a fairly efficient instantiation for $\Lring^\inner$ give by

$$ \Lfast^\inner = \Setst{ \sk_0 + 2^{128} \sk_1, \comring }{
 \eprint{ \exists d,\openring \textrm{\ s.t.\ } }{}
 % 0 < \sk_0,\sk_1 < 2^{128} \textrm{\ and\ } 
 \genfrac{}{}{0pt}{}{ \eprint{\rVRF.}{}\OpenRing(\comring,\openring) }{ \,\, = \sk_0 \genJ_0 + \sk_1 \genJ_1 + d \genJ_2 }
} \mathperiod $$

Applying our rerandomization \Reprove to $\pifast^\inner$ with opaque input
yields a zkSNARK $\pifast$ with the extra $\PedVRF.\OpenKey$ arithmetic to
have exactly the form $\piring$.

We explain later in \S\ref{sec:ring_hiding} how one could
choose $\chi_1$ independent before doing the trusted setup,
 and then wire $\chi_1$ into $\pifast$ inside $C$.
We could then prove $\pk = \sk\, \chi_1$ directly inside $\pifast^\inner$,
but doing so here requires slow non-native field arithmetic.

At this point, $\PedVRF.\Sign$ requires two scalar multiplications on $\ecE_1$
 and two on the somewhat faster $\ecE'$,
so together with rerandomization costing four scalar multiplications
on $\ecE_1$ and two on $\ecE_2$, our amortized prover time
 runs faster than 12 scalar multiplications on typical $\ecE_1$ curves. 
We expect the three pairings dominate verifier time, but
 verifiers also need five scalar multiplications on $\ecE_1$.

As an aside, one could construct a second faster curve with the same
group order as $\grE$, which speeds up two scalar multiplications
 in both the prover and verifier. 

Importantly, our fast ring VRF's amortised prover time now rivals
group signature schemes' performance \cite{group_sig_survey,}.
We hope this ends the temptation to deploy group signature like
 constructions where the deanonymization vectors matter.

% BEGIN TODO: Oana

\begin{theorem}\label{thm:knowledge_soundness}
\rVRF instantiated with \pifast and \PedVRF satisfies knowledge soundness.
\end{theorem}

\begin{proof}[Proof stetch.]
An extractor for \PedVRF reveals the opening of $X$ for us,
so our result follows from Lemma \ref{lem:knowledge_soundness}.
\end{proof}

% \begin{corollary}\label{cor:???}
% Our Pedersen ring VRF instantiated with \pifast satisfies ring unforgability and uniqueness.
% \end{corollary}

% \begin{theorem}\label{thm:pifast_anonymity}
% \rVRF instantiated with \pifast and \PedVRF satisfies zero-knowledge.
% \end{theorem}
%
% \begin{proof}[Proof stetch.]
% Assuming the same \comring, we know the zero-knowledge continuations
% are identically distributed by Lemma \ref{lem:unlinkable},
% even when reusing a zero-knowledge continuation $(X,A,B,C)$.
% It follows the typical simulator for \PedVRF ... WHAT???
% \end{proof}

% \begin{corollary}\label{cor:???}
% Our Pedersen ring VRF instantiated with \pifast satisfies ring anonymity.
% \end{corollary}

% END TODO: Oana

\end{comment}

%
%%\section{Ring updates}
\label{sec:ring_updates}

We discuss \pifast, or \pisk and \pipk, representing public keys
 in $\grE$ in $\grJ$ already,
% along with circuit implementation details of $\PedVRF.\{ \CommitKey, \OpenKey \}$,
but otherwise mostly treated the ring commitment scheme
$\rVRF.\{ \CommitRing, \CommitKey, \OpenKey \}$ like a black box.

Although our $\rVRF.\rSign$ runs fast, all users must update their
stored zkSNARK \pipk every time the ring $\ctx$ changes.
Almost any circuit works for \pipk though,
 which permits diverse optimizations depending upon usecase.


\subsection{Merkle trees} % {Poseidon}

Our $\rVRF.\{ \CommitRing, \CommitKey, \OpenKey \}$ could implement a
Merkle tree using zkSNARK friendly hash functions like Poseidon \cite{poseidon}.
%
All users need $O(\log |\ctx|)$ data with every update, which sounds
reasonable but not free.  There is a fast moving literature on securing
and optimizing zkSNARK friendly hash functions, with different techniques
being better suited to different zkSNARKs or even curves.

TODO: Arity 9 for 300 constraints?   % \cite{Groth16} vs plookup \cite{plookup}.

We leave deeper discussion of zkSNARK friendly Merkle to the literature.
Instead we spend this section focusing upon the diversity of circuit
designs that fit our framework.


\subsection{Vector commitments}

As noted in \S\ref{subsec:rvrf_side_channel}, our zkSNARK \pipk could use
polynomial based vector commitments \cite{KZG}. % or so called ``Verkle trees'' \cite{??Verkle??} too.
We need hidden opening locations of the sort discussed in Caulk \cite{caulk} and Caulk+ \cite{caulk+}.

TODO: Discuss Caulk

Interesstingly, we might avoid the extra proof-of-knowledge added
in \S\ref{subsec:rvrf_side_channel} because the KZG commitment itself
can provide the strucural proof.

TODO:  Is this still true with Caulk?


\subsection{Certificates} % \& revokation}

If an authority grants ring membership, then ring membership proofs
could simply verify some certificate by the authority, likely using
a signature on JubJub.

In this, we prefer a SNARK friendly random oracle,
because conventional random oracles cost like 30k constraints.
We also need a variable base scalar multiplication, which costs like
4k constraints, as well as a couple fixed base scalar multiplication.
A priori, these fixed base scalar multiplications cost roughly 700
constraints each, but ocasionally they cost only half this.   

We conjecture one fixed based scalar multiplication could be replaced
by adapting implicit certificate scheme technqiues,
 instead of simply a signature on a user provided key.

We typically need expiration dates in certificates, likely demanding
a range proof and maybe requiring that \pipk be recomputed more often.

% \subsection{Revokation}

As a rule, one needs some revokation path for certificates,
despite the underlying signature not being revokable. 
%
We suggest maitaining a seperate revokation list and then inside
\pipk prove non-membership in the revokation list.
% perhaps via \cite{???}.
In this way, we update \pipk only when the revokation list updates.
We expect this represents a significatn savings because the revokation
list could update far less often than the full ring \ctx itself.
% perhaps corresponding with expiration checks
% especially since ring membership cannot be traced across site so easily.

We already trust an authority with issuing certificates, so we trust
them with managing therevokation list too.  As such, our revokation list
non-membership proofs merely requires proving adjacency of the revoked
public keys lexicographically before and after our own public key.
If the revokation list requires secrecy, then VRFs could hide its ordering,
similarly to NSEC5 \cite{nsec5}.


\subsection{Append only logs}

If an append only log grants ring membership, then a recursive SNARK
could validate ring membership with each recursive addition being
relatively inexpensive.

In this, we need a $\pipk^0$ similar to \pipk as well as a
$\pipk^n$ that proves some $\comring_{n-1}$ to be the ancestor of
its own $\comring_n$ and recursively proves some $\pifast^{n-1}$ with
its own $\comring_{n-1}$ and the same $\sk$.
We expect half pairing cycles fit this usage nicely, although they complicate provers.

Append only logs still depand \pipk be reproven whenever
\ctx updates however, so they only reduce bandwidth they not CPU usage.
We shy away from such append only log optimizations, due to this
prover complexity and our desire for revokation, but
 they remain an interesting corner of the design space.

%%\section{Ring hiding}% {Hiding rings} % ring membership circuits}
\label{sec:ring_hiding}

At first, one imagines sites would accept few rings because each ring
gives some users multiple ``Sybil'' identities within the site.
In practice however, we think many sites benefit from accepting
multiple overlapping rings for convenience, reach, etc., but then
tollerate the resulting few ``Sybil'' users.

As sites accept more rings, we increase risks that each user's ring
\ctx reveals private user attributes, especially if
 users join many rings, sites accept many rings, and
 user agents manage the association poorly.
As a solution, we suggest tweaking \pifast to prove the ring itself
lies in some permitted set of rings, but hide the specific ring used.

We could achieve this using recursion inside \pifast of course,
but doing so lies out of scope.  We instead discuss using other
zk continuation techniques or similar.

\subsection{Unique circuit}

As a first step, if all rings use the same circuit, then we hide the
ring through openning a blinded polynomial commitment \cite{KZG}: 
In \S\ref{subsec:rvrf_faster}, our \pifast takes public input
 $X = \comring\, Y_0 + \compk$ where $\compk = \sk\, Y_1 + b \genB_\gamma$.
Instead of revealing \comring, we prove correctness of \comring in
 $X'' = \comring\, Y_0 + d'' \genB_\gamma$.
We prove $X''$ has this structure by opening a polynomial commitment
\cite{KZG}, but over a larger slower recursive elliptic curve
 like BW6 \cite{BW6}.

TODO: More details?

If using $\pisafe$ anyways then we could prove correctness for \comring
using $\pisafe'$ too, which saves pairings over adding KZK.

\subsection{Multi circuit}

Although $\gamma=1$ remains viable, all circuits wind up with
unique $\delta$ and hence unique $\ecE_2$ SRS element $[\delta]_2$.
We hide $[\delta]_2$ so we suggest proving correctness of $[\delta]_2$
using a blinded polynomial commitment \cite{KZG} over BW6 \cite{BW6},
except this time a multipicative blinding works better.

TODO: More details?

At this point, we have blinded and proven correct both the
ring commitment \comring and the circuit commitment $[\gamma]_2$.
A priori, \pifast chooses $\genG = Y_1$, which reveals the circuit too, like
$$ Y_1 = [{1\over\gamma} (\beta u_1(\tau) + \alpha v_1(\tau) + w_1(\tau))]_1 \mathperiod $$

Instead, we propose to stabalize the public input SRS elements:
We choose $Y_{1,\gamma}$ independent before selecting the circuit
 or running its trusted setup.
We then merely add an SRS element $Y_{1,\delta}$, for usage in $C$, that binds
 our independent $Y_{1,\gamma}$ to the desired definition, so
$$ Y_{1,\delta} := [{1\over\delta} (\beta u_1(\tau) + \alpha v_1(\tau) + w_1(\tau) - \gamma Y_{1,\gamma})]_1 \mathperiod $$
At this point, we replace $Y_1$ by $Y_{1,\gamma}$ everywhere and
 include $\comring \, Y_{1,\delta}$ inside $C$.

In this way, all ring membership circuits could share identical
public input SRS points $Y_{1,\gamma}$, and similarly $Y_0$ if desired.

% Interestingly the SRS ceremony could safely output points for both forms

\subsection{SnarkPack}

TODO: Handle $\pi$ hashes?



%% Handan writes: We cannot prove the UC security of our protocol with these options. Also, they are not formally described for a conference submission. We  should just say merkle tree or ring in the protocol description section
%%\section{Application: Identity}
\label{sec:app_identity}

Ring VRFs yield anonymous identity systems:
After a user and service establish a secure channel and
the server authenticates itself with certificates, then
the user authenticates themselves by providing an anonymous
VRF signature with input \msg being the service's identity,
thus creating an pseudonymous identified session with
a pseudonym unlinkable from other contexts.

We expand this identified session workflow with an extra
update operation suitable for our ring VRF's amortized prover.
We discuss only \pifast here but all techniques apply to \pisk and \pipk similarly. 

\begin{itemize}
\item {\em Register} --
 Adds users' public key commitments into some \ring,
 after verifying the user does not currently exist in \ring.
\item {\em Update} --
 User agents regenerate their stored SNARK $(\pk,\pifast^\inner)$ using
 $\SpecialG.\Preprove( (\sk_1,\sk_2,\openring); (\sk,\comring) )$
 each time \ring changes, perhaps even receiving \comring and \openring
 from some ring management service.
\item {\em Identify} --
 Our user agent first opens a standard TLS connection to a server \msg,
 both checking the server's name is \msg and checking certificate
 transparency logs, and then computes the shared session id \aux.
 Our user agent computes the user's identity
  $\mathtt{id} = \PedVRF.\Eval(\sk,\msg)$ on the server id \msg,
 Our user agent next rerandomizes \pifast, \compk, and \openpk using
 $\SpecialG.\Reprove( \pk, \pifast^\inner )$, computes
  $\sigma = \PedVRF.\Sign(\sk,\openpk,\msg,\aux \doubleplus \compk \doubleplus \pifast)$,
 and finally sends the server their ring VRF signature $(\compk,\pifast,\sigma)$
 % $\rVRF.\rSign(\sk,\openring,\msg,\aux)$ % $ = (\compk,\pifast,\sigma)$.
\item {\em Verify} -- 
 After receiving $(\compk,\pifast,\sigma)$ in channel \aux,
 the server named \msg checks $\SpecialG.\Verify( \comring, (\compk,\pifast) )$,
 checks the VRF signature, and obtains the user's identity $\mathtt{id}$, ala \\
 $\mathtt{id} = \PedVRF.\Verify(\compk,\msg,\aux \doubleplus \compk \doubleplus \pifast,\sigma)$.
\end{itemize}


\subsection{Browsers}

We must not link users' identities at different web sites, so user agents
should carefully limit cross site resource loading, referrer information, etc.
User agents could always load purely static resources, without metadata
like cookies or referrer information.
% especially purely content addressable resources.
At least Tor browser already takes cross site resource concerns seriously,
while Safari and Brave may limit invasive cross site resources too.
% In any case, one could always specify rules against cross site privacy invasions
% whenever writing ring VRF browser specifications.

We somewhat trust the CAs and CT log system with users' identities in
the above protocol, in that users could login to a site with fraudulent
credentials.  We think cross site restrictions limit this attack vector.
If stronger defenses are desired then instead of \msg being the site name,
\msg could be a public ``root'' key for the specific site, which then
also certifies its TLS certificate.  Ideally its secret key remains air gaped.


\subsection{AML/KYC}
\label{subsec:AML_KYC}

We shall not discuss AML/KYC in detail, because the entire field lacks
clear goals, and thus winds up being ineffective
 \cite{doi:10.1080/25741292.2020.1725366}.
% https://www.tandfonline.com/doi/full/10.1080/25741292.2020.1725366
% via https://twitter.com/ronaldpol/status/1491548352189587460
We do however observe that AML/KYC typically conflicts with security
and privacy laws like GDPR.  As a compromise between these regulations,
one needs a compliance party who know users' identities,
 while another separate service party knows the users' activities.
We propose a safer and more efficient solution:

Instead our compliance party becomes an identity issuer who maintains
a public \ring, and privately knows the users behind each public key.
As above, identity systems could employ \ring freely for diverse purposes.
If later asked or subpoenaed, users could prove their relevant identities
to investigators, or maybe prove which services they use and do not use. 

Interestingly \PedVRF could run ``backwards'' like
 $H_{\grE'}(\msg) \ne \sk^{-1} \, \PreOut$
to show a ring VRF output associated to $\PreOut$
does not belong to the user, without revealing the users'
identity $\Hout(\msg, \sk \, H_{\grE'}(\msg))$ to investigators. 

Our applications mostly ignore key multiplicity. 
AML/KYC demands suspects prove non-involvement using ring VRFs.

\begin{definition}\label{def:rvrf_exculpability}
We say \rVRF is {\em exculpatory} if we have an efficient algorithm
for equivalence of public keys, but a PPT adversary \adv cannot
find non-equivalent public keys $\pk_0,\pk_1$ with colliding VRF outputs.
% (perfectly or computationally)
% (either ever or with advantage negligible advantage in $\secparam$)
\end{definition}

A priori, our JubJub representations $\sk_0 \genJ_0 + \sk_1 \genJ_1$
used in \S\ref{subsec:rvrf_faster} and \S\ref{subsec:rvrf_side_channel}
costs us exculpability from Definition \ref{def:rvrf_exculpability}.
% Ad hoc rings ...
% Rings used for AML/KYC would be maintained by an authority and require
% some registration procedure, using government issued identity documents.

There is however a natural {\em exculpable public key} flavor $(\pk,\sigma)$,
in which
 $\sigma = \Sign(\sk, \CommitRing(\{ \pk \},\pk).\openring, \mathtt{ring\_name}, \mathtt{""})$.
The singleton ring $\{ \pk \}$ ensure that 
$\rVerify(\CommitRing(\{\pk\}), \mathtt{ring\_name}, \mathtt{""}, \sigma)$
uniquely determines the secret key, so exculpability holds
 if joining the ring requires $(\pk,\sigma)$.

% \begin{proposition}
% \end{proposition}


\subsection{Moderation}
\label{subsec:moderation}

All discussion or collaboration sites have behavioral guidelines and
moderation rules that deeply impact their culture and collective values.

Our ring VRFs enables a simple blacklisting operation:
If a user misbehaves, then sites could blacklist or otherwise penalizes
their site local identity $\mathtt{id}$.
As $\mathtt{id}$ remains unlinked from other sites, we avoid thorny
questions about how such penalties impact the user elsewhere, and thus
can assess and dispense justice more precisely. 

At the same time, there exist sites who must forget users' histories
eventually, like under some ``right to be forgotten'' principle, either
GDPR compliance or an ethical principle of social mistakes being ephemeral.

We obtain ephemeral identities if \msg consists of the site name plus
the current year and month, or some other approximate date.
In this way, users have only one stable $\mathtt{id}$ within the
approximate date range, but they obtain fresh $\mathtt{id}$s merely
by waiting until the next month.

We could adjust \PedVRF to simultaneously prove multiple VRF input-output
pairs $(\msg_j,\mathtt{id}_j)$.
As in \cite{PrivacyPass}, we merely delinearize \In and \PreOut in
\rSign and \rVerify like:
\begin{align*}
x &= H(\msg_j,\mathtt{id}_j,\ldots,\msg_j,\mathtt{id}_j) \\
\In &= \sum_j H_p(x,j) \, \In_j \\
\PreOut &= \sum_j H_p(x,j) \, \PreOut_j \\
\end{align*}

As doing so links these pairs together,
we could link together two or more ephemeral identities like this
to obtain a semi-permanent identity with user controlled revocation:
As login, our site demands two linked input-output pairs given by
 $\msg_1 = \mathtt{site\_name} \doubleplus \mathtt{current\_month}$ and
 $\msg_2 = \mathtt{site\_name} \doubleplus \mathtt{registration\_month}$,
so users could have multiple active pseudo-nyms given by $\mathtt{id}_2$,
but only one active pseudo-nym per month, enforced by deduplicating
 $\mathtt{id}_1$, which still prevents spam and abuse.

If instead our site associates pseudo-nyms to their most recently seen
$\mathtt{id}_1$, then we could link adjacent months, meaning $\msg_j$
is defined by the $j$th previous month, until reaching a previously used $\mathtt{id}_1$.
In this model, pseudo-nyms could be abandoned and replaced, but
abandoned pseudo-nyms cannot then be reclaimed without linking intervening dates.
Although more costly, sites could permanently bans a few problematic
users via the inequality proofs described in \S\ref{subsec:AML_KYC} too.

In these ways, sites encode important aspects of their moderation rules
into the ring VRF inputs they demand.  
% We expect this makes sites' values and culture more uniform, predictable, and transparent.


\subsection{Reduced pairings}
\label{sec:reduced_pairings}

At a high level, we distinguish moderation-like applications discussed
above, which resemble classic identity applications like AML/KYC, from
rate limiting applications discussed in the next section. 
%
In moderation-like applications, ring VRF outputs become long-term
stable identities, so users typically reidentify themselves many times
to the same sites, reusing the exact same \msg.

As an optimization, our zero-knowledge continuation could reuse the
same \compk and \pifast for the same \msg, so that verifiers could
memoize their verifications of \pifast.  We spend most verifier time
checking the Groth16 pairing equation, so this saves considerable CPU time. % assuming our cache wind up fast enough.

As a concrete example, our coefficients $r_1,r_2,b$ used for
rerandomization in \S\ref{sec:rvrf_cont} could be chosen
deterministically like $r_1,r_2,b \leftarrow H(\sk,\msg)$.
In this way, each (helpful) user's $\mathtt{id}$ has a unique \pifast,
which verifiers could memoize by storing
 $(\mathtt{id},H(\compk \doubleplus \pifast),\mathtt{dates})$
after their first verification, but then skipping the Groth16 check
 after merely rechecking the hash $H(\compk \doubleplus \pifast)$.

We could risk denial-of-service attacks by users who vary $r_1,r_2,b$ 
randomly however.  We therefore suggest $\mathtt{dates}$ record the last
several previous dates when $H(\compk \doubleplus \pifast)$ changed.
We rate limit or verify more lazily users with many nearby login dates


%%\section{Application: Rate limiting}
\label{sec:app_rate_limits}

We showed in \S\ref{sec:app_identity} how ring VRFs give users only
one unique identity for each input \msg.  
We explained in \S\ref{subsec:moderation} that choosing \msg to be
the concatenation of a base domain and a date gives users a stream of changing identities.
%
We next discuss giving users exactly $n > 1$ ring VRF outputs aka
``identities'' per date, as opposed to one unique identity 


% \subsection{Implementation}

As a trivial implementation, we could include a counter $k = 1 \ldots n$
in \msg, so $\msg = \mathtt{domain} \doubleplus \mathtt{date} \doubleplus k$.


\subsection{Avoiding linkage}

Our trivial implementation leaks information about ring VRF outputs'
 ownership by revealing $k$:
%
An adversary Eve observes two ring VRF signatures with the same
$\mathtt{domain}$ and $\mathtt{date}$ so
$\msg_i = \mathtt{domain} \doubleplus \mathtt{date} \doubleplus k_i$
for $i=1,2$, but with different outputs $\Out_1$ and $\Out_2$.
If $k_1 \ne k_2$ then Eve learns nothing, but if $k_1 = k_2$ then
 Eve learns that $sk_1 \ne \sk_2$, maybe representing different users. 

We do not necessarily always care if Eve learns this much information,
but scenarios exist in which one cares.  We therefore briefly describe
several mitigation:

If $n$ remains fixed forever, then we could simply let all users
register $n$ ring VRF public keys in \ring.
If $n$ fluctuates under an upper bound $N$, then we could create $N$
rings $\ring_i$ for $i = 1 \ldots N$, and
 then blind \comring in \pifast similarly to \S\ref{sec:ring_hiding}.

Although simple, these two approaches require users construct $n$ or $N$
different $\pipk$ proofs every time \ring updates.

Instead of proving ring membership of one public key, $\pipk$ could
prove ring membership of a Merkle commitment to multiple keys, so
users have $\pisk^1,\ldots,\pisk^N$ for each of their multiple keys.

% \smallskip

In principle, there exists ring VRFs that hide parts of their input
\msg, but still fit our abstract formulation in \S\ref{sec:overview}.
Although interesting, we caution these bring performance concerns not
discussed here, so deployments should consider if leaking $k$ suffices.


\subsection{Ration cards}
\label{subsec:app_ration_carts}

As a species, we expect $+3^{\circ}$C over the pre-industrial climate
by 2100 \cite{IPCC2022}, or more likely above $+4^{\circ}$C given
tipping points \cite{tipping2022}.  % https://www.youtube.com/watch?v=LxoyaCSWFGs
At these levels, we experience devastating famines as the Earth's
carrying capacity drops below one billion people \cite{carrying_capacity}.
In the near term, our shortages of resources, energy, goods, water,
and food shall steadily worsen over the next several decades, due to
climate change, ecosystem damage or collapse, and resource exhaustion
ala peak oil.  We expect synchronous crop failures around the 2040s
in particular \cite{climaterisk2021}. % https://nitter.it/ThierryAaron/status/1442442451541807109#m
% off topic: https://12ft.io/proxy?q=https%3A%2F%2Fwww.bbc.com%2Ftravel%2Farticle%2F20220816-why-theres-no-dijon-in-dijon-mustard
Invariably, nations manage shortages through rationing, like during WWI, WWII, and the oil shocks.  

Ring VRFs support anonymous rationing:
Instead of treating ring VRF outputs like identities,
we treat them like nullifiers which could each be spent exactly once.

\def\expiry{e}
We fix a set $U$ of limited resource types, overseen by
 an authority who certifies verifiers from a key $\mathtt{root}$.
We dynamically define an expiry date $\expiry_{u,d_0}$ and an availability $n_{u,d_0}$,
both dependent upon the resource $u \in U$ and current date $d_0$.
We typically want a randomness beacon $r_d$ too, which prevents
anyone learning $r_d$ much before date $d$. 
% Among other usages, this reduces damages from key compromises.
As ring VRF inputs, we choose
 $\msg = \mathtt{root} \doubleplus u \doubleplus r_d \doubleplus d \doubleplus k$
where $u \in U$ denotes a limited resource,
 $d$ denotes an non-expired date meaning $\expiry_{u,d_0} < d \le d_0$,
 and $1 \le k \le n_{u,d_0}$.
In this way, our rationing system controls both daily consumption
via $n_{u,d_0}$ and time shifted demand via expiry time $\expiry_{u,d_0}$.

Importantly, our rationing system retains ring VRF outputs as nullifiers,
filed under their associated date $d$ and resource $u$, so nullifiers
expire once $d \le \expiry_{u,d_0}$ which permits purging old data rapidly.

We remark that fully transferable assets could have constrained lifetimes
too, which similarly eases nullifier management when implements using
blind signatures, ZCash sapling, etc.  Yet, all these tokens require
an explicit issuance stage, while ring VRFs self-issue.

Among the political hurdles to rationing, we know certificates have
a considerable forgery problem, as witnessed by the long history of
fraudulent covid and TLS certificates.  It follows citizens would
justifiably protest to ration carts that operate by simple certificates.
Ring VRFs avoid this political unrest by proving membership in a public list.


\subsection{Multi-constraint rationing}
\label{subsec:multi_io}

% \cite{PrivacyPass}
As in \S\ref{subsec:moderation}, we could impose simultaneous rationing
constraints for multiple resources $u_1,\ldots,u_k$ by producing one
ring VRF signature in which \PedVRF proves correctness of pre-outputs
for multiple messages 
 $\msg_j = \mathtt{root} \doubleplus u_j \doubleplus r_d \doubleplus d \doubleplus k$ for $j=1 \ldots k$.

As an example, purchasing some prepared food product could require spending
rations for multiple base food sources, like making a cake from wheat, butter,
eggs, and sugar.  


\subsection{Decommodification}

There exist many reasons to decommodify important services, like
energy, water, or internet, beyond rationing real physical shortages.
Ring VRFs fit these cases using similar \msg formulations.

As an example, a municipal ISP allocates some limited bandwidth capacity
among all residents.  It allocates bandwidth fairly by verifying ring VRFs
signatures on hourly \msg and then tracking nullifiers until expiry.

Aside from essential government services, commercial service providers
typically offers some free service tier, usually because doing so
familiarizes users with their intimidating technical product.

Some free and paid tier examples include DuoLingo's hearts on mobile, 
continuous integration testing services, and many dating sites.

A priori, rate limiting cases benefit from unlinkability among individual
usages, not merely at some site boundary like moderation requires.
We thus use each ring VRF output only once, which prevents our cashing
trick of \S\ref{sec:reduced_pairings} from reducing verifier pairings.

Although rationing sounds valuable enough, we foresee services like ISP,
VPNs, or mixnets having many low value transactions.
In such cases, ring VRFs could authorize issuing a limited number of
fast simple single-use blind issued credentials, like blind signatures
ala GNU Taler \cite{taler} or PrivacyPass OPRF tokens \cite{PrivacyPass},
 which both solve the leakage of $k$ above too.
%
In principle, commercial service providers could sell the same tokens,
which avoids leaking whether the user uses the free or commercial tier.


\subsection{Delegation}

Almost all single-use blind signed tokens have an implicit delegation
protocol, in which token holders transfer token credentials without
sacrificing their own access.
As double spending remains possible, delegatees must trust delegators.
% PrivacyPass \cite{PrivacyPass} only supports this transfer style.
GNU Taler \cite{taler} argues against taxing such trusting transfers,
like when parents give their kids spending money, but enforces taxability
only when also preventing double spending.

In our rationing scheme, spenders authenticate their specific spending
operations inside the associated data \aux in a rVRF-AD signature.
As doing so requires knowing \sk, delegators place enormous trust in
delegatees, which likely precludes say parents delegating to children.

We could however achieve delegation by treating the ring VRF like a
certificate that authenticates another public key held by the delegatee.
In fact, delegators could limit delegatees uses too in this certificate,
like how GNU Taler achieves parental restrictions. % \cite{???}

We remark that \PedVRF has adaptor signatures aka implicit certificate mode:
A delegatee learns the full ring VRF signature, but the delegatee hides
the blinding factor signature $s_1$ in \PedVRF from downstream recipients,
and instead merely prove knowledge of $s_1$, say via
 a key exchange or another Schnorr signature with the base point $K$.
EC VRFs lack this mode.




%
\section{Applications}
\label{sec:app_short}

We briefly outline how ring VRFs could be used for
 identity, moderation, rationing, and games.


\eprint{\subsection{Identity}
\label{subsec:app_identity}}{\noindent{\textbf{Identity:}}}
Ring VRF outputs  provide users with stable identities across
arbitrarily many services given several conditions:
First, our ring VRF input should be stable for a given services,
 like by using services' urls.
Second, we demand an encrypted connection between the user agent and
the service, in which the service authenticates itself first,
 like by verifying TLS certificates.
Third, the user agent avoids identity leakage between different services,
 like by denying cross site resources.
Fourth, the server trusts the ring membership, like by trusting
 a third party who enforces a ring registration procedure.
Also, this third party updates users as the ring membership evolves.

An HTTPS workflow satisfying these conditions resembles:
\renewcommand{\pifast}{\ensuremath{\pi_{\ring}}}
\begin{itemize}
\item {\em Register} --
	Adds users' public key commitments into some \ring,
	after verifying the user does not currently exist in \ring.
\item {\em Update} --
	User agents regenerate their stored signature using
	$\SpecialG.\Preprove$
	each time \ring changes\eprint{, perhaps even receiving \comring and \openring
	from a ring management service.}{.}
\item {\em Identify} --
	Our user agent first opens a standard TLS connection to a server \msg,
	both checking the server's url is \msg and checking certificate
	transparency logs, and then computes the shared session id \aux.
	Our user agent computes the user's identity
	$\mathtt{id} = \PedVRF.\Eval(\sk,\msg)$ on the server id \msg,\eprint{
	Our user agent next rerandomizes \pifast, \compk, and \openpk using
	$\SpecialG.\Reprove( \pk, \pifast^\inner )$, computes
	$\sigma = \PedVRF.\Sign(\sk,\openpk,\msg,\aux \doubleplus \compk \doubleplus \pifast)$,
	and finally sends the server their ring VRF signature $(\compk,\pifast,\sigma)$}{ and generates ring VRF signature.}
	\eprint{Our user agent rejects identity requests from resources besides
	top/outer most frame.}{}
\item {\em Verify} -- 
	\eprint{After receiving $(\compk,\pifast,\sigma)$ in channel \aux,
	the server \msg checks $\SpecialG.\Verify( \comring, (\compk,\pifast) )$,
	checks the VRF signature, and obtains the user's identity $\mathtt{id}$, ala \\
	$\mathtt{id} = \PedVRF.\Verify(\compk,\msg,\aux \doubleplus \compk \doubleplus \pifast,\sigma)$.}{It verifies the ring VRF signature. If it verifies, it checks whether $ \mathtt{id}$ equals to the evaluation value generated from the verification process.}
\end{itemize}

\eprint{Anonymity depends largely upon certificate authentication, including
certificate transparency logs, in that users could otherwise login to
a site with fraudulent credentials.
%  We think cross site restrictions
%and \aux being the channel limit this attack vector somewhat though.
If stronger defences are desired then instead of \msg being the url,
\msg could be an air gapped public ``root'' key for the site or CA, which
then also certifies its TLS certificate.  }{}

\eprint{As an optimization, \Reprove could rerandomize deterministically based
upon $H(\sk,\msg)$, so servers could then cache $\pifast$ verification.
}{}

\eprint{}{\begin{comment}}
\subsection{AML/KYC}
\label{subsec:AML_KYC}

We shall not discuss AML/KYC in detail, because the entire field lacks
clear goals, and thus winds up being ineffective
\cite{doi:10.1080/25741292.2020.1725366}.
% https://www.tandfonline.com/doi/full/10.1080/25741292.2020.1725366
% via https://twitter.com/ronaldpol/status/1491548352189587460
We do however observe that AML/KYC typically conflicts with security
and privacy laws like GDPR.  As a compromise between these regulations,
one needs a compliance party who know users' identities,
while another separate service party knows the users' activities.
We propose a safer and more efficient solution:

Instead our compliance party becomes an identity issuer who maintains
a public \ring, and privately knows the users behind each public key.
As above, identity systems could employ \ring freely for diverse purposes.
If later asked or subpoenaed, users could prove their relevant identities
to investigators, or maybe prove which services they use and do not use. 

Interestingly \PedVRF could run ``backwards'' like
$H_{\grE'}(\msg) \ne \sk^{-1} \, \PreOut$
to show a ring VRF output associated to $\PreOut$
does not belong to the user, without revealing the users'
identity $\Hout(\msg, \sk \, H_{\grE'}(\msg))$ to investigators. 

Our applications mostly ignore key multiplicity. 
AML/KYC demands suspects prove non-involvement using ring VRFs.

\begin{definition}\label{def:rvrf_exculpability}
	We say \rVRF is {\em exculpatory} if we have an efficient algorithm
	for equivalence of public keys, but a PPT adversary \adv cannot
	find non-equivalent public keys $\pk_0,\pk_1$ with colliding VRF outputs.
	% (perfectly or computationally)
	% (either ever or with advantage negligible advantage in $\secparam$)
\end{definition}

A priori, our JubJub representations $\sk_0 \genJ_0 + \sk_1 \genJ_1$
used in \S\ref{subsec:rvrf_faster} and \S\ref{subsec:rvrf_side_channel}
costs us exculpability from Definition \ref{def:rvrf_exculpability}.
% Ad hoc rings ...
% Rings used for AML/KYC would be maintained by an authority and require
% some registration procedure, using government issued identity documents.

There is however a natural {\em exculpable public key} flavor $(\pk,\sigma)$,
in which
$\sigma = \Sign(\sk, \CommitRing(\{ \pk \},\pk).\openring, \mathtt{ring\_name}, \mathtt{""})$.
The singleton ring $\{ \pk \}$ ensure that 
$\rVerify(\CommitRing(\{\pk\}), \mathtt{ring\_name}, \mathtt{""}, \sigma)$
uniquely determines the secret key, so exculpability holds
if joining the ring requires $(\pk,\sigma)$.

% \begin{proposition}
% \end{proposition}
\eprint{}{\end{comment}}

%TODO:we don't prove anything related to multiple VRF input-output. We should make sure its security before adding it.
\eprint{\subsection{Moderation}
\label{subsec:moderation}}{\noindent\textbf{Moderation:}}
\eprint{%
All discussion or collaboration sites have behavioral guidelines and
moderation rules that deeply impact their culture and collective values.%
}{}
Our ring VRFs enables a simple blacklisting operation:
If a user misbehaves, then sites could blacklist or otherwise penalizes
their site local identity $\mathtt{id}$.
As $\mathtt{id}$ remains unlinked from other sites, we avoid thorny
questions about how such penalties impact the user elsewhere, and thus
can assess and dispense justice more precisely. 
At the same time, there exist sites who must forget users' histories
eventually, like under a ``right to be forgotten'' principle ala GDPR.
% or an ethical principles of social mistakes being ephemeral.

As users have distinct $\mathtt{id}$ for each \msg,
we obtain ephemeral identities if \msg consists of the url plus
the current week and month, or some other approximate date.
At this point, users have only one stable $\mathtt{id}$ within each
approximate date range, but they obtain fresh $\mathtt{id}$s merely
by waiting until the next week or month.

We then adjust \PedVRF to simultaneously prove multiple VRF input-output
pairs $(\msg_j,\mathtt{id}_j)$.  As in \cite{PrivacyPass}, we merely
delinearize $ \In  = H_\grE(\msg)$ and \PreOut in \rSign and \rVerify like:
\eprint{\begin{align*}
	x &= H(\msg_j,\mathtt{id}_j,\ldots,\msg_j,\mathtt{id}_j) \\
	\In &= \sum_j H_p(x,j) \, \In_j \\
	\PreOut &= \sum_j H_p(x,j) \, \PreOut_j \\
\end{align*}}{$ H(\msg_j,\mathtt{id}_j,\ldots,\msg_j,\mathtt{id}_j),
\In = \sum_j H_p(x,j),
\PreOut = \sum_j H_p(x,j) \, \PreOut_j $.}
In this way, \PedVRF proves the same secret key controls two or more
ephemeral identities, thereby constructing a stable identity from the
ephemeral identities.

At login, our site demands linked two input-output pairs given by
$\msg_1 = \mathtt{site\_name} \doubleplus \mathtt{current\_month}$ and
$\msg_2 = \mathtt{site\_name} \doubleplus \mathtt{registration\_month}$,
so users could have multiple active pseudo-nyms given by $\mathtt{id}_2$,
but only one active pseudo-nym per week, enforced by deduplicating
$\mathtt{id}_1$, which still prevents spam and abuse.
\eprint{Alternatively, we could associate users pseudo-nyms with their recently
seen $\mathtt{id}_1$ but link adjacent months.  In other words, we define
$\msg_j$ by the $j$th previous month, until reaching a previously used
$\mathtt{id}_1$.  In this model, pseudo-nyms could be abandoned, but
abandoned pseudo-nyms cannot then be reclaimed without linking intervening ones.}{}
% Although more costly, sites could permanently bans a few problematic
% users via the inequality proofs described in \S\ref{subsec:AML_KYC} too.
In these ways, sites encode important aspects of their moderation rules
into the ring VRF inputs they demand.  
% % We expect this makes sites' values and culture more uniform, predictable, and transparent.


\eprint{\section{Rate limiting}
\label{sec:app_rate_limits}}{\noindent\textbf{Rate limiting:}}
As a rate limiting device, we repeat this approximate date trick from
moderation, but also include a counter $k = 1 \ldots n$ in \msg, so
 $\msg = \mathtt{domain} \doubleplus \mathtt{date} \doubleplus k$.
Instead of treating ring VRF outputs like identities,
we now treat them like nullifiers which could each be spent exactly once,
 similarly to the nullifiers in ZCash or ecash systems.
We do leak information about nullifiers' ownership by revealing $k$ here:
An adversary Eve observes two ring VRF signatures with the same
$\mathtt{domain}$ and $\mathtt{date}$ so
$\msg_i = \mathtt{domain} \doubleplus \mathtt{date} \doubleplus k_i$
for $i=1,2$, but with different outputs $\Out_1$ and $\Out_2$.
If $k_1 \ne k_2$ then Eve learns nothing, but if $k_1 = k_2$ then
Eve learns that $sk_1 \ne \sk_2$, representing different users. 
We do not always care if Eve learns this much information, but users'
threat models should be clearly understood before making this choice.
In principle, we could hide $k$ if we replace \PedVRF by a flavor of
\Reval implemented using Groth16, 
 but which still fits our formulation in \S\ref{sec:overview}.
Indeed these \Reval choices could provide post-quantum anonymity, 
without expensive post-quantum soundness, perhaps interesting if leaking $k$ matters.


\eprint{\subsection{Ration cards}
\label{subsec:app_ration_carts}}{\noindent\textbf{Ration Cards:}}
\eprint{As a species, we expect $+3^{\circ}$C over the pre-industrial climate
by 2100 \cite{IPCC2022}, or more likely above $+4^{\circ}$C given
tipping points \cite{tipping2022}.  % https://www.youtube.com/watch?v=LxoyaCSWFGs
At these levels, we experience devastating famines as the Earth's
carrying capacity drops below one billion people \cite{carrying_capacity}.
In the near term, our shortages of resources, energy, goods, water,
and food shall steadily worsen over the next several decades, due to
climate change, ecosystem damage or collapse, and resource exhaustion
ala peak oil.  We expect synchronous crop failures around the 2040s
in particular \cite{climaterisk2021}. % https://nitter.it/ThierryAaron/status/1442442451541807109#m
% off topic: https://12ft.io/proxy?q=https%3A%2F%2Fwww.bbc.com%2Ftravel%2Farticle%2F20220816-why-theres-no-dijon-in-dijon-mustard
Invariably, nations manage shortages through rationing, like during WWI, WWII, and the oil shocks.  
}{}
Anonymous rationing works much like rate limiting, except with
 multiple resources, an issuing authority, and limited time shifting:
%
\def\expiry{e}
We fix a set $U$ of limited resource types, overseen by
an authority who certifies verifiers from a key $\mathtt{root}$.
We dynamically define an expiry date $\expiry_{u,d}$ and an availability $n_{u,d}$,
both dependent upon the resource $u \in U$ and date $d$.
% We typically want a randomness beacon $r_d$ too, which prevents
% anyone learning $r_d$ much before date $d$. 
As ring VRF inputs for the spend operation, we choose
$\msg = \mathtt{root} \doubleplus u \doubleplus d \doubleplus k$
where $u \in U$ denotes a limited resource,
the expiry check $d < \mathtt{today} \le \expiry_{u,d}$ passes,
and $1 \le k \le n_{u,d}$.
We also choose \aux to be a preliminary receipt signed by the merchant.
%
At this point, our merchant sends the ring VRF signature to the authority,
who enforces that each nullifier by spent at most once.
Our authority stores the nullifiers until expiry aka $d \le \expiry_{u,d_0}$.

% We do not discuss ring updates here, 

% We remark that fully transferable assets could have constrained lifetimes
% too, which similarly eases nullifier management when implements using
% blind signatures, ZCash sapling, etc.  Yet, all these tokens require
% an explicit issuance stage, while ring VRFs self-issue.

Among the political hurdles to rationing, certificates have
a \eprint{considerable}{} forgery problem, as witnessed by the \eprint{long}{} history of
fraudulent covid and TLS certificates.  It follows citizens would
justifiably protest to ration carts that operate by simple certificates.
Ring VRFs avoid this political unrest by proving membership in a public list.

\eprint{}{\begin{comment}}
\subsection{Multi-constraint rationing}
\label{subsec:multi_io}

% \cite{PrivacyPass}
As in \S\ref{subsec:moderation}, we could impose simultaneous rationing
constraints for multiple resources $u_1,\ldots,u_k$ by producing one
ring VRF signature in which \PedVRF proves correctness of pre-outputs
for multiple messages 
$\msg_j = \mathtt{root} \doubleplus u_j \doubleplus r_d \doubleplus d \doubleplus k$ for $j=1 \ldots k$.

As an example, purchasing some prepared food product could require spending
rations for multiple base food sources, like making a cake from wheat, butter,
eggs, and sugar.  

\subsection{Decommodification}

There exist many reasons to decommodify important services, like
energy, water, or internet, beyond rationing real physical shortages.
Ring VRFs fit these cases using similar \msg formulations.

As an example, a municipal ISP allocates some limited bandwidth capacity
among all residents.  It allocates bandwidth fairly by verifying ring VRFs
signatures on hourly \msg and then tracking nullifiers until expiry.

Aside from essential government services, commercial service providers
typically offers some free service tier, usually because doing so
familiarizes users with their intimidating technical product.

Some free and paid tier examples include DuoLingo's hearts on mobile, 
continuous integration testing services, and many dating sites.

A priori, rate limiting cases benefit from unlinkability among individual
usages, not merely at some site boundary like moderation requires.
We thus use each ring VRF output only once, which prevents our cashing
trick of \S\ref{sec:reduced_pairings} from reducing verifier pairings.

Although rationing sounds valuable enough, we foresee services like ISP,
VPNs, or mixnets having many low value transactions.
In such cases, ring VRFs could authorize issuing a limited number of
fast simple single-use blind issued credentials, like blind signatures
ala GNU Taler \cite{taler} or PrivacyPass OPRF tokens \cite{PrivacyPass},
which both solve the leakage of $k$ above too.
%
In principle, commercial service providers could sell the same tokens,
which avoids leaking whether the user uses the free or commercial tier.
\eprint{}{\end{comment}}



%
%% The next two lines define the bibliography style to be used, and
%% the bibliography file
\bibliographystyle{ACM-Reference-Format}
\bibliography{../climate,../commit,../anoncred,../sassafras,../identity,../vrf,../zkp,../ringsignatures}


%%
%% If your work has an appendix, this is the place to put it.
\appendix

\newcommand{\Gen}{\ensuremath{\mathsf{Gen}}}

\newcommand{\anonymouskeymap}{\ensuremath{\mathtt{anonymous\_key\_map}}}
\newcommand{\anonymouskeylist}{\mathcal{W}}
\renewcommand{\sim}{\simulator}

\begin{figure}
\footnotesize 
\begin{tcolorbox}[left=2pt,right=2pt]
	{  $ \fgvrf $ runs two PPT algorithms $ \Gen_W$ and $\Gen_{sign} $ during the execution.
	
		
				
			\textbf{[Key Generation.]} upon receiving a message $(\oramsg{keygen}, \sid)$ from a party $\user_i$, send $(\oramsg{keygen}, \sid, \user_i)$ to the simulator $\simulator$.
			Upon receiving a message $(\oramsg{verificationkey}, \sid, \pk)$ from $\simulator$, verify that $\pk$ has not been recorded before for $ \sid $; then, store in the table $\vklist$, under $\user_i$, the value $\pk$.
			Return $(\oramsg{verificationkey}, \sid, \pk)$ to $ \user_i$.
				
			%\textbf{[Malicious Key Generation.]} upon receiving a message $(\oramsg{keygen}, \sid, \pk)$ from $\simulator$, verify that $\pk$ was not yet recorded, and if so record in the table $\vklist$ the value $\pk$ under $\simulator$. Else, ignore the message.
				
			%\item[Honest Ring VRF Evaluation.] upon receiving a message $(\oramsg{eval}, \sid, \ring, \pk_i, m)$ from $\user_i$, verify that 
			%$\pk_i \in \ring$ 
			%and  
			%there exists $ \pk_i $ in $\vklist $ associated with $ \user_i $. If that was not the case, just ignore the request.
			%If there exists no $ W $ such that $ \anonymouskeymap[W] = (m, \ring, \pk_i) $, let $ W \leftsample \bin^\secparam $ and  $y \leftsample \bin^{\ell_\rVRF}$. Then, set $ \evaluationslist[m, W] = y$ and $ \anonymouskeymap[W] = (m, \ring,\pk_i) $.
			%Return $(\oramsg{evaluated}, \sid, \ring, m, W, y)$ to $ \user_i $.
			%The functionality does not check whether the evaluater's public key is in the ring because here we consider m, \ring as an input of the evaluation which is evaluated by a party who is not neccesarily in the ring. 
			\textbf{[Corruption:] } 
			upon receiving $ (\oramsg{corrupt}, \sid, \user_i) $ from $ \simulator $, remove $ \pk_i $ from $ \vklist[\user_i] $ and store $ \pk_i $ to $ \vklist $ under $ \sim $. Return $ (\oramsg{corrupted}, \sid,\user_i) $.
			
			\textbf{[Malicious Ring VRF Evaluation.]} upon receiving a message $(\oramsg{eval}, \sid, \pk_i, W, m)$ from $\sim$, if $ \pk_i $ is recorded under an honest party's identity or if there exists $ W'\neq W $ where $ \anonymouskeymap[m,W'] = \pk_i $, ignore the request.
			Otherwise, record in the table $\vklist$ the value $\pk_i$ under $\simulator$ if $ \pk_i $ is not in $ \vklist $. If  $\anonymouskeymap[m,W]  $ is not defined before, set $ \anonymouskeymap[m,W] = \pk_i $ and let   $y \leftsample \bin^{\ell_\rVRF}$ and set $ \evaluationslist[m, W] = y$.
			Then, set $ \evaluationslist[m, W] = y$, $ \anonymouskeymap[m,W] = \pk_i $ and obtain $ y =  \evaluationslist[m, W]$. Otherwise, obtain $ y = \evaluationslist[m, W] $.
			Return $(\oramsg{evaluated}, \sid,  m, \pk_i,W, y)$ to $ \user_i $.

			%upon receiving a message $(\oramsg{eval}, \sid, \pk_i, W, m)$ from $\sim$, if $ \pk_i $ is recorded under an honest party's identity or if there exists $ \anonymouskeymap[m, \pk_i] \neq W $ or if there exists a record for a key $ \pk \neq \pk_{i}$ such that $ \anonymouskeymap[m, \pk] = W $, ignore the request. Otherwise, record in the table $\vklist$ the value $\pk_i$ under $\simulator$ if $ \pk_i $ is not in $ \vklist $. If $ \anonymouskeymap[m,\pk_i]  $ is not defined, set $ \anonymouskeymap[m,\pk_i] = W $ and let   $y \leftsample \bin^{\ell_\rVRF}$ and set $ \evaluationslist[m, W] = y$.
			%Return $(\oramsg{evaluated}, \sid,  m, W, \evaluationslist[m, W])$ to $ \user_i $.
				
			\textbf{[Honest Ring VRF Signature and Evaluation.]} upon receiving a message $(\oramsg{sign}, \sid, \ring, \pk_i, m)$ from $\user_i$, verify that $\pk_i \in \ring$ and that there exists a public key $\pk_i$ associated to $\user_i$ in the table $ \vklist $. If that wasn't the case, just ignore the request. 	
			If there exists no $ W' $ such that $ \anonymouskeymap[m,W'] =  \pk_i $, let $ W \leftsample \{0,1\}^{w(\secparam)}$ and let $y \leftsample \bin^{\ell_\rVRF}$. If there exists $ W $ where $ \anonymouskeymap[m,W] $ is defined, then abort. Otherwise, set $ \anonymouskeymap[m,W] = \pk_i $ and set $ \evaluationslist[m, W] = y$.
			Obtain $ W, y $ where $ \anonymouskeymap[m,W] =\pk_i $ and $ \evaluationslist[m, W] = y$  and run  $ \Gen_{sign}(\ring, W) \rightarrow \sigma $. Verify that $ [m, W,\ring, \sigma, 0] $ is not recorded. If it is recorded, abort. Otherwise, record $ [m, W, \ring,\sigma, 1] $. Return $(\oramsg{signature}, \sid, \ring,W,m, y, \sigma)$ to $\user_i$.
			
			%\item[Malicious VRF evaluation.] upon receiving a message $(\oramsg{evalprove}, \sid, \ring, m)$ from $\simulator$, check that $\vklist$ has a public key associated to $\simulator$. If not, ignore the request. If $\evaluationslist[\ring, m][\simulator]$ is not set, sample $y \leftsample \bin^{\ell(\secparam)}$ and set $\evaluationslist[\ring, m][\simulator] \defeq y$ (and $\signaturelist[\ring,m]$ to $\emptyset$). If $\signaturelist[\ring, m]$ contains a proof (i.e., if $\signaturelist[\ring, m]$ is not empty), return $(\oramsg{evaluated}, \sid, y)$ to $\simulator$. Else, ignore the request.
			
			%\item[Verification.] upon receiving a message $(\oramsg{verify}, \sid, \ring, m, y, \sigma)$, from any party forward the message to the simulator. If there exists a $\pk_i$ among the values of \texttt{verification\_keys}, and there exists $\sigma \in \signaturelist[\ring, m]$, set $b = 1$. Else, set $b =0$. Finally, output $(\oramsg{verified}, \sid, \ring, m, y, \sigma, b)$.
			\textbf{[Malicious Requests of  Signatures and Outputs.]} upon receiving a message $ (\oramsg{request}, \sid, \ring, W, m) $ from $ \simulator $, obtain all existing valid signatures $ \sigma $ such that $ [m,W,\ring,\sigma, 1] $ is recorded and add them in a list $ \lst_{\sigma} $. 	Return $ (\oramsg{requests}, \sid, \ring, W,m, \lst_{\sigma},y)  $ to $ \simulator $.
			
			
			\textbf{[Ring VRF Verification.]} upon receiving a message $(\oramsg{verify}, \sid, \ring,W, m, \sigma)$ from a party, do the following: 
    		% \begin{list}[label={{C}}{{\arabic*}}, start = 1]
			% https://texblog.net/help/latex/ltx-260.html
			\newcounter{FunCond}
			\begin{list}{\hspace*{1pt} C\arabic{FunCond}}{\usecounter{FunCond}\setlength\leftmargin{0.15in}}
				\item If there exits a record $ [m,W,\ring,\sigma, b'] $, set $ b = b' $. (This condition guarantees the completeness and consistency.)
				%					\item Else if $ \pk  $ is an honest verification key where $ \anonymouskeymap[W] = (.,., \pk) $ and there exists no record $ [m, \ring, W, \sigma', 1] $ for any $ \sigma' $, then let $ b= 0  $.
				%					(This condition guarantees unforgeability meaning that if an honest party never signs a message $ m $ for a ring $ \ring $, then the verification fails.)\label{cond:forgery}
				
				%\item Else if there exists a record  such as $ [m,W,\ring,\sigma, b'] $, set $ b = b' $. (This condition guarantees consistency meaning that all identical verification requests will output the same $ b $) 
				\label{cond:consistency}
				\item Else if $ \anonymouskeymap[m,W]  $ is an honest verification key and  there exists a record $ [m, W,\ring, \sigma', 1] $ for any $ \sigma' $, then let $ b=1 $ and record $ [m, W,\ring,\sigma, 1] $. (This condition guarantees that if $ m $ is signed by an honest party for the ring $ \ring $ at some point, then the signature is $ \sigma' \neq \sigma $ which is generated by the adversary is valid) \label{cond:differentsignature}
				
				\item \label{cond:malicioussignature}Else relay the message $(\oramsg{verify}, \sid, \ring,W, m, \sigma)$ to $ \simulator $ and receive back the message $(\oramsg{verified}, \sid, \ring,W, m, \sigma, b_{\simulator}, \pk_\simulator)$.  Then check the following:

				\begin{enumerate}
					\item If $ W \notin \anonymouskeylist $ and $ |\anonymouskeylist[m, \ring]| > |\ring_{mal}| $ where $ \ring_{mal} $ is a set of malicious keys in $ \ring $, set $ b = 0 $.
					(This condition guarantees  uniqueness meaning that the number of verifying outputs that $ \sim $ can generate for $ m, \ring $ is at most the  number of malicious keys in $ \ring $.)\label{cond:uniqueness}.
					
					\item Else if $ \pk_\simulator $ is an honest verification key, set $ b = 0 $. (This condition guarantees unforgeability meaning that if an honest party never signs a message $ m $ for a ring $ \ring $)\label{cond:forgery}
					%\item \label{cond:forgerymalicious}Else if there exists $ \anonymouskeymap[W] = (m', \ring',.)  $ where $ (m', \ring') \neq (m, \ring) $ or $ \counter[m, \ring] > |\ring_m| $ where $ \ring_m $ is a set of keys in $ \ring $ which are not honest or $ b_{\simulator} = 0 $ or $ \pk_\simulator $ belongs to an honest party, set $ b = 0 $ and record $ [m, \ring,W,\sigma, 0] $. (This condition guarantees that if $ W $ is an anonymous key of a different message and ring or the number of anonymous keys of malicious parties in $ \ring $ is more than their number or     $ \simulator $ does not verify $ \sigma $, then the verification fails.)
					
					\item Else if there exists $ W' \neq W $ where  $ \anonymouskeymap[m,W'] = \pk_\simulator $, set $ b = 0 $. \label{cond:differentWforsamepk}
					\item Else set $ b = b_\sim$. \label{cond:simulatorbit}
				\end{enumerate}		

			\end{list}
			In the end,  record $ [m,W,\ring,\sigma, 0] $ if it is not stored. If $ b = 0 $, let $ \PreOut = \perp $. Otherwise,   do the following:
			\begin{itemize}
				\item if $ W \notin \anonymouskeylist[m,\ring] $, add $ W $ to $ \anonymouskeylist[m,\ring]  $.
				\item if $ \pk_\simulator $ is not recorded, record it in $ \vklist $ under $ \simulator $.
				\item if $ \evaluationslist[m,W] $ is not defined, set $ \evaluationslist[m, W]\leftsample \bin^{\ell_\rVRF}$, $ \anonymouskeymap[m,W]  = \pk_\simulator$.  Set $ \PreOut = \evaluationslist[m, W]$.
				\item otherwise, set $ \PreOut = \evaluationslist[m, W]$. 	
			\end{itemize}
			Finally, output $(\oramsg{verified}, \sid, \ring,W, m, \sigma, \PreOut, b)$ to the party.
			
	

	}
\end{tcolorbox}
\caption{Functionality $\fgvrf$.\label{f:gvrf}}
\end{figure}




\newcommand{\name}{rVRF}
\section{Security of Our Ring VRF Construction} 
\label{ap:ucproof}
\label{sec:ringvrfconstrnoPK}
\newcommand{\GG}{\grE}
\newcommand{\FF}{\F}
\newcommand{\hash}{H}
\newcommand{\hashG}{\hash_\grE}
\newcommand{\gen}{\mathsf{Gen}}
\newcommand{\hkeys}{\mathtt{h\_keys}}
\newcommand{\malkeys}{\mathtt{m\_keys}}
\newcommand{\rcom}{\mathcal{R}_{eval}}
\newcommand{\rsnark}{\Rring}
\newcommand{\counter}{\mathsf{counter}}
\newcommand{\bdv}{\mathcal{B}}
\newcommand{\abort}{\textsc{Abort}}
\newcommand{\pkeys}{\arraysym{public\_keys}}
\newcommand{\skeys}{\arraysym{secret\_keys}}
\newcommand{\keytransform}{T_{\mathsf{key}}}
%Before giving the security proof of our protocol, we give the protocol in Section \ref{sec:pederson_vrf} without the abstraction from $ \PedVRF $ for the sake of  clarity of the security proof.
%
%We instantiate parameter generation by constructing a group $\GG$ of order $ p $ and two generators $ \genG, \genB \in  \GG$.  We consider three hash functions: $ \hash, \hash_p: \{0,1\}^* \rightarrow \FF_p $ and a hash-to-group function $\hashG : \{0,1\}^* \rightarrow \GG$ and . \name \ works as follows:
%
%\begin{itemize}
%	\item $ \rVRF.\KeyGen(1^\kappa):  $ It selects randomly a secret key $ x \in \FF_p$ and computes the public key $ X = xG $. In the end, it outputs $ \sk = x $ and $ \pk = X $.
%	
%	%It also generates  PoK for the discrete logarithm of $ X $ for the relation $ \R_{dl} $, $ \NIZK.\Prove(\rdl, (x, (X, G, \GG))) \rightarrow \pi_{dl} $.
%	
%	%\begin{equation}
%	%	\rdl = \{(x,(X,G,\GG)): X,G \in \GG, x \in \FF_p, x = xG\}
%	%\end{equation}
%	
%	%For this, it does the following: $ a \leftsample \FF_p $, $ c = \hash_p(a\genG, X) $, $ s = a + cx $. 
%	
%	%	\item $ \rVRF.\eval(\sk, \ring, m) $: It lets $ P = \hashG(m, \ring) $ and computes $ W = xP  $. Then, it outputs $ y = \hash(m, \ring, W) $. So, the deterministic function $ F $ in our rVRF protocol is $ F(\sk, \ring, m) = H(m, \ring, x\hashG(m,\ring)) $.
%	%	
%	\item $ \rVRF.\Sign(\sk, \ring, m):$ It lets $ \In = \hashG(m) $ and computes the pre-output $ \PreOut= x\In$. The signing algorithm works as follows: 
%
%	\begin{itemize}
%		
%		\item It first commits to its secret key $
%		x$ i.e., $ \compk = X + \openpk \, \genB $ where $ \openpk \leftsample \FF_p $.
%		\item It generates a Chaum-Pedersen DLEQ proof $ \pi_{eval} $ showing the following relation by running the algorithm $ \NIZK_{\rcom}.\Prove(((\genG, \genB,\GG,\compk,\PreOut,\In); (x, \openpk))) $ which outputs $ \rightarrow \pi_{eval}$
%		\eprint{\begin{align}
%				\rcom= \{((x, \openpk), (\genG, \genB,\GG,\compk,\PreOut,\In)): 
%				\compk = x\genG + \openpk\, \genB, \PreOut = x \,\In \} \label{rel:commit} 
%		\end{align}}{
%		\begin{align}
%			\rcom= \{((x, \openpk), (\genG, \genB,\GG,\compk,\PreOut,\In)): \\
%			\compk = x\genG + \openpk\, \genB, \PreOut = x \,\In \} \label{rel:commit} \nonumber
%		\end{align}}
%		Here $ \Prove $ algorithm runs a non-interactive Chaum-Pedersen DLEQ proof with the Fiat-Shamir transform:  Sample random $r_1, r_2 \leftarrow \F_p$.
%		Let $R = r_1 \genG + r_2 K$, $R_m = r_1 \In$, and
%		$c = \hash_p(\ring, m, \PreOut,\compk,R,R_m)$.
%		Set $\pi_{eval} = (c,s_1,s_2)$ where $s_1 = r_1 + c x$ and $s_2= r_2 + c \, \openpk$.
%		\item %It obtains $ crs $ from $ \gcrs $ for the second proof by sending the message $ (\oramsg{learncrs}, \sid) $ to $ \gcrs $. Then,
%		%It constructs a Merkle tree $ \mathsf{MT} $ with the nodes $ X_i $ where  $ X_i \in \pk_i $ and $ \pk_i \in  \ring $. We denote its  root by $ \mathsf{root} $. In the end, 
%		It generates the second proof $ \pi_{ring} $ for the following relation with  the witness $ (\ring, x, \openpk) $. 		
%		
%		\eprint{\begin{equation}
%				%\rsnark = \{((\mathsf{copath}, X, \openpk),(G,\genB\GG,\mathsf{root}, \compk)): C-\openpk K = X, \mathsf{MT}.\Verify(\mathsf{copath}, X, \mathsf{root} ) \rightarrow 1\} \label{rel:snark}
%				\rsnark = \{(X, \openpk),(\genG,\genB,\GG,\ring, \compk)): \compk-\openpk \, \genB = X \in \ring\} \label{rel:snark}
%			\end{equation}
%		}{\begin{align}
%			\rsnark = \{(X, \openpk),(\genG,\genB,\GG,\ring, \compk)): \\\compk-\openpk \, \genB = X \in \ring\} \label{rel:snark} \nonumber
%		\end{align}
%	}
%		
%		%Here, $ \mathsf{copath} $ is a copath of the Merkle tree $ \mathsf{MT} $. $ \mathsf{MT}.\Verify(\mathsf{copath}, X, \mathsf{root} ) $ is a verification algorithm of the Merkle tree which verifies whether $ X $ is the one of the leaves of $ \mathsf{MT} $ i.e., compute a root $ \mathsf{root}' $ with $ X $ and $ \mathsf{copath} $ and output 1 if $ \mathsf{root} = \mathsf{root}' $.
%		
%		The second proof $ \pi_{ring} $ is generated by running 
%		$ \NIZK_{\rsnark}.\Prove(((\genG,\genB,\GG,\ring, \compk); (X, \openpk))) $ 
%	\end{itemize}
%	In the end, $ \rVRF.\Sign $ outputs $\sigma = (\pi_{eval}, \pi_{ring}, \compk, \PreOut) $.
%	
%	\item $ \rVRF.\Verify(\ring,\PreOut, m, \sigma) $: Given $  \sigma = (\pi_{eval}, \pi_{ring},\compk)  $ and $ \ring, \PreOut $,
%	% it first runs $ \NIZK.\Verify(\rdl,(X_i,\genG,\GG), \pi_{dl_i}) $ for each $ \pk_i= (X_i, \pi_{dl_i})  \in \ring $. If each of key in $ \ring $ verifies,
%	it runs $ \NIZK_{\rcom}.\Verify((\genG, \genB,\GG,\compk,\PreOut,\In), \pi_{eval} ) $ where $ P = \hashG(m) $. $ \NIZK_{\rcom}.\Verify $ works as follows: $ \pi_{eval} = (c,s_1, s_2) $, it lets $R' = s_1 \genG + s_2 \, \genB - c \,\compk$ and $R'_m = s_1 \hashG(m) - c \, \PreOut$. It
%	returns true if $c = \hash_p(\ring,m,\PreOut,\compk,R',R'_m)$. If  $ \NIZK_{\rcom}.\Verify((\genG, \genB,\GG,\compk,\PreOut,\In), \pi_{eval} ) $ outputs 1, it runs $ \NIZK_{\rsnark}.\Verify((\genG,\genB,\GG,\ring, \compk), \pi_{ring}) $. 
%	If all verification algorithms verify, it outputs $ 1 $ and the evaluation value $ y =  \hash(m,\PreOut)  $. Otherwise, it outputs $( 0, \perp) $.
%	
%\end{itemize}
%
%\subsection{Security Analysis}

Before we start to analyse our protocol, we should define the algorithm $ \gen_{sign} $  for $ \fgvrf $ and show that $ \fvrf $ with $ \gen_{sign} $ satisfies the anonymity defined in Definition \ref{def:anonymity}. $ \fgvrf $ that \name \ realizes runs  Algorithm \ref{alg:gensign} to generate honest signatures.



%\begin{algorithm}
%	\caption{$\gen_{W}(\ring,\pk,m)$}
%	\label{alg:genW}	 	
%	\begin{algorithmic}[1]
	%		\State$ W \leftsample\GG $
	%		%		\State \textbf{get} $ X \in \pk $
	%		%		\If{$\mathtt{DB}[m, \ring] = \perp  $}		
	%		%		\State{$ a \leftsample \FF_p $}		
	%		%		\State{$\mathtt{DB}[m, \ring] := a$}
	%		%		\EndIf
	%		%		\State$ a \leftarrow \mathtt{DB}[m, \ring] $
	%		%		
	%		%		\State \textbf{return} $ aX $
	%		\State \textbf{return} $ W $
	%	\end{algorithmic}
%	
%\end{algorithm}

\begin{algorithm}
	
	\caption{$\gen_{sign}(\ring,W,\{X,\pk\},\aux,\msg)$}
	\label{alg:gensign}	 	
	\begin{algorithmic}[1]
		\State $ c,s_1, s_2 \leftsample \FF_p $
		\State $ \pi_{eval}  \leftarrow (c,s_1, s_2)$
		\State $ \openpk \leftsample \FF_p $
		\State $ \compk =  x \genG + \openpk \, K$
		%\State $ \pi_{eval} \leftarrow \NIZK.\mathsf{Simulate}(\rcom, (G, \genB,\GG,\compk,W,\In)) $
		%\State \textbf{send} $(\oramsg{learn\_\tau},\sid)  $ to $ \gcrs $
		%\State \textbf{receive} $(\oramsg{trapdoor},\sid, \tau,crs)  $ from $ \gcrs $
		\State $ \comring, \openring \leftarrow \rVRF.\CommitRing(\ring, \pk) $
		\State $ \pi_{ring} \leftarrow \NIZK_{\rsnark}.\Prove((\comring, \compk); (\openpk, \openring)) $ 
		\State\Return$ \sigma = (\pi_{eval},\pi_{ring},\compk,\comring,W) $
	\end{algorithmic}
	
\end{algorithm}


\begin{lemma} \label{lem:anonymity} $ \fgvrf $ running Algorithm \ref{alg:gensign} satisfies anonymity defined in Definition \ref{def:anonymity} assuming that $ \NIZK_{\rsnark} $ is a zero-knowledge and Pedersen commitment is perfectly hiding.
\end{lemma}

\begin{proof}
	We simulate $ \fgvrf $ with Algorithm \ref{alg:gensign} against $ \mathcal{D} $. Assume that the advantage of $ \mathcal{D} $ is $ \epsilon $. We reduce the anonymity game to the following game where we change the simulation of $ \fgvrf $ by changing the Algorithm \ref{alg:gensign}. In our change, we let $ \pi_{ring} \leftarrow \NIZK_{\rsnark}.\mathsf{Simulate}(\genG,\genB,\GG,\comring, \compk) $. Since $ \NIZK_{\rsnark} $ is zero knowledge, there exists an algorithm  $ \NIZK_{\rsnark}.\mathsf{Simulate} $ which generates a proof which is indistinguishable from the proof generated from $ \NIZK_{\rsnark}.\Prove $. Therefore, our reduced game is indistinguishable from the anonymity game. Since in this game, no public key is used while generating the proof and $ W $ and $ \compk $ is perfectly hiding, the probability that  $ \mathcal{D} $ wins the game is $ \frac{1}{2} $. This means that $ \epsilon $ is negligible.		
\end{proof}

%We next show that \name \ realizes $ \fgvrf $  in the random oracle model under the assumption of the hardness of the decisional Diffie Hellman (DDH).

%The GDH problem is solving the computational DH problem by accessing the Diffie-Hellman oracle ($ \mathsf{DH}(.,.,.) $) which tells that given triple $ X,Y,Z $ is a DH-triple i.e., $ Z = xyG $ where $ X = xG $ and $ Y = yG $.

%\begin{definition}[$ n $-One-More Gap Diffie-Hellman (OM-GDH) problem]
%	Given   $ p $-order group $ \GG $ generated by $ G $, the challenges $ G, X = xG, P_1, P_2, \ldots, P_{n+1} $ and access to the DH oracle $ \mathsf{DH}(.,.,.) $ and the oracle $ \mathcal{O}_x(.) $ which returns $ x\In$ given input $ \In$, if a PPT adversary $ \mathcal{A} $ outputs $ xP_1, xP_2, \ldots, xP_{n+1} $ with the access of at most $ n $-times to the oracle $ \mathcal{O}_x $, then $ \mathcal{A}  $ solves the $ n $-OM-GDH problem. We say that $ n $-OM-GDH problem is hard in $ \GG $, if for all PPT adversaries, the probability of solving the $ n $-OM-GDH problem is negligible in terms of the security parameter.
%\end{definition}

\begin{theorem}
	Assuming that $ \hashG, \hash,\hash_p, \hash_\ring $ are random oracles,  the DDH problem is hard in the group structure $ (\GG, \genG,\genB, p) $, NIZK algorithms are zero-knowledge and knowledge sound and the commitment scheme is perfectly hiding and computationally binding, \name \ UC-realizes $\fgvrf$ running Algorithm \ref{alg:gensign}.
\end{theorem}

\begin{proof}
	We construct a simulator $ \simulator $ that simulates the honest parties in the execution of \name \ and simulates the adversary in $ \fgvrf $. 
	
		%\item \textbf{[Simulation of $ \gcrs $:] }When simulating $ \gcrs $, it runs $ \mathsf{SNARK}.\mathsf{SetUp}(\rsnark) $ which outputs a trapdoor $ \tau $ and $ crs $ instead of picking $ crs $ randomly from the distribution $ \distribution $. Whenever a party comes to learn the $ crs $, $ \simulator $ gives $ crs $ as  $ \gcrs $.
		
		 \noindent\textbf{[Simulation of $ \oramsg{keygen} $:]} Upon receiving $(\oramsg{keygen}, \sid, \user_i)$ from $\fgvrf$, $ \simulator $ obtains the a secret and public key pair $ x = (\sk,r)$ and $\pk $ by running $ \rVRF.\KeyGen $. It adds $ \pk $ to lists $ \hkeys $ and $ \vklist $ as a key of $ \user_i $. 
		 $ \simulator $ returns $(\oramsg{verificationkey}, \sid, x,\pk)$ to $\fgvrf$. 
		$ \simulator $ lets  $ \pkeys[X] = \pk$ and $ \skeys[X] =(\sk,r) $ where $ X = \sk \genG $.
		%Whenever the honest party $ \user $ is corrupted by $ \env, $ $ \simulator $ moves the key of $ \user $ to $ \malkeys $ from $ \hkeys $.
		
		 \noindent\textbf{[Simulation of corruption:]} Upon receiving a message $ (\oramsg{corrupted}, \sid, \user_i) $ from $ \fgvrf $, $ \simulator $ removes the public key $ \pk $ from $ \hkeys $ which is stored as a key of $ \user_i $ and adds $ \pk $ to $ \malkeys $.
		
		\noindent\textbf{[Simulation of the random oracles:]} We  describe how $ \simulator $ simulates the random oracles $ \hashG, \hash, \hash_p $ against the real world adversaries. 	
		
		$ \simulator $ simulates the random oracle $ \hashG $ as described in Figure \ref{oracle:HgnoPK}. \eprint{It selects a random element  $ h $ from $ \FF_p $ for each new input and outputs $ hG $ as an output of the random oracle $ \hashG $.}{} Thus, $ \simulator $ knows \emph{the discrete logarithm of each random oracle output of $\hashG  $}. 
		 The simulation of the random oracle $ \hash $ is less straightforward (See Figure \ref{oracle:HnoPK}).
		The value $ W $ can be a pre-output generated by $ \rVRF.\Eval $ or can be an anonymous key of  $ m $ generated by $ \fgvrf $ for an honest party. $ \simulator $ does not need to know about this at this point but $ \hash $ should output $ \evaluationslist[m,W] $ in both cases.	 
		%If $ W $ is a pre-output, $ \simulator $ needs to find corresponding malicious public key in the real world. If it is the case, $ W $ should be equal to $ x\hashG(m, \ring)= xhG $  where $ xG $ is a public key. 
		\eprint{	$ \simulator $ pretends $ W $ as if it is a pre-output. So, $ \simulator $ first obtains the discrete logarithm $ h $ of $ \hashG(m) $ from the $ \hashG $'s database and finds out a commitment key $ X^* = h^{-1}W $.}{}
		%If $ X^* $ has not been registered as a malicious key, it registers it to $ \fgvrf $. Thus, $ \simulator $ has a right to ask the output of the message $ m, \ring $ to $ \fgvrf $. 
	\eprint{	If  $ \skeys[X^*] $ is not empty, it replies by a randomly selected value from $ \FF_p $.
		Otherwise,
		$ \simulator $ checks if $ \pkeys[X^*] $ exists to see whether a corresponding public key of $ X^* $ exists. If it does not exist, $ \simulator $ picks a key $ \pk^* $ which is not stored in $ \pkeys $ and stores $ \pkeys[X^*] = \pk^* $. In any case, it obtains $ \evaluationslist[m,W] $ by sending a message $ (\oramsg{eval}, \sid,\pk^*,W,m) $ and replies with $ \evaluationslist[m,W] $.}{}
		Remark that in $ \hash $ if $ W $ is a pre-output generated by $ \adv $,  $ \fgvrf $ matches it with the evaluation value given by $ \fgvrf $. If $ W $ is an anonymous key of an honest party in the ideal world, $ \fgvrf $ still returns an honest evaluation value $ \evaluationslist[m,W] $ even if $ \simulator $ cannot know whether $ W $ is an anonymous key of an honest party in the ideal world. 
		% Remember that $ \fgvrf $ only replies to the evaluation message of $ \simulator $ if $ W $ is not mapped to another message, ring and public key $ (m', \ring', X')   $. $ W $ cannot be map to $ (m', \ring', X')   \neq  (m, \ring, X*)   $ because it would be aborted during the simulation $ \hashG $ if they were mapped to $ W $.
		During the simulation of $ \hash $, if $ \fgvrf $ aborts, then there exists $ W' \neq W $ such that $ \anonymouskeymap[m,W'] = \pk^* $. Remark that it is not possible because if it happens it means that $ hX^* = W' \neq W  $ where $ \pkeys[X^*] = \pk^* $, but also $ W = hX^* $. 
		Therefore, $ \simulator $ never aborts during the simulation of $ \hash $.
		
	\eprint{	We note that the anonymous keys for honest parties generated by $ \fgvrf $ are independent from honest commitment keys. Therefore, if $ X^* = h^{-1}W $ is an honest verification key, $ \simulator $ returns a random value because  $ \evaluationslist[m,W] $ is not defined or will not be defined in $ \fgvrf $ in this case except with a negligible probability. If it ever happens i.e., if $ \fgvrf $ selects randomly $ W = hX^* $, $ \env $ distinguishes the simulation via honest signature verification in the real world. So, this case is covered in our simulation in Figure \ref{oracle:H'}.}{}
		
		\begin{figure}
			\begin{minipage}{4cm}
			\centering
			\noindent\fbox{%
				\parbox{4cm}{%
					\underline{\textbf{Oracle $ \hashG $}} \\
					\textbf{Input:} $ m $ \\
					\textbf{if} $\mathtt{oracle\_queries\_gg}[m] = \perp  $
					
					\tab{$ h \leftsample \FF_p $}
					
					%					\tab{\textbf{for all} $ X \in \ring $}
					%					
					%					\tab{$ W =  hX $}
					%					
					%					\tabdbl{\textbf{if} $ W \in \anonymouskeylist $: \textsc{Abort}}
					%					
					%					\tabdbl{\textbf{else:} \textbf{add} $ W $ \textbf{to} $ \anonymouskeylist $}
					
					\tab{$ P \leftarrow hG $} 
					
					\tab{$\mathtt{oracle\_queries\_gg}[m] := h$}
					
					\textbf{else}:
					
					\tab{$ h \leftarrow \mathtt{oracle\_queries\_gg}[m] $}
					
					\tab{$ P \leftarrow hG$}
					
					\textbf{return $ \In$}
					
			}}	
			\caption{The random oracle $ \hashG $}
			\label{oracle:HgnoPK}
		
	\end{minipage}
\hfill
	\begin{minipage}{7cm}
			\centering
			
			\noindent\fbox{%
				\parbox{7cm}{%
					\underline{\textbf{Oracle $ \hash$}} \\
					\textbf{Input:} $ m,W $ 
					
					\textbf{if} $ \mathtt{oracle\_queries\_h}[m, W] \neq \perp $
					
					\tab{\textbf{return $  \mathtt{oracle\_queries\_h}[m,  W] $}}
					
					%					\textbf{send} $ (\oramsg{request}, \sid, \emptyset,W, m) $ \textbf{to} $ \fgvrf $
					%					
					%					\textbf{receive} $ (\oramsg{requests}, \sid, \emptyset, W, m, \setsym{L}_\sigma, y) $ \textbf{from} $ \fgvrf $
					
					%					\textbf{if} $ y = \perp $
					
					{$ P \leftarrow \hashG(m) $}
					
					{$ h \leftarrow \mathtt{oracle\_queries\_gg}[m] $}
					
					{$ X^* := h^{-1}W $ // candidate commitment key} 
					
					{{\textbf{if} $\skeys[X^*] = \perp$ }} 
					
					\tab{\textbf{if} $ \pkeys[X^*]  = \perp$}
					
					\tabdbl{$ \pk^* \leftsample \grE $} %TODO put the space of pk's
					
					\tabdbl{$ \pkeys[X^*] \leftarrow \pk^* $}
					
					\tab{\textbf{send} $ (\oramsg{eval}, \sid, W, \pkeys[X^*] , m) $ \textbf{to} $ \fgvrf $}
					
					\tab{\textbf{if} $ \fgvrf $ ignores: \abort}
					
					\tab{\textbf{receive} $ (\oramsg{evaluated}, \sid, W, m, y) $ \textbf{from} $ \fgvrf $}
					
					\tab{$  \mathtt{oracle\_queries\_h}[m, W]:=y $}
					
					{\textbf{else:} }
					
					\tab{$ y \leftsample \FF_p $}
					
					\tab{$  \mathtt{oracle\_queries\_h}[m,  W]:=y $}
					%					{\textbf{else:} $ \mathtt{oracle\_queries\_h}[m, \ring, W]  = \perp$}
					%					
					%					%\tab{\textbf{return} \textsc{Abort}}
					%					\tab{$ y \leftsample \bin^\lambda $}
					%					
					%					\tab{$\mathtt{oracle\_queries\_h}[m, \ring, W] := y $}
					
					%	\textbf{else:} $  \mathtt{oracle\_queries\_h}[m,  W]:=y $
					
					\textbf{return $  \mathtt{oracle\_queries\_h}[m,  W] $}
					
			}}	
			\caption{The random oracle $ \hash $}
			\label{oracle:HnoPK}
		\end{minipage}
		\end{figure}
		
		The simulation of the random oracle $ \hash_p $ (See Figure \ref{oracle:H'}) checks whether the random oracle query $ (\ring,m,W,\compk,R,R_m) $ is an $ \rcom $ verification query before answering. \eprint{For this, it checks whether $ \fgvrf $ has a recorded valid signature for the message $ m $ and the ring $ \ring $ with the anonymous key $ W $. If there exists such valid signature where $ \compk $ is part of it, $ \simulator $ checks whether the first proof of the signature $ (c,s_1, s_2) $ generates $ R, R_m $ as in $ \rVRF.\Verify $ in order to make sure that it is a $ \rcom $ verification query. If it is the case, it assigns $ c $ as an answer of $ \hash_p (\ring,m,W,\compk,R,R_m) $ so that $ \rcom $ verifies. However, if this input has already been set to another value which is not equal to $ c $ or $ W $ is a pre-output of an honest key, then $ \simulator $ aborts because the output of the real world for this signature and the ideal world will be different.}{}
		We remind that if an anonymous key $ W $ of an honest party  for a message $ m $ sampled by $ \fgvrf $ equals to a pre-output generated by $ \rVRF.\Sign $  for the same honest party's key and the message $ m $, then $ \env $ can distinguish the ideal and real world outputs because the evaluation value in the ideal world and real world for $ m,W $ will be different because of the simulation of the random oracle $ \hash $ i.e., $ \mathtt{oracle\_queries\_h}[m,W] \neq \evaluationslist[m,W] $.  Therefore, $ \simulator $ aborts if it is ever happen.
		
		\begin{figure}
			\centering
			
			\noindent\fbox{%
				\parbox{\columnwidth}{%
					\underline{\textbf{Oracle $ \hash_p $}} \\
					\textbf{Input:} $ (\aux', \msg,\compk,W,R,R_m) $ \\					
					
					\textbf{parse} $ \aux' $ as $\tmpaux $
					
					\textbf{send} $ (\oramsg{request\_signatures},\sid, \aux, W,\msg) $
					
					\textbf{receive} $ (\oramsg{signatures},\sid,\msg, \setsym{L}_\sigma) $
					
					\textbf{if} $ \exists \sigma \in \setsym{L}_\sigma $ where $ \compk   \in \sigma $ \textbf{and} $ \NIZK_{\Rring}.\Verify((\compk,\comring);\piring) \rightarrow 1 $
					
					\tab{\textbf{get} $ \pi_1 = (c,s_1, s_2) \in \sigma $} 
					
					\tab{\textbf{if} $ R = s_1\genG + s_2 \genB -c\compk, R_m = s_1 \hashG(m) - c W $}  
					
					\tabdbl{$ h := \mathtt{oracle\_queries\_gg}[m,W] $ }
					
					\tabdbl{\textbf{if} $ \mathtt{oracle\_queries\_h\_CP}[\aux, m,\compk,W,R,R_m]  = \perp $}
					
					\tabdbldbl{$ \mathtt{oracle\_queries\_h\_CP}[\aux, m,\compk,W,R,R_m]  := c$}
					
					\tabdbl{\textbf{else if} $( \mathtt{oracle\_queries\_h\_CP}[\aux, m,\compk,W,R,R_m]  \neq c $ }
					
					\tabdbl{\textbf{or} $ X^* = h^{-1}W \in \hkeys) $: \abort}
					
					
					\textbf{if} $ \mathtt{oracle\_queries\_h\_CP}[\aux, m,\compk,W,R,R_m]  = \perp $
					
					\tab{$ c \leftsample \FF_p $}
					
					\tab{$ \mathtt{oracle\_queries\_h\_CP}[\aux, m,\compk,W,R,R_m]  := c$}
					
					{\textbf{return} $ \mathtt{oracle\_queries\_h\_CP}[\aux, m,\compk,W,R,R_m] $}
					
			}}
			\caption{The random oracle $ \hash_p $}
			\label{oracle:H'}
		\end{figure}
		
		
		%		\item \textbf{[Simulation of $ \oramsg{sign} $]} 
		%		The simulator has a table  $\preoutputlist $ to keep the pre-outputs that it selects for each input and the ring of public keys. 
		%		Upon receiving $(\oramsg{sign}, \sid, \ring, m, y)$  from the functionality $\fgvrf$, $ \simulator $ generates the signature $ \sigma $ as follows:
		%		
		%		For the first proof, it samples $ c, s_1, s_2 \in \FF_p $ and $ \compk, W \in \GG$. Then, it lets the first proof be $\pi_1 =  (c, s_1, s_2) $. 
		%		In addition, it sets $ R = sG+ \delta K+ c\compk $ and $ R_m = s \hashG(m, \ring)+ cW $ and maps the input $ \ring,m, W,\compk, R, R_m$ to $ c $ in the table of the random oracle $ \hash_p $ so that $ \pi_1 $ verifies in the real-world execution.  
		%		It adds $ W $ to the list $ \preoutputlist[m, \ring] $.
		%		
		%		$ \simulator $ gets the trapdoor $ \tau $ that it generated during the simulation of $ \gcrs $ to simulate the second proof. Then, it runs $ \mathsf{SNARK}.\mathsf{Simulate}(\rsnark,\tau, crs) $ and obtains $ \pi_2 $.
		%		
		%		In the end, $ \simulator $  responds by sending the message $(\oramsg{signature}, \sid, \ring, m, \sigma = (\pi_1, \pi_2, \compk, W))$ to the $ \fgvrf $.  It also lets $ \mathtt{oracle\_queries\_h}[m, \ring, W] $ be $ y $, if it is not defined yet. If it is defined with another value $ y' \neq y $, then it aborts.
		%TODO: Talk about this abort case happens with a negl probability. 
		
		
		 \noindent\textbf{[Simulation of $ \oramsg{verify} $]} Upon receiving  $(\oramsg{verify}, \sid, \ring,W, \aux,\msg, \sigma)$ from the functionality $\fgvrf$, $ \simulator $ runs the two NIZK verification algorithms run for $ \rcom, \rsnark $ with the input $ \comring, \msg, \sigma, W $ described in $ \rVRF.\Verify $ algorithm of ring VRF protocol if $ \sigma $ can be parsed as $ (\pi_1,\pi_2, \compk, \comring) $. If  all verify, it sets $ b_{\simulator} =1 $. Otherwise it sets $ b_{\simulator} =0  $.
		
		\noindent If $ b_\simulator = 1 $, it sets $ X = h^{-1} W$ where $ h = \mathtt{oracle\_queries\_gg}[m] $. Then it obtains $ \pk  = \pkeys[X]$ if it exists. If it does not exist, it picks a $ \pk  $ which is not stored in $ \pkeys $ and sets $ \pkeys[X] = \pk $. Then sends  $ (\oramsg{verified}, \sid, \ring, W,\aux, m, \sigma, b_\simulator, \pkeys[X]) $ to $ \fgvrf $ and receives back $ (\oramsg{verified}, \sid, \ring, W, \aux, m, \sigma, y, b) $. 
			
			- If $ b \neq b_\simulator $, it means that the signature is not a valid signature in the ideal world, while it is in the real world. So, $ \simulator $ aborts in this case.
				If $ \fgvrf $ does not verify a ring signature even if  it is verified in the real world, $ \fgvrf $ is in either C3-\ref{cond:uniqueness}, \ref{cond:forgery} or C3-\ref{cond:differentWforsamepk}.
				If $ \fgvrf $ is in C3-\ref{cond:uniqueness}, it means that $ \counter[m,\ring] > |\ring_m| $. If $ \fgvrf $ is in C3-\ref{cond:forgery}, it means that $ \pk$ belongs to an honest party but this honest party never signs $ m $ for  $ \ring $. So, $ \sigma $ is a forgery.	 If $ \fgvrf $ is in C3- \ref{cond:differentWforsamepk}, it means that there exists $ W' \neq W $ where $ \anonymouskeymap[m,W'] = \pk$. If $ [m,W'] $ is stored before, it means that $ \simulator $ obtained $ W' = hX $ where $ h = \mathtt{oracle\_queries\_h}[m] $ but it is impossible to happen since $ W = hX $.
				
			  - If $ b = b_\simulator $, it sets $ \mathtt{oracle\_queries\_h}[m,W] = y $, if it is not defined before.
				% In short, if $ \simulator $ aborts because $ b\neq b_\simulator $ it means either $ W $ of an honest party is not unique and $ \adv $ in the real world generates a forgery signature of $ (m, \ring, \sigma) $ with $ W $ or the adversary in the real world generates anonymous keys for $ (m, \ring) $ more than the number of adversarial keys in $ \ring $.
				%				 
				
				%	\item If $ b = b_\simulator $, set $ \mathtt{oracle\_queries\_h}[m, W] = y $. Here, if $ \sigma $ is a signature of an honest party, $ \simulator $ sets its output with respect to the output selected by $ \fgvrf $. 
				%    Remark that we do not need to set $ \mathtt{oracle\_queries\_h\_CP} $ because it already verifies in the real world.
			
		 \noindent If $ b_\simulator = 0 $, it sets $ \pk = \perp $ and sends  $ (\oramsg{verified}, \sid, \ring, W,\aux, m, \sigma, b_\simulator, X) $ to $ \fgvrf $. Then, $ \simulator $ receives back $ (\oramsg{verified}, \sid, \ring, W, \aux,m, \sigma, \perp, 0) $. 
			%			\begin{itemize}
				%				\item If $ b \neq b_\simulator $, it means that it was a signature of an honest party and $ \NIZK.\Verify $ for $ \rcom $ does not validate in the real world. So, $ \simulator $ sets $ \mathtt{oracle\_queries\_h}[m, \ring,W] = y $ and $ \mathtt{oracle\_queries\_h\_CP}[\ring, m, W, \compk, R', R_m'] = c $ where $ R' = s\genG + \delta K+ c\compk  $, $ R_m = s \hashG(m,\ring) + cW$. 
				%				Now, the signature verifies in the real world as well.
				%				\item If $ b = b_\simulator $, $ \simulator $ doesn't need to do anything.
				%			\end{itemize}
			


	Now, we need to show that the outputs of honest parties in the ideal world are indistinguishable from the honest parties in the real world. 
	
	\begin{lemma}\label{lem:honestoutput}
			The outputs of honest parties in the real protocol \name\ are indistinguishable from the output of the honest parties in $ \fgvrf $.
	\end{lemma}
		
		\begin{proof}
			\eprint{Clearly, the evaluation outputs of the ring signatures in the ideal world identical to the real world protocol because  the outputs are randomly selected by $ \fgvrf $ as the random oracle $ \hash $ in the real protocol.}{The evaluation outputs are indistinguishable.} The only difference between two world's output is the ring signatures of honest parties (See Algorithm \ref{alg:gensign}) since the pre-output $ W $ and $ \pi_{eval} $ are generated differently in Algorithm \ref{alg:gensign} than $ \rVRF.\Sign $. \eprint{The distribution of $ \pi_{eval} = (c,s_1, s_2) $ and $ \compk $ generated by Algorithm \ref{alg:gensign} and the distribution of $ \pi_{eval} = (c,s_1, s_2) $ and $ \compk $ generated by $ \rVRF.\Sign $ are from uniform distribution so they are indistinguishable.}{The distribution of $ \pi_{eval}  $ and $ \compk $ in two worlds are the same.} So, we are left to show that the anonymous key $ W $ selected randomly from $ \grE $ and pre-output $	 W $ generated by $ \rVRF.\Sign $ are indistinguishable given $ \pk  $. 
			
			\textbf{Case 1 ($ \pk \neq x \genG$):} If  $ \pk \neq x \genG$, then  $ \pk $ is uniformly random and independent from $ x $. Therefore, $ \env $ can distinguish ideal world honest signatures from the real world honest signatures at most with probability $ \frac{1}{2} $.
			
			\textbf{Case 2 ($ \pk = x \genG$):} We  show this under the assumption that the DDH problem  is hard.  In other words, we show that if there exists a distinguisher $ \mathcal{D} $ that distinguishes honest signatures in the ideal world and honest signatures in the real protocol then we construct another adversary $ \bdv $ which breaks the DDH problem. 
			We use the hybrid argument to show this.
			We define hybrid simulations $ H_{i} $ where  the signatures of first $ i $ honest parties are computed as described in $ \rVRF.\Sign $ and the rest are computed as in $ \fgvrf $. Without loss of generality, $ \user_1, \user_2, \ldots, \user_{n_h} $ are the honest parties. Thus, $ H_0 $ is equivalent to the honest of the ideal protocol  and $ H_{n_h}  $ is equivalent to  honest signatures in the real world.  We construct an adversary $ \bdv $ that breaks the DDH problem given that there exists an adversary $ \mathcal{D} $ that distinguishes hybrid games $ H_i $ and $ H_{i + 1} $ for $ 0 \leq i < n_h $. $\bdv $ receives the DDH challenges $ X,Y, Z \in \GG $ from the DDH game and simulates the game against $ \mathcal{D} $ as follows. 
			Then $ \bdv $ runs a simulated copy of $ \env $ and starts to simulate $ \fgvrf $ and $ \simulator $ for $ \env $. For this, it first runs the simulated copy of $ \adv $ as $ \simulator $ does. $  \bdv $ publishes $ \GG, \genG = Y, \genB $ as parameters of the ring VRF protocol. $\bdv $ generates the public key of all  honest parties' key as usual by running $ \rVRF.\KeyGen$ as $ \simulator $ does except party $ \user_{i+1} $. It lets the public key of $ \user_{i + 1} $ be $ X $.
		
%			\begin{algorithm}
%				
%				\caption{$\gen^{ind}_{sign}(\ring,\openring,W,\{X,\pk\},\aux,\msg)$}
%				\label{alg:gensignind}	 	
%				\begin{algorithmic}[1]
%					\State $ \compk \leftsample \grE$
%					\State $ \pi_{eval}  \leftarrow \NIZK_{\rcom}.\Simulate(\compk,W,\hashG(\msg)) $
%					%\State $ \pi_{eval} \leftarrow \NIZK.\mathsf{Simulate}(\rcom, (G, \genB,\GG,\compk,W,\In)) $
%					%\State \textbf{send} $(\oramsg{learn\_\tau},\sid)  $ to $ \gcrs $
%					%\State \textbf{receive} $(\oramsg{trapdoor},\sid, \tau,crs)  $ from $ \gcrs $
%					\If{$ \pk = \OpenRing(\comring,\openring) $}
%					\State $ \pi_{ring} \leftarrow \NIZK_{\rsnark}.\Simulate((\comring, \compk)) $ 
%					\State\Return$ \sigma = (\pi_{eval},\pi_{ring},\compk,\comring,W) $
%					\EndIf
%				\end{algorithmic}
%				
%			\end{algorithm}
		
			
			While simulating $ \fgvrf $, $\bdv $ simulates the ring signatures of first $ i $ parties by running $ \rVRF.\Sign $ and the parties $ \user_{i+2}, \ldots, \user_{n_h} $ by running Algorithm \ref{alg:gensign} where $ W $ is selected randomly. The simulation of $ \user_{i + 1} $ is different.  Whenever $ \user_{i+1} $ needs to sign a message $ m$, it obtains $ \In = \hashG(m) = hY $ from $ \mathtt{oracle\_queries\_gg} $ and lets $ W = hZ $. Then it lets $ \compk = X + \openpk\genB $, lets  $ \pieval \rightarrow \NIZK_{\rcom}.\Simulate(\compk,W,\hashG(m)) $ and  $ \pi_{ring} \leftarrow \NIZK_{\rsnark}.\Simulate((\comring, \compk)) $.  Remark that if $ (X,Y,Z)$ is a DH triple (i.e., $  \mathsf{DH}(X,Y,Z) \rightarrow 1 $), $ \user_{i+1} $ is simulated as in \name \ because $ W = x\In$ in this case. Otherwise, $ \user_{i+1} $ is simulated as in the ideal world because $ W $ is random. So, if $  \mathsf{DH}(X,Y,Z)  \rightarrow 1$, $\simulator $ simulates $ H_{i+1} $. Otherwise, it simulates $ H_{i} $. In the end of the simulation, if $ \mathcal{D} $ outputs $ i $, $\simulator $ outputs $ 0 $ meaning $  \mathsf{DH}(X,Y,Z) \rightarrow 0$. Otherwise, it outputs $ i + 1 $. The success probability of $\simulator $ is equal to the success probability of $\mathcal{D} $ which distinguishes $ H_i $ and $ H_{i +1} $. Since DDH problem is hard, $\simulator $ has negligible advantage in the DDH game. So, $ \mathcal{D} $ has a negligible advantage too. Hence, from the hybrid argument, we can conclude that $ H_0    $ which corresponds the output of honest parties in  the ring VRF protocol and $ H_q  $ which corresponds to  the output of honest parties in ideal world are indistinguishable.
			
			
		\end{proof}	
		This concludes the proof of showing the output of honest parties in the ideal world are indistinguishable from the output of the honest parties in the real protocol.
		Next we show that the simulation executed by $ \simulator $ against $ \adv $ is indistinguishable from the real protocol execution.
		
		\begin{lemma} \label{lem:simulation-ind}
			The view of $ \adv $ in its interaction with the simulator $ \simulator $ is indistinguishable from the view of $ \adv $ in its interaction with real honest parties.
		\end{lemma}
		
		
		\begin{proof}
			The  simulation against the real world adversary $ \adv $ is identical to the real protocol except the output of the honest parties and cases where $ \simulator $ aborts. 
			We show that the abort cases happen with a negligible probability during the simulation. $ \simulator $ aborts during the simulation of random oracles \eprint{$ \hash $ and}{} $ \hash_p $ and during the simulation of verification.\eprint{ We have already explained that the abort case during the simulation of $ \hash $ cannot happen.}{} The abort case happens in the simulation of $ \hash_p $ if $ W = hX $ where $ X = \sk\genG $ or if $ \mathtt{oracle\_queries\_h\_CP}[\comring,m,W,\compk,R,R_m] $ has already been defined by a value which is different than $ c $. \eprint{The first case happens in $ \hash_p $ if $ \fgvrf $ selects a random $ W \in \GG$ for an anonymous key of $ m, \pk  $ for the honest party with the other key $ X  $ and the random oracle $ \hashG $ selects a random $ h \in \FF_p  $ where $ \hashG(m) = hG $ and $ W = hX $. Clearly, this can happen with a negligible probability in $ \secparam $. The 
			second case happens in $ \hash_p $ if $ \adv $ queries with the input $ (\comring,m,W,\compk,R,R_m) $ before $ (\pi_1,\pi_2,\compk,\comring,W) $ generated by $ \gen_{sign} $. Since $ \compk $ is randomly selected by $ \fgvrf $, the probability that $ \adv $ guesses $ \compk $ before it is generated is negligible.}{They can be clearly happen with negligible probability.}
			Now, we are left with the abort case during the verification.
			For this, we show that if there exists an adversary $ \adv $ which makes $ \simulator $ abort during the simulation, then we construct another adversary $ \bdv $ which breaks either the CDH problem or the binding property of $ \rVRF.\KeyGen $.
			
			Consider a CDH game in a prime $ p $-order group  $ \grE $ with the challenges $ \genG,U, V \in \grE$. The CDH challenges are given to the simulator $ \bdv $. Then $ \bdv $ runs a simulated copy of $ \env $ and starts to simulate $ \fgvrf $ and $ \simulator $ for $ \env $. For this, it first runs the simulated copy of $ \adv $ as $ \simulator $ does. $ \bdv $ provides $ (\grE, p, \genG , \genB) $ as a public parameter of the ring VRF protocol to $ \adv $.	For the public keys of honest parties, $ \bdv $ picks a random $ r_x\in \FF_p $ and sets $ X =r_xV$. If $ \rVRF.\KeyGen $ is defined as $ \pk = \sk \genG $, it lets $ \pk $ be $ X $ otherwise it picks a random public key $ \pk $. Whenever $ \bdv $ needs to generate a ring signature for $ m $ for an honest party with a public key $ \pk $ mapped to $X $, it behaves exactly as $ \fgvrf $ except that it runs   Algorithm \ref{alg:gensignbdv} to generate the signature. 
%			Remark that $ \bdv$  never needs to know the secret key of honest parties to simulate them since $ \bdv $ selects anonymous keys randomly  and generates the ring signatures  without the secret keys. Since the public key generated by $ \rVRF.\KeyGen $ is random and independent from the secret key, $ \bdv $'s key generation is indistinguishable from $ \simulator $'s key generation.
			
			%			
			%			\begin{algorithm}
				%				\caption{$\gen_{W}(X, m)$}
				%				\label{alg:genWbdv}	 	
				%				\begin{algorithmic}[1]
					%					\If{$ DB_W[m, X] = \perp $}
					%					\State $ W \leftsample \GG$
					%					\State $ DB_W[m, X] := W $
					%				%	\State \textbf{add} $ W $ to list $ \anonymouskeylist[m,\ring] $
					%					\EndIf
					%					\State \textbf{return} $ DB_W[m, X] $
					%				\end{algorithmic}
				%			\end{algorithm}
			
			
			\begin{algorithm}
				\caption{$\gen_{sign}(\ring,W,\{X,\pk\},\aux,m)$}
				\label{alg:gensignbdv}	 	
				\begin{algorithmic}[1]
%					\State $ c,s_1, s_2 \leftsample \FF_p $
%					\State $ \pi_{eval}  \leftarrow (c,s_1, s_2)$
					\State $ \openpk \leftsample \FF_p $
					\State $ \compk =  X + \openpk \, K$
					\State $ \pieval \leftarrow \NIZK_{\mathcal{R}_{eval}}.\Simulate(\compk, W,\hashG(m)) $
					\State $ \comring, \openring \leftarrow \rVRF.\CommitRing(\ring) $
					\State $ \piring \leftarrow \NIZK_{\rsnark}.\Simulate(\comring, \compk) $ 
%					\State $ R' = s\genG +\delta K + c\compk$
%					\State $ R_m = s\hashG(m) + c W $
%					\State $ \aux' = \tmpaux $
%					\State $ \mathtt{oracle\_queries\_h\_CP}[\aux',m, \compk,W,R',R'_m] = c$						
					\State\Return$ \sigma = (\pi_{eval},\pi_{ring},\compk,\comring,W) $
				\end{algorithmic}
				
			\end{algorithm}
			
			
			The  signature of an honest party by $ \fgvrf$ the  signature generated by $ \bdv $ are the same. The only difference is that now $ \bdv $ does not need to set $ \hash_p $ so that $ \pieval $ verifies because $ \Gen_{sign} $ in Algorithm \ref{alg:gensignbdv} does it while simulating the proof for $ \rel_{eval} $. \eprint{Therefore, the simulation of $ \hash_p $ is simulated as a usual random oracle by $ \bdv $.}{$ \hash_p $ is simulated as a usual random oracle by $ \bdv$.}				
			$ \bdv $  sets up $ \evaluationslist[m,W] $ by querying queries $ m,W $ to  $ \hash $ described. \eprint{$ \bdv $ simulates the random oracle $ \hash $ as a usual random oracle.
			The only difference from the simulation of $ \hash $ by $ \simulator $ is that $ \bdv $ does not ask for the output of $ \hash(m,W) $ to $ \fgvrf $ but it does not change the simulation because now $\fgvrf $ asks for it.  Remark that since $ \hashG $ is not simulated as in Figure \ref{oracle:HgnoPK}, $ \bdv $ cannot check whether $ W $ is an anonymous key generated by an honest secret key or not.  However, it does not need this information because $ \hash $ is simulated as a usual random oracle. $ \bdv $ also simulates $ \hash_\ring $ for the ring commitments as a usual random oracle. 			
			Simulation of $ \hashG $ by $ \bdv $ returns $ hU $ instead of $ hG $. The simulation of $ \hashG $ is indistinguishable from the simulation of $ \hashG $ in Figure \ref{oracle:HgnoPK}. }{}
			
%			\begin{figure}
%				\centering
%				
%				\noindent\fbox{%
%					\parbox{\columnwidth}{%
%						\underline{\textbf{Oracle $ \hash$}} \\
%						\textbf{Input:} $ m,W $ 
%						
%						\textbf{if} $ \mathtt{oracle\_queries\_h}[m,  W] = \perp $
%						
%						\tab{$ y \leftsample \{0,1\}^{\ell_\rVRF} $}
%						
%						\tab{$  \mathtt{oracle\_queries\_h}[m,  W]:=y $}
%						
%						
%						\textbf{return $  \mathtt{oracle\_queries\_h}[m, W] $}
%						
%				}}	
%				\caption{The random oracle $ \hash $}
%				\label{oracle:HbyB}
%			\end{figure}
			
%			\begin{figure}
%				\centering
%				
%				\noindent\fbox{%
%					\parbox{\columnwidth}{%
%						\underline{\textbf{Oracle $ \hashG $}} \\
%						\textbf{Input:} $ m$ \\
%						\textbf{if} $\mathtt{oracle\_queries\_gg}[m] = \perp  $
%						
%						\tab{$ h \leftsample \FF_p $}
%						
%						
%						\tab{$ P \leftarrow hU $} 
%						
%						\tab{$\mathtt{oracle\_queries\_gg}[m] := h$}
%						
%						\textbf{else}:
%						
%						\tab{$ h \leftarrow \mathtt{oracle\_queries\_gg}[m] $}
%						
%						\tab{$ P \leftarrow hU$}
%						
%						\textbf{return $ \In$}
%						
%				}}	
%				\caption{The random oracle $ \hashG $}
%				\label{oracle:HgbyB}
%			\end{figure}

			
			During the simulation, when $ \adv $ outputs a signature $ \sigma = (\pi_{eval},\pi_{ring},\compk,\comring,W) $ of message $ m $ with $ \aux $ which is not recorded in $ \fgvrf $'s record, $ \bdv $ runs $ \rVRF.\Verify(\comring,m, \aux, \sigma) $. If it verifies, it finds the corresponding ring $ \ring $ of $ \comring $ by checking the random oracle $ \hash_\ring $'s database. Remark that there exists $ \ring $ where Merkle tree root of $ \ring $ is $ \comring $ because if it was not the case $ \sigma $ would not verify which also checks  $ \pi_{ring} $. 
			Then it runs the extractor algorithm of $ \NIZK_{\Rring} $ and obtains $ X = \compk  - \openpk \, \genB $. 
			If $ \pk = \rVRF.\OpenRing(\comring,\openring) $ is not an honest key then $\bdv $ adds $ W $  to $ \anonymouskeylist[m, \ring] $.  If $ \pk $ is not a malicious key but $ X $ is  generated for honest parties by $ \bdv $ while simulating $ \simulator $, $ \bdv $ aborts \footnote{This case never happens if $ \pk  $ is defined $ \sk\genG $}. The abort case happen with a negligible probability because all the outputs seen by the adversary are independent from $ X $.
			Otherwise, it  runs the extractor algorithm of $ \NIZK_{\rcom} $ and obtains $(\hat{\sk},\hat{\openpk} )$ such that $ \compk = \hat{\sk}\genG + \hat{\openpk} \, \genB $ and $ W = \hat{\sk} \hashG(m) $. If  $ W \notin \anonymouskeylist[m, \ring] $, $ \bdv $ increments  $ \counter[m,\ring] $ and adds $ W $ to $ \anonymouskeylist[m,\ring] $ for $ \Rring $.

			If $ X  $ is a key which is generated by $ \bdv $ and $ X = \hat{\sk}G $, $ \bdv $ solves the CDH problem as follows: $ W = \hat{\sk} h U $ where $ h = \mathtt{oracle\_queries\_gg}[m] $. Since $ X = r V $, $ W = \hat{\sk}huG =rhuV $. So, $ \bdv $ outputs $ r^{-1}h^{-1}W $ as a CDH solution and simulation ends. Remark that this case happens when $ \simulator $ aborts because of \ref{cond:forgery}.
			
			If $ \anonymouskeylist[m,\ring] \geq |\ring_{mal}| =t$, $ \bdv $ obtains all the signatures $ \{\sigma_i\}_{i =1}^t $ that make $ \bdv $ to add an anonymous key to $ \anonymouskeylist[m,\ring] $. Then it solves the CDH problem as follows: Remark that this case happens when $ \simulator $ aborts because of \ref{cond:uniqueness}.
			
			For all $ \sigma_j = (\pieval,\piring,\compk_j,W_j) \in \{\sigma_i\}_{i =1}^t $, $ \bdv $ runs extractor for $ \Rring $ and obtains $\openring_j, \openpk_j$. Then it obtains the public key $ \pk_j = \rVRF.\OpenRing(\ring, \openring_j) $ where $ \pk_j \in \ring $ and  $ X_j = \compk- \openpk \genB $. Then it adds $ X_j $ to a list $ \setsym{X}  $ and $ \pk_j $ to a set $ \setsym{PK} $. One of the following cases happens:
			
		
			- All $ X_j $ in $ \setsym{X} $ are different and $ |\setsym{PK}| \leq t $, $ \bdv $ aborts: Each $ \pk \in \setsym{PK} $ commits to a secret key $ \sk $. Since it is a binding commitment there exists one opening $ r $ except with a negligible probability. Since $ \piring $ verifies in $ \Rring $ whether $ \rel_{\pk} $ is satisfied, if  $ X_j $ in $ \setsym{X} $ are different and $ |\setsym{PK}| \leq t $, means that the binding property is broken. Therefore, $ \bdv  $ aborts with a negligible probability. We note $ \bdv $ can be in this case only if $ \pk \neq \sk \genG $.
			
			- All $ X_j $ in $ \setsym{X} $ are different and $ |\setsym{PK}| > t $: If $ \bdv $ is in this case, it means that there exists one commitment public key $ X_a \in \setsym{X} $ which belongs to an honest party or . Then $ \bdv $ runs the extractor algorithm of $ \NIZK_{\rcom} $ and obtains $ \hat{\sk}_a, \hat{\openpk} $ such that $ \compk_a = \hat{\sk}_a\genG + \hat{\openpk}_a \,\genB $ and $ W_a = \hat{\sk}_a \hashG(m) $.  If $ \bdv $ is in this case, $ \hat{\sk}_a\genG\neq X_a $ because otherwise it would solve the CDH as described before. Therefore, $ \openpk_a \neq \hat{\openpk}_a $. Since $ X_a + \openpk_a \, \genB = \hat{\sk}_a\genG + \hat{\openpk}_a \,\genB  $ and $ X_a = r_aV $ where $ r_a $ is generated by $ \bdv $ during the key generation process, $ \bdv $ obtains a representation of $ V = \gamma \genG + \delta \genB $ where $ \gamma = \hat{\sk}_ar^{-1}_a  $ and $ \delta = (\hat{\openpk}_a -\openpk)\,r_a^{-1} $. Then $ \bdv $ stores $ (\gamma, \delta) $ to a list $ \mathsf{rep} $. If $ \mathsf{rep} $ does not include another element $ (\gamma', \delta')  \neq (\gamma, \delta) $, $ \bdv $ rewinds $ \adv $ to the beginning with a new random coin.  Otherwise, it obtains $ (\gamma', \delta') $ which is another representation of $ V $ i.e., $ V = \gamma' \genG + \delta' \genB $. Thus, $ \bdv $ can find discrete logarithm of $ V $ on base $ G $ which is $ v = \gamma + \delta \theta $ where $ \theta = (\gamma - \gamma')(\delta' - \delta)^{-1} $. $ \bdv $ outputs $ vU $ as a CDH solution.
				
				
			- There exists at least two $ X_a,X_b \in \setsym{X} $ where $ X_a = X_b $. $ \bdv $ runs the extractor algorithm of $ \NIZK_{\rcom} $ for $ \pi_{ring_a} $ and $ \pi_{ring_b}  $ and obtains $(\hat{\sk}_a,\hat{\openpk}_a )$ and $(\hat{\sk}_b,\hat{\openpk}_b )$, respectively\eprint{ such that $ \compk_a = \hat{\sk}_a\genG + \hat{\openpk}_a \, \genB \compk_b = \hat{\sk}_b\genG + \hat{\openpk}_b \, \genB $ and $ W_a = \hat{\sk}_a \hashG(m), W_b = \hat{\sk}_b \hashG(m) $}{}. Since $ W_a \neq W_b $, $ \hat{\sk}_a \neq \hat{\sk}_b $.  So, $ \bdv $ can obtain  two different and non trivial representation of $ X_a = X_b $ i.e., $ X_a = X_b = \hat{\sk}_a\genG + (\hat{\openpk}_a - \openpk_a) \, \genB = \hat{\sk}_b\genG + (\hat{\openpk}_b - \openpk_b) \, \genB  $. Thus, $ \bdv $ finds the discrete logarithm of $ K = U $ in base $ G $ which is $ u = \frac{\hat{\sk}_a - \hat{\sk}_b}{\hat{\openpk}_a -\openpk_a -\hat{\openpk}_b + \openpk_b} $. $ \bdv $ outputs $ uV $ as a CDH solution.

			
			
			
			
			
			
			
			
			
			
			
			
			%	
			%		
			%		
			%			$ \bdv $ solves CDH if $ \bdv $ is in the abort case of simulation of $ \hash$ in Figure \ref{oracle:H} by outputting $ r^{-1}h^{-1}W $ is the CDH solution of $ U,V $. $ r^{-1}h^{-1}W $ is the CDH solution because $ vU $ is a solution of $ CDH $ where $ V =  $
			%		
			%		is in this case $ \bdv $ outputs  $ r^{-1}h^{-1}W $  where $ X^* = rV $ and $ h = \mathtt{oracle\_queries\_gg}[m,\ring] $ and simulation ends. Remark that if $ \bdv $ aborts during the simulation of $ \hash $ it means that $ X^* $ belongs to an honest party and $ X^* =  h^{-1}W = rV = rvG$.  Therefore, $ r^{-1}h^{-1}W $ is the CDH solution of $ U,V $.
			%				  
			%		During the simulation if $ \bdv $ sees a valid forgery ring signature  $ m, \ring, \sigma = (\pi_{eval}, \pi_{ring}, C, W) $ where $ W $ is an anonymous key generated by $ \bdv $ for $ (m',\ring') \neq (m, \ring) $, $ \bdv $ aborts. $ \Pr[x\hashG(m',\ring') = W; xG \in \ring'| \ring',W] $ is negligible because $ \hashG $ is a random oracle.
			%		%TODO exact probability
			%				  
			
			%		During the simulation if $ \bdv $ sees a forgery ring signature  $ m, \ring, \sigma = (\pi_{eval}, \pi_{ring}, C, W) $ where $ X = h^{-1}W $ is an honest key, then $ \bdv $ does the following: It runs the extractor algorithms on $ \pi_{ring} $ i.e., $ \ext(\rsnark,..) $ and obtains $ X' \in \ring $ and $ \openpk' $ where $ C = X' + \openpk' K $ and $ \pi_{eval} $ i.e., $ \ext(\rcom,..)  \rightarrow x, \openpk$ where $ C = x\genG + \openpk K$ and $ W = x\hashG(m, \ring)= xhV$.  Then, $ \bdv $ outputs the CDH of $ U, V $ which is $r^{-1}xU  $. This is correct CDH solution because $ X= rV = xG $, $ V= r^{-1}xG $.
			%		We remark  a forgery signature corresponds to the abort case of $ \simulator $ during the verification because $ \fgvrf $ is in \ref{cond:forgerymalicious}, $ \pk_\simulator  $ is an honest party's key. 
			
			%		During the simulation if $ \bdv $ sees $  k > |\ring_m| $-valid and malicious ring signatures $ \{\sigma_1, \sigma_2, \ldots,\sigma_k\} $ of the message $ m$ signed by $\ring $ whose anonymous keys are $ \{W_1, W_2, \ldots, W_k\} $, respectively , it runs $ \ext(\rsnark,..) $ for each valid malicious signatures $ \sigma_i $(signatures that are not generated by $ \bdv $) and obtains $ \openpk'_i, X'_i \in \ring $. In this case, the one of following two cases must happen:
			%		
			%		\begin{itemize}
				%			\item There exists $ X'\in \ring$ which is an honest key. In this case, $ \bdv  $ runs $ \pi_{eval} $ i.e., $ \ext(\rcom,..)  \rightarrow x, \openpk$ and stores $ r V = x^*\genG + (\openpk - \openpk')K = x^*\genG + b K$ to $DB $. If $ DB $ is empty, rewind $ \adv $ to the beginning of the simulation. If it is not empty i.e., there exists $ \hat{r}X = \hat{\sk} \genG + \hat{b}K $, then $ \bdv $ first checks whether $ r = \hat{r} $. If it is the case, it aborts. If it is not the case, it finds the discrete logarithm of $ G=U $ on base $ K $ which is $ t = \frac{\hat{r}^{-1}\hat{b}^*-{r}^{-1}b}{{r}^{-1}\hat{\sk}-\hat{r}^{-1}\hat{\sk}} $.  Then, it outputs the CDH of $ U,V $ which is $ tV $. 
				%			We remark that $ \bdv $ aborts after rewinding with a negligible probability because it selects $ r $ randomly.
				%			%TODO exact probability
				%			\item  There exists $ X' \in \ring$ which is the  output of two different signatures.
				%		\end{itemize}
			%		
			%		
			%		During the simulation if $ \bdv $ sees a valid ring signature  $ m, \ring, \sigma = (\pi_{eval}, \pi_{ring}, C, W) $ where $ X = h^{-1}W \notin \ring$, then $ \bdv $ does the following: It runs the extractor algorithms on $ \pi_{ring} $ i.e., $ \ext(\rsnark,..) $ and obtains $ X' \in \ring $ and $ \openpk' $ where $ C = X' + \openpk' K $ and $ \pi_{eval} $ i.e., $ \ext(\rcom,..)  \rightarrow x, \openpk$ where $ C = x\genG + \openpk K$ and $ W = x\hashG(m, \ring)= xhV$. In this case, $ X \neq X' $, so $ \openpk \neq \openpk' $. 
			%				  
			%				  
			%		\begin{itemize}
				%			\item If $ X' $ is honest, then store $ r V = x^*\genG + (\openpk - \openpk')K = x^*\genG + b K$ to $DB $. If $ DB $ is empty, rewind $ \adv $ to the beginning of the simulation. If it is not empty i.e., there exists $ \hat{r}X = \hat{\sk} \genG + \hat{b}K $, then $ \bdv $ first checks whether $ r = \hat{r} $. If it is the case, it aborts. If it is not the case, it finds the discrete logarithm of $ G=U $ on base $ K $ which is $ t = \frac{\hat{r}^{-1}\hat{b}^*-{r}^{-1}b}{{r}^{-1}\hat{\sk}-\hat{r}^{-1}\hat{\sk}} $.  Then, it outputs the CDH of $ U,V $ which is $ tV $. 
				%			We remark that $ \bdv $ aborts after rewinding with a negligible probability because it selects $ r $ randomly.
				%			%TODO exact probability
				%				  	
				%			\item If $ X' $ is  a malicious key, $ \bdv $ runs the extractor algorithm on PoK proof $ \pi_{dl} $ of $ X' $ i.e., $ \ext(\rdl,..,) $ which outputs $ x' $ where $ X' = x'G $. Since $ x' \neq x^* $, $ \bdv  $ has a Pedersen commitment $ C $ with two openings so it can find the discrete logarithm of $ K$ on base $ G $ which is $ t = \frac{x^* - x'}{\openpk' - \openpk^*} $.  In the end, it outputs the CDH of $ X,Y $ which is $ tU $. 
				%
				%		 \end{itemize}
			%		
			%	
			So, the probability of $ \bdv $ solves the CDH problem is equal to the probability of $ \adv $ breaks the forgery or uniqueness in the real protocol. Therefore,  if there exists $ \adv $ that makes $ \simulator$ aborts during the verification, then we can construct an adversary $ \bdv $ that solves the CDH problem except with a negligible probability.
			
			
				  
		\end{proof}
		This completes the security proof of our ring VRF protocol.\qed
	\end{proof}


\eprint{\section{Ring VRF Variations}
\label{sec:morefuncs}
In this section, we give a ring VRF functionality which gives more security properties than the basic ring VRF functionality $ \fgvrf $ that we define in Figure \ref{f:gvrf}.
%
%\newcommand{\faux}{\fgvrf^{\mathsf{aux}}}
%\subsection{Ring VRF with Associated DATA}
%We define a variation of ring VRF which signs also an associated data (aux) along with a message. It is very similar to $ \fgvrf $. It additionally requires unforgeability notion for $ \aux $ as well. See Figure \ref{f:aux} for the details.
%\begin{figure}
%\begin{tcolorbox}
%	{  $ \faux $ runs two PPT algorithms $\gen_{sign} $ during the execution.
%		
%		\begin{description}
%			
%			\item[Key Generation.] Same as in $ \fgvrf $.
%			
%			\item[Malicious Ring VRF Evaluation.] Same as in $ \fgvrf $.
%			
%			\item[Honest Ring VRF Signature.] upon receiving a message $(\oramsg{sign}, \sid, \ring, \pk_i, m, \underline{aux})$ from $\user_i$, verify that $\pk_i \in \ring$ and that there exists a public key $\pk_i$ associated to $\user_i$ in the table $ \vklist $. If that wasn't the case, just ignore the request. 	
%			If there exists no $ W' $ such that $ \anonymouskeymap[m,W'] =  \pk_i $, let $ W \leftarrow \setsym{S}_W$. Then, let $y \leftsample \setsym{S}_{eval}$ and set $ \anonymouskeymap[W] = (m,\pk_i) $ and set $ \evaluationslist[m, W] = y$.
%			%					\begin{itemize}
%				%						%\item If there exists $ W \in  \anonymouskeymap  $, abort.
%				%						\item Else 
%				%						%TODO define what \in \anonymouskeymap mean
%				%					\end{itemize}
%			%			    \end{itemize}
%		Obtain $ W, y $ where  $ \evaluationslist[m, W] = y$, $ \anonymouskeymap[m,W] = \pk_i $ and run  $ \gen_{sign}(\ring, W, m,\underline{aux}) \rightarrow \sigma $. Verify that $ [m, \underline{aux},W, \sigma, 0] $ is not recorded. If not, abort. Otherwise, record $ [m,\underline{aux}, W, \sigma, 1] $. Return $(\oramsg{signature}, \sid, \ring,W,m,\underline{aux}, y, \sigma)$ to $\user_i$.
%		\item[Ring VRF Verification.] Same as in $ \fgvrf $ except that $ \faux $ checks records $  [m,\underline{aux},W,\ring,\sigma, b]   $ in the places where $ \fgvrf $ checks $ [m,W,\ring,\sigma, b] $.
%	\end{description}
%	
%}
%\end{tcolorbox}
%\caption{Functionality $\faux$.\label{f:aux}}
%\end{figure}



%
%\begin{theorem}
%\name \ with AD over the group structure $ (\GG,p,\genG,\genB) $ realizes $ \faux $ in Figure \ref{f:aux} in the random oracle model assuming that NIZK is zero-knowledge and knowledge extractable, the decisional Diffie-Hellman (DDH) problem are hard in $ (\GG,p,\genG,\genB)  $. 
%\end{theorem}
%
%\begin{proof}
%The proof is similar to the proof of Theorem \ref{thm:rvrf}. $ \gen_{sign} $ as well as $ \mathtt{oracle\_queries\_h\_schnor} $ simulated by $ \bdv $ takes the input $ aux $ in Algorithm \ref{alg:gensignbdv}.
%\end{proof}




\newcommand{\frvrfsec}{\fgvrf^s}
\subsection{Secret Ring VRF}
We also define another version of $ \fgvrf $ that we call $ \frvrfsec $. $ \frvrfsec $ operates as $ \fgvrf $. In addition, it also lets a party generate a secret  element to check whether it satisfies a certain relation i.e., $ ((m,y), (\eta, \pk_i)) \in \rel $ where $ \eta $ is the secret random element. If it satisfies the relation, then $ \frvrfsec $ generates a proof. Proving works as $ \mathcal{F}_{zk} $ \cite{zkfunc} except that a part of the witness ($ \eta $) is generated randomly by the functionality. $ \frvrfsec $ is useful in applications where a party wants to show that the random output $ y $ satisfies a certain relation without revealing his identity.

\begin{figure}
	%	\sassafras{\scriptsize}{\scriptsize}
	\begin{tcolorbox}
		{
			%\par\hrulefill\\
			$ \frvrfsec$ for a relation $ \mathcal{R} $ behaves exactly as $ \fgvrf $. Differently, it has an algorithm $ \gen_{\pi} $ and  it additionally does the following:
			\begin{description}
				\item [Secret Element Generation of Malicious Parties.]upon receiving a message $(\oramsg{secret\_rand}, \sid, \ring,\pk,W, m)$ from $\simulator$, verify that $ \anonymouskeymap[m,W] =  \pk_i $. If that was not the case, just ignore the request. If $ \evaluationsecretlist[m,W] $ is not defined, obtain $ y = \evaluationslist[m, W] $. Then, run $ \gen_{\eta}(m,\pk_i, y) \rightarrow \eta  $ and store $ \evaluationsecretlist[m,W] = \eta $. Obtain $ \eta = \evaluationsecretlist[m,W] $  and return $(\oramsg{secret\_rand}, \sid, \ring, W, \eta)$ to $ \user_i $.
				
				\item[Secret Random Element Proof.] upon receiving a message $(\oramsg{secret\_rand}, \sid, \pk, W, m)$ from $\user_i$, verify that $ \anonymouskeymap[m,W] =  \pk_i $. If that was not the case, just ignore the request. If $ \evaluationsecretlist[m,W] $ is not defined, run  $ \gen_{\eta}(m,\pk_i, y) \rightarrow \eta  $ and store $ \evaluationsecretlist[m,W] = \eta $. Obtain $ \eta \leftarrow\evaluationsecretlist[m,W] $ and $ y \leftarrow \evaluationslist[m,W] $. If $ ((m, y),(\eta,\pk_i)) \in \mathcal{R} $,  run  $ \gen_\pi(W, m) \rightarrow \pi $ and add $ \pi $ to a list $ \proofzklist[m, W] $. Else, let $ \pi  $ be $ \perp $. Return $(\oramsg{secret\_rand}, \sid, W, \eta, \pi)$ to $ \user_i $.
				
				\item[Secret Verification.] upon receiving a message $(\oramsg{secret\_verify}, \sid, W, m, \pi)$, relay the message to $ \simulator $ and receive $(\oramsg{secret\_verify}, \sid, W, m, \pi, \pk,\eta)$. Then,
				
				\begin{itemize}
					\item if $ \pi \in \proofzklist[m,W,\ring] $, set $ b = 1 $.
					\item else if $ \evaluationsecretlist[W,m] = \eta$ and $ ((m, y, \ring),(\eta,\pk_i)) \in \mathcal{R} $, set $ b = 1 $ and add to the list $ \proofzklist[m,W,\ring] $.
					\item else set $ b = 0 $.
				\end{itemize}
				Send $(\oramsg{verification}, \sid, \ring, W, m, \pi, b)$ to $ \user_i $.
			\end{description}
		}
	\end{tcolorbox}
	\caption{Functionality  $ \frvrfsec $.\label{f:gvrfzk}}
\end{figure}
}{}
\section{$\SpecialG$ as Instantiation of $\ZKCont$}
\label{appendix_specialg}

Below we describe $ \SpecialG $ in more details. We start by giving a reminder about   
Quadratic Arithmetic Program (QAP)~\cite{LegoSNARK}, ~\cite{GGPR13} and related $\relRQ$ in a standard way. 

\begin{definition}[QAP] 
\label{def:QAP}
A Quadratic Arithmetic Program (QAP) $\cQ = (\cA, \cB, \cC, t(X))$ of size $m$ 
and degree $d$ over a finite field $\F_q$ is defined by three sets of polynomials $\cA = \{a_i(X)\}_{i=0}^m$, 
$\cB = \{b_i(X)\}_{i=0}^m$, $\cC = \{c_i(X)\}_{i=0}^m$, each of degree less than $d-1$ and a target degree $d$ polynomial $t(X)$. Given 
$\cQ$ we define $\relRQ$ as the set of pairs $((\bary, \barx); \baromega) \in \F_q^{l} \times \F_q^{n-l} \times \F_q^{m-n}$ for which it 
holds that there exist a polynomial $h(X)$ of degree at most $d-2$ such that:
$$(\sum_{k=0}^m v_k \cdot a_k(X)) \cdot (\sum_{k=0}^m v_k \cdot b_k(X)) = (\sum_{k=0}^m v_k \cdot c_k(X)) + h(X)t(X) \ \ (\ast)$$ 
where $\barv = (v_0, \ldots, v_m) = (1, x_1, \ldots, x_n, w_1, \ldots w_{m-n})$ and $\bary = (x_1, \ldots, x_l)$ and 
$\barx = (x_{l+1}, \ldots, x_n)$ and $\baromega = (w_1, \ldots, w_{m-n})$. 
\end{definition}

\noindent Given notation provided in section~\ref{sec:background}, in particular elliptic curve $\ecE$, its pairing $e$ and 
the related source, target groups and generators, we introduce%Let $\mathbb{F}_q$ be a prime field, 
%let $G_1$, $G_2$, $G_T$ be as defined in section~\ref{??}, let $e$, $g$, $h$ be defined as $\ldots$. Let $t(X)$ and
%$\{u_i(X),v_i(X),w_i(X)\}_{i=0}^m$ be polynomials in $\F_q[X]$, let $\ldots$ be $\ldots$ such that there exists $h(X) \in \F_q[X]$ with
% $$ \sum_{i=0}^m a_iu_i(X) \cdot  \sum_{i=0}^m a_iv_i(X)  = \sum_{i=0}^m a_iw_i(X) + h(X)t(X)  \ (\ast)$$
%Then let $\relone = \{ (;) \ | \ (;)  (\ast) \}$

\begin{definition}[Specialised Groth16 ($\SpecialG$)]
\label{insta:sg16} Let $\relRQ$ be as mentioned above. We call 
specialised Groth16 for relation $\relRQ$ the following: instantiation of the zero-knowledge continuation notion from Definition~\ref{def:zk_cont}:
\begin{itemize}
\item $\SpecialG.\Setup: (1^{\lambda}, \relRQ) \mapsto (\crs, \tw,\pp)$. \\ 
\noindent Let $\alpha, \beta, \gamma, \delta, \tau, \eta  \xleftarrow{\$} \F_q^{*}$. 
Let $\tw = (\alpha, \beta, \gamma, \delta, \tau, \eta)$. \\ 
Let $\crs = (\barsig_1, \barsig_2)$ where 
\begin{align*}
\barsig_1 = (&\alpha \cdot \gone, \beta \cdot \gone, \delta \cdot \gone, \{\tau_i \cdot \gone\}_{i=0}^{d-1}, \\
& \left\{\frac{\beta a_i(\tau)+ \alpha b_i(\tau)+ c_i(\tau)}{\gamma} \cdot \gone \right\}_{i=1}^n, \frac{\eta}{\gamma} \cdot \gone, \\ 
&\left\{\frac{\beta a_i(\tau)+ \alpha b_i(\tau)+ c_i(\tau)}{\delta} \cdot \gone \right\}_{i=n+1}^m, \\
& \left\{\frac{1}{\delta}\sigma^it(\sigma)\cdot \gone \right\}_{i=0}^{d-2}, \frac{\eta}{\delta}\cdot \gone), \\
\barsig_2 = (&\beta \cdot \gtwo, \gamma \cdot \gtwo, \delta \cdot \gtwo, \{\tau^i \cdot \gtwo\}_{i=0}^{d-1}). 
\end{align*} 

%\item $\SpecialG.\Gen: (\crs, \relRQ) \mapsto (\pp, \crspk, \crsvk)$ where \\
%$\crspk = \left([\barsig_1]_1, [\beta]_2, [\delta]_2, \left\{[\tau^i]_2\right\}_{i=0}^{d-1}\right)$  \\ 
%$\crsvk = \left([\alpha]_1, \left\{ \left[ \frac{\beta a_i(\tau)+ \alpha b_i(\tau)+ c_i(\tau)}{\gamma} \right]_1 \right\}_{i=1}^{l}, 
%[\beta]_2, [\gamma]_2, [\delta]_2\right)$  \\ 
$\pp = \left( \left \{ \frac{\beta a_i(\tau)+ \alpha b_i(\tau)+ c_i(\tau)}{\gamma} \cdot \gone \right \}_{i=l+1}^{n}, \frac{\eta}{\gamma} \cdot \gone \right)$. \\
\noindent Moreover, for simplicity and later use, we call \\
$\Kgamma = \frac{\eta}{\gamma} \cdot \gone$  and $\Kdelta = \frac{\eta}{\delta} \cdot \gone$.


%{We should say what is the difference from Groth16.Setup or it is the same. I think in general in $ \SpecialG $, you should tell which part is from Groth16 or Legosnark and where we %change it while describing the algorithms. It will be much clear for the reader to verify. You have notes in the end but I think it is better to have it while describing since you can tell %more right away from the algorithm than in the end of everything}


\item $\SpecialG.\Preprove: (\crs, \bar{y}, \bar{x}, \baromega_1, \relRQ) \mapsto (X', \pi', b')$ such that \\
\begin{align*}
&\ \ \ \ \ \ \ \   b' = 0; r, s\xleftarrow{\$} \F_p; X' = \sum_{i=l+1}^{n} v_i \cdot  \frac{\beta a_i(\tau)+ \alpha b_i(\tau)+ c_i(\tau)}{\gamma} \cdot \gone;  \\
&\ \ \ \ \ \ \ \  o = \alpha + \sum_{i=0}^{m} v_i \cdot a_i(\tau) + r \cdot \delta; u = \beta + \sum_{i=0}^{m} v_i \cdot b_i(\tau) + s \cdot \delta; \\ 
&\ \ \ \ \ \ \ \   v = \frac{\sum_{i=n+1}^{m} (v_i (\beta a_i(\tau)+ \alpha b_i(\tau)+ c_i(\tau))) + h(\tau)t(\tau)}{\delta}   + \\ 
& \ \ \ \ \ \ \ \ \ \ \ \ \  \ \   + o\cdot s + u \cdot r - r \cdot s \cdot \delta; \\
&\ \ \ \ \ \ \ \ \ \pi' = (o \cdot \gone, u \cdot \gtwo, v \cdot \gone), 
\end{align*}
where $\bary = (x_1, \ldots, x_l)$, $\barx = (x_{l+1}, \ldots, x_n)$, \\
$\baromega = (w_1, \ldots, w_{m-n})$, $\barv = (1, x_1, \ldots, x_n, w_1, \ldots, w_{m-n})$ (same as per definition of QAP).

\item $\SpecialG.\Reprove: (\crs, X', \pi', b', \relRQ) \mapsto (X, \pi, b)$  such that
\begin{align*}
&\ \ \ \ \ \ \ \  b , r_1, r_2  \xleftarrow{\$} \F_p, X = X' + (b- b') \Kgamma, \pi = (O, U, V), \\
&\ \ \ \ \ \ \ \ O = \frac{1}{r_1} O', U = r_1 U' + r_1r_2 \delta \gtwo, V = V' + r_2O'  - (b - b') \Kdelta \mathperiod
\end{align*}
\noindent where $\pi' = (O', U', V')$.
 
\item $\SpecialG.\VerifyCom: (\pp, X, \barx, b) \mapsto 0/1$ where the output is 1 iff the following holds
$$X = \sum_{i=l+1}^{n} x_i \cdot  \frac{\beta a_i(\tau)+ \alpha b_i(\tau)+ c_i(\tau)}{\gamma} \cdot \gone  + b \Kgamma,$$
where $\barx = (x_{l+1}, \ldots, x_n)$, $ 0 \leq l \leq n-1$. 

\item $\SpecialG.\Verify: (\crs, \bar{y}, X, \pi, \relRQ) \mapsto 0/1$ where the output is 1 iff the following holds 
$$e(O,U) = e(\alpha \cdot \gone, \beta \cdot \gtwo) \cdot e(X + Y, \gamma \cdot \gtwo) \cdot e(V, \delta \cdot \gtwo),$$
where $\pi = (O, U, V)$, $Y = \sum_{i=1}^{l} x_i \cdot \frac{\beta a_i(\tau)+ \alpha b_i(\tau)+ c_i(\tau)}{\gamma}  \cdot \gone$ 
and $\bary = (x_1, \ldots, x_l)$.

\item $\SpecialG.\Sim: (\tw, \bary, \relRQ) \mapsto (\pi, X)$ where \\ $x, o, u \xleftarrow {\$} \F_p$ and let 
$\pi = (o \cdot \gone, u  \cdot \gtwo , v \cdot \gone)$ where \\ $v = \frac{o\cdot u - \alpha \beta - \sum_{i=1}^{l} x_i (\beta a_i(\tau)+ \alpha b_i(\tau)+ c_i(\tau))- x}{\delta}  $ and, 
by definition $\bary = (x_1, \ldots, x_l)$. Note that $\pi$ is a simulated proof for transparent input $\bary$ 
and commitment $X = x \cdot \gone$.
\end{itemize} 
\end{definition}



\end{document}
\endinput
%%
%% End of file `sample-sigconf.tex'.
