\section{Introduction}

We introduce a novel cryptographic primitive called \emph{a ring verifiable random function (ring VRF)}. Ring VRF  operates in a manner akin to both VRF \cite{vrf_micali}  and ring signatures \cite{ring_accountable,ring_efficient,ring_linkable,ring_noRO,ring_sublinear}, leveraging the properties of uniqueness, pseudorandomness, and anonymity. In ring VRF, a user can generate a ring VRF output, which is a \emph{unique} pseudorandom number, with their key and  input  similar to VRF. They also sign the input with a set of public keys  (ring) including their key, similar to the ring signatures, to generate a ring signature which assures that the output is the unique output of the input generated with one of the public keys. Thus, the verification process does not reveal the owner of the  output except that their key is in the public key set.


When utilizing a ring VRF scheme, it may become necessary to generate multiple ring VRF signatures of multiple inputs for the same ring. An ideal solution to this would be generating a new signature by simply rerandomizing an existing signature of a different input, all while maintaining both verifiability and anonymity.
To this end, we introduce a  new notion called \emph{zero-knowledge continuations}. It provides a way to efficiently prove a statement with a simple transformation of an existing proof of the same statement. After this transformation the new proof remains unlinkable to the other proofs. This powerful tool help a ring VRF signer to effortlessly produce a new ring VRF signature for the same ring, simply by rerandomizing. Thus, it enables the construction of more efficient ring VRF schemes.

The distinctive properties of ring VRF such as pseudorandomness, anonymity and uniqueness offer  an efficient alternative for anonymous access control systems. Imagine an identity system where a user registers with their public key. Assuming that the system maintains a fixed input for a given service (e.g., urls) and provides a public commitment of the registered public keys, a registered user  can create a ring VRF output using the fixed input and their key, which serves as their pseudonym.  The user can then use this pseudonym as an identity while accessing a service provided by the system. At the same time, they can prove that their pseudonyms  are  legitimate  all without revealing their true identity. Namely, they generate a ring VRF signature which shows that their pseudonym is associated with one of the registered users. In this way, the identity system protects the user's  privacy. Moreover, the  system is protected against the Sybil behaviour, as the ring VRF protocol ensures that a user can produce only one pseudonym per input. This protection enables the system to ban certain pseudonyms in cases of abusive behaviours. Thus, the abusive user loses the access since they cannot generate  another legitimate pseudonym for this particular service.
In current anonymous systems, user accountability is primarily addressed through two main approaches: (1) allowing users to authenticate for a fixed duration \cite{limited_authentication1,limited_authentication2,limited_authentication3}, and (2) incorporating mechanisms for privacy revocation administered by a central authority \cite{revocation1,revocation2,revocation3,revocation4}, or through privacy revocation using anonymous committees \cite{anonymous-committee1,anonymous-committee2}.
In contrast, Ring VRF offers a straightforward and efficient solution for user accountability when compared to existing methods as it neither imposes limitations on user behaviours nor necessitates the involvement of central authorities or anonymous committees to revoke the privacy of a malicious user.

In addition to facilitate anonymous authentication, ring VRF  serves as a potent tool for the concept of proof-of-personhood (PoP) \cite{pop2008,pop2017,pop2020} to establish a connection between the physical entities and virtual identities by preserving the accountability and anonymity of the entity. 
In this context, an individual can physically enrol in a designated system (e.g., an issuer responsible for issuing identity cards) with a ring VRF key. Then, they present their ring VRF outputs as a virtual identity to another system, with the ring VRF signature serving as verifiable evidence of their physical existence. This process maintains their anonymity across different contexts since each ring VRF output of a corresponding system is unlinkable. 
The use cases of a ring VRF  can be extended in various potential  applications such as rate limiting systems, rationing systems or secret leader elections but we focus in this paper to define the security model of a ring VRF  and design efficient constructions. 

In short, our contributions in this paper are as follows:
 \begin{itemize}
 	\item We formally define the security of a ring VRF in the universal composability (UC) model. For this, we construct a functionality $ \fgvrf $ and verify the security properties that $ \fgvrf $ provides.
 	
 	\item We introduce a novel notion called zero-knowledge continuations  which defines the transformation of a valid proof into another valid and unlinkable proof of the same statement through efficient operations. Essentially, this allows a prover to generate an initially costly proof and subsequently reuse it by simply rerandomizing it,  while maintaining unlinkability with other proofs. 
 	%This novel notion provides us with a practical tool for developing more efficient ring VRF constructions.
 	
 	\item We construct two distinct  ring VRF frameworks. The first framework is designed to be utilized with a non-interactive zero-knowledge (NIZK) proving system with our specific relations. The second framework is more specialized, allowing instantiation with any zero-knowledge continuations. This second framework offers an efficient solution for ring VRF applications that necessitate the generation of multiple signatures for the same ring. We show that both of the frameworks are UC-secure.
 	
 	\item 	We construct a protocol called SpecialG  which  is a simple transformation of any Groth16 proof into a new proof by deploying the rerandomization idea of LegoSNARK ccGro16 \cite{LegoSNARK}. We show that SpecialG is a zero-knowledge continuation, making it suitable for deployment in instantiating our second framework.
 	 
  
 \end{itemize}
