
\section{Zero-knowledge continuations}
\label{sec:rvrf_cont}

We now develop our zero-knowledge continuation technique, called 
{\em special Groth16}, and build ring VRFs with fast amortized prover time.
Anonymity demands we finalize the amortized ``continuation'' multiple
times, with each time being unlinkable to the others, meaning our rVRF
stays zero-knowledge even with the continuation being reused.

% $$ \Lring = \Setst{ \compk, \comring }{
%  \exists \openpk,\openring \textrm{\ s.t.\ } 
%  \genfrac{}{}{0pt}{}{\PedVRF.\OpenKey(\compk,\openpk) \quad}{\,\, = \rVRF.\OpenRing(\comring,\openring)}
% } \mathperiod $$

% \smallskip
\subsection{Rerandomization}
% \label{sec:rvrf_groth16}

Zero-knowledge invariably comes from random blinding factors.
Zero-knowledge continuations need rerandomizable zkSNARKs,
meaning Groth16 \cite{Groth16}, but beyond rerandomization their
unlinkability demands hiding public inputs.
In our case, we ``specialize'' Groth16 to permit alteration of \openpk
in the $\PedVRF.\OpenKey$ invocation without reproving our heavy
$\rVRF.\OpenRing$ invocation.

In Groth16 \cite{Groth16}, we have an SRS $S$ consisting of curve
points in $\grE_1$ and $\grE_2$ that encode the circuit being proven.
We follow \cite{Groth16} in discussing the SRS $S$ in terms of
its ``toxic waste''
 $(\alpha,\beta,\delta,\gamma,\tau^1,\tau^2,\ldots) \in \F_q^*$.
In other words, we could write say $[ f(\tau)/\delta ]_1$ or $[\cdots]_2$
to denote an element of our SRS $S$ in $\grE_1$ or $\grE_2$ respectively,
computed by scalar multiplication of the Groth16 generators from
the toxic waste $\tau$ and $\delta$,
 but for which nobody knows the underlying $\tau$ or $\delta$ anymore.

In the SRS $S$, we distinguish the verifiers' string of elements
 $\chi_1,\ldots,\chi_k, [\alpha]_1 \in \grE_1$ and
 $[\beta]_2, [\gamma]_2, [\delta]_2 \in \grE_2$.
% as separate from the provers' much longer string of elements in $\grE_1$ and $\grE_2$.
A Groth16 \cite{Groth16} proof takes the form 
 $\pi = (A,B,C) \in \grE_1 \times \grE_2 \times \grE_1$.
A verifier then produces a $X = \sum_i^k x_i \chi_i \in \grE_1$ from
 the public inputs $x_i$ and then checks 
$$ e(A,B) = e([\alpha]_1, [\beta]_2) \cdot
 e(X, [\gamma]_2) \cdot e(C, [\delta]_2) \mathperiod $$

We need the rerandomization algorithm from \cite[Fig.~1]{RandomizationGroth16}:
% to build a zero-knowledge continuation:
% https://eprint.iacr.org/2020/811
% https://github.com/arkworks-rs/groth16/pull/16/files
% \algo{rerandomize}
An existing SNARK $(A,B,C)$ is transformed into a fresh
SNARK $(A',B',C')$ by sampling random $r_1,r_2 \in \F_p$ and computing
% $$
% A' = {1 \over r_1} A, \qquad
% B' = r_1 B + r_1 r_2 [\delta]_2, \qquad
% C' = C + r_2 A \mathperiod
% $$
$$ \begin{aligned}
A' &= {1 \over r_1} A \\
B' &= r_1 B + r_1 r_2 [\delta]_2 \\
C' &= C + r_2 A \mathperiod \\
\end{aligned} $$
At this point, our $x_i$ remain identical after rerandomization,
so $X$ links $(A,B,C)$ to $(A',B',C')$.
Alone rerandomization cannot alter public inputs $x_i$, so
we instead need an opaque public input point $X$, which then becomes
part of our proof and incurs its own separate proof of correctness.

We build {\em special Groth16} aka \SpecialG by adding one fresh
basepoint $\genB_\gamma$ independent from all others,
 including the $H_{\grE}(\msg)$ in \PedVRF.%
\footnote{Apply the underlying $H_\grE$ to an input outside the \msg domain for example.}
In the trusted setup, we build one additional prover SRS elements
$$ \genB_\delta := {\gamma\over\delta} \genB_\gamma $$
% Although $\genB_\gamma$ is independent,  we create $\genB_\delta$ during the trusted setup,  so the toxic waste $\gamma$ and $\delta$ remain secret.
After $\genB_\delta$ is created, our toxic waste $\gamma$ and $\delta$
disappear and subversion resistance could be checked like 
$$ e(\genB_\gamma, [\gamma]_2) = e(\genB_\delta, [\delta]_2) \mathperiod $$

We now have a zero-knowledge continuation $\pi = (X,A,B,C)$ from which
our algorithm $\SpecialG.\Reprove : (X,A,B,C) \mapsto ((X',A',B',C'); b)$ produces an
unlinkable instance $\pi' = (X',A',B',C')$ by
 first sampling random $b,r_1,r_2 \in \F_p$ and then computing
$$ \begin{aligned}
X' &= X + b \genB_\gamma \\
A' &= {1 \over r_1} A \\
B' &= r_1 B + r_1 r_2 [\delta]_2 \\
C' &= C + r_2 A + b \genB_\delta \mathperiod \\
\end{aligned} $$
As our two $b$ terms cancel in the pairings, our special Groth16
rerandomization reduces to the standard Groth16 rerandomization
construction above,
 except with $X$ now an opaque Pedersen commitment.

% TODO:  Should we be saying opaque less and Pedersen more below?

Along side opaque inputs in $X = \sum_i^k x_i \chi_i$,
our verifier should typically enforce specific values by assembling
a few {\em transparent} inputs $Y = \sum_i^l y_i \Upsilon_i$ themselves.
In particular, our ring VRF verifiers should enforce the commitment
\comring for $\ring$, even if they outsource computing \comring.
We thus write $\SpecialG.\Preprove : (\bar{y}, \bar{x}; \bar{\omega}) \mapsto (X,A,B,C)$
where $(A,B,C) = \primalgo{Groth16}.\Prove(\bar{y}, \bar{x}; \bar{\omega})$,
so a full \Prove algorithm works by composing \Preprove and \Reprove.

At this point $\SpecialG.\Verify(\bar{y}; (X',A',B',C') )$
 computes $X' + Y = X' + \sum_i^l y_i \Upsilon_i$ and checks
 the tuple $(X' + Y,A',B',C')$ like Groth16 does,
$$ e(A',B') = e([\alpha]_1, [\beta]_2) \cdot
 e(X' + Y, [\gamma]_2) \cdot e(C', [\delta]_2) \mathperiod $$
As our verifier does not build $X'$ themselves, we prove nothing
with this pairing equation unless the verifier separately checks
 a proof-of-knowledge that $X' = b \genB_\gamma + \sum_i^k x_i \chi_i$
 for some unknown $b,\bar{x}$.

\begin{lemma}\label{lem:unlinkable}
Our rerandomization procedure % $(X,A,B,C) \mapsto (X',A',B',C')$
transforms honestly generated zero-knowledge continuations $(X,A,B,C)$
into identically distributed zero-knowledge continuations $(X',A',B',C')$,
with identical opaque inputs $x_1,\ldots,x_k$ and
 identical transparent inputs $y_1,\ldots,y_l$.
\end{lemma}

\begin{proof}[Proof idea.]
Adapt the proof of Theorem 3 in \cite[Appendix C, pp. 31]{RandomizationGroth16}.
\end{proof}

% \begin{corollary}\label{cor:unlinkable}
%	If $\sigma'$ and $\sigma''$ are \PedVRF{}s then ???
% \end{corollary}

All told, our opaque rerandomization trick converts any conventional
Groth16 zkSNARK $\pi$ for $\rVRF.\OpenRing$ into a zkSNARK $\pi'$
with inputs split into a transparent part $\bar{y}$ vs opaque unlinkable part $X$.
% We explore two concrete $\pi$ proposals below.

Importantly, rerandomization requires only
 four scalar multiplications on $\ecE_1$ and
 two scalar multiplications on $\ecE_2$,
which  BLS12 curves make roughly equivalent to
 eight scalar multiplications on $\ecE_1$.

\begin{lemma}\label{lem:knowledge_soundness}
Assuming AGM plus the $(2n-1,n-1)$-DLOG assumption,
any zero-knowledge continuation with circuit size less than $n$
satisfies knowledge soundness.
\end{lemma}

\begin{proof}[Proof idea.]
As our \Prove is composed from \Preprove and \Reprove, our challenger
learns the actual public input wire values and blinding factors.
Adapt the proof of Theorem 2 in \cite[\S3, pp. 9]{RandomizationGroth16},
observing that $K_\gamma$ and $K_\delta$ never interact with other elements. 
%TODO: Alistair or Oana, Do we even need the first sentense here?  nything more to say about the second?
\end{proof}

In fact, one could prove zero-knowledge continuations satisfy
weak white-box simulation extractability,  % under similar restrictions,
much like Theorem 1 in \cite[\S3, pp. 8 \& 11]{RandomizationGroth16}.
%TODO:  Alistair or Oana, what the hell did I mean by this?  -Jeff
We depend upon the specific simulator though, which itself increases
our dependence upon the usage of the zero knowledge continuation.


\subsection{Continuation}
\label{subsec:rvrf_faster}

% TODO \PedVRF.\OpenKey(\compk,\openpk)

\def\longeq{=\mathrel{\mkern-10mu}=}% {=\joinrel=} % https://tex.stackexchange.com/questions/35404/is-there-a-wider-equal-sign
We describe a much faster choice \pifast for \piring
that sets $x_1 \longeq \sk$ and $x_0 \longeq \comring$ so that taking
 $\genG \longeq \chi_1$, $\genB \longeq \genB_\gamma$, and $\openpk \longeq b$
in \PedVRF yields an incredibly fast amortized ring VRF prover.
Also \PedVRF itself proves knowledge of $X' =  \sk\, \chi_1 + b \genB_\gamma $,
 as required by $\SpecialG\Verify$.
% $$ X' + Y = \comring\, \Upsilon_1 + \sk\, \chi_1 + b \genB_\gamma $$

A priori, we do not know $\chi_1$ during the trusted setup for $\pifast$,
which prevents computing $\pk = \sk\, \chi_1$ inside $\pifast$.
Instead, we propose $\ring$ contain commitments to $\sk$ over
some Jubjub curve $\ecJ$.  

We know the large subgroup $\grJ$ of $\ecJ$ typically has smaller prime
order $p_\grJ$ than $\grE$, itself due to $\ecJ$ being an Edwards curve. 
We thus choose $\sk_0,\sk_1 < p_\grJ$ so that
 $\PedVRF.\sk = \sk_0 + \sk_1 \, 2^{128} \mod p_\grE$, and so
our $\rVRF.\KeyGen$ \eprint{returns}{shall now return}
a secret key of the form $\rVRF.\sk = (\sk_0,\sk_1,d)$
 with $d \leftsample \F_{p_\grJ}$ and
a public key of the form
 $\rVRF.\pk = \sk_0\, \genJ_0 + \sk_1\, \genJ_1 + d \genJ_2$,
for some independent $\genJ_0,\genJ_1,\genJ_2$. % (see \S\ref{subsec:AML_KYC}).
\footnote{Interestingly we avoid range proofs for $\sk_1$ and $\sk_2$ by this independence.}

$$ \Lfast^\inner = \Setst{ \sk_0 + 2^{128} \sk_1, \comring }{
 \eprint{ \exists d,\openring \textrm{\ s.t.\ } }{}
 % 0 < \sk_0,\sk_1 < 2^{128} \textrm{\ and\ } 
 \genfrac{}{}{0pt}{}{ \eprint{\rVRF.}{}\OpenRing(\comring,\openring) }{ \,\, = \sk_0 \genJ_0 + \sk_1 \genJ_1 + d \genJ_2 }
} $$ % \mathperiod 

Applying our rerandomization \Reprove to $\pifast^\inner$ with opaque input
yields a zkSNARK $\pifast$ with the extra $\PedVRF.\OpenKey$ arithmetic to
have exactly the form $\piring$.

We explain later in \S\ref{sec:ring_hiding} how one could
choose $\chi_1$ independent before doing the trusted setup,
 and then wire $\chi_1$ into $\pifast$ inside $C$.
In this case, we could prove $\pk = \sk\, \chi_1$ inside $\pifast$, but then
non-native arithmetic makes $\pifast$ far slower.

At this point, $\PedVRF.\Sign$ requires two scalar multiplications on $\ecE_1$
 and two on $\ecE'$,
so together with rerandomization costing four scalar multiplications
on $\ecE_1$ and two on $\ecE_2$, our amortized prover time
 comes under 12 scalar multiplications on typical $\ecE_1$ curves. 
We expect the three pairings dominate verifier time, but
 verifiers also need five scalar multiplications on $\ecE_1$.

As an aside, one could construct a second faster curve with the same
group order as $\grE$, which speeds up two scalar multiplications
 in both the prover and verifier. 

Importantly, our fast ring VRF' amortized prover time now rivals
group signature schemes' performance.  We hope this ends the temptation
to deploy group signature like constructions where the deanonymization vectors matter.

\begin{theorem}\label{thm:knowledge_soundness}
\rVRF instantiated with \pifast and \PedVRF satisfies knowledge soundness.
\end{theorem}

\begin{proof}[Proof stetch.]
An extractor for \PedVRF reveals the opening of $X$ for us,
so our result follows from Lemma \ref{lem:knowledge_soundness}.
\end{proof}

% BEGIN TODO: Oana

% \begin{corollary}\label{cor:???}
% Our Pedersen ring VRF instantiated with \pifast satisfies ring unforgability and uniqueness.
% \end{corollary}

% \begin{theorem}\label{thm:pifast_anonymity}
% Our ring VRF \rVRF using \pifast and \PedVRF satisfies zero-knowledge.
% \end{theorem}
%
% \begin{proof}[Proof stetch.]
% Assuming the same \comring, we know the zero-knowledge continuations
% are identically distributed by Lemma \ref{lem:unlinkable}.
% It follows the typical simulator for \PedVRF ... WHAT???
% \end{proof}

% \begin{corollary}\label{cor:???}
% Our Pedersen ring VRF instantiated with \pifast satisfies ring anonymity.
% \end{corollary}

% END TODO: Oana

