
\def\eprintsmallskip{\smallskip}{}%
We introduce a new cryptographic primitive, aptly named
\emph{ring verifiable random functions (ring VRF)}.
% Anonymized
%\eprint{Ring VRFs are}{We introduce ring VRFs, which are}
Ring VRFs are ring signatures that also prove correct evaluation
of a pseudo-random function (PRF) by a signer, while hiding the  signer's
identity within some set of signers, known as the ring. 
We design a ring VRF protocol that has an efficient instantiation with our novel {\em zero-knowledge continuation} technique.
% \eprint{We propose ring VRFs as a natural fulcrum around which a diverse array of zkSNARK circuits turn, making them an ideal target for optimization and eventually standards.}{}
We define the security of ring VRFs in the universally composable (UC) model and show that our protocol is UC secure.

\eprintsmallskip
%We need a motivation or a reason why we build/need zero-knowledge cont. technique. So I suggest the following explanation. In the previous version, the connection was unclear.
A signer in our ring VRF construction generates two proofs: one for the correct evaluation of the message and one for the ring membership. We succeed to have the former with discrete logarithm equality (DLEQ) proof. However, the latter requires a more complex relation. Therefore,  we demonstrate our zero-knowledge continuation technique for the ring membership proof.  Our technique lets a signer reuse the same ring membership proof with different messages but the same ring. 
%NOTE Is the next sentence clear for everyone. Maybe we should restate it
In a nutshell, it works by adjusting a Groth16 trusted setup to hide public inputs when rerandomizing the Groth16.  
Our optimization reduces the prover's computations to eight scalar multiplications in the first group and two scalar multiplications in the second group of the bilinear pairing used in Groth16.
%NOTE: I would remove the claim below because we don't have any comparison related to group signatures in the paper.
%, making it the only ring signature
%with performance competitive with group signatures.

\eprintsmallskip
Ring VRF has functional real-world applications.
A ring VRF signer produces a unique identity for any given context via signed message but remains
unlinkable between different contexts.  These unlinkable but unique
pseudonyms provide a better balance between user privacy and service provider or social interests than attribute-based credentials.
%NOTE 'a far better' is a strong claim that if we write it we should strongly backup. Also IRMA is not something very well known, so it is better to remove it from here and talk in intro since it does not give much information
%a far better balance between user privacy and service
%provider or social interests than attribute-based credentials like %IRMA credentials.
In addition,
ring VRFs support anonymous rationing or rate-limiting resource
consumption that winds up vastly more flexible and efficient than the protocols that rely on compensation.
We explain the usage of ring VRF in these applications in the end.
