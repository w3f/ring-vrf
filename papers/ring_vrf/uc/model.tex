\section{Security Model of Ring VRF}


In this section, we describe the security of our new cryptographic primitive ring VRF. First, we describe the basic ring VRF in the real world and in the ideal world. Second, we extend the basic ring VRF by adding a new property that we call secret evaluation. 


\paragraph{Ring VRF (Basic Definition):} A ring VRF operates like a VRF but only proves its key comes from a specific list without giving any information about which specific key. We define the ring VRF functionality $ \fgvrf $ in Figure \ref{f:gvrf}. The functionality lets parties generate a key (Key Generation), evaluate a message with the party's key (VRF Evaluation), prove that the evaluation is executed by one of the keys (VRF evaluation and proof) and verify the evaluation without knowing the key used for the evaluation (Verification). We also define linking procedures in $ \fgvrf $ to link an evaluation and a proof with its associated key. So, if a party wants to reveal its identity at some point, it can use the linking process to prove that the evaluation is executed with its key (Linking proof). Later on, anyone can verify the linking proof (Linking verification).
%TODO Secret Evaluation
\begin{figure}\scriptsize
	\begin{tcolorbox}
		{  $ \fgvrf $ runs two PPT algorithms $ \gen_W$ and $\gen_{sign} $ during the execution.
			
			 \begin{description}
				
				\item[Key Generation.] upon receiving a message $(\msg{keygen}, \sid)$ from a party $\user_i$, send $(\msg{keygen}, \sid, \user_i)$ to the simulator $\simulator$.
				Upon receiving a message $(\msg{verificationkey}, \sid, \pkrvrf)$ from $\simulator$, verify that $\pkrvrf$ has not been recorded before; then, store in the table $\vklist$, under $\user_i$, the value $\pkrvrf$.
				Return $(\msg{verificationkey}, \sid, \pkrvrf)$ to $ \user_i$.
				
				\item[Malicious Key Generation.] upon receiving a message $(\msg{keygen}, \sid, \pkrvrf)$ from $\simulator$, verify that $\pkrvrf$ was not yet recorded, and if so record in the table $\vklist$ the value $\pkrvrf$ under $\simulator$. Else, ignore the message.
				
				%\item[Honest Ring VRF Evaluation.] upon receiving a message $(\msg{eval}, \sid, \pkring, \pkrvrf_i, m)$ from $\user_i$, verify that 
				%$\pkrvrf_i \in \pkring$ 
				%and  
				%there exists $ \pkrvrf_i $ in $\vklist $ associated with $ \user_i $. If that was not the case, just ignore the request.
				%If there exists no $ W $ such that $ \anonymouskeymap[W] = (m, \pkring, \pkrvrf_i) $, let $ W \sample \bin^\secpar $ and  $y \sample \bin^{\ell_\rvrf}$. Then, set $ \evaluationslist[m, W] = y$ and $ \anonymouskeymap[W] = (m, \pkring,\pkrvrf_i) $.
				%Return $(\msg{evaluated}, \sid, \pkring, m, W, y)$ to $ \user_i $.
				%The functionality does not check whether the evaluater's public key is in the ring because here we consider m, \pkring as an input of the evaluation which is evaluated by a party who is not neccesarily in the ring. 
				\item[Malicious Ring VRF Evaluation.] upon receiving a message $(\msg{eval}, \sid, \pkring, \pkrvrf_i, W, m)$ from $\sim$, verify that $ \pkrvrf_i $ has not been recorded under an honest party's key.
			    If it is the case record in the table $\vklist$ the value $\pkrvrf_i$ under $\simulator$. Else, ignore the request.  If $ \counter[m,\pkring] $ does not exist, initiate $ \counter[m,\pkring] = 0 $.
			    If there exists $ W $ such that $ \anonymouskeymap[W] = (m',\pkring', \pkrvrf)$ then do the following:
			    \begin{itemize}
			    	\item if $(m', \pkring', \pkrvrf) \neq  (m, \pkring, \pkrvrf_i)  $, ignore the request,
			    	\item else obtain $ y = \evaluationslist[m, W]   $. 
			    \end{itemize}
				If there exists no $ W $ such that $ \anonymouskeymap[W] = (m, \pkring, \pkrvrf_i) $, let   $y \sample \bin^{\ell_\rvrf}$ and increment $ \counter[m,\pkring] $. Then, set $ \evaluationslist[m, W] = y$, $ \anonymouskeymap[W] = (m, \pkring,\pkrvrf_i) $ .
				Return $(\msg{evaluated}, \sid, \pkring, m, W, y)$ to $ \user_i $.
				
				
				
				
				\item[Honest Ring VRF Signature.] upon receiving a message $(\msg{sign}, \sid, \pkring, \pkrvrf_i, m)$ from $\user_i$, verify that $\pkrvrf_i \in \pkring$ and that there exists a public key $\pkrvrf_i$ associated to $\user_i$ in the table $ \vklist $. If that wasn't the case, just ignore the request. 	
				If there exists no $ W' $ such that $ \anonymouskeymap[W'] = (m, \pkring, \pkrvrf_i) $, run $ \gen_W(\pkring, \pkrvrf_i, m) \rightarrow W$. Then, let $y \sample \bin^{\ell_\rvrf}$ and set $ \anonymouskeymap[W] = (m, \pkring,\pkrvrf_i) $ and set $ \evaluationslist[m, W] = y$.
%					\begin{itemize}
%						%\item If there exists $ W \in  \anonymouskeymap  $, abort.
%						\item Else 
%						%TODO define what \in \anonymouskeymap mean
%					\end{itemize}
%			    \end{itemize}
				Obtain $ W, y $ where  $ \evaluationslist[m, W] = y$, $ \anonymouskeymap[W] = (m, \pkring,\pkrvrf_i) $ and run  $ \gen_{sign}(\pkring, W, m) \rightarrow \sigma $. Verify that $ [m, W, \sigma, 0] $ is not recorded. If not, abort. Otherwise, record $ [m, W, \sigma, 1] $. Return $(\msg{signature}, \sid, \pkring,W,m, y, \sigma)$ to $\user_i$.
				
				%\item[Malicious VRF evaluation.] upon receiving a message $(\msg{evalprove}, \sid, \pkring, m)$ from $\simulator$, check that $\vklist$ has a public key associated to $\simulator$. If not, ignore the request. If $\evaluationslist[\pkring, m][\simulator]$ is not set, sample $y \sample \bin^{\ell(\secpar)}$ and set $\evaluationslist[\pkring, m][\simulator] \defeq y$ (and $\signaturelist[\pkring,m]$ to $\emptyset$). If $\signaturelist[\pkring, m]$ contains a proof (i.e., if $\signaturelist[\pkring, m]$ is not empty), return $(\msg{evaluated}, \sid, y)$ to $\simulator$. Else, ignore the request.
				
				%\item[Verification.] upon receiving a message $(\msg{verify}, \sid, \pkring, m, y, \sigma)$, from any party forward the message to the simulator. If there exists a $\pkrvrf_i$ among the values of \texttt{verification\_keys}, and there exists $\sigma \in \signaturelist[\pkring, m]$, set $b = 1$. Else, set $b =0$. Finally, output $(\msg{verified}, \sid, \pkring, m, y, \sigma, b)$.
				\item[Ring VRF Verification.] upon receiving a message $(\msg{verify}, \sid, \pkring,W, m, \sigma)$ from a party, relay the message $(\msg{verify}, \sid, \pkring,W, m, \sigma)$ to $ \simulator $ and receive back the message $(\msg{verified}, \sid, \pkring,W, m, \sigma, b_{\simulator}, \pkrvrf_\simulator)$. Then do the following: 
				\begin{enumerate}[label={{Cond.-} }{{\arabic*}}, start = 1]
					\item If there exits a record $ [m,W,\sigma, 1] $, set $ b = 1 $. (This condition guarantees the completeness meaning that if  $ W $ is an anonymous key that is generated for the ring $ \pkring $ and  the message $ m $ and the signature $ \sigma $ is legitimately generated for $ m, W $, then the verification succeeds.)
					\item Else if $ \pkrvrf  $ is an honest verification key where $ \anonymouskeymap[W] = (.,., \pkrvrf) $ and there exists no record $ [m, \pkring, W, \sigma', 1] $ for any $ \sigma' $, then let $ b= 0  $.
					(This condition guarantees unforgeability meaning that if an honest party never signs a message $ m $ for a ring $ \pkring $, then the verification fails.)\label{cond:forgery}
					\item Else if there exists a record  such as $ [m,W,\sigma, b'] $, set $ b = b' $. (This condition guarantees consistency meaning that all identical verification requests will output the same $ b $) \label{cond:consistency}
					\item Else if $ \pkrvrf  $ is an honest verification key where $ \anonymouskeymap[W] = (.,., \pkrvrf) $ and there exists a record $ [m, W, \sigma', 1] $ for any $ \sigma' $, then let $ b= b_{\simulator} $ and record $ [m, W,\sigma, b_{\simulator}] $. (This condition guarantees that if $ m $ is signed by an honest party for the ring $ \pkring $ at some point and the signature is $ \sigma' \neq \sigma $, then the decision of verification is up to the adversary) \label{cond:differentsignature}
					\item \label{cond:forgerymalicious}Else if there exists $ \anonymouskeymap[W] = (m', \pkring',.)  $ where $ (m', \pkring') \neq (m, \pkring) $ or $ \counter[m, \pkring] > |\pkring_m| $ where $ \pkring_m $ is a set of keys in $ \pkring $ which are not honest or $ b_{\simulator} = 0 $ or $ \pkrvrf_\simulator $ belongs to an honest party, set $ b = 0 $ and record $ [m, \pkring,W,\sigma, 0] $. (This condition guarantees that if $ W $ is an anonymous key of a different message and ring or the number of anonymous keys of malicious parties in $ \pkring $ is more than their number or     $ \simulator $ does not verify $ \sigma $, then the verification fails.)
					\item Else set $ b = 1 $. Set $ \evaluationslist[m, W]\sample \bin^{\ell_\rvrf}$, $ \anonymouskeymap[W]  = (m, \pkring, \pkrvrf_\simulator)$ and $ \counter[m, \pkring, \pkrvrf_\simulator] = 0 $ if they are not defined before. Increment $ \counter[m, \pkring, \pkrvrf_\simulator]  $. \label{cond:advsignature}
				\end{enumerate}
				In the end, if $ b = 0 $, set $ \out = \emptyset $. Otherwise,  set $ \out = \evaluationslist[m, W]$. 		Finally, output $(\msg{verified}, \sid, \pkring,W, m, \sigma, \out, b)$ to the party.
				
			\end{description}
		
			
		}
	\end{tcolorbox}
	\caption{Functionality $\fgvrf$.\label{f:gvrf}}
\end{figure}
	


\begin{figure}\scriptsize
	\begin{tcolorbox}
		{  This part of $ \fgvrf $ for the parties who want to show that they generate a particular ring signature.
			
		
			\begin{description}
				\item[Linking signature.] upon receiving a message $(\msg{link}, \sid, \pkring, \pkrvrf_i, W, m,\sigma)$ from $\user_i$, check that $\pkrvrf_i $ is associated to $\user_i$ in $ \vklist $, $ \anonymouskeymap[W] = (m, \pkring, \pkrvrf_i) $ and 
				check whether $ [m, W, \sigma, 1] $ is stored. If any of them fails, ignore the request. Otherwise,
				send $(\msg{link}, \sid, \pkring, W, m, y)$ to $\simulator$. Upon receiving $(\msg{linkproof}, \sid, \pkring, W, m, y, \hat \sigma)$ from $\simulator$, verify that $ [m, \pkring, \pkrvrf_i, \sigma, \hat{\sigma}, 0] $ is not stored in $ \Linklist $. If not, abort. Otherwise,  record $\hat\sigma$ to $[m, \pkring, \pkrvrf_i,\sigma, \hat{\sigma}, 1]$ to $ \Linklist $ and return $(\msg{linked}, \sid, \pkring, \pkrvrf_i,W, m, y,\sigma, \hat\sigma)$ to $\user_i$.
				%\item[Malicious linking proof.] upon receiving a message $(\msg{link}, \sid, \pkring, m, y)$ from $\simulator$, check that $\vklist$ has a key set for $\simulator$, and that it is in $R$.
				%Check that $\evaluationslist[\pkring, m][\simulator] = y$.
				%If any of the above is not satisfied, ignore the request.
				%Return $(\msg{linked}, \sid, y)$ to $\simulator$.
				\item[Linking verification.] upon receiving a message $(\msg{verifylink}, \sid, \pkrvrf_i, \pkring, W, m,\sigma,\hat\sigma)$ from any party forward the message to the simulator and receive back  the message $(\msg{verified}, \sid, \pkrvrf_i, \pkring, W,m, \sigma,\hat\sigma,  b_{\simulator})$. Then do the following:
				
				\begin{itemize}
					\item If there exits a record $ [m, \pkring,\pkrvrf_i,\sigma,\hat\sigma, 1] $ in $ \Linklist $, set $ b = 1 $ and $ \pk = \pkrvrf $. (This condition guarantees the completeness.)
					\item Else if $ \pkvrf_i $ is a key of an honest party and there exits no record such as $ [m, \pkring,\pkrvrf_i,\sigma,\hat\sigma',  1] $ for any  $  \hat\sigma'$, then set $ b = 0 $ and record $ [m, \pkring,\pkrvrf_i,\sigma,\hat\sigma,  0] $. (This condition guarantees unforgeability meaning that if an honest party never signs a message $ m $ in the linking signature, then the verification fails.)
					\item Else if there exists a record  such as $ [m, \pkring,\pkrvrf_i,\sigma,\hat\sigma,  b'] $, set $ b = b' $. 
					\item Else set $ b = b_{\simulator} $ and record $ [m, \pkring,\pkrvrf_i,\sigma,\hat\sigma,  1] $. 
				\end{itemize}
				
				Return $(\msg{verified}, \sid, \pkrvrf_i, \pkring, m, \hat\sigma, b).$ to the party.
			\end{description}
		}
	\end{tcolorbox}
	\caption{Functionality $\fgvrf$.\label{f:gvrf}}
\end{figure}



In a nutshell, the functionality $\fgvrf$, when given as input a message $m$ and a key set $\pkring$ of participant, allows to create $n$ possible different outputs pseudo-random that appear independent from the inputs. The output can be verified to have been computed correctly by one of the participants in $\pkring$ without revealing who they are. At a later stage, the author of the VRF output can prove that the output was generated by them and no other participant could have done so.

Below, we define the real-world execution of the ring VRF.
\begin{definition}[Ring-VRF (rVRF)]\label{def:ringvrf}
	Ring VRF is a VRF with a deterministic function $ F(.,.):\{0,1\}^\kappa \times\{0,1\}^* \rightarrow \{0,1\}^{\ell_\rvrf} $ and with the following algorithms:
	
	\begin{itemize}
		\item $ \rvrf.\keygen(1^\kappa) \rightarrow (\skrvrf,\pkrvrf)$ where $ \kappa $ is the security parameter,
	\end{itemize}
	Given list of public keys $ \pkring = \set{\pkrvrf_1, \pkrvrf_2, \ldots, \pkrvrf_n}$, a message $ m \in \{0,1\}^* $
	\begin{itemize}
		\item $ \rvrf.\eval(\skrvrf_i, \pkring, m)\rightarrow y $ where $ y = F(\skrvrf,\pkring,m) $,
		\item $ \rvrf.\evalprove(\skrvrf_i, \pkring, m)\rightarrow (F(\skrvrf,\pkring,m),\pi) $ where  $ \pi $ is a proof for the evaluation.
		\item $ \rvrf.\verify(\pkring, m, y,\pi) \rightarrow  b$ where $ b \in \{0,1\} $. $ b =1 $ means verified and $ b = 0 $ means not verified.
		\item $ \rvrf.\link(\skrvrf_i, \pkring,m,y, \pi) \rightarrow \pi_{\link} $ where  $ \pi^\link $ is a proof linking the public key of the producer of $ y $. 
		\item $ \rvrf.\link\verify(\pkring, \skrvrf_i, m, y, \pi,, \pi_{\link})\rightarrow b$ where $ b \in \{0,1\} $. $ b =1 $ means verified and $ b = 0 $ means not verified.
	\end{itemize}
	
\end{definition}
\paragraph{Ring VRF with Secret Evaluation:} 


Below, we define the real-world execution of the ring VRF with secret evaluation.
\begin{definition}[Ring-VRF (rVRF)]\label{def:ringvrfse}
	Ring VRF with secret evaluation is two VRFs with a deterministic function $ F(.,.):\{0,1\}^\kappa \times\{0,1\}^* \rightarrow \{0,1\}^{\ell_\rvrf} $ and$ F_s(.,.):\{0,1\}^\kappa \times\{0,1\}^* \rightarrow \{0,1\}^{\ell_\rvrf} $. It consists of the algorithms of ring VRF defined in Definition \ref{def:ringvrf} and additionally the following algorithms:
	
	Given list of public keys $ \pkring = \set{\pkrvrf_1, \pkrvrf_2, \ldots, \pkrvrf_n}$, a message $ m \in \{0,1\}^* $
	\begin{itemize}
		\item $ \rvrf.\secreteval(\skrvrf_i, \pkring, m)\rightarrow \omega $ where $ \omega = F_s(\skrvrf,\pkring,m) $,
		\item $ \rvrf.\secretprove(\skrvrf_i, \pkring, m)\rightarrow \pi / \perp $ where  $ \pi $ is a proof for the secret evaluation. If $ (F_s(\skrvrf,\pkring,m), (\pkring, m)) \notin \rel  $, it outputs $ \perp $.
		\item $ \rvrf.\secretverify(\pkring, m,\pi) \rightarrow  b$ where $ b \in \{0,1\} $. $ b =1 $ means verified and $ b = 0 $ means not verified.
	\end{itemize}
	
\end{definition}

\begin{figure}
	\scriptsize
	\begin{tcolorbox}
	{
			%\par\hrulefill\\
			 $ \fgvrf^{zk} $ for a relation $ \mathcal{R} $ behaves exactly as $ \fgvrf $ and  it additionally does the following:
			\begin{description}
				\item[Secret Evaluation.] upon receiving a message $(\msg{secreteval}, \sid, \pkring,\pkvrf_i, m)$ from $\user_i$, verify that $\pkrvrf_i$ is in $ \pkring $ and that there exists a key $\user_i$ with an associated public key $\pkrvrf_i$ in $\vklist$. If that was not the case, just ignore the request.
				If $\evaluationslist[\pkring, m][\pkrvrf_i]$ is empty, then sample a new $y \sample \bin^{\ell(\secpar)}$ and  store $\evaluationslist[\pkring, m][\pkrvrf_i] = y$. In any case, sample a random element $ \eta  \sample \bin^{\ell(\secpar)} $ and store $ \evaluationsecretlist[\pkring, m][\pkvrf_i] = \eta $. Create a set $ \proofzklist[\pkring,m][\pkvrf_i] = \emptylist $, if it does not exist. Return $(\msg{evaluated}, sid, y, \eta)$ to $ \user_i $.
				
				\item[Secret evaluation and proof.] upon receiving a message $(\msg{secretprove}, \sid, \pkring, \pkvrf_i, m)$ from $\user_i$, verify that $\pkrvrf_i \in \pkring $ and  that there exists a key $ \user_i $ with an associated public key $ \pkrvrf_i $ in $ \vklist $. (If that wasn't the case, just ignore the request.). Obtain $ (y, \eta) $ from $\evaluationslist[\pkring, m][\pkvrf_i]$ and $\evaluationsecretlist[\pkring, m][\pkvrf_i]$. If they are not defined execute the steps in VRF evaluation and secret evaluation.
				If $ ((m, y, \pkring),(\eta,\pkvrf_i)) \notin \mathcal{R} $, ignore the request.  Otherwise, send $(\msg{zkprove}, \sid, \pkring, m, y)$ to $\simulator$. Upon receiving $(\msg{zkproof}, \sid, \pkring, m, y \pi^{zk})$ as a response from $\simulator$,  add $ \pi^{zk} $ to  the set $\proofzklist[\pkring, m][\pkvrf_i]$. Return $(\msg{zkproof}, \sid, y, \eta, \pi^{zk})$ to $\user_i$.
				
				
				\item[Secret Verification.] upon receiving a message $(\msg{secretverify}, \sid, \pkring, m, \pi^{zk})$, from a party, check whether there exists  $ \pkvrf_i \in \vklist $ such that $ \proofzklist[\pkring,m][\pkvrf_i] = \pi^{zk} $, $ \evaluationslist[\pkring,m][\pkrvrf_i] = y $ and $ \evaluationsecretlist[\pkring,m][\pkrvrf_i] $. If there exists, set $b = 1$. Else, set $b =0$. Finally, output $(\msg{verification}, \sid, \pkring, m, \pi^{zk}, b)$.
			\end{description}
		}
	\end{tcolorbox}
	\caption{Functionality  $ \fgvrf^{zk} $.\label{f:gvrfzk}}
\end{figure}
