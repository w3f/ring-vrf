\documentclass{article}

\usepackage{hyperref}
\usepackage{pdfpages}

\title{Ring VRFs from zk continuations: \\ Ethical identity and anonymous rationing}
\author{Jeffrey Burdges \and Handan Kilinc-Alper \and Alistair Stewart \and Sergey Vasilyev}
\date{}

\begin{document}

\maketitle

Jeffrey Burdges would love to present our zero-knowledge continuation
technique, and how it enables extremely efficient ring verifiable
random functions (ring VRFs).

\smallskip

Our zero-knowledge continuation reuses a previously proven Groth16 SNARK
without reproving.  We wire this SNARKs inputs into another signature,
or another SNARK, without linking the multiple reuses.

\smallskip

Almost all prover time could therefore by amortized away by the application,
dramatically improving performance.  In particular, our ring VRF has a marginal
prover CPU cost of roughly 12 scalar multiplications!

\smallskip

Jeff would also discuss how ring VRFs yield one of the safest identity
schemes, as well as their applications to rate limiting,
including the idea of anonymous ration cards.

\medskip

At \quad \url{https://www.youtube.com/watch?v=0i_5RW3tkVM} \quad you'll
find Jeff's 20 minute talk on this material at the
ZK Summit in Berlin.

\medskip

For RWC, Jeff envisions doing a more polished talk, which better clarifies
the problem space, and focuses more on ring VRF input structure,
specifically:
\begin{itemize}
\item The abuse risks of attribute based credentials like IRMA.
 And why ring VRFs yield a safer identity scheme.
\item How verifier abuse demands verifier certification, which IRMA does
 not address, but ring VRFs do.  In particular, a root key in the VRF input
 permits certifying verifiers without adding CA complexities.
\item We'll better describe how a ``ring'' based credential improves
 the audit trail, and hence user trust, over anonymized certificate schemes
 like IRMA.  We think this becomes extremely important if used for rationing,
 as anonymous credentials won't admit analogs of certificate transparency.
\end{itemize}

\noindent Additional refinements include:
\begin{itemize}
\item An RWC audience likely wants clearer, but not deeper, comments
 on the trusted setup than required for ZK Summit.
\item We regret abandoning its jokes, but we shall omit our secret single
 leader election (SSLEs) application Sassafras, as it takes us too far
 away from the privacy story, and RWC's audience feels less interested in
 consensus protocols.  We'll see if cards against humanity still fits.
\item We repeatedly cautioned the blockchain heavy ZK Summit audience
 that zero-knowledge continuations cannot amortize crypto-currency
 UTXOs, but this feels unnecessary for a RWC audience.
\end{itemize}


\includepdf[pages=-]{pets.pdf}


\end{document}


\endinput



Anonymized ring VRFs are ring signatures that prove correct evaluation
of some authorized signer's PRF while hiding the specific signer's
identity within some set of possible signers, known as the ring.

% \eprint{We propose ring VRFs as a natural fulcrum around which a diverse array of zkSNARK circuits turn, making them an ideal target for optimization and eventually standards.}{}

We demonstrate a reusable {\em zero-knowledge continuation} technique,
which works by adjusting a Groth16 trusted setup to hide public inputs
when rerandomizing the Groth16.  We then build ring VRFs that amortize
expensive ring membership proofs across many ring VRF signatures.
%
Incredibly, our ring VRF needs only eight $\mathcal{G}_1$ and two
$\mathcal{G}_2$ scalar multiplications, making it the only ring signature
with performance competitive with group signatures.

Ring VRFs produce a unique identity for any give context but remain
unlinkable between different contexts.  These unlinkable but unique
pseudonyms provide a far better balance between user privacy and service
provider or social interests than attribute based credentials like IRMA.

Ring VRFs support anonymously rationing or rate limiting resource
consumption that winds up vastly more efficient than purchases via money-like protocols.


